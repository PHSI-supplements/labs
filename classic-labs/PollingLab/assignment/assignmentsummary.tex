Please familiarize yourself with the entire assignment before beginning.
There are three parts to this assignment.

The first part, which is intended to be completed during lab time under the guidance of a TA, is to populate the keypad's lookup table (Section~\ref{subsec:populatekeypad}), determine the base addresses for the memory-mapped I/O register banks used in this assignment (Section~\ref{subsec:baseAddresses}), and introduce code to detect keypresses on the numeric keypad (Section~\ref{subsec:detectKeyAction}).
After that, there are two major parts to this assignment, and they can be completed in either order.

\subsection{Memory-Mapped Input/Output}

The starter code contains functions that provide access to the buttons, switches, keypad, LEDs, and display module.
Initially, these functions make use of functions available in the CowPi library.
In Section~\ref{sec:MemMapIO}, you will re-implement these functions using memory-mapped I/O\@.

\subsection{Implementing a Simple System using Polling}

In Section~\ref{sec:SimpleSystem}, you will implement a simple system that makes use of your hardware kit's buttons, switches, keypad, LEDs, and display module.
You will implement this system by polling inputs.

\subsection{Separation of Concerns}

The code that interfaces directly with the hardware is in one file (\textit{io\_functions.c}), and the logic to implement your simple embedded system is in another file (\textit{number\_builder.c}).
With embedded systems, it isn't always easy to have such a clean separation of concerns, but when you can, you should.
As you complete this assignment, place all of your system's logic in \textit{number\_builder.c},
and place all of the code that accesses the memory-mapped I/O registers in \textit{io\_functions.c}.

\subsection{Constraints} \label{subsec:constraints}

You may use any features that are part of the C standard if they are supported by the compiler.
You may use the constants and functions provided in the starter code
(to receive credit for the memory-mapped I/O portion of this lab, you will need to re-implement the I/O functions).

You must detect inputs using polling;
you may not use interrupts.

All of your memory-mapped I/O code must go in \textit{io\_functions.c}.
All of your system's logic code must go in \textit{number\_builder.c}.

\subsubsection{Constraints on the Arduino core}

You may use the \function{delayMicroseconds()} function to introduce a 1--2$\mu$s delay in \function{get_keypress()} and to introduce a 1$\mu$s delay in \function{send_halfbyte()};
you may not use \function{delayMicroseconds()} for any other purpose.
You may use the \function{millis()} function to determine how much time (in milliseconds) has elapsed.

You may not use any other libraries, functions, macros, types, or constants from the Arduino core.\footnote{\url{https://www.arduino.cc/reference/en/}}

\subsubsection{Constraints on AVR-libc}

You may not use any AVR-specific functions, macros, types, or constants of avr-libc.\footnote{\url{https://www.nongnu.org/avr-libc/user-manual/index.html}}

\subsubsection{Constraints on the CowPi library}

To receive credit for the memory-mapped I/O portion of this lab, all input and
output must be accomplished using the memory-mapped I/O registers.
The starter code calls \function{cowpi_setup()} for you.
You also may use \href{https://cow-pi.readthedocs.io/en/latest/CowPi/inputs.html#debouncing}{\function{cowpi_debounce_byte()}}.
You may not use any other functions described in the \href{https://cow-pi.readthedocs.io/en/latest/library.html}{CowPi Library} portion of the Cow~Pi datasheet.

You may use any functions that send characters or commands to the display module that are described in the \href{https://cow-pi.readthedocs.io/en/latest/library.html}{Functions for Direct Control of HD44780-Controlled Display Modules} portion of the Cow~Pi datasheet;
however, we very strongly advise against doing so.
We expect that you will find that using \function{fprintf()} is less error-prone.
Regardless, your \function{send_halfbyte()} function must be implemented using memory-mapped I/O registers.

\subsubsection{Constraints on other libraries}

You may not use any libraries beyond those explicitly identified here.


