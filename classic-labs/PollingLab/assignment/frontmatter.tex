In this assignment, you will write functions to use the input and output devices on \runtimeenvironment.
You will then use those functions to implement the ability to build a number, similar to entering a value on a calculator, using polling.

\begin{figure}[h]
    \centering
    \includegraphics[width=10cm]{AreWeThereYet}
    \caption{Polling. \tiny Image by 20th Century Fox Television}
\end{figure}

The instructions are written assuming you will edit the code in the Arduino IDE or in VS Code with the PlatformIO extension, and assuming that you will run it on \runtimeenvironment, constructed according to the pre-lab instructions.
%If you wish, you may edit the code in a different environment; however, our ability to provide support for problems with other IDEs is limited.

\tableofcontents

\section*{Learning Objectives}

After successful completion of this assignment, students will be able to:
\begin{itemize}
\item Use memory-mapped I/O to obtain inputs from peripheral devices
\item Use memory-mapped I/O to send outputs to peripheral devices
\item Poll an input to determine when it has taken on a value of interest
\item Scan a matrix keypad
% \item Use the Serial-Parallel Interface (SPI) protocol
\item Use the Inter-Integrated Circuit (I$^2$C) protocol
% \item Display a value or a message on a collection of 7-segment displays
\item Display a value or a message on a dot-matrix character display
\end{itemize}

\subsection*{Continuing Forward}

We will use the hardware kit for the remaining labs.
In the labs after this one, you will not be required to use the memory-mapped I/O registers (but you may!);
%however, we will assume that you know how to scan a matrix keypad, and that you can build and display a value from keypad inputs.
however, we will assume that you have mastered this assignment's learning objectives and will build on that mastery.

\section*{During Lab Time}

During your lab period, the TAs will provide a refresher on bitmasks, both to read inputs and to use the read-modify-write pattern for outputs.
They will also discuss some subtle semantics problems that students have encountered in the past.
Finally, the TAs will guide the class through the first modifications to the starter code that you must make.
During the remaining time, the TAs will be available to answer questions.