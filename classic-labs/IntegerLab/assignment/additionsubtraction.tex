Now that you've warmed up to bitwise operations and implementing operations without using C's built-in operations, let us turn your attention to arithmetic.
Before you can add two $n$-bit values, you must be able to add two 1-bit values.

\subsection{One Bit Full Adder} \label{subsec:one-bit-full-adder}

In the \function{one_bit_full_addition()} function, you will implement a 1-bit full adder;
that is, an adder that takes two operand bits and a carry-in bit, and it produces a sum bit and a carry-out bit.

The function takes one argument, a structure containing five fields.
As described in Section~\ref{subsubsec:alu.h}, these five fields are the operand bits \lstinline{a} and \lstinline{b}, the carry-in bit \lstinline{c_in}, the sum bit \lstinline{sum}, and the carry-out bit \lstinline{c_out}.
When the structure is passed in to the function, only \lstinline{a}, \lstinline{b}, and \lstinline{c_in} are populated.
Your task is to populate the \lstinline{sum} and \lstinline{c_out} fields, and return the structure.

Implement a 1-bit full adder using bitwise operations.
Because the fields are guaranteed to be strictly 1 or 0, you do not need to apply any of the test functions to reduce them to 1 or 0.

\subsubsection*{Check your work}

Compile and run \texttt{\textbf{\textit{./integerlab}}}, trying all possible values.
When you enter the inputs for your 1-bit full adder, only the least significant bit of each operand will be used.
For example:
\begin{verbatim}
    Enter ... "add1 <binary_value1> <binary_value2> <carry_in>" ...: add1 0 0 0
    expected: 0 + 0 + 0 = 0 carry 0
    actual:   0 + 0 + 0 = 0 carry 0

    Enter ... "add1 <binary_value1> <binary_value2> <carry_in>" ...: add1 0 0 1
    expected: 0 + 0 + 1 = 1 carry 0
    actual:   0 + 0 + 1 = 1 carry 0
\end{verbatim}

Check your code with all eight possible inputs, comparing your actual results with the expected results.


\subsection{Ripple-Carry Adder} \label{subsec:ripple-carry-adder}

Use your 1-bit full adder to implement a 32-bit ripple-carry adder.
As a reminder, in a ripple-carry adder, the carry-out bit from bit position $n$ becomes the carry-in bit for bit position $n+1$.

Use whatever code that you need, that does not violate any of this assignment's constraints, to populate the input fields of a \lstinline{one_bit_adder_t} variable and pass that variable to \function{one_bit_full_addition()}.
Use the \lstinline{sum} field to contribute to the 32-bit sum and the \lstinline{c_out} bit as the \lstinline{c_in} bit of the next bit position.
Repeatedly do this until you have added all 32-bit positions, resulting in the 32-bit sum.

\subsubsection*{Check your work}

Compile and run \texttt{\textbf{\textit{./integerlab}}}, trying a few values.
When you enter the inputs for your 32-bit adder, the operands will be interpreted as hexadecimal values even if you omit the leading ``0x'', and only the least-significant bit of the carry-in will be used.
For example:
\begin{verbatim}
    Enter ... "add32 <hex_value1> <hex_value2> <carry_in>" ...: add32 0x1a 0x22 0
    expected: 0x0000001A + 0x00000022 + 0 = 0x0000003C
    actual:   0x0000001A + 0x00000022 + 0 = 0x0000003C

    Enter ... "add32 <hex_value1> <hex_value2> <carry_in>" ...: add32 1a 22 1
    expected: 0x0000001A + 0x00000022 + 1 = 0x0000003D
    actual:   0x0000001A + 0x00000022 + 1 = 0x0000003D
\end{verbatim}

Check your code with other values, comparing your actual results with the expected results.


\subsection{16-Bit Addition}

The \function{add()} function, along with the other arithmetic functions, returns an \lstinline{alu_result_t} structure.

Having implemented a 32-bit adder, you can use it for your 16-bit addition function.
The 16-bit sum will be the lower 16 bits of the 32-bit adder's sum;
place this sum in the \lstinline{alu_result_t} variable's \lstinline{result} field.

Your remaining task for addition is to examine the 32-bit sum to determine whether your \textit{16-bit} addition overflowed.
Assume that the operands are unsigned 16-bit integers and determine whether overflow occurred;
set the \lstinline{alu_result_t} variable's \lstinline{unsigned_overflow} flag accordingly.
Now assume that the operands are signed 16-bit integers and determine whether overflow occurred;
set the \lstinline{alu_result_t} variable's \lstinline{signed_overflow} flag accordingly.


\subsubsection*{Check your work}

Compile and run \texttt{\textbf{\textit{./integerlab}}}, trying a few values.
For example:
\begin{verbatim}
    Enter ... a two-operand arithmetic expression... or "quit": 3 + 15
    UNSIGNED ADDITION
        expected result (hexadecimal): 0x0003 + 0x000F = 0x0012
        expected result (unsigned):    3 + 15 = 18	overflow: false
        actual result (hexadecimal):   0x0003 + 0x000F = 0x0012
        actual result (unsigned):      3 + 15 = 18	overflow: false
    SIGNED ADDITION
        expected result (hexadecimal): 0x0003 + 0x000F = 0x0012
        expected result (signed):      3 + 15 = 18	overflow: false
        actual result (hexadecimal):   0x0003 + 0x000F = 0x0012
        actual result (signed):        3 + 15 = 18	overflow: false

    Enter ... a two-operand arithmetic expression... or "quit": 0x6000 + 0x3000
    UNSIGNED ADDITION
        expected result (hexadecimal): 0x6000 + 0x3000 = 0x9000
        expected result (unsigned):    24576 + 12288 = 36864	overflow: false
        actual result (hexadecimal):   0x6000 + 0x3000 = 0x9000
        actual result (unsigned):      24576 + 12288 = 36864	overflow: false
    SIGNED ADDITION
        expected result (hexadecimal): 0x6000 + 0x3000 = 0x9000
        expected result (signed):      24576 + 12288 = -28672	overflow: true
        actual result (hexadecimal):   0x6000 + 0x3000 = 0x9000
        actual result (signed):        24576 + 12288 = -28672	overflow: true
\end{verbatim}

If you are performing this lab on \runtimeenvironment, then the expected overflow flags are obtained directly from flags set in the processor's ALU and are authoritative.\footnote{
    If you are not performing this lab on \runtimeenvironment\ and receive the compile-time warning ``Some of the code to determine the *expected* supplemental\_result and *expected* flags is not yet defined'' then the expected overflow flags should not be trusted.
}

Check your code with other values, comparing your actual results with the expected results.
Use positive and negative operands.
Generate sums that produce signed overflow, sums that produce unsigned overflow, and sums that produce neither.


\subsection{16-Bit Subtraction}

Having implemented a 32-bit adder, you can use it for your 16-bit subtraction function.
If you use the adder as discussed in Chapter~3 and in lecture, the 16-bit difference will be the lower 16 bits of the 32-bit adder's sum;
place this difference in the \lstinline{alu_result_t} variable's \lstinline{result} field.

Your challenges for this function will be using the ripple-carry adder to perform subtraction, and detecting signed- and unsigned overflow.
Note that C will promote 16-bit operands to 32 bits when they're passed as arguments the 32-bit adder;
this promotion includes extending the sign bit into the upper 16 bits.
If you prevent this sign extension (that is, if you ensure the upper 16 bits are all 0s) then you will be able to apply the overflow rules discussed in Chapter~3 and in lecture.


\subsubsection*{Check your work}

Compile and run \texttt{\textbf{\textit{./integerlab}}}, trying a few values.
For example:
\begin{verbatim}
    Enter ... a two-operand arithmetic expression... or "quit": 15 - 25
    UNSIGNED SUBTRACTION
        expected result (hexadecimal): 0x000F - 0x0019 = 0xFFF6
        expected result (unsigned):    15 - 25 = 65526	overflow: true
        actual result (hexadecimal):   0x000F - 0x0019 = 0xFFF6
        actual result (unsigned):      15 - 25 = 65526	overflow: true
    SIGNED SUBTRACTION
        expected result (hexadecimal): 0x000F - 0x0019 = 0xFFF6
        expected result (signed):      15 - 25 = -10	overflow: false
        actual result (hexadecimal):   0x000F - 0x0019 = 0xFFF6
        actual result (signed):        15 - 25 = -10	overflow: false

    Enter ... a two-operand arithmetic expression... or "quit": 0x100 - 0x7F
    UNSIGNED SUBTRACTION
        expected result (hexadecimal): 0x0100 - 0x007F = 0x0081
        expected result (unsigned):    256 - 127 = 129	overflow: false
        actual result (hexadecimal):   0x0100 - 0x007F = 0x0081
        actual result (unsigned):      256 - 127 = 129	overflow: false
    SIGNED SUBTRACTION
        expected result (hexadecimal): 0x0100 - 0x007F = 0x0081
        expected result (signed):      256 - 127 = 129	overflow: false
        actual result (hexadecimal):   0x0100 - 0x007F = 0x0081
        actual result (signed):        256 - 127 = 129	overflow: false
\end{verbatim}

As with addition, if you are performing this lab on \runtimeenvironment, then the expected overflow flags are obtained directly from flags set in the processor's ALU\@.

Check your code with other values, comparing your actual results with the expected results.
Use positive and negative operands.
Generate differences that produce signed overflow, sums that produce unsigned overflow, and sums that produce neither.
