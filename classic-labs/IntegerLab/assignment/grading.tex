When you have completed this assignment, upload \textit{basetwo.c} and \textit{alu.c} to \filesubmission.

\policyforcodethatdoesnotcompile

\integerlablatepolicy

\subsection*{Rubric}

This assignment is worth 40 points.
All functional requirements will be checked on \runtimeenvironment.
If you developed your code on your own system, be sure that it performs as expected on \runtimeenvironment.\footnote{
    We cannot grade your code on your personal system, so be sure that your code runs correctly on the system we will grade it on.
}
\begin{description}
    \rubricitem{1}{The \function{exponentiate()} function produces the correct powers of two.}
    \rubricitem{1}{The \function{lg()} function produces the correct base-2 logarithms.}
    \rubricitem{1}{The \function{is_negative()} function correctly determines whether its argument has a negative value when interpreted as a signed integer.}
    \rubricitem{1}{The \function{equal()} function correctly determines whether its arguments are equal to each other.}
    \rubricitem{1}{The \function{not_equal()} function correctly determines whether its arguments are not equal to each other.}
    \rubricitem{1}{The \function{logical_not()} function correctly produces the logical inverse of its argument.}
    \rubricitem{1}{The \function{logical_and()} function correctly produces the logical conjunction of its arguments.}
    \rubricitem{1}{The \function{logical_or()} function correctly produces the logical disjunction of its arguments.}
    \rubricitem{1}{The \function{one_bit_full_addition()} function correctly determines the \lstinline{sum} and \lstinline{c_out} bits for a 1-bit full adder.}
    \rubricitem{5}{The \function{ripple_carry_addition()} function correctly implements a 32-bit ripple-carry adder.}
    \rubricitem{2}{The \function{add()} function correctly performs 16-bit integer addition.}
    \rubricitem{2}{The \function{add()} function correctly detects unsigned integer overflow and signed integer overflow.}
    \rubricitem{3}{The \function{subtract()} function correctly performs 16-bit integer subtraction.}
    \rubricitem{2}{The \function{subtract()} function correctly detects unsigned integer overflow and signed integer overflow.}
    \rubricitem{1}{The \function{less_than()} function correctly determines whether its first argument is strictly less than its second argument.}
    \rubricitem{1}{The \function{at_most()} function correctly determines whether its first argument is less than or equal to its second argument.}
    \rubricitem{1}{The \function{at_least()} function correctly determines whether its first argument is greater than or equal to its second argument.}
    \rubricitem{1}{The \function{greater_than()} function correctly determines whether its first argument is strictly greater than its second argument.}
    \rubricitem{3}{The \function{multiply_by_power_of_two()} function correctly multiplies its first argument by its second argument when the second argument is 0 or is a power of two.}
    \rubricitem{3}{The \function{unsigned_multiply()} function correctly provides the 16-bit product when it multiplies its first argument by its second argument when they are interpreted as unsigned integers.}
    \rubricitem{2}{The \function{unsigned_multiply()} function correctly provides the 32-bit full product spread across the \lstinline{supplemental_result} and \lstinline{result} fields.}
    \rubricitem{3}{The \function{unsigned_divide()} function correctly provides the 16-bit quotient when it divides its first argument by its second argument (or correctly reports division by zero) when they are interpreted as unsigned integers and the second argument is 0 or is a power of two.}
    \rubricitem{2}{The \function{unsigned_divide()} function correctly provides the 16-bit remainder.}
    \bonusitem{1}{The \function{signed_multiply()} function correctly performs signed integer multiplication by addressing the underlying reason that signed and unsigned multiplication need to be different.}
    \bonusitem{1}{The \function{signed_divide()} function correctly performs signed integer division by addressing the underlying reason that signed and unsigned division need to be different.}
    \item[Penalties]
    \penaltyitem{1}{For each of these functions that violates an assignment constraint:
        \function{exponentiate()}, \function{lg()}, \function{is_negative()}, \function{equal()}, \function{not_equal()},
        \function{logical_not()}, \function{logical_and()}, \function{logical_or()},
        \function{less_than()}, \function{at_most()}, \function{at_least()}, \function{greater_than()}.}
    \penaltyitem{15}{If \function{one_bit_full_addition()} violates an assignment constraint.}
    \penaltyitem{14}{If \function{one_bit_full_addition()} does not violate an assignment constraint but \function{ripple_carry_addition()} does.}
    \penaltyitem{4}{If \function{one_bit_full_addition()} and \function{ripple_carry_addition()} do not violate an assignment constraint but \function{add()} does.}
    \penaltyitem{5}{If \function{one_bit_full_addition()} and \function{ripple_carry_addition()} do not violate an assignment constraint but \function{subtract()} does.}
    \penaltyitem{8}{If \function{multiply_by_power_of_two()} violates an assignment constraint.}
    \penaltyitem{5}{If \function{multiply_by_power_of_two()} does not violate an assignment constraint but \function{unsigned_multiply()} does.}
    \penaltyitem{5}{If \function{unsigned_divide()} violates an assignment constraint.}
    \item[] \textit{Assignment constraint violations by helper functions will be assessed against the required function(s) that they help.}
    \item[no bonus] If \function{signed_multiply()} or \function{signed_divide()} violate an assignment constraint or fail to address the underlying reason that the signed and unsigned implementations need to be different.
    \spaghetticodepenalties{1}
\end{description}