Please familiarize yourself with the entire assignment before beginning.
There are three parts to this assignment.

%The first part, which is intended to be completed during lab time under the guidance of a TA, is to populate the keypad's lookup table (Section~\ref{subsec:populatekeypad}), determine the base addresses for the memory-mapped I/O register banks used in this assignment (Section~\ref{subsec:baseAddresses}), and introduce code to detect keypresses on the numeric keypad (Section~\ref{subsec:detectKeyAction}).
%After that, there are two major parts to this assignment, and they can be completed in either order.
%
%\subsection{Memory-Mapped Input/Output}
%
%The starter code contains functions that provide access to the buttons, switches, keypad, LEDs, and display module.
%Initially, these functions make use of functions available in the CowPi library.
%In Section~\ref{sec:MemMapIO}, you will re-implement these functions using memory-mapped I/O\@.
%
%\subsection{Implementing a Simple System using Polling}
%
%In Section~\ref{sec:SimpleSystem}, you will implement a simple system that makes use of your hardware kit's buttons, switches, keypad, LEDs, and display module.
%You will implement this system by polling inputs.
%
%\subsection{Separation of Concerns}
%
%The code that interfaces directly with the hardware is in one file (\textit{io\_functions.c}), and the logic to implement your simple embedded system is in another file (\textit{number\_builder.c}).
%With embedded systems, it isn't always easy to have such a clean separation of concerns, but when you can, you should.
%As you complete this assignment, place all of your system's logic in \textit{number\_builder.c},
%and place all of the code that accesses the memory-mapped I/O registers in \textit{io\_functions.c}.
%
\subsection{Constraints} \label{subsec:constraints}

You may use any features that are part of the C standard if they are supported by the compiler.
You may use the constants and functions provided in the starter code.

You may re-use your code that uses memory-mapped I/O registers,
or you can use functions provided by the CowPi library to read from and write to pins.

You may \textit{not} poll the matrix keypad nor the pushbuttons to determine if they have been pressed.
You must use interrupts to determine if a key or button has been pressed.
Once a press has been detected, you may scan the matrix keypad or read the pushbuttons to determine which key or button has been pressed.

You may poll the left switch to determine if its position has changed;
however, the specification has been written such that your code should only need to occasionally check the switch's position rather than polling it for changes.

% While it is possible to configure the SPI hardware to generate an interrupt after the content of the SPI Data Register is transmitted,
% you may use your \function{send_data()} function that polls the \texttt{SPIF} bit,
% or you may use the Cow Pi library's default implementation.

While it is possible to configure the I$^2$C hardware to generate an interrupt after the content of the TWI Data Register is transmitted,
you may use your \function{send_halfbyte()} function that polls the \texttt{TWINT} bit,
or you may use the Cow Pi library's default implementation.

\subsubsection{Constraints on the Arduino core}

You may use \function{attachInterrupt()} and \function{digitalPinToInterrupt()} to register interrupt handlers.
You may (but are not required to) use \function{Serial.print()} and \function{Serial.println()} instead of \function{printf()} if you wish.
If and only if you use your keypad scanner function from PollingLab, you may use the \function{delayMicroseconds()} function to introduce a 1--2$\mu$s delay in that function if necessary.

You may not use any other libraries, functions, macros, types, or constants from the Arduino core.\footnote{
    \url{https://www.arduino.cc/reference/en/}
}

\subsubsection{Constraints on AVR-libc}

You may use any AVR-specific functions, macros, types, or constants of avr-libc.\footnote{
    \url{https://www.nongnu.org/avr-libc/user-manual/index.html}
}

\subsubsection{Constraints on the CowPi library}

You may use any functions that are described in Section~2 of the Cow Pi datasheet,
and you may use any data structures provided by the CowPi library.

\subsubsection{Constraints on other libraries}

You may not use any libraries beyond those explicitly identified here.