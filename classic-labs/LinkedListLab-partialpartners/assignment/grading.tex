When you have completed this assignment, upload \textit{answers.txt} and \textit{challenge-response.c} to \filesubmission.

\policyforcodethatdoesnotcompile

\pointerlablatepolicy

\subsection*{Rubric}

This assignment is worth 25 points.
All functional requirements will be checked on \runtimeenvironment.
If you developed your code on your own system, be sure that it performs as expected on \runtimeenvironment.\footnote{
    We cannot grade your code on your personal system, so be sure that your code runs correctly on the system we will grade it on.
}
\begin{description}
    \rubricitem{2}{Student's answers in \textit{answers.txt} demonstrate an understanding of the bug in Section~\ref{subsec:uninitializedvariables}'s code and how to correct it.}
    \rubricitem{3}{Student's answer in \textit{answers.txt} demonstrate an understanding of the bug in Section~\ref{subsec:danglingPointers}'s code.}
    \rubricitem{1}{\function{create_node} creates and initializes a \lstinline{struct node} as specified.}
    % Depending on the insert_word code, 0 or 1 is a reasonable initial value for the occurrences field
    \item[\hspace{.5cm}]\function{insert_after}:
    \begin{description}
        \rubricitem{3}{correctly places a new node in a list by updating the \lstinline{next} pointers.}
        \rubricitem{1}{also updates the \lstinline{previous} pointers.}
    \end{description}
    \rubricitem{2}{\function{word_to_lowercase} returns a copy of the input string with uppercase letters replaced with lowercase letters.}
    \item[\hspace{.5cm}]\function{insert_word}:
    \begin{description}
        \rubricitem{2}{creates a new node at the appropriate location in the list when the word is not already present in the list (where the appropriate location is immediately after the last word to have been added).}
        \rubricitem{1}{does not create a new node but instead updates the number of occurrences, when the word is present in the list.}
    \end{description}
    \rubricitem{3}{\function{build_list} opens a file for reading, builds a list by reading one line at a time and the word that is passed to \function{insert_word()}, and closes the file after the last line has been read, when the words in the file are pre-sorted.}
    \rubricitem{2}{\function{respond()} produces the correct response word in accordance with the specified rules when the words in the file are pre-sorted.}
    \rubricitem{2}{Your challenge-response code can build a list and generate the correct response when the words in the file are unsorted.}
    \rubricitem{2}{Your challenge-response code can generate a list from a file of up to 80,000 words and generate the correct response within 20 seconds when the words in the file are pre-sorted.}
    \rubricitem{1}{Your challenge-response code can generate a list from a file of up to 80,000 words and generate the correct response within 20 seconds when the words in the file are unsorted.}
    \item[Penalties]
    \penaltyitem{1}Newline characters are included in the word strings when building a list.
    \spaghetticodepenalties{1}
\end{description}
