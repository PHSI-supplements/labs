When you have completed this assignment, upload \textit{alarm.c}, \textit{sensor.c}, \textit{user\_controls.c}, and \textit{shared\_variables.h} to \filesubmission.

%\policyforcodethatdoesnotcompile
\subsection*{No Credit for Uncompilable Code}
If the TA cannot create an executable from your code, then your code will be assumed to have no functionality.\footnote{
    At the TA's discretion, if they can make your code compile with \textit{one} edit (such as introducing a missing semicolon) then they may do so and then assess a 10\% penalty on the resulting score.
    The TA is under no obligation to do so, and you should not rely on the TA's willingness to edit your code for grading.
    If there are multiple options for a single edit that would make your code compile, there is no guarantee that the TA will select the option that would maximize your score.
}
Before turning in your code, be sure to compile and test your code on your Cow~Pi with the original driver code and the original header file(s).

\interruptlablatepolicy

\subsection*{Rubric}

This assignment is tentatively worth 50 points.
(I might increase it to 60.)

\subsection*{TODO}

%This assignment is worth 60 points.
%
%\begin{description}
%    \item[User Controls] %13
%    \rubricitem{1}{The switches control the mode of operation as specified}
%    \rubricitem{2}{The user can request a ping when the system is in Single Pulse mode}
%    \rubricitem{2}{The user can request another ping when the system is in Single Pulse mode}
%    \rubricitem{1}{The user is prompted to enter a new threshold range when the system is in Threshold Adjustment mode}
%    \rubricitem{2}{The user can enter a new threshold range when the system is in Threshold Adjustment mode}
%    \rubricitem{2}{Valid threshold ranges are those between 50cm and 400cm, inclusive}
%    \rubricitem{1}{The user is given a helpful error message after entering an invalid threshold range}
%    \rubricitem{1}{The user is re-prompted to enter a threshold range after entering an invalid threshold range}
%    \rubricitem{1}{The user is shown a confirmation message after entering a valid threshold range}
%
%    \item[Sensor] %16
%    \rubricitem{2}{There is code to initiate an ultrasound pulse}
%    \rubricitem{3}{There is code to detect the length of the pulse}
%    \rubricitem{3}{The pulse's length is measured to a precision of no greater than 1\textmu s\footnote{
%        The assignment describes obtaining a precision of \textonehalf\textmu s, which is, of course, acceptable. The 4\textmu s precision offered by the Arduino \function{micros()} function is \textit{not} acceptable.
%    }}
%    \rubricitem{3}{The pulse's length is measured as accurately as possible\footnote{
%        This means you use interrupts to detect the rising and falling edges of the pulse.
%    }}
%    \rubricitem{3}{The code correctly recognizes the that no object has been detected, if no object reflects the ultrasound pulse}
%    \rubricitem{2}{The distance to an object is correctly calculated from the pulse's length}
%    \penaltyitem{16}{The Arduino \function{pulseIn()} and/or \function{pulseInLong()} function, or a third-party library, is used to time the ultrasonic echo sensor's pulse.}
%
%    \item[Alarm] %16
%    \rubricitem{1}{There is code to generate an audible tone}
%    \rubricitem{1}{The system continuously generates the tone when in Continuous Tone mode}
%    \rubricitem{2}{The audible tone has a frequency of 5kHz}
%    \rubricitem{2}{A chirp is an audible tone lasting 50ms}
%    \rubricitem{2}{A chirp occurs for a detected object that is closer than the threshold range}
%    \rubricitem{2}{A chirp only occurs for a detected object that is closer than the threshold range}
%    \rubricitem{2}{A strobe is an illumination of the right LED for 50ms}
%    \rubricitem{2}{A strobe occurs for any detected object}
%    \rubricitem{2}{A strobe only occurs for a detected object}
%    \penaltyitem{16}{The Arduino \function{tone()} and/or \function{noTone()} function, or a third-party library, is used to generate tones on the piezoelectric disc.}
%
%    \item[Object Detection] %14
%    \rubricitem{2}{When there is no in-range object, the system displays a message to that effect}
%    \rubricitem{2}{When an object is detected (in Single-Pulse mode or Normal Operation mode), the system displays the distance to the object}
%    \rubricitem{2}{When an object is detected in Normal Operation mode the rate of approach is displayed}
%    \rubricitem{2}{When an object is detected in Normal Operation mode, the rate of approach is updated at least once every second}
%    \rubricitem{2}{When an object is detected in Single-Pulse mode, the system generates exactly one alarm}
%    \rubricitem{2}{When an object is detected in Normal Operation mode, the system repeatedly generates alarms}
%    \rubricitem{2}{When the system repeatedly generates alarms, the time between alarms is as specified in Table~\ref{tab:alarmPeriods}}
%
%    \item[Code Quality]
%    \rubricitem{1}{The code is clean, well-organized, has good variable and function names, and is otherwise understandable}
%    \spaghetticodepenalties{1}
%
%    \item[Bonuses]
%    \bonusitem{1}{Use the actual $ADC\_register\_value$ when performing distance and speed calculations.}
%    \bonusitem{2}{Get assignment checked-off by TA or professor during office hours before it is due (you cannot get both check-off bonuses)}
%    \bonusitem{1}{Get assignment checked-off by TA at \textit{start} of your scheduled lab immediately after it is due (your code must be uploaded to \filesubmission\ \textit{before} it is due; you cannot get both bonuses)}
%\end{description}
%
%This assignment is worth 80 points. \\
%
%Rubric:
%\begin{description}
%    \rubricitem{2}{Combinations are displayed as three 2-hex-digit numbers separated
%    by dashes.}
%    \rubricitem{4}{Numbers are entered using the matrix keypad.}
%    \rubricitem{2}{The lock is locked when powered-up.}
%    \rubricitem{4}{When the lock is locked, it displays the combination-entry
%    display, initially showing empty numbers (only dashes).}
%    \rubricitem{4}{The cursor indicating which number is being entered is
%    represented as blinking decimal points in the relevant number position.}
%    \rubricitem{4}{The user can move the cursor between number positions by pressing
%    the right button.}
%    \rubricitem{4}{After entering a combination, the user can submit the combination
%    by pressing the left button.}
%    \rubricitem{2}{If the user tries to submit an incomplete combination, the lock
%    displays ``error'' and then returns to the combination-entry display.}
%    \rubricitem{4}{When the user unlocks the lock, it displays ``lab open''.}
%    \rubricitem{2}{When the user mis-enters the combination, it displays ``bad try''
%    and the attempt number.}
%    \rubricitem{4}{After the first two bad tries, the user is given another
%    opportunity to enter the combination.}
%    \rubricitem{4}{After the third bad try, the system displays ``alert!'', the
%    external LED rapidly blinks, and the lock becomes unresponsive.}
%    \rubricitem{4}{The user can begin changing the combination by moving both
%    switches to the right and pressing the left button.}
%    \rubricitem{4}{When the user begins changing the combination, the lock displays
%    ``enter'' and then shows the combination-entry display.}
%    \rubricitem{4}{After entering a new combination, the user can begin confirming
%    the combination by sliding the left switch to the left and pressing the left
%    button.}
%    \rubricitem{4}{When the user begins confirming the combination, the lock
%    displays ``re-enter'' and then shows the combination-entry display.}
%    \rubricitem{4}{After entering the confirming combination, the user can compare
%    the new combination with the confirmed combination by sliding the right switch
%    to the left and pressing the left button.}
%    \rubricitem{4}{If the user successfully confirmed the new combination, the lock
%    displays ``changed'', and the new combination is the correct combination
%    for future unlocking attempts.}
%    \rubricitem{4}{If the user failed to confirm the new combination, the lock
%    displays ``nochange'', and the previously-correct combination remains the
%    combination for future unlocking attempts.}
%    \rubricitem{4}{The user can lock the lock by moving both switches to the left
%    and \textit{either} double-clicking the right button, \textit{or} pressing
%    both buttons at the same time. (Only one of these methods needs to be
%    implemented).}
%    \rubricitem{2}{The inputs that are explicitly specified as having no effect in
%    certain situations are ignored in those situations.}
%    \rubricitem{4}{The correct combination persists while Arduino is powered-down.}
%    \rubricitem{2}{The source code is well-organized and is readable.}
%    \bonusitem{2}{Get assignment checked-off by TA or professor during office hours
%    before it is due. (You cannot get both bonuses.)}
%    \bonusitem{1}{Get assignment checked-off by TA at \textit{start} of your
%    scheduled lab immediately after it is due. (Your code must be uploaded to
%    \filesubmission\ \textit{before} it is due. You cannot get both bonuses.)}
%    \spaghetticodepenalties{1}
%\end{description}

Students' scores may be adjusted up or down as necessary if the team had an inequitable distribution of effort.