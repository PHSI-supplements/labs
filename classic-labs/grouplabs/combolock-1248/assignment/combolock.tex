%%
%% CombinationLock GroupLab (c) 2022-24 Christopher A. Bohn
%%
%% Licensed under the Apache License, Version 2.0 (the "License");
%% you may not use this file except in compliance with the License.
%% You may obtain a copy of the License at
%%     http://www.apache.org/licenses/LICENSE-2.0
%% Unless required by applicable law or agreed to in writing, software
%% distributed under the License is distributed on an "AS IS" BASIS,
%% WITHOUT WARRANTIES OR CONDITIONS OF ANY KIND, either express or implied.
%% See the License for the specific language governing permissions and
%% limitations under the License.
%%

%%
%% labs/common/assignment.tex
%% (c) 2021-22 Christopher A. Bohn
%%
%% Licensed under the Apache License, Version 2.0 (the "License");
%% you may not use this file except in compliance with the License.
%% You may obtain a copy of the License at
%%     http://www.apache.org/licenses/LICENSE-2.0
%% Unless required by applicable law or agreed to in writing, software
%% distributed under the License is distributed on an "AS IS" BASIS,
%% WITHOUT WARRANTIES OR CONDITIONS OF ANY KIND, either express or implied.
%% See the License for the specific language governing permissions and
%% limitations under the License.
%%

\documentclass[12pt]{article}

\usepackage{fullpage}
\usepackage{fancyhdr}
\usepackage[procnames]{listings}
\usepackage{hyperref}
\usepackage{textcomp}
\usepackage{bold-extra}
\usepackage[dvipsnames]{xcolor}
\usepackage{etoolbox}

% These are placeholder commands and will be renewed in each lab

\newcommand{\labnumber}{}
\newcommand{\labname}{Lab \labnumber\ Assignment}
\newcommand{\shortlabname}{}
\newcommand{\duedate}{}

% Individual or team effort

\newcommand{\individualeffort}{This is an individual-effort project. You may
    discuss concepts and syntax with other students, but you may discuss
    solutions only with the professor and the TAs. Sharing code with or copying
    code from another student or the internet is prohibited.}
\newcommand{\teameffort}{This is a team-effort project. You may discuss concepts
    and syntax with other students, but you may discuss solutions only with your
    assigned partner(s), the professor, and the TAs. Sharing code with or
    copying code from a student who is not on your team, or from the internet,
    is prohibited.}
\newcommand{\freecollaboration}{In addition to the professor and the TAs, you
    may freely seek help on this assignment from other students.}
\newcommand{\collaborationrules}{}

% Software engineering (if you care about that)

\providebool{allowspaghetticode}

\newcommand{\softwareengineeringfrontmatter}{
    \ifboolexpe{not bool{allowspaghetticode}}{
        \section*{No Spaghetti Code Allowed}
        In the interest of keeping your code readable, you may \textit{not} use
        any \lstinline{goto} statements, nor may you use any
        \lstinline{continue} statements, nor may you use any \lstinline{break}
        statements to exit from a loop, nor may you have any functions
        \lstinline{return} from within a loop.
    }{}
}

\newcommand{\spaghetticodepenalties}[1]{
    \ifboolexpe{not bool{allowspaghetticode}}{
        \penaltyitem{1}{for each \lstinline{goto} statement,
            \lstinline{continue} statement, \lstinline{break} statement used to
            exit from a loop, or \lstinline{return} statement that occurs within
            a loop.}
    }{}
}

% You shouldn't need to customize these,
% but you can if you like

\lstset{language=C, tabsize=4, upquote=true, basicstyle=\ttfamily}
\newcommand{\function}[1]{\textbf{\lstinline{#1}}}
\setlength{\headsep}{0.7cm}
\hypersetup{colorlinks=true}

\newcommand{\pagelayout}{
    \pagestyle{fancy}
    \fancyhf{}
    \lhead{\coursenumber}
    \chead{\ Lab \labnumber: \labname}
    \rhead{\courseterm}
    \cfoot{\shortlabname-\thepage}
}

\newcommand{\labidentifier}{
    \title{\ Lab \labnumber}
    \author{\labname}
    \date{Due: \duedate}
    \maketitle

    \textit{\collaborationrules}
}

% deprecated
\newcommand{\startdocument}{
    \pagelayout
	\begin{document}
	\labidentifier
}

\newcommand{\rubricitem}[2]{\item[\underline{\hspace{1cm}} +#1] #2}
\newcommand{\bonusitem}[2]{\item[\underline{\hspace{1cm}} Bonus +#1] #2}
\newcommand{\penaltyitem}[2]{\item[\underline{\hspace{1cm}} -#1] #2}
\newcommand{\checkoffitem}[1]{\item (\phantom{xxx}) #1}
\newcommand{\precheckoffitem}[1]{\item [] (\phantom{xxx}) #1}

%%
%% labs/common/semester.tex
%% (c) 2021-22 Christopher A. Bohn
%%
%% Licensed under the Apache License, Version 2.0 (the "License");
%% you may not use this file except in compliance with the License.
%% You may obtain a copy of the License at
%%     http://www.apache.org/licenses/LICENSE-2.0
%% Unless required by applicable law or agreed to in writing, software
%% distributed under the License is distributed on an "AS IS" BASIS,
%% WITHOUT WARRANTIES OR CONDITIONS OF ANY KIND, either express or implied.
%% See the License for the specific language governing permissions and
%% limitations under the License.
%%


% Customize the semester (or quarter) and the course number

\newcommand{\courseterm}{Spring 2022}
\newcommand{\coursenumber}{CSCE 231}

% Customize how a typical lab will be managed;
% you can always use \renewcommand for one-offs

\newcommand{\runtimeenvironment}{your account on the \textit{csce.unl.edu} Linux server}
\newcommand{\filesource}{Canvas or {\footnotesize$\sim$}cse231 on \textit{csce.unl.edu}}
\newcommand{\filesubmission}{Canvas}

% Customize for the I/O lab hardware

\newcommand{\developmentboard}{Arduino Nano}

%\newcommand{\serialprotocol}{SPI}
\newcommand{\serialprotocol}{I2C}

%\newcommand{\displaymodule}{MAX7219digits}
%\newcommand{\displaymodule}{MAX7219matrix}
\newcommand{\displaymodule}{LCD1602}

\setbool{usedisplayfont}{true}

\newcommand{\obtaininghardware}{
    The EE Shop has prepared ``class kits'' for CSCE 231; your class kit costs \$20. The EE Shop is located at 122 Scott
    Engineering Center and is open M-F 7am-4pm. You do not need an appointment. You may pay at the window with cash,
    with a personal check, or with your NCard. If you wish to pay by credit card, you must make the purchase from
    \url{https://marketplace.unl.edu/ees/engineering-class-kits/csce231-kit.html} the day before you visit the EE
    Shop.\footnote{The price listed on the website is \$18.65; after sales tax is added, your total will be \$20.}
}

% Update to reflect the CS2 course(s) at your institute

\newcommand{\cstwo}{CSCE~156, RAIK~184H, or SOFT~161}

% Do you care about software engineering?

\setbool{allowspaghetticode}{false}

% Which assignments are you using this semester, and when are they due?

\newcommand{\pokerlabnumber}{1}
\newcommand{\pokerlabcollaboration}{Except as noted in Section~\ref{StudyTheCode}, \individualeffort}
\newcommand{\pokerlabdue}{Week of January 24, before the start of your lab section}

\newcommand{\keyboardlabnumber}{2}
\newcommand{\keyboardlabcollaboration}{\individualeffort}
\newcommand{\keyboardlabdue}{Week of January 31, before the start of your lab section}

\newcommand{\pointerlabnumber}{3}
\newcommand{\pointerlabcollaboration}{\individualeffort}
\newcommand{\pointerlabdue}{Week of February 7, before the start of your lab section}

\newcommand{\integerlabnumber}{4}
\newcommand{\integerlabcollaboration}{\individualeffort}
\newcommand{\integerlabdue}{Week of February 14, before the start of your lab section}

\newcommand{\floatlabnumber}{5}
\newcommand{\floatlabcollaboration}{\individualeffort}
\newcommand{\floatlabdue}{soon}

\newcommand{\addressinglabnumber}{6}
\newcommand{\addressinglabcollaboration}{\individualeffort}
\newcommand{\addressinglabdue}{Week of February 28, before the start of your lab section}

%bomblab was 7
%attacklab was 8

\newcommand{\pollinglabnumber}{9}
\newcommand{\pollinglabcollaboration}{\individualeffort}
\newcommand{\pollinglabdue}{Week of April 11, before the start of your lab section}
\newcommand{\pollinglabenvironment}{your \developmentboard-based class hardware kit}

\newcommand{\ioprelabnumber}{\pollinglabnumber-prelab}
\newcommand{\ioprelabcollaboration}{\freecollaboration}
\newcommand{\ioprelabdue}{Before the start of your lab section on April 5 or 6}

\newcommand{\interruptlabnumber}{10}
\newcommand{\interruptlabcollaboration}{\individualeffort}
\newcommand{\interruptlabdue}{Week of April 18, before the start of your lab section}
\newcommand{\interruptlabenvironment}{your \developmentboard-based class hardware kit}

\newcommand{\capstonelab}{ComboLock}    % this will come into play when we generalize capstonelab
\newcommand{\capstonelabnumber}{11}
\newcommand{\capstonelabcollaboration}{\teameffort}
\newcommand{\capstonelabdue}{Week of May 2, Before the start of your lab section\footnote{See Piazza for the due dates of teams with students from different lab sections.}}
\newcommand{\capstonelabenvironment}{your \developmentboard-based class hardware kit}

\newcommand{\memorylabnumber}{12}
\newcommand{\memorylabcollaboration}{This is an individual-effort project. You may discuss the nature of memory technologies and of memory hierarchies with classmates, but you must draw your own conclusions.}
\newcommand{\memorylabdue}{Week of May 2, at the end of your lab section}
\newcommand{\memorylabenvironment}{your \developmentboard-based class hardware kit and your account on the \textit{csce.unl.edu} Linux server}

% Labs not used this semester

\newcommand{\concurrencylabnumber}{XX}
\newcommand{\concurrencylabcollaboration}{\individualeffort}
\newcommand{\concurrencylabdue}{not this semester}

\newcommand{\ssbcwarmupnumber}{XX}
\newcommand{\ssbcwarmupcollaboration}{\freecollaboration}
\newcommand{\ssbcwarmupdue}{not this semester}

\newcommand{\ssbcpollingnumber}{XX}
\newcommand{\ssbcpollingcollaboration}{\individualeffort}
\newcommand{\ssbcpollingdue}{not this semester}

\newcommand{\ssbcinterruptnumber}{XX}
\newcommand{\ssbcinterruptcollaboration}{\individualeffort}
\newcommand{\ssbcinterruptdue}{not this semester}

%%
%% labs/common/storylines.tex
%% (c) 2020-22 Christopher A. Bohn
%%
%% Licensed under the Apache License, Version 2.0 (the "License");
%% you may not use this file except in compliance with the License.
%% You may obtain a copy of the License at
%%     http://www.apache.org/licenses/LICENSE-2.0
%% Unless required by applicable law or agreed to in writing, software
%% distributed under the License is distributed on an "AS IS" BASIS,
%% WITHOUT WARRANTIES OR CONDITIONS OF ANY KIND, either express or implied.
%% See the License for the specific language governing permissions and
%% limitations under the License.
%%

\newcommand{\MeetArchie}{
    You're relaxing at your favorite hangout when another customer catches your attention.
    He's rather large (dare I say, \textit{mammoth}), a bit hairy, and looking frustrated in front of his laptop.
    ``I'm Archie,'' he says, ``and I'm trying to teach myself this card game called \textit{Poker}.
    I found this source code that I thought I could use to understand Poker better, but the code is incomplete, and I don't entirely understand what's there.
    Could you explain the code to me, please?'
}

\newcommand{\GetHired}{
    Archie's face lights up in a very big smile.
    ``Thanks!''
    After pausing in thought for a moment, he says, ``Say, I've got a new startup company that could really use your help.
    Are you interested?
    It'll be exciting!''
}

\usepackage{amsmath}
%\usepackage{array,color,colortbl}
\definecolor{LightGreen}{rgb}{0.88,1,0.88}
\usepackage{multicol}
\usepackage{CJKutf8}
\usepackage{gensymb}
\usepackage{graphicx}
\usepackage{multirow}
\usepackage{listings}
\usepackage{tikz}
\setlength{\columnsep}{2cm}
\newcommand{\power}{power~({\color{red}\textbf{+}}) rail}
\newcommand{\ground}{ground~({\color{blue}\textbf{--}}) rail}



\captionsetup{width=.8\linewidth}

\lstset{language=c, numbers=left, showstringspaces=false,
    moredelim = [s][\ttfamily]{/*}{*/} % I shouldn't need this parameter!
}

\renewcommand{\labnumber}{\capstonelabnumber}
\renewcommand{\labname}{Implementing a Combination Lock}
\renewcommand{\shortlabname}{combolock -- grouplab}
\renewcommand{\collaborationrules}{\capstonelabcollaboration}
\renewcommand{\duedate}{\capstonelabdue}
\renewcommand{\runtimeenvironment}{\capstonelabenvironment}
\pagelayout
\begin{document}
    \labidentifier

    \pdfbookmark[1]{Frontmatter}{frontmatter}                           The purpose of this assignment is to give you more confidence in C programming
and to begin your exposure to the underlying bit-level representation of data.

The instructions are written assuming you will edit and run the code on
\runtimeenvironment. If you wish, you may edit and run the code
in a different environment; be sure that your compiler suppresses no warnings,
and that if you are using an IDE that it is configured for C and not C++.

\section*{Learning Objectives}

After successful completion of this assignment, students will be able to:
\begin{itemize}
    \item Use the ASCII table to determine the corresponding integer values of C
    \lstinline{char} values.
    \item Apply arithmetic operators and comparators to C \lstinline{}{char} values.
    \item Construct and use a bitmask.
    \item Use bitwise operators and bit shift operators to create and modify values.
\end{itemize}

\subsection*{Continuing Forward}

Your experience with viewing values as bit patterns will be applicable in
future labs, as will bit masks and bit operations. Some of the functions you
write in this lab will be used in the next lab.

\section*{During Lab Time}

During your lab period, the TAs will demonstrate how to read the ASCII table and
will provide a refresher on bitwise AND, bitwise OR, and left- and right-shifts.
During the remaining time, the TAs will be available to answer questions.

Before leaving lab, \textit{at a minimum} \dots


    \softwareengineeringfrontmatter

    \section*{Scenario}                                                 \CombinationLockNeeded

    \section{Assignment Summary}                                        Please familiarize yourself with the entire assignment before beginning.
There are three parts to this assignment.

\subsection{Why are There Letters on Telephone Keypads?}

Once upon a time, telephone exchanges were staffed by operators who would use patch cords on a switchboard to connect callers.
If you needed to make a local-area call to someone whose phone was serviced by a different exchange, then you needed to tell your operator which exchange to connect to.
Letters were assigned to digits so that easy-to-remember -- and audibly-distinctive -- mnemonics could be formed such that the first two letters of the mnemonic that correspond to the 2-digit exchange identifier.
For example, the 86 exchange would use a mnemonic that started with a `T', `U', or `V' and that has as a second letter an `M', `N', or `O' --
so 867--5309 might be ``University 7--5309''.

The presence of letters on telephone dials and (later) telephone keypads allowed for custom phone numbers that used words formed by the available letters.
For example, a bank's phone number might be 472--2265, aka 472--BANK.
Less fictionally, 1--800--FLOWERS was used by a company that partnered with florists to allow people to have bouquets delivered anywhere in the U.S\@.

When the Short Message Service protocol was introduced to allow text-based communication by taking advantage of unused bytes in the handshake between cellular phones and the cell network, naturally the letters that were already present on the keypad were used to tap out messages.

While QWERTY keyboards on smartphones have largely replaced the 10-digit keypad for text entry, the letters remain, waiting for the next clever use\dots

\subsection{Constraints} \label{subsec:constraints}

You may \textit{not} poll the matrix keypad nor the pushbuttons to determine if they have been pressed.
You must use interrupts to determine if a key or button has been pressed.
Once an interrupt has fired, you may scan the matrix keypad or read the pushbuttons to determine which key has been pressed or whether the button has been pressed or released.

You may use any features that are part of the C standard if they are supported by the compiler.
You may use the constants and functions provided in the starter code.

\subsubsection{Constraints on the Arduino core}

You may \textit{not} use any libraries, functions, macros, types, or constants from the Arduino core.

%\subsubsection{Constraints on AVR-libc}
%
%You may use any AVR-specific functions, macros, types, or constants of avr-libc.\footnote{
%    \url{https://www.nongnu.org/avr-libc/user-manual/index.html}
%}

\ifdefstring{\processor}{ATmega328P}{
    \subsubsection{Constraints on AVR-libc}

    You may not use any AVR-specific functions, macros, types, or constants of avr-libc.\footnote{\url{https://www.nongnu.org/avr-libc/user-manual/index.html}}
}{}
\ifdefstring{\processor}{RP2040}{
% TODO: parameterize this (when we eventually port to the bare-metal Arduino toolchain, and to the Pico SDK)
    \subsubsection{Constraints on MBED OS}

    You may not use any functions, macros, types or constants from MBED that are not part of the C standard.\footnote{\url{https://os.mbed.com/docs/mbed-os/v6.16/introduction/index.html}}
}{}

\subsubsection{Constraints on the CowPi library}

You may use any functions provided by the CowPi\footnote{
    \url{https://cow-pi.readthedocs.io/en/latest/library.html}
}
and the CowPi\_stdio\footnote{
    \url{https://cow-pi.readthedocs.io/en/latest/stdio.html}
} libraries,
and you may use any data structures\footnote{
    \url{https://cow-pi.readthedocs.io/en/latest/microcontroller.html}
} provided by the CowPi library.

\subsubsection{Constraints on other libraries}

You may \textit{not} use any libraries beyond those explicitly identified here.


    \section{Getting Started} \label{sec:GettingStarted}                Download the zip file or tarball from \filesource.
Once downloaded, unpackage the file and open the project in your IDE\@.

\subsection{Description of RangeFinder Files}

\subsubsection{combolock.c, interrupt\_support.h, interrupt\_support.cpp, display.h, display.cpp}

Do not edit \textit{combolock.c}, \textit{interrupt\_support.h}, \textit{interrupt\_support.cpp}, \textit{display.}, or \textit{display.cpp}.

These files contain code to simplify interrupt management, and functions to place text on the display module.

\subsubsection{rotary-encoder.h \& rotary-encoder.c}

Do not edit \textit{rotary-encoder.h}

The \textit{rotary-encoder.c} file is where you will process inputs from the rotary encoder.

\subsubsection{servomotor.h \& servomotor.c}

Do not edit \textit{servomotor.h}

The \textit{servomotor.c} file is where you will control the servomotor.

\subsection{lock-controller.h \& lock-controller.c}

Do not edit \textit{lock-controller.h}

The \textit{lock-controller.c} file is where you will implement the logic for the combination lock.

%\subsubsection{shared\_variables.h}
%
%The \textit{shared\_variables.h} header file is where you will place any types that you define and where you will externalize any global variables that need to be used by more than one \textit{.c} file.
%
%It also contains a structure that you may use to access an analog-digital converter's registers,
%and it contains meaningfully-named constants to refer to specific pins you will use in this assignment.


\subsection{Assemble the Hardware}

\textcolor{red}{\textbf{BEFORE YOU PROCEED FURTHER:}}
\begin{description}
    \checkoffitem{Add the new hardware to your Cow~Pi as described in Appendix~\ref{sec:hardwareMods-mk4b}.}
\end{description}


    \vspace{1cm}

    After you have assembled the hardware, you can work on the lab entirely with your partner, perhaps pair programming,
    or you can decide to have one partner work on Section~\ref{sec:rotaryEncoder} and the other work on Section~\ref{sec:servo}.
    Regardless, you will need to work together on Section~\ref{sec:integration}, as this is when you will integrate the work from Sections~\ref{sec:rotaryEncoder}--\ref{sec:servo}.

    \section{Test Mode Specification} \label{sec:testMode}              If the \textbf{right switch} is in the \textit{left position} when the system boots, then the system shall enter its test mode.
The test mode is used to demonstrate the correct handling of the \textbf{rotary encoder} and of the \textbf{servomotor}.

\subsection*{When the system is in its test mode:}
%! suppress = LabelConvention
\begin{enumerate}
    \item \label{spec:encoderCounts} The system shall display the number of times that the \textbf{rotary encoder} is turned clockwise and the number of times that the \textbf{rotary encoder} is turned counterclockwise.
%        \begin{itemize}
%            \item Optionally, the system may also display the quadrature values as the \textbf{rotary encoder is turned}.
%        \end{itemize}
    \item \label{spec:servoCenter} If the \textbf{right pushbutton} is \textit{pressed}, the servomotor shall take its \textit{center position}.
    \item \label{spec:servoFollow} If the \textbf{right pushbutton} is \textit{not pressed}, the servomotor shall follow the \textbf{left switch}:
        \begin{itemize}
            \item If the \textbf{left switch} is in the \textit{left position}, the servomotor shall rotate \textit{fully clockwise}.
            \item If the \textbf{left switch} is in the \textit{right position}, the servomotor shall rotate \textit{fully counterclockwise}.
        \end{itemize}
    \item \label{spec:servoString} The system shall display ``SERVO: center'', ``SERVO: left'', or ``SERVO: right'' according to the command given to the servo per Requirements~\ref{spec:servoCenter}--\ref{spec:servoFollow}.
\end{enumerate}


    \section{Decoding the Rotary Encoder's Direction} \label{sec:rotaryEncoder}     \subsection{Theory of Operation}

The rotary encoder consists of a shaft that can rotate without stop in either direction,
and detents that hold the shaft in position when you are not rotating it.
Electrically, it has a pair of wipers, each of which is connected to a pin at one end, and which share a common pin at the other end.
By connecting the common pin to ground (as we have) and connecting the wipers' pins to pulled-up input pins on the microcontroller (as we have), then we can read the logic values of the wipers.

As the shaft rotates, the pair of wipers each cycle through a square wave.
The two wipers' square waves form a \textit{quadrature};
that is, they are 90\textdegree out of phase with each other, as shown in Figure~{fig:quadrature}.
By tracking which pin changes value first or second, we can determine which direction the shaft is rotating.

\begin{figure}
    \begin{center}
        \begin{tikzpicture}[x=2mm, y=2mm]
            \draw (0,0) -- ++(2.5,0) -- ++(0,5) -- ++(5,0) -- ++(0,-5) -- ++(5,0) -- ++(0,5) -- ++(5,0) -- ++(0,-5) -- ++(5,0) -- ++(0,5) -- ++(5,0) -- ++(0,-5) -- ++(2.5,0);
            \draw (-5, 2.5) node[left]{B};
            \draw (-1, 0) node[left]{\small 0};
            \draw (-1, 5) node[left]{\small 1};
            \draw (0,0) ++(0,-10) -- ++(0,5) -- ++(5,0) -- ++(0,-5) -- ++(5,0) -- ++(0,5) -- ++(5,0) -- ++(0,-5) -- ++(5,0) -- ++(0,5) -- ++(5,0) -- ++(0,-5) -- ++(5,0);
            \draw (-5, -7.5) node[left]{A};
            \draw (-1, -10) node[left]{\small 0};
            \draw (-1, -5) node[left]{\small 1};
            \draw[-Latex] (10, 10) -- ++(5,0) node[right]{\footnotesize clockwise};
            \draw[Latex-] (10, 8) -- ++(5,0) node[right]{\footnotesize counterclockwise};
            \draw[dashed] (3.75, -11) -- ++(0, 21) node[above] {detent};
            \draw[dashed] (3.75, -11) ++(10, 0) -- ++(0, 17);
            \draw[dashed] (3.75, -11) ++(20, 0) -- ++(0, 17);
        \end{tikzpicture}
        \caption{Quadrature encoding of the rotational position in a rotary encoder.} \label{fig:quadrature}
    \end{center}
\end{figure}

\subsection{Changes to \textit{rotary-encoder.c}}

Open \textit{rotary-encoder.c}.
Locate the \function{get_quadrature()} function.
The \texttt{A} component of the signal from Figure~{fig:quadrature} is on input pin~16, and the \texttt{B} component is on input pin~17.
\begin{description}
    \checkoffitem{Read the quadrature from the input pins into a variable.}
    \checkoffitem{Shift the bits in that variable so that the \texttt{A} component is in bit~0 and the \texttt{B} component is bit 1, and return the shifted variable.}
\end{description}

We will track the movement of the rotary encoder's shaft using the state machine depicted in Figure~\ref{fig:rotaryEncoderStateMachine}.
The convention we will use is that the states are named after the logic values of the quadrature's two components: \texttt{\textit{B\_A}}.
For example, if the bits returned by \function{get_quadrature()} are 0b0000'0010 then the state machine should be in state \texttt{HIGH\_LOW}.

\begin{figure}
    \begin{center}
        \begin{tikzpicture}
            \node[state, initial]   at (0, 10)  (highHigh){HIGH\_HIGH};
            \node[state]            at (5, 5)   (highLow){HIGH\_LOW};
            \node[state]            at (-5, 5)  (lowHigh){LOW\_HIGH};
            \node[state]            at (0, 0)   (lowLow){LOW\_LOW};
            \draw   (highHigh)  edge[-Latex, bend left]     node{\rotatebox{-45}{\textcolor{blue}{quadrature==0b10}}}   (highLow)
                    (highHigh)  edge[-Latex, bend right]    node{\rotatebox{45}{\textcolor{blue}{quadrature==0b01}}}    (lowHigh)
                    (highLow)   edge[-Latex, bend left]     node{\rotatebox{-45}{\textcolor{blue}{quadrature==0b11}}}   (highHigh)
                    (highLow)   edge[-Latex, bend left]     node{\rotatebox{45}{\parbox{5cm}{\centering\textcolor{blue}{quadrature==0b00 \&\& last\_state==HIGH\_HIGH}\\\textcolor{red}{rotating clockwise} } } }           (lowLow)
                    (lowHigh)   edge[-Latex, bend right]    node{\rotatebox{45}{\textcolor{blue}{quadrature==0b11}}}    (highHigh)
                    (lowHigh)   edge[-Latex, bend right]    node{\rotatebox{-45}{\parbox{5cm}{\centering\textcolor{blue}{quadrature==0b00 \&\& last\_state==HIGH\_HIGH}\\\textcolor{red}{rotating counterclockwise} } } }   (lowLow)
                    (lowLow)    edge[-Latex, bend right]    node{\rotatebox{-45}{\parbox{5cm}{\centering\textcolor{blue}{\phantom{x} \\ quadrature==0b01 \&\& last\_state==HIGH\_LOW}}}}   (lowHigh)
                    (lowLow)    edge[-Latex, bend left]     node{\rotatebox{45}{\parbox{5cm}{\centering\textcolor{blue}{\phantom{x} \\ quadrature==0b10 \&\& last\_state==LOW\_HIGH}}}}    (highLow)
                    ;
        \end{tikzpicture}
        \caption{State machine to determine a rotary encoder's direction of rotation.}\label{fig:rotaryEncoderStateMachine}
    \end{center}
\end{figure}

Locate the \function{initialize_rotary_encoder()}
\begin{description}
    \checkoffitem{Initialize the \lstinline{state} variable.}
\end{description}

For test mode, we'll use \lstinline{clockwise_count} and \lstinline{counterclockwise_count} to count the number of times that the shaft is turned in each direction for Requirement~\ref{spec:encoderCounts}.
For the combination lock, we'll use \lstinline{direction} to track which direction the shaft was turned.
Once the direction has been provided to the lock controller, we want to set \lstinline{direction} back to \lstinline{STATIONARY} so that a single turn isn't interpreted as many turns.

Location \function{count_rotations()}.
\begin{description}
    \checkoffitem{Add code to populate the buffer with the counts as required by Requirement~\ref{spec:encoderCounts}.
        Clearly label which count is which.
        \begin{itemize}
            \item Due to the limited amount of space on the display, you'll probably want to abbreviate, such as ``CW'' for clockwise and ``CCW'' for counterclockwise.
        \end{itemize}
    }
    \checkoffitem{Compile and upload your code, and confirm that a descriptive string is displayed with the counts (both of which should be 0 for now).}
\end{description}

Locate \function{get_direction()}.
\begin{description}
    \checkoffitem{Add code to save a copy of \lstinline{direction}.}
    \checkoffitem{Add code to set \lstinline{direction} to \lstinline{STATIONARY}.}
    \checkoffitem{Return the copy of \lstinline{direction}.}
\end{description}

Locate \function{handle_quadrature_interrupt()} and examine the state machine depicted in Figure~\ref{fig:rotaryEncoderStateMachine}.
You will implement the state machine in \function{handle_quadrature_interrupt()},
but first make sure that you understand how the state transitions driven by the quadrature signal in Figure~\ref{fig:quadrature} allow us to determine the direction of rotation.

\subsubsection*{But what about debouncing?}

You \textit{should} be wondering about debouncing.
After all, the rotary encoder relies on mechanical contacts.
In fact the \texttt{A} and \texttt{B} wipers \textit{do} experience switch bounce.
Fortunately, the state machine provides us a way to implement debouncing without using \function{cowpi_debounce_byte()} or \function{debounce_interrupt()}.

Consider the transition from \texttt{HIGH\_HIGH} to \texttt{HIGH\_LOW}:
\begin{itemize}
    \item This could be due to the user turning the shaft clockwise, causing the \texttt{A} component to drop from 1 to 0, or
    \item This could be due to switch bounce
        \begin{itemize}
            \item The user has turned the shaft counterclockwise
            \item The last quadrature change from that action is the \texttt{A} component rising from 0 to 1
            \item Switch bounce has caused the \texttt{A} component to drop from 1 to 0
        \end{itemize}
\end{itemize}
Now consider the transition from \texttt{HIGH\_LOW} to \texttt{HIGH\_HIGH}:
\begin{itemize}
    \item This could be due to the user rotating the shaft counterclockwise, causing the \texttt{A} component to rise from 0 to 1, or
    \item This could be due to the \texttt{A} wiper bouncing back as part of the aforementioned switch bounce
\end{itemize}
The reasoning is similar for transitions between \texttt{HIGH\_HIGH} and \texttt{LOW\_HIGH}.

Now consider the transition from \texttt{HIGH\_LOW} to \texttt{LOW\_LOW}.
Continuing the clockwise turn, the \texttt{B} component drops from 1 to 0.
If the change in \texttt{B}'s value were due to switch bounce, then we shall deny the state transition.
How?
By remembering which state the state machine was in before it was in \texttt{HIGH\_LOW}.

If the previous state was \texttt{HIGH\_HIGH} then \texttt{we know that the shaft is turning clockwise}.
On the other hand, if the previous state was \texttt{LOW\_LOW} then we know that the \texttt{B} wiper is bouncing.
Thus, we will only allow a transition from \texttt{HIGH\_LOW} to \texttt{LOW\_LOW} if the previous state was \texttt{HIGH\_HIGH}.
We will similarly limit all transitions into and out of \texttt{LOW\_LOW}.

(We cannot similarly limit the transitions into and out of \texttt{HIGH\_HIGH} because after the shaft stops at the detent, the user can continue to turn the shaft in the same direction, or they could turn the shaft in the opposite direction.)

\begin{description}
    \checkoffitem{Implement the state machine from Figure~\ref{fig:rotaryEncoderStateMachine} in \function{handle_quadrature_interrupt()}.}
    \checkoffitem{Add code to the transitions into \texttt{LOW\_LOW} to update \lstinline{clockwise_count}, \lstinline{counterclockwise_count}, and \lstinline{direction} as appropriate.}
\end{description}

Locate the \function{initialize_rotary_encoder()} function:
\begin{description}
    \checkoffitem{Uncomment this line: \\
    \lstinline{//    register_pin_ISR((1 << A_WIPER_PIN) | (1 << B_WIPER_PIN), handle_quadrature_interrupt);}
    }
    \checkoffitem{Compile and upload your code, and confirm that the clockwise count and counterclockwise count update on the display as specified by Requirement~\ref{spec:encoderCounts}.}
        \begin{itemize}
            \item It is acceptable for the counter to fail to update for the occasional turn if you turn the shaft quickly, but slow turns should always register, and most fast turns should register.
        \end{itemize}
\end{description}


    \section{Controlling the Servomotor} \label{sec:servo}              \subsection{Theory of Operation}

The servomotor is controlled with a signal that consists of periodic pulses.
The desired position of the servomotor is encoded as with width of each pulse.
This encoding, \textit{pulse width modulation} (PWM) is sufficiently useful that most microcontrollers can be configured to generate such a signal by setting a few of their memory-mapped I/O registers.
We will not do that.
Instead, we will directly form the signal by setting an output pin to 1 or 0 as needed.
Figure~\ref{fig:pwm} shows the relevant parameters of a PWM signal.

\begin{figure}
    \begin{center}
        \begin{tikzpicture}[x=2mm, y=2mm]
            \draw (0,0) -- (5,0) -- ++(0,5) -- ++(5,0) -- ++(0,-5) -- (25,0) -- ++(0,5) -- ++(5,0) -- ++(0,-5) -- (45,0)-- ++(0,5) -- ++(5,0) -- ++(0,-5) -- (65,0)-- ++(0,5) -- ++(5,0) -- ++(0,-5) -- (75,0);
            \draw (-1, 0) node[left]{\small 0};
            \draw (-1, 5) node[left]{\small 1};
            \draw (25, -3) -- ++(0,-6);
            \draw (45, -3) -- ++(0,-6);
            \draw[Latex-Latex] (26, -6) -- (44,-6) node[pos=.5]{\footnotesize \parbox{1.8cm}{\centering signal\\period}};
            \draw (45, 8) -- ++(0, 6);
            \draw (50, 8) -- ++(0, 6);
            \draw[Latex-] (44, 11) -- ++(-3, 0);
            \draw[Latex-] (51, 11) -- ++(3, 0);
            \draw (47.5, 11) node{\footnotesize \parbox{1.8cm}{\centering pulse\\width}};
        \end{tikzpicture}
        \caption{A PWM signal periodically sends a pulse. The $pulse\ width$ encodes the information being transmitted; the $signal\ period$ is how often the pulse is sent.}\label{fig:pwm}
    \end{center}
\end{figure}

A servomotor expects a pulse to be sent every 20ms (20,000\textmu s); this is the \textit{signal period}.
The \textit{pulse width} can vary between 500\textmu s and 2500\textmu s.
A 500\textmu s pulse directs the servomotor to rotate fully clockwise,
and a 2500\textmu s pulse directs the servomotor to rotate fully counterclockwise.
The angle varies linearly with the pulse width (for example, 1500\textmu is the central position).

\subsection{Changes to \textit{servomotor.c}}

Open \textit{servomotor.c}.

Locate these lines of code:
\begin{lstlisting}
#define PULSE_INCREMENT_uS  (INT32_MAX)
#define SIGNAL_PERIOD_uS    (INT32_MAX)
\end{lstlisting}
\begin{description}
    \checkoffitem{For \lstinline{SIGNAL_PERIOD_uS}, replace \lstinline{INT32_MAX} with the period (measured in microseconds) that the servomotor requires of its signal.}
    \checkoffitem{Select a factor common to the signal period and to each of the pulse widths necessary to realize Requirements~\ref{spec:servoCenter}--\ref{spec:servoFollow}.}
    \checkoffitem{For \lstinline{PULSE_INCREMENT_uS}, replace \lstinline{INT32_MAX} with that factor (measured in microseconds).
        This will be your timer interrupt period.}
\end{description}

Locate the \function{test_servo()} function.
\begin{description}
    \checkoffitem{Add code to call the \function{center_servo()}, \function{rotate_full_clockwise()}, and \\ \function{rotate_full_counterclockwise()} functions as necessary for Requirements~\ref{spec:servoCenter}--\ref{spec:servoFollow}.}
    \checkoffitem{Add code to populate \lstinline{buffer} with the strings required by Requirement~\ref{spec:servoString}.}
    \checkoffitem{Compile and upload your code, and confirm that the correct strings are displayed.}
\end{description}

Locate the \function{center_servo()}, \function{rotate_full_clockwise()}, and \function{rotate_full_counterclockwise()} functions.
\begin{description}
    \checkoffitem{Populate these functions with code to set the value of \lstinline{pulse_width_us} to the necessary pulse width as necessary to encode the servo's desired rotational position into the pulse width.}
\end{description}

Locate the \function{handle_timer_interrupt()} function.
\begin{description}
    \checkoffitem{Add variables to track the time until the next rising edge of the pulse and the next falling edge of the pulse.}
    \checkoffitem{Add code so that when the time until the next rising edge is 0\textmu s:
        \begin{itemize}
            \item Start the pulse by setting output pin~22 to 1.
            \item Update your variable that tracks the time until the next rising edge, to the time until the \textit{next} pulse should start.
            \item Update your variable that tracks the time until the next falling edge, to the time until \textit{this} pulse should finish.
        \end{itemize}
    }
    \checkoffitem{Add code so that when the time until the next falling edge is 0\textmu s:
        \begin{itemize}
            \item Finish the pulse by setting output pin~22 to 0.
        \end{itemize}
    }
    \checkoffitem{Add code to update the time remaining until the next rising edge and the time until the next falling edge, to reflect the time that has elapsed between timer interrupts.}
\end{description}

Locate the \function{initialize_servo()} function:
\begin{description}
    \checkoffitem{Uncomment this line: \\
        \lstinline{//    register_periodic_timer_ISR(0, PULSE_INCREMENT_uS, handle_timer_interrupt);}
    }
\end{description}

\vspace{1cm}

\textcolor{red}{Pre-emptive debugging.} Consider:
\begin{description}
    \checkoffitem{Multiply the definitions of \lstinline{SIGNAL_PERIOD_uS} and \lstinline{PULSE_INCREMENT_uS} by 1000.}
    \checkoffitem{Multiply the assignments to \lstinline{pulse_width_us} by 1000.}
    \checkoffitem{In \function{handle_timer_interrupts()}, replace the updates to pin~22 with updates to pin~20.}
\end{description}
This will have the effect of replacing ``microseconds'' with ``milliseconds'' and of replacing the servomotor with the right LED\@.
\begin{description}
    \checkoffitem{Compile and upload your code, and confirm that the correct right LED lights up and dims in the correct pattern, considering that you have replaced ``microseconds'' with ``milliseconds''.}
    \checkoffitem{Restore the original definitions, assignments, and pin number.}
\end{description}

\vspace{1cm}

\begin{description}
    \checkoffitem{Compile and upload your code, and confirm that the servomotor rotates to the correct positions as specified by Requirements~\ref{spec:servoCenter}--\ref{spec:servoFollow}.}
\end{description}


    \section{Combination Lock Specification} \label{sec:spec}           %! suppress = LabelConvention
\begin{enumerate}
    \item \textit{Initial Conditions:} The cart shall initially be stationary (speed 0), facing north (heading 0\textdegree), with the wheels pointing forward, with the brake off, and with the motor disengaged.
    \item \textit{No ``Reverse'' Gear:} The cart shall not be able to travel backwards: its speed must always be a non-negative value.
    \item \textit{Maximum Speed:} The cart has a maximum speed of 3,000~furlongs per fortnight.\footnote{
            A furlong is $\frac{1}{8}$ of a mile, approximately 200m. A fortnight is 14 days.
        }
        To prevent damage to the drivetrain, the system shall not allow the cart to attempt to go faster than that.
    \item \textit{Newton's First Law:} When no keys are pressed, the car shall hold its speed and shall travel in a straight line if its speed is positive.
    \item \label{spec:importantData} \textit{Important Data:} The system will keep track of the carts speed, heading, and distance travelled since the last system reset.
        \begin{enumerate}
            \item The speed will be tracked to a precision of 1~furlong per fortnight.
            \item The heading will be tracked to a precision of 0.000,001\textdegree.
                \begin{itemize}
                    \item The heading is the traditional compass heading, with right turns increasing the heading and left turns decreasing the heading, modulo~360\textdegree.
                \end{itemize}
            \item The distance will be tracked to a precision of 0.001~microfurlongs.
        \end{enumerate}
    \item \label{spec:standardDisplay} \textit{Standard Display:} Under normal operation, the display module shall always display the current speed (in furlongs per fortnight), the current heading (in degrees), and the distance travelled (in microfurlongs) since the system was last reset.
        While the system will store values more precisely, the displayed values shall be \textit{rounded down} to an integer.
        For example, a cart that has travelled 456.4 microfurlongs and is currently travelling 123.0~furlongs per fortnight at a heading of 270.8\textdegree\ would show: \\
        \display{
            \colorbox{LightGreen}{Spd\phantom{x}123fpf\phantom{fxxx}Hdg} \vspace{-1mm}\\
            \colorbox{LightGreen}{Odo\phantom{x}456\textmu fur\phantom{xx}270}
        }
        (This is an example only; no particular layout is specified.)
        \begin{itemize}
            \item The display must show all significant digits of the vehicle's speed, and the speed must be clearly indicated with a label and/or with units.
            \item The display must display at least the lower four decimal digits of the rounded distance travelled, and the distance must be clearly indicated with a label and/or with units.
                The odometer may ``roll over'' at 10,000 microfurlongs or at a greater value if space is available for more digits.
                The odometer must never show a negative distance.
            \item The display must show all significant digits of the vehicle's direction, and the direction must be clearly indicated with a label and/or with a ``degrees'' indicator (`\textdegree').
        \end{itemize}
    \item \label{spec:dPad} \textit{Controlling the Cart:} The \textbf{numeric keypad} shall serve as a \textit{directional pad}.
    \begin{enumerate}
        \item \label{spec:pressOnlyOneKey} The user will press at most one key at a time;
            the user shall never press two or more keys at the same time.
        \item When the user presses the `2' key, the cart shall accelerate if the motor is engaged.
            The increase in speed is per-press.
            Continuously holding the `2' key shall not increase the speed further;
            the only way to increase the speed further is to press the `2' key again.
            \begin{itemize}
                \item The cart shall not accelerate when the motor is disengaged.
                \item If the user attempts to accelerate to a speed faster than the maximum speed, then the speed shall be the maximum speed.
            \end{itemize}
        \item When the user presses the `8' key, the cart shall decelerate.
            The decrease in speed is per-press.
            Continuously holding the `8' key shall not decrease the speed further;
            the only way to decrease the speed further is to press the `2' key again.
            \begin{itemize}
                \item If the user attempts to decelerate to a negative speed, then the speed shall be 0.
            \end{itemize}
        \item When the user presses the `0' key, the cart shall brake.
            Braking results in the speed immediately being set to 0, and is also required to engage and disengage the motor.
        \item \label{spec:turnLeft} When the user presses the `4' key, the cart shall turn left.
            The cart shall continue to turn left while the user continues to press the `4' key.
            The turn is complete when the user releases the `4' key.
        \item \label{spec:turnRight} When the user presses the `6' key, the cart shall turn right.
            The cart shall continue to turn right while the user continues to press the `6' key.
            The turn is complete when the user releases the `6' key.
        \item Pressing any other key shall have no effect.
    \end{enumerate}
    \item \textit{Engaging/Disengaging the Motor:} The \textbf{left switch} shall be used to engage and disengage the motor.
        \begin{enumerate}
            \item The motor shall only change between being engaged and being disengaged as a result of the left switch's position when the user releases the brake (the `0' key).
            \item When the user releases the brake, if the left switch is in the \textit{left position} then the motor shall disengage.
            \item When the user releases the brake, if the left switch is in the \textit{right position} then the motor shall engage.
        \end{enumerate}
    \item \textit{Gear Selector:} The \textbf{right switch} shall be used to change between \textit{low gear} and \textit{high gear}.
        \begin{enumerate}
            \item When the right switch is in the \textit{left position}, the cart shall be in \textit{low gear}.
                Pressing the `2' button shall cause the cart to accelerate by 1~furlong per fortnight.
                Pressing the `8' button shall cause the cart to decelerate by 1~furlong per fortnight.
            \item When the right switch is in the \textit{right} position, the cart shall be in \textit{high gear}.
                Pressing the `2' button shall cause the cart to accelerate by 3~furlongs per fortnight.
                Pressing the `8' button shall cause the cart to decelerate by 3~furlongs per fortnight.
        \end{enumerate}
    \item \label{spec:TurnSignalControls} \textit{Turn Signal Controls:} The \textbf{pushbuttons} shall serve as turn signal selectors.
        \begin{itemize}
            \item The user shall indicate their preference to turn left by pressing the \textbf{left pushbutton};
                the user shall indicate their preference to turn right by pressing the \textbf{right pushbutton}.
        \end{itemize}
    \item \label{spec:TurnSignals} \textit{Turn Signals:} One function of the \textbf{LEDs} is as turn signals.
        \begin{enumerate}
            \item If the user has indicated a preference to turn left, then the \textbf{left LED} shall blink;
                if the user has indicated their preference to turn right, then the \textbf{right LED} shall blink.
            \item \label{spec:blinkRate} The blink rate shall be repeating 750ms on, 750ms off.
            \item The turn signal shall continue to blink until the user has completed the turn.
                \begin{itemize}
                    \item Per Requirements~\ref{spec:turnLeft}--\ref{spec:turnRight}, releasing the `4' or `6' key shall cause a blinking turn signal to stop blinking
                \end{itemize}
            \item An LED shall not blink when its turn signal has not been selected.
        \end{enumerate}
    \item \textit{Brake Lights:} The other function of the \textbf{LEDs} is as brake indicators.
        \begin{enumerate}
            \item If no turn signal has been selected, then both LEDs shall illuminate steadily for as long as the user is braking.
            \item If a turn signal has been selected, then the corresponding LED shall blink, and the LED for the non-selected direction shall illuminate steadily.
            \item If the user is not braking, then neither LED shall illuminate steadily.
        \end{enumerate}
    \item \textit{Turn Rate:} While the `4' or `6' key is pressed, the cart shall change its heading at the rate of $\pm$0.3\textdegree~per microfurlong.
        \begin{itemize}
            \item For example, if the cart is travelling at the rate of 500~furlongs per fortnight, then it travels 0.5~microfurlongs each nanofortnight, and so a right turn would result in a heading change of 0.15~degrees.
            \item If the cart is stationary (the speed is 0), then pressing the `4' or `6' key would not cause the cart to change its heading.
        \end{itemize}
    \item \label{spec:dataRefreshRate} \textit{Data Refresh Rate:} Every nanofortnight,\footnote{
            A nanofortnight is 1.2096ms. We will approximate this as \textbf{1.208ms}, an error under 0.15\%, resulting in a loss of fewer than 30 minutes over a fortnight.
        } the system shall update the distance that the cart has travelled and its heading.
    \item \label{spec:displayRefreshRate} \textit{Display Refresh Rate:} The system shall update the \textbf{display module} evey 256 nanofortnights.
    \item \label{spec:diagnosticDisplays} \textit{Diagnostic Displays:} The letter keys can be used to show diagnostic data.
        The display module shall be refreshed at the same rate specified in Requirement~\ref{spec:displayRefreshRate}.
        \begin{enumerate}
            \item When the user presses the `A' key, the system shall display the standard display described in Requirement~\ref{spec:standardDisplay}.
            \item When the user presses the `B' key, the system shall display the full variable used to track the cart's speed.
                Other diagnostic data may also be displayed.
            \item When the user presses the `C' key, the system shall display the full variable used to track the cart's distance.
                Other diagnostic data may also be displayed.
            \item When the user presses teh `D' key, the system shall display the full variable used to track the cart's direction.
                Other diagnostic data may also be displayed.
        \end{enumerate}
    \item \label{spec:responsive} The system shall always be responsive to user input.
        \begin{itemize}
            \item While one pushbutton is being pressed, the system does not need to respond to the other being pressed
            \item As noted in Requirement~\ref{spec:pressOnlyOneKey}, the user will never press two keys at the same time
            \item But for those exceptions, there shall be no noticeable lag when responding to an input
        \end{itemize}
\end{enumerate}

    \section{Implementing the Combination Lock} \label{sec:integration} The remaining portion of this assignment integrates the work you put into Sections~\ref{sec:initialSoftware}, \ref{sec:distance}, and \ref{sec:sound}.

If you and your partner decided to separately work on Sections~\ref{sec:distance} and \ref{sec:sound}, and if you're still waiting for your partner to finish their section, there is some work in this section that you can get started on.
For example, the user interface portion of Section~\ref{subsubsec:obtainingThreshold} can be while waiting for your partner to finish; however, completing the Threshold Adjustment mode will require the ``distance'' variable from Section~\ref{subsec:distanceSinglePulseOperation}, and making use of the threshold range in Section~\ref{subsubsec:applyingThreshold} requires a working alarm from Section~\ref{subsec:soundSinglePulseOperation}.
Similarly, if you have a working distance sensor, you can do Sections~\ref{subsubsec:repeatedPulses} and \ref{subsubsec:approachRate} without a working alarm, but you will need a working alarm for Section~\ref{subsubsec:repeatedAlarm}.

\subsection{Single-Pulse Operation} \label{subsec:integrationSpeedSinglePulseOperation}

Your distance sensor code responds to a ping request by initiating an ultrasonic pulse, which is what it should do.
Your alarm code responds to a ping request by chirping the piezobuzzer and strobing the right LED;
however, the alarm should only sound if an object has been detected.
There is another incompatibility between the two subsystems:
both set the ping request variable to \lstinline{false} after responding, which means that only one can actually respond to the ping request.

You can resolve this by introducing another shared variable that represents an alarm request.
\begin{description}
    \checkoffitem{When an object is detected, the sensor subsystem should request an alarm.}
    \checkoffitem{When an alarm is requested, the alarm subsystem should chirp the piezobuzzer, strobe the LEDs, and set the alarm request to \lstinline{false} (instead of the ping reqeust).}
\end{description}


\subsection{Threshold Adjustment}

When in Single-Pulse Operation mode, the system now alarms once whenever an object is detected.
Requirement~\ref{spec:singlePulseResponse}, however, says that the LEDs should strobe whenever an object is detected,
but the piezobuzzer should chirp only when the detected object is closer than the threshold range.
\begin{description}
    \checkoffitem{Introduce a shared variable to store the threshold range with an initial value of 400cm, the greatest valid threshold range.}
\end{description}


\subsubsection{Obtaining the Threshold Range} \label{subsubsec:obtainingThreshold}

Requirement~\ref{spec:thresholdAdjustment} describes the user interaction for inputting the threshold range.
Implement this requirement in \textit{user\_controls.c}.

(Note: while you \textit{should} write your code to safely handle a user attempting to enter more than three digits, we will not attempt to do so during grading.)

\begin{description}
    \checkoffitem{After the user has entered a valid threshold range, assign that value to the shared variable.}
\end{description}


\subsubsection{Applying the Threshold Range} \label{subsubsec:applyingThreshold}

You now have the threshold range, externalized from \textit{user\_controls.c}, and the distance to an object, externalized from \textit{sensor.c}.

\begin{description}
    \checkoffitem{Modify the alarm code so that, when in Single-Pulse Operation mode, the alarm strobes the LEDs whenever an object is detected, but it only chirps the piezobuzzer when the distance is no greater than the threshold range.}
\end{description}


\subsection{Normal Operation}

You have now completed every operating mode except ``Normal Operations.''
Look over Requirement~\ref{spec:normalOperation}.

\subsubsection{Repeated Ultrasonic Pulses} \label{subsubsec:repeatedPulses}

When in Normal Operation mode, the sensor does not wait for a ping request.
Instead, it can (and should) send a ping whenever the sensor is in its ``ready'' state.

Modify your code so that:
\begin{description}
    \checkoffitem{When the system is in Normal Operation mode, it initiates a pulse whenever the sensor is ``ready.''}
    \checkoffitem{Re-compute the distance and update the display with each pulse.}
\end{description}

\subsubsection{Compute the Rate of Approach} \label{subsubsec:approachRate}

Since the distance sensor cannot determine the change of direction to the detected object (indeed, it cannot determine the direction), you cannot calculate the object's speed.
You can, however, calculate the longitudinal component of its speed -- that is, how fast it is approaching the sensor.
(A retreating object would, of course, have a negative rate of approach.)
Since the sensor does not detect doppler shift, you must calculate the rate of approach based on the difference between two distance measurements.
There are two options to compute the rate of approach in centimeters per second.
Choose one.
(Or, if you can think of a third approach, you can choose it.)

\begin{description}
    \item[Compare two distances separated in time by 1~second] \phantom{ } \\
        If you have a working alarm subsystem, then you have a timer interrupt firing every 100\textmu s.
        For every 10,000 times that the alarm timer's ISR is triggered, 1~second has passed.
        If you externalize the number of times that the ISR runs, then your code in \textit{sensor.c} can subtract the current distance from the distance calculated 1~second ago.
        The specification does not require that the rate of approach be updated more frequently than once per second, so this is an acceptable approach.
        The only complication is that you'll need to make sure that the user doesn't see an erroneous speed value in the second between obtaining the very first distance after detection and obtaining the second distance.

    \item[Compare the pulse duration for two adjacent measurements] \phantom{ } \\
        Alternatively, you can update the speed every 65,536\textmu s.
        The actual distance travelled will be too small to measure with an integer number of centimeters, so we shall instead rely on the differences in the pulse durations $\tau_1 and \tau_2$.
        Regardless of whether you're using the old hardware or the new hardware, this calculation will need to use 64-bit integers for the intermediate terms.
        If we assume that the wall-clock times of reflection are exactly 65,536\textmu s apart,\footnote{
            If the object is moving, then the actual difference in the wall-clock times of reflection won't be 65,536\textmu s apart; however, the error resulting from this simplifying assumption is less than the rounding error.
        } and that the air temperature does not change appreciably between detections then:
            \begin{description}
                \item[Old Hardware] \phantom{ } \\
                    \begin{align*}
                        speed & = \frac{\Delta distance}{time} %= \frac{\left( \tau_2 \times \frac{18,025 cm}{2^{21}} - \tau_1 \times \frac{18,025 cm}{2^{21}} \right)}{65,536\mu s}
                          = \frac{\left( \tau_2 - \tau_1 \right) \times 18,025 cm}{2^{21} \times 65,536\mu s} \\
                        & \\
                        & =  \left( \tau_2 - \tau_1 \right) \times 281,640,625 \div 2^{31} \ \frac{cm}{s}
                    \end{align*}
                \item[New Hardware] \phantom{ } \\
                    \begin{align*}
                        speed & = \frac{\Delta distance}{time} = \frac{\left( \tau_2 - \tau_1 \right) \times \left( 256,108,888 - 121,907 \times ADC\_register\_value \right) cm}{2^{33} \times 65,536\mu s} \\
                        & \\
                        & =  1,000,000 \times \left( \tau_2 - \tau_1 \right) \times \left( 256,108,888 - 121,907 \times ADC\_register\_value \right) \div 2^{47} \ \frac{cm}{s}
                    \end{align*}
            \end{description}
    \checkoffitem{Add code to \function{manage_sensor()} to calculate the rate of approach when the system is in Normal Operations.}
    \checkoffitem{Whenever you have calculated a new rate of approach, update the display with that rate of approach.}
\end{description}

\subsubsection{Repeated Alarm} \label{subsubsec:repeatedAlarm}

Your remaining task is to repeatedly activate the alarm when in Normal Operations.
You probably noticed the \lstinline{total_period} variable in \textit{alarm.c}, immediately below the \lstinline{on_period} variable.
Just as \lstinline{on_period} is used to determine how long the piezobuzzer should emit its tone and the LEDs should be illuminated, the \lstinline{total_period} is used to determine how much time should transpire between activations of the alarm.
(If you wish, you can imagine a \lstinline{off_period} variable whose value is always \lstinline{total_period} - \lstinline{on_period}.)

\paragraph{Repeatedly activate the alarm}
\begin{description}
    \checkoffitem{Add code so that, when an object is detected while the system is in Normal Operations mode, the alarm will repeatedly activate.}
        \begin{itemize}
            \item We recommend that you use the \lstinline{total_period} variable as part of that solution.
        \end{itemize}
\end{description}
\paragraph{Vary the time between activations}
\begin{description}
    \checkoffitem{Using Table~\ref{tab:alarmPeriods} and the distance to the detected object, change \lstinline{total_period}'s value so that the time between alarm activations is correct.}
\end{description}


    \section{Turn-in and Grading}                                       \filesubmission

\policyforcodethatdoesnotcompile

\latepolicy

\subsection*{Rubric}

This assignment is worth 20 points.
\begin{description}
    \rubricitem{4}{\textit{problem1.c} produces the specified output.}
    \rubricitem{4}{\function{iz_digit()} in \textit{problem2.c} determines whether
    or not a character is a digit.}
    \rubricitem{4}{\function{decapitalize()} in \textit{problem2.c} converts
    uppercase letters to lowercase and leaves other characters unchanged.}
    \rubricitem{4}{\function{is_even()} in \textit{problem3.c} determines whether
    a number is even or odd.}
    \item[\hspace{1cm}]\function{produce_multiple_of_ten()} in \textit{problem3.c}
    has the following:
    \begin{description}
        \rubricitem{1}{Code to assign the value 5 to the variable \lstinline{five}}
        \rubricitem{1}{Code to divide an even number by 2}
        \rubricitem{1}{Code to subtract 1 from an odd number}
        \rubricitem{1}{Correct functionality}
    \end{description}
    \item[Penalties]
    \penaltyitem{4}{for each solution that depends on a prohibited character.}
    \penaltyitem{4}{for each solution that hard-codes a return value instead of attempting to solve the specified problem}
    \softwareengineeringpenalties
\end{description}


    \section*{Epilogue}                                                 \CombinationLockInstalled

    \textit{The end...?}

%    \newpage\appendix
    \appendix

    \section{Appendix: Lab Checkoff}                                    You are not required to have your assignment checked-off by a TA or the professor.
If you do not do so, then we will perform a functional check ourselves.
In the interest of making grading go faster, we are offering a small bonus % to get your assignment checked-off at the start of your scheduled lab time immediately after it is due.
% Because checking off all students during lab would take up most of the lab time, we are offering a slightly larger bonus
if you complete your assignment early and get it checked-off by a TA or the professor during office hours.

\subsection*{TODO}

%\begin{enumerate}
%    \precheckoffitem{Position your Cow Pi's storage box upright, a little more than 1~meter from the Cow Pi.}
%    \precheckoffitem{Place both switches in the left position.}
%    \precheckoffitem{Upload your code to your Cow Pi, and leave your code open in the IDE.}
%    \precheckoffitem{Confirm that the system detects the box and not something closer (such as a computer or the table surface).}
%
%    \checkoffitem{Show and explain to the TA how your code generates a tone with a frequency of 5kHz; that is, it has a period of 200\textmu s.}
%    \checkoffitem{Place the right switch in the right position, putting the system in Continuous Tone mode.
%        The system generates a continuous 5kHz tone.}
%    \item[] (TA, confirm that the tone is 5kHz by code inspection and by ear; confirm with the HuskerScope spectrum analyzer if you aren't sure.) \\
%        \textit{+1 There is code to generate an audible tone} \\
%        \textit{+1 The system continuously generates the tone when in Continuous Tone mode} \\
%        \textit{+2 The audible tone has a frequency of 5kHz}
%
%    \checkoffitem{Place the left switch in the right position, putting the system in Threshold Adjustment mode.
%        The system prompts for a new threshold range. \\
%        \textit{+1 The user is prompted to enter a new threshold range when the system is in Threshold Adjustment mode}}
%    \checkoffitem{Enter a range of 25, using the `\#' key to indidicate that you have fininished entering the value.
%        The system displays a helpful error message explaining that this is not a valid threshold range.
%        The system then prompts the user for a new threshold range. \\
%        \textit{+1 The user is given a helpful error message after entering an invalid threshold range} \\
%        \textit{+1 The user is re-prompted to enter a threshold range after entering an invalid threshold range}}
%    \checkoffitem{Enter a range of 450, using the `\#' key to indidicate that you have fininished entering the value.
%        The system displays a helpful error message explaining that this is not a valid threshold range.
%        The system then prompts the user for a new threshold range.}
%    \checkoffitem{Enter a range of 75, using the `\#' key to indidicate that you have fininished entering the value.
%        The system displays a message confirming the new threshold range. \\
%        \textit{+2 Valid threshold ranges are those between 50cm and 400cm, inclusive} \\
%        \textit{+1 The user is shown a confirmation message after entering a valid threshold range} \\
%        \textit{+2 The user can enter a new threshold range when the system is in Threshold Adjustment mode}}
%
%    \checkoffitem{Place the right switch in the left position, putting the system in Single Pulse mode.
%        The system might indicate that no object has been detected yet; however, this is not required information before initiating a ping.}
%    \checkoffitem{Show and explain to the TA how your code initiates a pulse.}
%    \checkoffitem{Show and explain to the TA how your code measures the length of a pulse.}
%    \checkoffitem{Show and explain to the TA how your code achieves the required precision (no greater than 1\textmu s) and accuracy (immediately detect pulse edges without waiting for code in the main loop to poll the pin).}
%    \checkoffitem{Press the pushbutton to initiate a pulse.
%        The right LED strobes once.
%        The piezodisc does not chirp.
%        The system displays the correct distance to the wall (or book or other object). \\
%        \textit{+2 There is code to initiate an ultrasound pulse} \\
%        \textit{+3 There is code to detect the length of the pulse} \\
%        \textit{+3 The pulse's length is measured to a precision of no greater than 1\textmu s} \\
%        \textit{+3 The pulse's length is measured as accurately as possible} \\
%        \textit{+2 The user can request a ping when the system is in Single Pulse mode} \\
%        \textit{+2 The distance to an object is correctly calculated from the pulse's length} \\
%        \textit{+2 When an object is detected, the system displays the distance to the object} \\
%        \textit{+2 When an object is detected in Single-Pulse mode, the system generates exactly one alarm} \\
%        \textit{+2 A strobe is an illumination of the right LED for 50ms} \\
%        \textit{+2 A strobe occurs for any detected object}}
%    \checkoffitem{Slightly change the distance between the Cow Pi and the target object.
%        Press the pushbutton to initiate a pulse.
%        The LED strobes once.
%        The piezodisc does not chirp.
%        The system displays the new distance to the wall (or book or other object). \\
%        \textit{+2 The user can request another ping when the system is in Single Pulse mode}}
%
%    \checkoffitem{Place the left switch in the left position, putting the system in Normal Operation mode.
%        The system displays the distance to the target object, and it displays an approach rate of 0cm/s.
%        The LED strobes once per seccond (100ms), but the piezodisc does not chirp. \\
%        \textit{+1 The switches control the mode of operation as specified} \\
%        \textit{+2 When an object is detected in Normal Operation mode, the system repeatedly generates alarms}}
%    \checkoffitem{Slowly move the Cow Pi closer to the wall, or slowly move the book (or other object) closer to the Cow Pi.
%        As you do so, vary the rate of approach slighly to demonstrate that the rate of approach updates.
%        The displayed distance changes with the decreasing distance to the target object.
%        The system displays a plausible, positive rate of approach that updates at least once every second. \\
%        \textit{+2 When an object is detected in Normal Operation mode the rate of approach is displayed} \\
%        \textit{+2 When an object is detected in Normal Operation mode, the rate of approach is updated at least once every second}}
%    \checkoffitem{As the distance between the Cow Pi and the target object decreases, note that:
%        \begin{description}
%            \item[When the distance falls below 100cm] the LED strobes more frequently, once every 750ms (\textthreequarters sec)
%            \item[When the distance falls below 75cm] the piezodisc chirps every 750ms
%            \item[When the distance falls below 50cm] the LED strobes, and the piezodisc chirps, every 500ms (\textonehalf sec)
%            \item[When the distance falls below 25cm] the LED strobes, and the piezodisc chirps, every 250ms (\textonequarter sec)
%            \item[When the distance falls below 10cm] the LED strobes, and the piezodisc chirps, every 125ms ($\frac{1}{8}$ sec)
%        \end{description}
%        \textit{+2 A chirp is an audible tone lasting 50ms} \\
%        \textit{+2 When the system repeatedly generates alarms, the time between alarms is as specified}}
%
%    \checkoffitem{Place the both switches in the right position, putting the system in Threshold Adjustment mode.
%        The system prompts for a new threshold range.}
%    \checkoffitem{Enter a range of 55.
%        The system displays a message confirming the new threshold range.}
%    \checkoffitem{Place the both switches in the left position, putting the system in Normal Operation mode.
%        The system displays the distance to the target object, and it displays an approach rate of 0cm/s.}
%    \checkoffitem{Slowly move the Cow Pi away from the wall, or slowly move the book (or other object) away from the Cow Pi.
%        The displayed distance changes with the decreasing distance to the target object.
%        The system displays a plausible, negative rate of approach.}
%    \checkoffitem{As the distance between the Cow Pi and the target object decreases, the alarms become less urgent.
%        Note that as the distance increases above 55cm, the piezodisc stops chirping but the LED continues to strobe. \\
%        \textit{+2 A chirp occurs for a detected object that is closer than the threshold range} \\
%        \textit{+2 A chirp only occurs for a detected object that is closer than the threshold range}}
%
%    \checkoffitem{Reorient the Cow Pi, or remove the book (or other object) so that there are no in-range objects to detect.
%        The system displays a message indicating that no object is detected.
%        The LED does not strobe, and the piezodisc does not chirp.\\
%        \textit{+3 The code correctly recognizes the that no object has been detected, if no object reflects the ultrasound pulse} \\
%        \textit{+2 When there is no in-range object, the system displays a message to that effect} \\
%        \textit{+2 A strobe only occurs for a detected object}}
%
%    \checkoffitem{Show the TA any code they have not yet examined. \\
%        \textit{+1 The code is clean, well-organized, has good variable and function names, and is otherwise understandable}}
%\end{enumerate}
%
%
%\begin{description}
%    \precheckoffitem{Establish that the code you are demonstrating is the code
%    you submitted to to \filesubmission.}
%    \begin{itemize}
%        \item If you are getting checked-off during lab time, show the TA that the
%        file was submitted before it was due.
%        \item Download the file into your ComboLab directory. If necessary,
%        rename it to \textit{ComboLab.ino}.
%    \end{itemize}
%    \precheckoffitem{Upload \textit{ComboLab.ino} to your \developmentboard\ and open the
%    Serial Monitor.}
%\end{description}
%
%\begin{enumerate}
%    \checkoffitem{The combination screen is displayed
%    \texttt{\phantom{88}-\phantom{88}-\phantom{88}}) with the cursor (two
%    decimal points) blinking in the left-most position. No numbers are
%    displayed in the combination.} \\
%    \textit{+2 The lock is locked when powered-up.} \\
%    \textit{+2 When locked, shows combo-entry display, initially without numbers
%        (power-up).} \\
%    \textit{+2 The cursor is represented with decimal points in the relevant
%    position.} \\
%    \textit{+2 The cursor blinks.}
%    \checkoffitem{Place both switches in the left position.}
%    \checkoffitem{Press the right button twice. The cursor moves into the middle
%    position and then the right-most position.} \\
%    \textit{+3 The user can move the cursor using the right button.}
%    \checkoffitem{Press the right button again. the cursor moves into the left-most
%    position.} \\
%    \textit{+1 The cursor wraps-around from the last number to the first
%    number.}
%    \checkoffitem{Press 1, then A. The display shows \texttt{1.A.-\phantom{88}-\phantom{88}}}. \\
%    \textit{+1 Combinations use 2-hex-digit numbers.} \\
%    \textit{+4 Numbers are entered using the keypad.}
%    \checkoffitem{Press the left button. The display shows \texttt{error} and then
%    \texttt{1.A.-\phantom{88}-\phantom{88}}.} \\
%    \textit{+1 Submit ``error'' combination with left button.} \\
%    \textit{+1 Incomplete combination produces error message.} \\
%    \textit{+1 After the error message, the combo-entry display returns.}
%    \checkoffitem{Finish entering an incorrect combination. The display shows all
%    three numbers, separated by dashes.} \\
%    \textit{+1 Combinations are three numbers separated by dashes.}
%    \checkoffitem{Press the left button. The display shows \texttt{badtry 1} and
%    then the combo-display display.} \\
%    \textit{+1 Submit ``bad try'' combination with left button.} \\
%    \textit{+1 Wrong combination produces bad-try message.} \\
%    \textit{+2 After the first bad try, the combo-entry display returns.}
%    \checkoffitem{Enter another incorrect combination and press the left button. The
%    display shows \texttt{badtry 2} and then the combo-display display.} \\
%    \textit{+1 The ``bad try'' number increments.} \\
%    \textit{+2 After the second bad try, the combo-entry display returns.}
%    \checkoffitem{Enter another incorrect combination and press the left button. The
%    display shows \texttt{badtry 3} and then \texttt{alert!}. The external
%    LED rapidly blinks.} \\
%    \textit{+1 After the third bad try, the sytem displays an alert message.} \\
%    \textit{+2 After the third bad try, the external LED rapidly blinks.}
%    \checkoffitem{Press buttons and keys a few times. Nothing happens except that
%    the alert message is still displayed and the LED still blinks.} \\
%    \textit{+1 After the third bad try, the system is non-responsive.}
%    \checkoffitem{Press the \developmentboard's RESET button (on the top of the \developmentboard).}
%    \checkoffitem{Enter the correct combination and press the left button. The
%    system displays \texttt{lab open}.} \\
%    \textit{+2 Submit correct combination with left button.} \\
%    \textit{+4 When the user locks the lock, it displays ``lab open.''}
%    \checkoffitem{Double-check that both switches are in the left position. Press
%    the left button. Nothing happens. Press the right button. Nothing
%    happens.} \\
%    \textit{+1 Single-presses of only one button have no effect when the lock is unlocked and the switches are not in the right position.}
%    \checkoffitem{Place both switches in the right position and press the left
%    button. The system displays \texttt{enter} then
%    \texttt{\phantom{88}-\phantom{88}-\phantom{88}}.} \\
%    \textit{+4 Start changing combo by pushing left button while both switches
%    are to the right.} \\
%    \textit{+2 When starting to change the combo, ``enter'' is displayed.} \\
%    \textit{+2 After displaying ``enter,'' the combo-entry display is shown.}
%    \checkoffitem{Enter a new combination.}
%    \checkoffitem{Press the left button. Nothing happens.} \\
%    \textit{+\textonehalf\ Left button has no effect unless left switch is
%    moved.}
%    \checkoffitem{Place the left switch in the left position and press the left
%    button. The system displays \texttt{re-enter} then
%    \texttt{\phantom{88}-\phantom{88}-\phantom{88}}.} \\
%    \textit{+4 Start confirming by moving the left switch and pressing the left
%    button.} \\
%    \textit{+2 When starting to change the combo, ``re-enter'' is displayed.} \\
%    \textit{+2 After displaying ``re-enter,'' the combo-entry display is shown.}
%    \checkoffitem{Enter a non-matching combination.}
%    \checkoffitem{Press the left button. Nothing happens.} \\
%    \textit{+\textonehalf\ Left button has no effect unless right switch is
%    moved.}
%    \checkoffitem{Place the right switch in the left position and press the left
%    button. The system displays \texttt{nochange}. Though unspecified, it is
%    acceptable to display \texttt{lab open} after displaying \texttt{nochange} for at least one second.} \\
%    \textit{+4 Compare combos by moving the right switch and pressing the left
%    button.} \\
%    \textit{+2 When the combos do not match, ``nochange'' is displayed.}
%    \checkoffitem{Double-check that both switches are in the left position. Either
%    double-click the right button \textit{or} press both buttons at the same
%    time. The system displays \texttt{closed} and then \texttt{\phantom{88}-\phantom{88}-\phantom{88}}.} \\
%    \textit{+4 Re-lock the lock using one of the specified techniques.} \\
%    \textit{+2 When locked, shows combo-entry display, initially without numbers
%        (re-locked).}
%    \checkoffitem{Enter the original, correct combination and press the left button.
%    The system displays \texttt{lab open}.} \\
%    \textit{+2 When the combos do not match, the previously-correct combo
%    remains the correct combo.}
%    \checkoffitem{Move both switches to the right, enter a new combination, move the
%    left switch to the left, and press the left button. The system displays
%    \texttt{re-enter} then \texttt{\phantom{88}-\phantom{88}-\phantom{88}}.}
%    \checkoffitem{Enter the matching combination, move the right switch to the left,
%        and press the right button. The system displays \texttt{nochange}. Though
%        unspecified, it is acceptable to display \texttt{lab open} after
%        displaying \texttt{changed} for at least one second.} \\
%    \textit{+2 When the combos match, ``changed'' is displayed.}
%    \checkoffitem{Double-check that both switches are in the left position. Either
%    double-click the right button \textit{or} press both buttons at the same
%    time. The system displays \texttt{closed} and then \texttt{\phantom{88}-\phantom{88}-\phantom{88}}.}
%    \checkoffitem{Enter the new combination and press the left button. The system
%    displays \texttt{lab open}.} \\
%    \textit{+2 When the combos match, the new combo is the correct combo.}
%    \checkoffitem{Press the \developmentboard's RESET button.}
%    \checkoffitem{Enter the new combination and press the left button. The system
%    displays \texttt{lab open}.} \\
%    \textit{+4 The correct combination persists while the Arduino is
%    powered-down.}
%\end{enumerate}

    \newpage

    \section{Installing Necessary Hardware Components} \label{sec:hardwareMods-mk4b}    You will need to add a couple of hardware components to your Cow Pi circuit before you can start this lab.

\subsection{Necessary Components}

Figure~\ref{fig:components-mk4b} shows the components you will need for the range finder and alarm.

\begin{figure}
    \centering
    \includegraphics[width=10cm]{hardware/mk4b/components} % TODO: change to mk4b
    \caption{Components needed for the Range Finder \label{fig:components-mk4b}}
\end{figure}

You will need:
\begin{itemize}
    \item Your Cow Pi hardware circuit
    \item A piezoelectric ``passive buzzer''
    \begin{itemize}
        \item Piezoelectric devices can convert electric energy into mechanical energy, and vice-versa
        \item You will use a piezoelectric device to create an audible alarm
    \end{itemize}
    \item An ultrasonic echo sensor
    \begin{itemize}
        \item \textcolor{red}{If your sensor is still in the red bubblewrap packaging, there are also two resistors in the packaging. We will not use these, but we might have a use for them in a future semester. Please place them in the Cow~Pi's storage box.}
        \item The two prominent drums on this device are ultrasonic transducers, one of which converts electricity to 40kHz sound (well above the range of human hearing), and the other of which converts 40kHz sound into electricity
        \item You will measure the time between the ultrasound being transmitted and its reflecting being received to determine the distance to whatever is reflecting the ultrasound
    \end{itemize}
    \item Six 20cm male-to-male wires (you can separate these wires from a multi-wire cable)
\end{itemize}

There is a labeled header on the left side of the Cow~Pi;
we will use this to connect the hardware components to the RP2040 microcontroller.

\subsection{The Mini-Breadboard on the Cow Pi}

A key feature of solderless breadboards, such as the mini-breadboard on your Cow~Pi, are the groups of 5 holes (Figure~\ref{fig:breadboard-mk4b}).
Each group of five is a \textit{terminal strip}.

\begin{figure}
    \centering
    \includegraphics[height=4cm]{hardware/mk4b/breadboard}
    \caption{Terminal strips on a mini-breadboard \label{fig:breadboard-mk4b}}
\end{figure}

The five holes in a terminal strip are electrically connected to each other but are electrically isolated from the other terminal strips.\footnote{
    They are isolated for DC signals, and parasitic reactance is negligible for AC signals below about 10~kHz.
}
For example, in Figure~\ref{fig:breadboard-mk4b}, all holes in the terminal strip that is highlighted in blue are connected to each other,
but they are not connected to the holes in the adjacent terminal strip highlighted in green.
Similarly, they are not connected to the holes in the red terminal strip on the other side of the gutter.

A consequence of this is that any components' connectors that are inserted into a terminal strip are connected to the connectors of other components that are inserted into the same terminal strip.

If you want to learn more, a very good overview of solderless breadboards can be found here: \url{https://learn.adafruit.com/breadboards-for-beginners?view=all}

Figure~\ref{fig:breadboard-with-components-mk4b} shows where you will insert the hardware components for the group project.
(The precise placement is not critical, so long as you make a note of which terminal strips you use.)

\begin{figure}
    \centering
    \includegraphics[height=4cm]{hardware/mk4b/breadboard-with-components}
    \caption{Terminal strips on a mini-breadboard \label{fig:breadboard-with-components-mk4b}}
\end{figure}


\subsection{Connecting the ultrasonic echo sensor}

Take a look at the ultrasonic echo sensor (Figure~\ref{fig:ultrasonic-mk1f}).
Notice that it has four pins, labeled \texttt{Gnd}, \texttt{Echo}, \texttt{Trig}, and \texttt{Vcc}.

\begin{figure}
    \centering
    \subfloat[The back side of the ultrasonic sensor.]{
        \includegraphics[height=4cm]{hardware/components/ultrasonic_rear}
%        \label{fig:ultrasonicRear}
    }
    \hfil
    \subfloat[The front side of the ultrasonic sensor.]{
        \includegraphics[height=4cm]{hardware/components/ultrasonic_front}
%        \label{fig:ultrasonicFront}
    } \\
    \subfloat[Inserting the ultrasonic sensor.]{
        \includegraphics[height=4cm]{hardware/mk4b/insert-ultrasonic}
    } \\
    \subfloat[One end of the board's connection to the sensor.]{
        \includegraphics[height=4cm]{hardware/mk4b/connect-ultrasonic-1}
    }
    \hfil
    \subfloat[The other end of the board's connection to the sensor.]{
        \includegraphics[height=4cm]{hardware/mk4b/connect-ultrasonic-2}
    }
    \caption{The ultrasonic echo sensor. \label{fig:ultrasonic-mk4b}}
\end{figure}

\begin{description}
    \checkoffitem{Insert the ultrasonic echo sensor into unused rows on the breadboard, pointing away from you.
        Leave room behind the sensor to connect a wire into each of the same terminal strips that the sensor uses.}
    \checkoffitem{Insert four 20cm wires, one into each of the same terminal strips that the sensor uses.
        Make a note of which color wire corresponds to which of the sensor's pins.}
    \checkoffitem{Insert the other end of the \texttt{Vcc} wire into a \texttt{5V} slot}
    \checkoffitem{Insert the other end of the \texttt{Gnd} wire into a \texttt{GND} slot}
    \checkoffitem{Insert the other end of the \texttt{Echo} wire into the \texttt{GP16} slot}
    \checkoffitem{Insert the other end of the \texttt{Trig} wire into the \texttt{GP17} slot}
\end{description}

The ultrasonic echo sensor is now connected to the Raspberry Pi Pico's pins \texttt{GP16}--\texttt{GP17}.
The starter code will configure \texttt{GP17} (\lstinline{TRIGGER}) to be an output pin and \texttt{GP16} (\lstinline{ECHO}) to be an input pin.


\subsection{Connecting the Piezobuzzer}

\begin{figure}
    \centering
    \subfloat[Inserting the piezobuzzer]{
        \includegraphics[height=4cm]{hardware/mk4b/insert-piezo}
    }
    \hfil
    \subfloat[The front side of the ultrasonic sensor.]{
        \includegraphics[height=4cm]{hardware/mk4b/connect-piezo}
    }
    \caption{The piezoelectric ``passive buzzer''. \label{fig:piezobuzzer-mk4b}}
\end{figure}

\begin{description}
    \checkoffitem{Insert the piezobuzzer into unused rows on the breadboard.
        Leave room to connect a wire into each of the same terminal strips that the piezobuzzer.}
    \checkoffitem{Insert two 20cm wires, one into each of the same terminal strips that the piezobuzzer uses.}
    \checkoffitem{Insert the other end of one of the wires into a \texttt{GND} slot
        \begin{itemize}
            \item For \textit{this} particular device, the polarity does not matter, so it doesn't matter which wire is connected to \texttt{GND}
        \end{itemize}}
    \checkoffitem{Insert the other end of the other wire into the \texttt{GP22} slot}
\end{description}

The piezodisc can now be operated through the Raspberry Pi Pico's pin \texttt{GP22}.
The starter code will configure \texttt{GP22} (\lstinline{BUZZER}) to be an output pin.


\vspace{1cm}

Your Cow~Pi is now ready for you to design and code the software for a range finder and alarm.


\end{document}
