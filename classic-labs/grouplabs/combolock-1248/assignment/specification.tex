%! suppress = LabelConvention
\newcounter{myenumcounter}
\setcounter{myenumcounter}{\value{enumi}}

If the \textbf{right switch} is in the \textit{right position} when the system boots, then the system shall enter its combination lock mode.

\subsection*{When the system is in its combination lock mode:}

\begin{enumerate}[start=\value{enumi}]
    \setcounter{enumi}{\value{myenumcounter}}
    \item \label{spec:combinationDefined} A combination shall consist of three numbers in a particular order.
        \begin{enumerate}
            \item Each of the three numbers shall consist of exactly two decimal digits.
            \item It shall be possible for any two of the numbers to be duplicates of each other.
                It shall be possible for all three numbers to be duplicates of each other.
                For example, 12-12-12 is a valid combination.
            \item Single-digit numbers must have leading zeroes.
                For example, 01-02-03 is a valid combination, but 1-2-3 is not.
            \item Each of the three numbers can assume any value between 00 and 15, inclusive.
                \begin{itemize}
                    \item We are artificially limiting the range of possible values to simplify verifying the solution's correctness.
                \end{itemize}
        \end{enumerate}
    \item When displayed on the \textbf{display module}, the combination's first number shall occupy the two leftmost digits;
        the second number shall occupy the middle two digits;
        and the third number shall occupy the two rightmost digits.
        Dashes shall be displayed between the numbers, for example: \texttt{12-03-10}.
        If one or more of the positions has not had a number entered, then the digits for the unentered positions(s) shall be blank.
        For example, if the first number has been entered but not the second and third, then the display will show \texttt{12-\phantom{88}-\phantom{88}}.
%    \item The combination's numbers shall be entered using the \textbf{matrix keypad}.
%        Buttons with decimal numerals (\textit{0-9}) or alphabetic letters (\textit{A-D}) shall be interpreted as having the corresponding hexadecimal numeral;
%        the button with the octothorp (\#) shall be interpreted as having the hexadecimal numeral \textit{E};
%        and the button with the asterisk (*) shall be interpreted as having the hexadecimal numeral \textit{F}.
    \item The combination's numbers shall be entered using the \textbf{rotary encoder}.
        \begin{enumerate}
            \item When entering a number, the initial number shall be the last number that had been entered.
                If no number had previously been entered, then the initial number shall be 00.
            \item When the \textbf{rotary encoder} is turned \textit{clockwise}, the number shall increase.
                If the number increases past 15, it shall overflow to 00.
            \item When the \textbf{rotary encoder} is turned \textit{counterclockwise}, the number shall decrease.
                If the number decreases past 00, it shall underflow to 15.
        \end{enumerate}
    \item When the system is powered-up, and after the system is reset, it shall be LOCKED, regardless of its state before losing power,
        and the display shall have all numbers blank: \texttt{\phantom{88}-\phantom{88}-\phantom{88}}.
    \item When the system is LOCKED, the \textbf{left LED} shall be \textit{on}, and the \textbf{right LED} shall be \textit{off}.
    \item When the system is LOCKED, the \textbf{servomotor} shall be turned fully \textit{clockwise}.
    \item When the system is LOCKED, the system shall display the combination-entry display.
    \item \label{spec:enterCombination} When the system displays the combination-entry display, it shall display the currently-entered combination (which might not be the correct combination),
        and the user shall be able to enter and change the currently-entered combination.
        \begin{enumerate}
            \item The first number shall be entered by turning the \textbf{rotary encoder} \textit{clockwise}.
            \item After entering the first number, when the user turns the \textbf{rotary encoder} \textit{counterclockwise}, the first number shall be fixed, and the second number shall be entered.
            \item The second number shall be entered by turning the \textbf{rotary encoder} \textit{counterclockwise}.
            \item After entering the second number, when the user turns the \textbf{rotary encoder} \textit{clockwise}, the second number shall be fixed, and the third number shall be entered.
            \item The third number shall be entered by turning the \textbf{rotary encoder} \textit{clockwise}.
            \item After entering the third number, if the user turns the \textbf{rotary encoder} \textit{counterclockwise}, then the entered combination shall be cleared.
            \item After entering the third number, if the user presses the \textbf{left pushbutton}, then the entered combination shall be evaluated.
        \end{enumerate}
    \item \label{spec:verifyCombination} The entered combination is \textit{correct} if and only if the following are true:
        \begin{enumerate}
            \item The first number is the correct first number, the second number is the correct second number, and the third number is the correct third number.
            \item The first number was entered by passing the correct first number at least twice and then stopped at the correct first number on the \textit{third} (or greater) time that number had been visible.
            \item The second number was entered by passing the correct second number once and then stopped at the correct second number on the \textit{second} time that number had been visible.
            \item The third number was entered by stopping at the correct third number the \textit{first} time that number had been visible.
        \end{enumerate}
    \item If the entered combination is correct, then the system shall be UNLOCKED\@.
    \item When the system is UNLOCKED, the \textbf{right LED} shall be \textit{on}, and the \textbf{left LED} shall be \textit{off}.
    \item When the system is UNLOCKED, the \textbf{servomotor} shall be turned fully \textit{counterclockwise}.
    \item When the system is UNLOCKED, the system shall \textit{not} display the combination but instead shall display ``OPEN''.
    \item \label{spec:incorrectCombination} If the entered combination is incorrect, a warning message shall be displayed, both LEDs shall blink twice, the entered combination shall be cleared, and the system shall remain LOCKED\@.
        \begin{itemize}
            \item The warning message shall be ``bad try 1'' or ``bad try 2'', depending on whether it was their first or second attempt.
            \item The blinks may be implemented with timed busy-waits.
        \end{itemize}
    \item If the user makes three incorrect attempts, the system shall be ALARMED\@.
        \begin{enumerate}
            \item When the system is ALARMED, it shall display \texttt{alert!} and both \textbf{LEDs} shall blink on-and-off every quarter-second.
                \begin{itemize}
                    \item The blinks may be implemented with timed busy-waits.
                \end{itemize}
            \item When the system is ALARMED, it shall not accept further input until it has been reset or has been powered-down and then powered-up.
        \end{enumerate}
    \item \label{spec:abandonedCombination} \underline{Optionally}, a partially-entered combination can be abandoned, causing the partially-entered combination to be cleared, under these conditions:
        \begin{itemize}
            \item When entering the second number, 00 is displayed three times
            \item When entering the third number, 00 is displayed twice
        \end{itemize}
    \item \label{spec:changeCombination} When the system is UNLOCKED, the user shall be able to change the combination.
    \begin{enumerate}
        \item When the user places the \textbf{left switch} in the \textit{right position} and then presses the \textbf{right pushbutton}, the system shall be CHANGING\@.
        \item When the system is CHANGING, it shall display \texttt{enter} and the combination-entry display,
            and the user shall be able to enter the proposed new combination by pressing six digits on the \textbf{numeric keypad}.
        \item After the user has entered a new combination, a second combination-entry display shall be shown,
            and the user shall be able to re-enter the proposed new combination by pressing six digits on the \textbf{numeric keypad}.
        \item When the user places the \textbf{left switch} in the \textit{left position}:
            \begin{enumerate}
                \item If either of the two entered combinations are incomplete, the system shall display ``no change''.
                \item If the two entered combinations do not match, the system shall display ``no change''.
                \item If any of the numbers in the combination is greater than 15, the system shall display ``no change''.
                \item Otherwise, the system shall display ``changed'', and the correct combination shall change to the new combination.
                \item The system shall be UNLOCKED\@.
            \end{enumerate}
    \end{enumerate}
    \item \label{spec:locking} When the system is UNLOCKED, the user shall be able to lock the system by pressing both \textbf{pushbuttons} simultaneously.
        \begin{itemize}
            \item After doing so, the system shall be LOCKED.
        \end{itemize}
%    \item \label{spec:persistentCombination} The correct combination shall be persistent even while the system is unpowered.
%        For example, if the user changes the combination, powers-down the system, and powers-up the system, then the correct combination shall still be the combination that they had changed it to before powering-down the system.
    \item \label{spec:persistentCombination} The correct combination shall be persistent even when the system is reset.
        For example, if the user changes the combination and presses the \textbf{RESET button}, then the correct combination shall still be the combination that they had changed it to before resetting the system.
\end{enumerate}