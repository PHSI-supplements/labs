You are not required to have your assignment checked-off by a TA or the professor.
If you do not do so, then we will perform a functional check ourselves.
In the interest of making grading go faster, we are offering a small bonus % to get your assignment checked-off at the start of your scheduled lab time immediately after it is due.
% Because checking off all students during lab would take up most of the lab time, we are offering a slightly larger bonus
if you complete your assignment early and get it checked-off by a TA or the professor during office hours.

\begin{enumerate}
    \checkoffitem{Place both switches in the left position and apply power (or reset) the Cow Pi.}
    \vspace{1cm}

    % Rotary Encoder Test Mode -- 10

    \checkoffitem{Confirm that the number of clockwise \& counterclockwise turns is displayed and clearly labeled, with the number of turns holding at 0. \\
        \textit{+1 The number of clockwise and counterclockwise turns is displayed, starting at 0 for each}
    }
    \checkoffitem{Gently turn the rotary encoder one notch clockwise. \\
        \textbf{The number of clockwise turns increases to 1; the number of counterclockwise turns remains 0} \\
        \textit{+1 Turning clockwise increases the clockwise count}
    }
    \checkoffitem{Gently turn the rotary encoder one notch clockwise. \\
        \textbf{The number of clockwise turns increases to 2; the number of counterclockwise turns remains 0} \\
        \textit{+1 Turning clockwise again increases the clockwise count}
    }
    \checkoffitem{Gently turn the rotary encoder two notches clockwise. \\
        \textbf{The number of clockwise turns increases to 4; the number of counterclockwise turns remains 0} \\
        \textit{+1 Continuous turning clockwise increases the clockwise count for each detent}
    }
    \checkoffitem{Gently turn the rotary encoder one notch counterclockwise. \\
        \textbf{The number of counterclockwise turns increases to 1; the number of clockwise turns remains 4} \\
        \textit{+1 Turning counterclockwise increases the clockwise count}
    }
    \checkoffitem{Gently turn the rotary encoder one notch counterclockwise. \\
        \textbf{The number of counterclockwise turns increases to 2; the number of clockwise turns remains 4} \\
        \textit{+1 Turning counterclockwise again increases the clockwise count}
    }
    \checkoffitem{Gently turn the rotary encoder two notches counterclockwise. \\
        \textbf{The number of counterclockwise turns increases to 4; the number of clockwise turns remains 4} \\
        \textit{+1 Continuous turning counterclockwise increases the clockwise count for each detent} \\
        \textit{+1 No evidence of bouncing} \\
        \textit{+1 The displayed turns are not opposite of the actual direction turned}
    }
    \checkoffitem{Confirm that the number of clockwise \& counterclockwise turns being displayed holds at 4. \\
        \textit{+1 When the rotary encoder is not turned, the number of turns shown on the display does not change}
    }
    \vspace{1cm}

    % Servomotor Test Mode -- 10

    \checkoffitem{Confirm that the display shows ``SERVO: left''.}
    \checkoffitem{Press the left pushbutton \\
        \textbf{The servo arm moves 90\textdegree counterclockwise} \\
        \textbf{The display shows ``SERVO: center''} \\
        \textit{+1 The servo arm is initially following the left switch} \\
        \textit{+1 Pressing the left pushbutton centers the servo arm}
    }
    \checkoffitem{Release the left pushbutton. \\
        \textbf{The servo arm moves 90\textdegree clockwise, back to its original position} \\
        \textbf{The display shows ``SERVO: left''} \\
        \textit{+1 Releasing the button allows the servo arm to following the left switch} \\
        \textit{+1 The servo can rotate clockwise}
    }
    \checkoffitem{Toggle the left switch to the right position. \\
        \textbf{The servo arm moves 180\textdegree counterclockwise} \\
        \textbf{The display shows ``SERVO: right''} \\
        \textit{+1 The servo can rotate counterclockwise} \\
        \textit{+1 The servo's position is correctly displayed}
    }
    \checkoffitem{Toggle the left switch to the left position. \\
        \textbf{The servo arm moves 180\textdegree clockwise} \\
        \textbf{The display shows ``SERVO: left''} \\
        \textit{+1 The servo can rotate clockwise again}
    }
    \checkoffitem{Toggle the left switch to the right position. \\
        \textbf{The servo arm moves 180\textdegree counterclockwise} \\
        \textbf{The display shows ``SERVO: right''} \\
        \textit{+1 The servo can rotate counterclockwise again} \\
        \textit{+1 The servo fully deflects} \\
        \textit{+1 The servo does not ``stutter'' (except when the system boots)}
    }
    \vspace{1cm}

    \checkoffitem{Examine the combination shown on the display module.
        If any numbers are greater than 15, press the right pushbuton to reset the combination.}
    \checkoffitem{Make note of the combination.}
    \checkoffitem{Place the right switch in the right position, and press the RESET button.}
    \vspace{1cm}

    % Combination Lock -- 30 //

    \checkoffitem{The system is now in ``Combination Lock'' mode. \\
        \textbf{The left LED is lit} \\
        \textbf{The display shows \texttt{\phantom{xx}-\phantom{xx}-\phantom{xx}}} \\
        \textbf{The servo is deflected fully clockwise} \\
        \textit{+1 The system is locked when powered-up} \\
        \textit{+\textonehalf\ When the system is locked, the left LED is lit} \\
        \textit{+\textonehalf\ When the system is locked, the display shows the combination-entry display, initially showing empty numbers} \\
        \textit{+2 When the system is locked, the servo is deflected fully clockwise}
    }
    \checkoffitem{Turn the rotary encoder clockwise until the \underline{first} time that the first number reaches the correct first number. \\
        \textbf{The display shows \texttt{\phantom{05}-\phantom{xx}-\phantom{xx}}} (if the correct combination is 05-10-15) \\
        \textit{+\textonehalf\ Combination numbers increase with clockwise turns}
    }
    \checkoffitem{Turn the rotary encoder counterclockwise until the \underline{first} time that the second number reaches the correct second number. \\
        \textbf{The display shows \texttt{\phantom{05}-\phantom{10}-\phantom{xx}}} (if the correct combination is 05-10-15) \\
        \textit{+\textonehalf\ Combination numbers decreases with counterclockwise turns}
    }
    \checkoffitem{Turn the rotary encoder clockwise until the \underline{first} time that the third number reaches the correct third number. \\
        \textbf{The display shows \texttt{\phantom{05}-\phantom{10}-\phantom{15}}} (if the correct combination is 05-10-15) \\
        \textit{+1 Combination numbers are display as three 2-digit numbers separated by dashes}
    }
    \checkoffitem{Press the left pushbutton. \\
        \textbf{The display shows \texttt{bad try 1} and both LEDs blink twice}, then \\
        \textbf{An empty combination-entry is shown and the left LED is lit} \\
        \textit{+1 The combination is evaluated when the user presses the left pushbutton} \\
        \textit{+2 The system does not unlock if the correct combination is entered, but was entered incorrectly} \\
        \textit{+1 When the user mis-enters the combination, it displays ``bad try'' and the attempt number, and blinks both LEDs twice} \\
        \textit{+\textonehalf\ After the first bad try, the user is given another opportunity to enter the combination}
    }
    \checkoffitem{Turn the rotary encoder clockwise until the \underline{third} time that the first number reaches 10 (or some other incorrect first number). \\
        \textbf{The display shows \texttt{\phantom{10}-\phantom{xx}-\phantom{xx}}} \\
        \textit{+\textonehalf\ Combination numbers overflow from 15 to 00}
    }
    \checkoffitem{Turn the rotary encoder counterclockwise until the \underline{second} time that the first number reaches 15 (or some other incorrect second number). \\
        \textbf{The display shows \texttt{\phantom{10}-\phantom{15}-\phantom{xx}}} \\
        \textit{+\textonehalf\ Combination numbers underflow from 00 to 15}
    }
    \checkoffitem{Turn the rotary encoder clockwise until the \underline{first} time that the first number reaches 05 (or some other incorrect third number), and press the left pushbutton. \\
        \textbf{The display shows \texttt{\phantom{10}-\phantom{15}-\phantom{00}}}, then \\
        \textbf{The display shows \texttt{bad try 2} and both LEDs blink twice}, then \\
        \textbf{An empty combination-entry is shown and the left LED is lit} \\
        \textit{+2 The system does not unlock for incorrect combinations} \\
        \textit{+\textonehalf\ After the second bad try, the user is given another opportunity to enter the combination}
    }
    \checkoffitem{Enter the combination thrice-clockwise~05, twice-counterclockwise~10, once-clockwise~15 (or the correct number if it isn't 05-10-15), and press the left pushbutton. \\
        \textbf{The display shows \texttt{\phantom{05}-\phantom{10}-\phantom{15}}}, then \\
        \textbf{The display shows \texttt{OPEN} and the right LED is lit}, and \\
        \textbf{The servo deflects fully counterclockwise} \\
        \textit{+2 The system unlocks for the correct combination entered correctly} \\
        \textit{+\textonehalf\ When the system is unlocked, the right LED is lit} \\
        \textit{+\textonehalf\ When the system is unlocked, the display shows ``OPEN''} \\
        \textit{+2 When the system is unlocked, the servo is deflected fully counterclockwise}
    }
    \checkoffitem{Place the left switch in the right position, and press the right pushbutton. \\
        \textbf{The display shows \texttt{enter} and \texttt{\phantom{xx}-\phantom{xx}-\phantom{xx}}} \\
        \textit{+1 The user can begin changing the combination by moving the left switch to the right and pressing the right pushbutton} \\
        \textit{+1 When the user begins changing the combination, the lock displays ``enter'' and then shows the combination-entry display}
    }
    \checkoffitem{Using the numeric keypad, press \texttt{1 2 0 6 0 1} \\
        \textbf{The display shows \texttt{enter} and \texttt{\phantom{12}-\phantom{06}-\phantom{01}} and \texttt{\phantom{xx}-\phantom{xx}-\phantom{xx}}} \\
        \textit{+1 The user enters the new combination by pressing six digits on the numeric keypad} \\
        \textit{+1 The user confirms the new combination in a second combination-entry display, using the numeric keypad}
    }
    \checkoffitem{Using the numeric keypad, press \texttt{1 2 0 6 0 1}, and place the left switch in the left position. \\
        \textbf{The display shows \texttt{enter} and \texttt{\phantom{12}-\phantom{06}-\phantom{01}} and \texttt{\phantom{12}-\phantom{06}-\phantom{01}}}, then \\
        \textbf{The display shows \texttt{changed}} \\
        \textit{+1 The new combination is checked for validity by returning the left switch to the left position}
    }
    \checkoffitem{Press both pushbuttons simultaneously. \\
        \textbf{The left LED is lit}, and \\
        \textbf{The display shows \texttt{\phantom{xx}-\phantom{xx}-\phantom{xx}}}, and \\
        \textbf{The servo deflects fully clockwise} \\
        \textit{+2 The system is re-locked by pressing both pushbuttons simultaneously.}
    }
    \checkoffitem{Enter the \underline{original} correct combination (which is no longer correct) three times. \\
        \textbf{The system displays \texttt{alert!}}, and \\
        \textbf{Both LEDs continuously blink}, and \\
        \textbf{The system becomes unresponsive} \\
        \textit{+1 After the third bad try, the system displays ``alert!'', both LEDS continously blink, and the lock becomes unresponsive}
    }
    \checkoffitem{Press the RESET button. \\
        \textbf{The left LED is lit}, and \\
        \textbf{The display shows \texttt{\phantom{xx}-\phantom{xx}-\phantom{xx}}}, and \\
        \textbf{The servo deflects fully clockwise}
    }
    \checkoffitem{Enter the thrice-clockwise~12, twice-counterclockwise~06, once-clockwise~01 \\
        \textbf{The display shows \texttt{OPEN} and the right LED is lit}, and \\
        \textbf{The servo deflects fully counterclockwise} \\
        \textit{+1 Valid proposed combinations are saved in memory that persists between resets}
    }
    \checkoffitem{Place the left switch in the right position, and press the right pushbutton. Using the numeric keypad, press \texttt{0 2 1 3 0 4}. For the confirmation, press \texttt{0 2 1 4 0 3}. Place the left switch in the left position. \\
        \textbf{The display shows \texttt{no change}} \\
        \textit{+1 If the proposed combination and its confirmation do not match, the change is rejected.}
    }
    \checkoffitem{Place the left switch in the right position, and press the right pushbutton. Using the numeric keypad, press \texttt{0 2 1 6 0 4}. For the confirmation, press \texttt{0 2 1 6 0 4}. Place the left switch in the left position. \\
        \textbf{The display shows \texttt{no change}} \\
        \textit{+1 If the proposed combination has an out-of-range value, the change is rejected.}
    }
    \vspace{1cm}

    \checkoffitem{Press both pushbuttons simultaneously to re-lock the combination lock.}
    \checkoffitem{Turn thrice-clockwise~12. Turn \underline{thrice}-counterclockwise~00 (going past the second occurrence of 06). \\
        \textbf{The combination is abandoned without generating a ``bad try''.} \textbf{This \underline{should} happen by going back to an empty combination-entry display but is not required for this point.} (Since you're turning counterclockwise, the first number should \underline{not} be generating.) \\
        \textit{BONUS +1 A combination can be abandoned on the second number.}
    }
    \checkoffitem{Turn thrice-clockwise~12. Turn twice-counterclockwise~06. Turn \underline{twice}-clockwise~00 (going past the first occurrence of 01). \\
        \textbf{The combination is abandoned without generating a ``bad try''.} \\
        \textit{BONUS +1 A combination can be abandoned on the second number.} \\
        If \textbf{the combination-entry display resets, and the continuing clockwise motion generates the first number.} \\
        \textit{BONUS +1 When entering the third number, if the rotation goes too far then seamlessly transitions to entering the first number.}
    }

\end{enumerate}

%    \checkoffitem{Show the TA any code they have not yet examined. \\
%        \textit{+1 The code is clean, well-organized, has good variable and function names, and is otherwise understandable}}
