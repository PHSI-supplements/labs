Download the zip file or tarball from \filesource.
Once downloaded, unpackage the file and open the project in your IDE\@.

\subsection{Description of RangeFinder Files}

\subsubsection{combolock.c, interrupt\_support.h, interrupt\_support.cpp, display.h, display.cpp}

Do not edit \textit{combolock.c}, \textit{interrupt\_support.h}, \textit{interrupt\_support.cpp}, \textit{display.}, or \textit{display.cpp}.

These files contain code to simplify interrupt management, and functions to place text on the display module.

\subsubsection{rotary-encoder.h \& rotary-encoder.c}

Do not edit \textit{rotary-encoder.h}

The \textit{rotary-encoder.c} file is where you will process inputs from the rotary encoder.

\subsubsection{servomotor.h \& servomotor.c}

Do not edit \textit{servomotor.h}

The \textit{servomotor.c} file is where you will control the servomotor.

\subsection{lock-controller.h \& lock-controller.c}

Do not edit \textit{lock-controller.h}

The \textit{lock-controller.c} file is where you will implement the logic for the combination lock.

%\subsubsection{shared\_variables.h}
%
%The \textit{shared\_variables.h} header file is where you will place any types that you define and where you will externalize any global variables that need to be used by more than one \textit{.c} file.
%
%It also contains a structure that you may use to access an analog-digital converter's registers,
%and it contains meaningfully-named constants to refer to specific pins you will use in this assignment.


\subsection{Assemble the Hardware}

\textcolor{red}{\textbf{BEFORE YOU PROCEED FURTHER:}}
\begin{description}
    \checkoffitem{Add the new hardware to your Cow~Pi as described in Appendix~\ref{sec:hardwareMods-mk4b}.}
\end{description}
