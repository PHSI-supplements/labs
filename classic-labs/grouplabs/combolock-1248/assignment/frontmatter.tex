In this assignment, you will write code for your Cow Pi that will use new electronic devices to interact with the physical world.
Specifically, you will implement an electronic combination lock.

%The instructions are written assuming you will edit the code in the Arduino IDE and run it on \runtimeenvironment, constructed according to the pre-lab instructions.
%If you wish, you may edit the code in a different environment; however, our ability to provide support for problems with other IDEs is limited.

\tableofcontents

\section*{Learning Objectives}

After successful completion of this assignment, students will be able to:
\begin{itemize}
    \item Work collaboratively on a hardware/software project
    \item Design and implement a simple embedded system
    \item Expand their programming knowledge by consulting documentation
\end{itemize}

\section*{During Lab Time}

During your lab period, coordinate with your group partner(s) to decide on your working arrangements.
Unless you're only going to work on the assignment when you're together, you may want to set up a private Git repository that is shared with your partner(s).
With your partner(s), add the new hardware as described in Appendix~\ref{sec:hardwareMods-mk4b}.
Then, think through your system's design and begin implementing it.
The TAs will be available for questions.
