Please familiarize yourself with the entire assignment before beginning.
There are multiple parts to this assignment.

If one of the partners does not make an equitable contribution, we will adjust both partners' scores accordingly.

The assignment is written so that if you and your partner decide to pair program, you can complete Sections~\ref{sec:hardwareMods}--\ref{sec:integration} in sequence.
The assignment is also written so that you and your partner can work on Sections~\ref{sec:hardwareMods} and \ref{sec:initialSoftware} together, then split up with one student working on Section~\ref{sec:distance} and the other working on Section~\ref{sec:sound}, and then get back together again to work on Section~\ref{sec:integration}.

If you are in the unfortunate position of having a partner who does not contribute to the project, we do not expect you to complete the full assignment by yourself and will take the circumstances into account when grading.
In this situation, you should prioritize Section~\ref{sec:distance} over Section~\ref{sec:sound}, as you will be able to complete more of Section~\ref{sec:integration} with a working distance sensor but no alarm, than the other way around.

\subsection{Constraints} \label{subsec:constraints}

You may use the constants and functions provided in the starter code.
You may use any features that are part of the C standard if they are supported by the compiler.
You may use code written by you and/or your partner.

\subsubsection{Constraints on the Arduino core}

You may use \function{digitalWrite()} to write to pins Dxx and Dxx (or you may use memory-mapped I/O).
You may use the \function{delayMicroseconds()} function to introduce delays not to exceed $10\mu s$.

You may not use any other libraries, functions, macros, types, or constants from the Arduino core.\footnote{
    \url{https://www.arduino.cc/reference/en/}
}
This prohibition includes, but is not limited to, the \function{tone()}, \function{noTone()}, \function{pulseIn()}, and \function{pulseInLong()} functions.

\subsubsection{Constraints on AVR-libc}

You may use any AVR-specific functions, macros, types, or constants of avr-libc.\footnote{
    \url{https://www.nongnu.org/avr-libc/user-manual/index.html}
}

\subsubsection{Constraints on the CowPi library}

You may use any functions provided by the CowPi\footnote{
    \url{https://cow-pi.readthedocs.io/en/latest/library.html}
}
and the CowPi\_stdio\footnote{
    \url{https://cow-pi.readthedocs.io/en/latest/stdio.html}
} libraries,
and you may use any data structures\footnote{
    \url{https://cow-pi.readthedocs.io/en/latest/microcontroller.html}
} provided by the CowPi library.


\subsubsection{Constraints on other libraries}

You may not use any libraries beyond those explicitly identified here.