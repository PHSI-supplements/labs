Download the zip file or tarball from \filesource.
Once downloaded, unpackage the file and open the project in your IDE\@.

\subsection{Description of RangeFinder Files}

\subsubsection{RangeFinder.cpp, interrupt\_support.h, interrupt\_support.cpp, outputs.h, outputs.cpp}

Do not edit \textit{RangeFinder.cpp}, \textit{interrupt\_support.h}, \textit{interrupt\_support.cpp}, \textit{outputs.h}, or \textit{outputs.cpp}.

These files contain code to simplify interrupt management and to Functions to place text on the display module.

\subsubsection{user\_controls.h \& user\_controls.c}

Do not edit \textit{user\_controls.h}

The \textit{user\_controls.c} file is where you will get inputs from the user.

\subsubsection{sensor.h \& sensor.c}

Do not edit \textit{sensor.h}

The \textit{sensor.c} file is where you will control the ultrasonic echo sensor and compute any detected objects' distance and speed.

\subsubsection{alarm.h \& alarm.c}

Do not edit \textit{alarm.h}

The \textit{alarm.c} file is where you will control the piezoelectric disc and manage the chirping and strobing of a proximity alarm.

\subsubsection{shared\_variables.h}

The \textit{shared\_variables.h} header file is where you will place any types that you define and where you will externalize any global variables that need to be used by more than one \textit{.c} file.

It also contains a structure that you may use to access an analog-digital converter's registers,
and it contains meaningfully-named constants to refer to specific pins you will use in this assignment.


\subsection{Combining Code for Old and New Hardware}

If you and your partner are both using the old hardware, you do not need to worry about combining code for old and new hardware.
Similarly, if you and your partner are both using the new hardware, you do not need to worry about combining code for old and new hardware.

If, however, one partner is using the old hardware and the other is using the new hardware,
then you will need to include compiler preprocessor directives to detect which microcontroller the program will run on,
and execute the correct code for that microcontroller.
These differences primarily will be configuring the timers and calculating distances.

C's preprocessor accepts directives that modify a temporary copy of a file before compiling it.
Two that you've already seen are \lstinline{#include} and \lstinline{#define}.
The \lstinline{#if} directive and its relatives can be used to include/exclude code based on compile-time conditions.
If one partner is using the old hardware and the other is using the new hardware, use the pattern
\begin{lstlisting}
    // code for both old and new hardware goes here
#if defined (ARDUINO_AVR_NANO)
    // code for the old hardware goes here
#elif defined (ARDUINO_RASPBERRY_PI_PICO)
    // code for the new hardware goes here
#endif
    // code for both old and new hardware goes here
\end{lstlisting}

A concise example can be found in \function{refresh_display()} in \textit{outputs.cpp}:

\begin{lstlisting}[firstnumber=108]
void refresh_display(void) {
    if (display_needs_refreshed) {
#if defined (ARDUINO_AVR_NANO)
        fprintf(display, "\f%s\n%s\f", rows[0], rows[1]);
#elif defined (ARDUINO_RASPBERRY_PI_PICO)
        obdDumpBuffer(&display, backbuffer);
#endif
        display_needs_refreshed = false;
    }
}
\end{lstlisting}

\subsection{Assemble the Hardware and Edit \textit{platformio.ini}}

\textcolor{red}{\textbf{BEFORE YOU PROCEED FURTHER:}}
\begin{description}
    \checkoffitem{Edit \textit{platformio.ini} so that the section for your hardware (old/new hardware) is uncommented, and the other section (old/new hardware) is commented-out.}
    \checkoffitem{Add the new hardware to your Cow~Pi as described in Appendices~\ref{sec:hardwareMods-mk1f}--\ref{sec:hardwareMods-mk4b}.}
\end{description}


