When you have completed this assignment, upload \textit{lock-controller.c}, \textit{rotary-encoder.c}, and \textit{servomotor.c} to \filesubmission.

%\policyforcodethatdoesnotcompile
\subsection*{No Credit for Uncompilable Code}
If the TA cannot create an executable from your code, then your code will be assumed to have no functionality.\footnote{
    At the TA's discretion, if they can make your code compile with \textit{one} edit (such as introducing a missing semicolon) then they may do so and then assess a 10\% penalty on the resulting score.
    The TA is under no obligation to do so, and you should not rely on the TA's willingness to edit your code for grading.
    If there are multiple options for a single edit that would make your code compile, there is no guarantee that the TA will select the option that would maximize your score.
}
Before turning in your code, be sure to compile and test your code on your Cow~Pi with the original driver code and the original header file(s).

\interruptlablatepolicy

\subsection*{Rubric}

This assignment is worth 50 points.

\begin{description}
    \item[Rotary Encoder (test mode)] % 10
    \rubricitem{1}{The number of clockwise and counterclockwise turns is displayed, starting at 0 for each}
    \rubricitem{1}{When the rotary encoder is not turned, the number of turns shown on the display does not change}
    \rubricitem{4}{When the rotary encoder is turned clockwise, the displayed number of clockwise turns increases by one for each detent}
    \rubricitem{4}{When the rotary encoder is turned counterclockwise, the displayed number of counterclockwise turns increases by one for each detent}

    \item[Servomotor (test mode)] % 10
    \rubricitem{1}{The servo's position is correctly displayed}
    \rubricitem{3}{When (and only when) the left pushbutton is pressed, the servo moves to the center position}
    \rubricitem{3}{When the left switch is in the \underline{left} position (and the pushbutton is not pressed), the servo deflects fully \underline{clockwise}}
    \rubricitem{3}{When the left switch is in the \underline{right} position (and the pushbutton is not pressed), the servo deflects fully \underline{clockwise}}

    \item[Combination Lock] %30
    \rubricitem{1}{Combinations are displayed as three 2-digit numbers separated by dashes}
    \rubricitem{2}{Combinations are entered using the rotary encoder}
    \rubricitem{1}{The system is locked when powered-up}
    \rubricitem{1}{When the system is locked, the left LED is lit, and the display shows the combination-entry display, initially showing empty numbers (only dashes)}
    \rubricitem{2}{When the system is locked, the servo is deflected fully clockwise}
    \rubricitem{1}{The combination is evaluated when the user presses the left pushbutton}
    \rubricitem{2}{The system unlocks only if the correct combination is entered}
    \rubricitem{4}{The system unlocks only if the combination is entered correctly (3x clockwise, 2x counterclockwise, 1x clockwise)}
    \rubricitem{1}{When the system is unlocked, the right LED is lit, and the display shows ``OPEN''}
    \rubricitem{2}{When the system is unlocked, the servo is deflected fully counterclockwise}
    \rubricitem{1}{When the user mis-enters the combination, it displays ``bad try'' and the attempt number, and blinks both LEDs twice}
    \rubricitem{1}{After the first two bad tries, the user is given another opportunity to enter the combination}
    \rubricitem{1}{After the third bad try, the system displays ``alert!'', both LEDS continously blink, and the lock becomes unresponsive}
    \rubricitem{1}{The user can begin changing the combination by moving the left switch to the right and pressing the right pushbutton}
    \rubricitem{1}{When the user begins changing the combination, the lock displays ``enter'' and then shows the combination-entry display}
    \rubricitem{1}{The user enters the new combination by pressing six digits on the numeric keypad}
    \rubricitem{1}{The user confirms the new combination in a second combination-entry display, using the numeric keypad}
    \rubricitem{1}{The new combination is checked for validity by returning the left switch to the left position}
    \rubricitem{2}{Invalid proposed combinations are rejected with a ``no change'' message}
    \rubricitem{1}{Valid proposed combinations are saved in memory that persists between resets}
    \rubricitem{2}{The system is re-locked by pressing both pushbuttons simultaneously.}
    \bonusitem{3}{A partially-entered combination can be abandoned by turning the rotary encoder sufficiently past the correct numbers}
    \bonusitem{2}{Get assignment checked-off by TA or professor during office hours before it is due}% (you cannot get both bonuses)}
%    \bonusitem{1}{Get assignment checked-off by TA at \textit{start} of your scheduled lab immediately after it is due (your code must be uploaded to \filesubmission\ \textit{before} it is due. You cannot get both bonuses)}
    \spaghetticodepenalties{1}
\end{description}

Students' scores may be adjusted up or down as necessary if the team had an inequitable distribution of effort.