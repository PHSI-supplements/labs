Open \textit{lock-controller.c}.

\subsection{Limited Starter Code}

The starter code in \textit{lock-controller.c} is sparse:
an array to store the combination, a function to retrieve the combination, and a function to force the combination to take on a pre-defined setting (think of it as a ``factory reset'').
There are also stubs for two additional functions.

For your lock controller, you won't need to use \function{get_combination()} because you can directly access the \lstinline{combination} array,
and you should not use \function{force_combination_reset()} because it serves no purpose in the context of the system's specification.

The functions you \textit{will} use are \function{initialize_lock_controller()} and \function{control_lock()}, as well as any additional functions that you write and call from these functions.
When the system is in its combination lock mode, the effective \function{main} code looks like:
\begin{lstlisting}
    int main() {
        initialize_rotary_encoder();
        initialize_servo();
        initialize_lock_controller();
        for(;;) {
            control_lock();
        }
    }
\end{lstlisting}
(If you look at \textit{combolock.c}, you'll see that there's more to it than that, but this simplification is enough to make the point that\dots)
Any code that you want to run \textit{once} should go in \function{initialize_lock_controller()}, and any code that you want to run repeatedly in the main control loop should go in \function{control_lock()} (or be called by \function{control_lock()}.)


\subsection{Storing the Combination Between Resets, and Resetting the Combination}

As long as you store the correct combination in the \lstinline{combination} array (and store the user entries in other arrays), you will have satisfied Requirement~\ref{spec:persistentCombination}.

An earlier version of this assignment used a different microcontroller that had a small EEPROM built into it, and so students were able to save the combination and access it even after power had been removed and restored.
The RP2040 microcontroller doesn't have a built-in EEPROM for data, so we will get as close as possible to persisting the combination.

The declaration of the \lstinline{combination} array includes a directive that it should be placed in memory that doesn't get initialized during a system boot.
The effect of this is that, as long as the microcontroller continues to have power, the \lstinline{combination} array will retain its contents even if the microcontroller gets reset.

\subsubsection*{What if the microcontroller loses power?}

If the microcontroller loses power, the combination \textit{will} change.
If you place the system in its test mode, then the currently-stored combination will be displayed.
Because the combination after restoring power might not have numbers that are strictly less than 16,
if you press the \textbf{right pushbutton} while in test mode, the combination will reset to 05-10-15.


\subsection{Recommendations}

I recommend that you use polling to get human-scale inputs (\textit{i.e.}, button \& key presses and switch toggles).
I recommend that you use the functions from the CowPi library to do so (\function{cowpi_get_keypress()}, \function{cowpi_left_button_is_pressed()}, etc.)
Debounce these inputs if you need to -- see the PollingLab code for examples of how to use the debounce functions, and/or read the datasheet (\url{https://cow-pi.readthedocs.io/en/latest/CowPi/inputs.html#debouncing}).

You can implement blinking using timed busy-waits, similar to the busy-wait you placed in \function{get_keypress()} in PollingLab, but considerably longer than 1\textmu s.
Other than when you're blinking the LEDs, the system must be responsive to valid user inputs.
(You may ignore invalid inputs.)

Use \function{display_string()} to send strings to the display module.

You do \textbf{not} need to set up globally-shared variables to communicate between \textit{lock-controller.c}, \textit{rotary-encoder.c}, and \textit{servomotor.c}.
If you believe that you do, see me and if I agree that you do then I'll show you how to set that up.
However:
\begin{itemize}
    \item The lock controller should not need to expose any information to the rotary encoder and the servomotor.
    \item The lock controller needs to find out which direction the rotary encoder's shaft had been turned, if it had been turned.
        This can be obtained through the \function{get_direction()} function.
    \item The lock controller needs to instruct the servomotor to turn one direction or another.
        This can be accomplished through the \function{rotate_full_clockwise()} and \\ \function{rotate_full_counterclockwise()} functions.
\end{itemize}
