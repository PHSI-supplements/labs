In this assignment, you will write code for \runtimeenvironment\ that will use interrupts from external devices and from a timer to simulate controlling a remote-controlled cart.

\begin{figure}[h]
    \centering
    \includegraphics[width=10cm]{MomMomMom}
    \caption{Interrupts. \tiny Image by 20th Century Fox Television}
\end{figure}

The instructions are written assuming you will edit the code in the Arduino IDE or in VS~Code with the PlatformIO plugin, and run it on \runtimeenvironment, constructed according to the pre-lab instructions.

\tableofcontents

\section*{Learning Objectives}

After successful completion of this assignment, students will be able to:
\begin{itemize}
    \item Use tables from a datasheet to determine the bit vectors needed to configure I/O devices
    \item Configure a hardware timer to generate interrupts
    \item Register an interrupt service routine for an interrupt vector
    \item Register an interrupt service routine using a higher abstraction
    \item Use interrupt-driven I/O to realize simple requirements
\end{itemize}

\subsection*{Continuing Forward}

After completing PollingLab and InterruptLab, you will be ready for the Group Project, in which you will design and implement a simple embedded system.

\section*{During Lab Time}

During your lab period, the TAs will demonstrate how to use tables from a datasheet to construct bit vectors.
The TAs will also guide the class through the first modifications to the starter code that you must make.
During the remaining time, the TAs will be available to answer questions.