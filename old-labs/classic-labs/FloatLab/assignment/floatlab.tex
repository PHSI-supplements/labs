%%
%% FloatLab (c) 2019-23 Christopher A. Bohn
%%
%% Licensed under the Apache License, Version 2.0 (the "License");
%% you may not use this file except in compliance with the License.
%% You may obtain a copy of the License at
%%     http://www.apache.org/licenses/LICENSE-2.0
%% Unless required by applicable law or agreed to in writing, software
%% distributed under the License is distributed on an "AS IS" BASIS,
%% WITHOUT WARRANTIES OR CONDITIONS OF ANY KIND, either express or implied.
%% See the License for the specific language governing permissions and
%% limitations under the License.
%%

%%
%% labs/common/assignment.tex
%% (c) 2021-22 Christopher A. Bohn
%%
%% Licensed under the Apache License, Version 2.0 (the "License");
%% you may not use this file except in compliance with the License.
%% You may obtain a copy of the License at
%%     http://www.apache.org/licenses/LICENSE-2.0
%% Unless required by applicable law or agreed to in writing, software
%% distributed under the License is distributed on an "AS IS" BASIS,
%% WITHOUT WARRANTIES OR CONDITIONS OF ANY KIND, either express or implied.
%% See the License for the specific language governing permissions and
%% limitations under the License.
%%

\documentclass[12pt]{article}

\usepackage{fullpage}
\usepackage{fancyhdr}
\usepackage[procnames]{listings}
\usepackage{hyperref}
\usepackage{textcomp}
\usepackage{bold-extra}
\usepackage[dvipsnames]{xcolor}
\usepackage{etoolbox}

% These are placeholder commands and will be renewed in each lab

\newcommand{\labnumber}{}
\newcommand{\labname}{Lab \labnumber\ Assignment}
\newcommand{\shortlabname}{}
\newcommand{\duedate}{}

% Individual or team effort

\newcommand{\individualeffort}{This is an individual-effort project. You may
    discuss concepts and syntax with other students, but you may discuss
    solutions only with the professor and the TAs. Sharing code with or copying
    code from another student or the internet is prohibited.}
\newcommand{\teameffort}{This is a team-effort project. You may discuss concepts
    and syntax with other students, but you may discuss solutions only with your
    assigned partner(s), the professor, and the TAs. Sharing code with or
    copying code from a student who is not on your team, or from the internet,
    is prohibited.}
\newcommand{\freecollaboration}{In addition to the professor and the TAs, you
    may freely seek help on this assignment from other students.}
\newcommand{\collaborationrules}{}

% Software engineering (if you care about that)

\providebool{allowspaghetticode}

\newcommand{\softwareengineeringfrontmatter}{
    \ifboolexpe{not bool{allowspaghetticode}}{
        \section*{No Spaghetti Code Allowed}
        In the interest of keeping your code readable, you may \textit{not} use
        any \lstinline{goto} statements, nor may you use any
        \lstinline{continue} statements, nor may you use any \lstinline{break}
        statements to exit from a loop, nor may you have any functions
        \lstinline{return} from within a loop.
    }{}
}

\newcommand{\spaghetticodepenalties}[1]{
    \ifboolexpe{not bool{allowspaghetticode}}{
        \penaltyitem{1}{for each \lstinline{goto} statement,
            \lstinline{continue} statement, \lstinline{break} statement used to
            exit from a loop, or \lstinline{return} statement that occurs within
            a loop.}
    }{}
}

% You shouldn't need to customize these,
% but you can if you like

\lstset{language=C, tabsize=4, upquote=true, basicstyle=\ttfamily}
\newcommand{\function}[1]{\textbf{\lstinline{#1}}}
\setlength{\headsep}{0.7cm}
\hypersetup{colorlinks=true}

\newcommand{\pagelayout}{
    \pagestyle{fancy}
    \fancyhf{}
    \lhead{\coursenumber}
    \chead{\ Lab \labnumber: \labname}
    \rhead{\courseterm}
    \cfoot{\shortlabname-\thepage}
}

\newcommand{\labidentifier}{
    \title{\ Lab \labnumber}
    \author{\labname}
    \date{Due: \duedate}
    \maketitle

    \textit{\collaborationrules}
}

% deprecated
\newcommand{\startdocument}{
    \pagelayout
	\begin{document}
	\labidentifier
}

\newcommand{\rubricitem}[2]{\item[\underline{\hspace{1cm}} +#1] #2}
\newcommand{\bonusitem}[2]{\item[\underline{\hspace{1cm}} Bonus +#1] #2}
\newcommand{\penaltyitem}[2]{\item[\underline{\hspace{1cm}} -#1] #2}
\newcommand{\checkoffitem}[1]{\item (\phantom{xxx}) #1}
\newcommand{\precheckoffitem}[1]{\item [] (\phantom{xxx}) #1}

%%
%% labs/common/semester.tex
%% (c) 2021-22 Christopher A. Bohn
%%
%% Licensed under the Apache License, Version 2.0 (the "License");
%% you may not use this file except in compliance with the License.
%% You may obtain a copy of the License at
%%     http://www.apache.org/licenses/LICENSE-2.0
%% Unless required by applicable law or agreed to in writing, software
%% distributed under the License is distributed on an "AS IS" BASIS,
%% WITHOUT WARRANTIES OR CONDITIONS OF ANY KIND, either express or implied.
%% See the License for the specific language governing permissions and
%% limitations under the License.
%%


% Customize the semester (or quarter) and the course number

\newcommand{\courseterm}{Spring 2022}
\newcommand{\coursenumber}{CSCE 231}

% Customize how a typical lab will be managed;
% you can always use \renewcommand for one-offs

\newcommand{\runtimeenvironment}{your account on the \textit{csce.unl.edu} Linux server}
\newcommand{\filesource}{Canvas or {\footnotesize$\sim$}cse231 on \textit{csce.unl.edu}}
\newcommand{\filesubmission}{Canvas}

% Customize for the I/O lab hardware

\newcommand{\developmentboard}{Arduino Nano}

%\newcommand{\serialprotocol}{SPI}
\newcommand{\serialprotocol}{I2C}

%\newcommand{\displaymodule}{MAX7219digits}
%\newcommand{\displaymodule}{MAX7219matrix}
\newcommand{\displaymodule}{LCD1602}

\setbool{usedisplayfont}{true}

\newcommand{\obtaininghardware}{
    The EE Shop has prepared ``class kits'' for CSCE 231; your class kit costs \$20. The EE Shop is located at 122 Scott
    Engineering Center and is open M-F 7am-4pm. You do not need an appointment. You may pay at the window with cash,
    with a personal check, or with your NCard. If you wish to pay by credit card, you must make the purchase from
    \url{https://marketplace.unl.edu/ees/engineering-class-kits/csce231-kit.html} the day before you visit the EE
    Shop.\footnote{The price listed on the website is \$18.65; after sales tax is added, your total will be \$20.}
}

% Update to reflect the CS2 course(s) at your institute

\newcommand{\cstwo}{CSCE~156, RAIK~184H, or SOFT~161}

% Do you care about software engineering?

\setbool{allowspaghetticode}{false}

% Which assignments are you using this semester, and when are they due?

\newcommand{\pokerlabnumber}{1}
\newcommand{\pokerlabcollaboration}{Except as noted in Section~\ref{StudyTheCode}, \individualeffort}
\newcommand{\pokerlabdue}{Week of January 24, before the start of your lab section}

\newcommand{\keyboardlabnumber}{2}
\newcommand{\keyboardlabcollaboration}{\individualeffort}
\newcommand{\keyboardlabdue}{Week of January 31, before the start of your lab section}

\newcommand{\pointerlabnumber}{3}
\newcommand{\pointerlabcollaboration}{\individualeffort}
\newcommand{\pointerlabdue}{Week of February 7, before the start of your lab section}

\newcommand{\integerlabnumber}{4}
\newcommand{\integerlabcollaboration}{\individualeffort}
\newcommand{\integerlabdue}{Week of February 14, before the start of your lab section}

\newcommand{\floatlabnumber}{5}
\newcommand{\floatlabcollaboration}{\individualeffort}
\newcommand{\floatlabdue}{soon}

\newcommand{\addressinglabnumber}{6}
\newcommand{\addressinglabcollaboration}{\individualeffort}
\newcommand{\addressinglabdue}{Week of February 28, before the start of your lab section}

%bomblab was 7
%attacklab was 8

\newcommand{\pollinglabnumber}{9}
\newcommand{\pollinglabcollaboration}{\individualeffort}
\newcommand{\pollinglabdue}{Week of April 11, before the start of your lab section}
\newcommand{\pollinglabenvironment}{your \developmentboard-based class hardware kit}

\newcommand{\ioprelabnumber}{\pollinglabnumber-prelab}
\newcommand{\ioprelabcollaboration}{\freecollaboration}
\newcommand{\ioprelabdue}{Before the start of your lab section on April 5 or 6}

\newcommand{\interruptlabnumber}{10}
\newcommand{\interruptlabcollaboration}{\individualeffort}
\newcommand{\interruptlabdue}{Week of April 18, before the start of your lab section}
\newcommand{\interruptlabenvironment}{your \developmentboard-based class hardware kit}

\newcommand{\capstonelab}{ComboLock}    % this will come into play when we generalize capstonelab
\newcommand{\capstonelabnumber}{11}
\newcommand{\capstonelabcollaboration}{\teameffort}
\newcommand{\capstonelabdue}{Week of May 2, Before the start of your lab section\footnote{See Piazza for the due dates of teams with students from different lab sections.}}
\newcommand{\capstonelabenvironment}{your \developmentboard-based class hardware kit}

\newcommand{\memorylabnumber}{12}
\newcommand{\memorylabcollaboration}{This is an individual-effort project. You may discuss the nature of memory technologies and of memory hierarchies with classmates, but you must draw your own conclusions.}
\newcommand{\memorylabdue}{Week of May 2, at the end of your lab section}
\newcommand{\memorylabenvironment}{your \developmentboard-based class hardware kit and your account on the \textit{csce.unl.edu} Linux server}

% Labs not used this semester

\newcommand{\concurrencylabnumber}{XX}
\newcommand{\concurrencylabcollaboration}{\individualeffort}
\newcommand{\concurrencylabdue}{not this semester}

\newcommand{\ssbcwarmupnumber}{XX}
\newcommand{\ssbcwarmupcollaboration}{\freecollaboration}
\newcommand{\ssbcwarmupdue}{not this semester}

\newcommand{\ssbcpollingnumber}{XX}
\newcommand{\ssbcpollingcollaboration}{\individualeffort}
\newcommand{\ssbcpollingdue}{not this semester}

\newcommand{\ssbcinterruptnumber}{XX}
\newcommand{\ssbcinterruptcollaboration}{\individualeffort}
\newcommand{\ssbcinterruptdue}{not this semester}

%%
%% labs/common/storylines.tex
%% (c) 2020-22 Christopher A. Bohn
%%
%% Licensed under the Apache License, Version 2.0 (the "License");
%% you may not use this file except in compliance with the License.
%% You may obtain a copy of the License at
%%     http://www.apache.org/licenses/LICENSE-2.0
%% Unless required by applicable law or agreed to in writing, software
%% distributed under the License is distributed on an "AS IS" BASIS,
%% WITHOUT WARRANTIES OR CONDITIONS OF ANY KIND, either express or implied.
%% See the License for the specific language governing permissions and
%% limitations under the License.
%%

\newcommand{\MeetArchie}{
    You're relaxing at your favorite hangout when another customer catches your attention.
    He's rather large (dare I say, \textit{mammoth}), a bit hairy, and looking frustrated in front of his laptop.
    ``I'm Archie,'' he says, ``and I'm trying to teach myself this card game called \textit{Poker}.
    I found this source code that I thought I could use to understand Poker better, but the code is incomplete, and I don't entirely understand what's there.
    Could you explain the code to me, please?'
}

\newcommand{\GetHired}{
    Archie's face lights up in a very big smile.
    ``Thanks!''
    After pausing in thought for a moment, he says, ``Say, I've got a new startup company that could really use your help.
    Are you interested?
    It'll be exciting!''
}


\renewcommand{\labnumber}{\floatlabnumber}
\renewcommand{\labname}{Floating Point Representation and Arithmetic Lab}
\renewcommand{\shortlabname}{floatlab}
\renewcommand{\collaborationrules}{\floatlabcollaboration}
\renewcommand{\duedate}{\floatlabdue}

\pagelayout
\begin{document}
    \labidentifier\

    \pdfbookmark[1]{Frontmatter}{frontmatter}                           The purpose of this assignment is to give you more confidence in C programming
and to begin your exposure to the underlying bit-level representation of data.

The instructions are written assuming you will edit and run the code on
\runtimeenvironment. If you wish, you may edit and run the code
in a different environment; be sure that your compiler suppresses no warnings,
and that if you are using an IDE that it is configured for C and not C++.

\section*{Learning Objectives}

After successful completion of this assignment, students will be able to:
\begin{itemize}
    \item Use the ASCII table to determine the corresponding integer values of C
    \lstinline{char} values.
    \item Apply arithmetic operators and comparators to C \lstinline{}{char} values.
    \item Construct and use a bitmask.
    \item Use bitwise operators and bit shift operators to create and modify values.
\end{itemize}

\subsection*{Continuing Forward}

Your experience with viewing values as bit patterns will be applicable in
future labs, as will bit masks and bit operations. Some of the functions you
write in this lab will be used in the next lab.

\section*{During Lab Time}

During your lab period, the TAs will demonstrate how to read the ASCII table and
will provide a refresher on bitwise AND, bitwise OR, and left- and right-shifts.
During the remaining time, the TAs will be available to answer questions.

Before leaving lab, \textit{at a minimum} \dots


    \softwareengineeringfrontmatter

    \section*{Scenario}                                                 \WriteAnFPU

    \section{Assignment Summary}                                        Please familiarize yourself with the entire assignment before beginning.
There are three parts to this assignment.

\subsection{Why are There Letters on Telephone Keypads?}

Once upon a time, telephone exchanges were staffed by operators who would use patch cords on a switchboard to connect callers.
If you needed to make a local-area call to someone whose phone was serviced by a different exchange, then you needed to tell your operator which exchange to connect to.
Letters were assigned to digits so that easy-to-remember -- and audibly-distinctive -- mnemonics could be formed such that the first two letters of the mnemonic that correspond to the 2-digit exchange identifier.
For example, the 86 exchange would use a mnemonic that started with a `T', `U', or `V' and that has as a second letter an `M', `N', or `O' --
so 867--5309 might be ``University 7--5309''.

The presence of letters on telephone dials and (later) telephone keypads allowed for custom phone numbers that used words formed by the available letters.
For example, a bank's phone number might be 472--2265, aka 472--BANK.
Less fictionally, 1--800--FLOWERS was used by a company that partnered with florists to allow people to have bouquets delivered anywhere in the U.S\@.

When the Short Message Service protocol was introduced to allow text-based communication by taking advantage of unused bytes in the handshake between cellular phones and the cell network, naturally the letters that were already present on the keypad were used to tap out messages.

While QWERTY keyboards on smartphones have largely replaced the 10-digit keypad for text entry, the letters remain, waiting for the next clever use\dots

\subsection{Constraints} \label{subsec:constraints}

You may \textit{not} poll the matrix keypad nor the pushbuttons to determine if they have been pressed.
You must use interrupts to determine if a key or button has been pressed.
Once an interrupt has fired, you may scan the matrix keypad or read the pushbuttons to determine which key has been pressed or whether the button has been pressed or released.

You may use any features that are part of the C standard if they are supported by the compiler.
You may use the constants and functions provided in the starter code.

\subsubsection{Constraints on the Arduino core}

You may \textit{not} use any libraries, functions, macros, types, or constants from the Arduino core.

%\subsubsection{Constraints on AVR-libc}
%
%You may use any AVR-specific functions, macros, types, or constants of avr-libc.\footnote{
%    \url{https://www.nongnu.org/avr-libc/user-manual/index.html}
%}

\ifdefstring{\processor}{ATmega328P}{
    \subsubsection{Constraints on AVR-libc}

    You may not use any AVR-specific functions, macros, types, or constants of avr-libc.\footnote{\url{https://www.nongnu.org/avr-libc/user-manual/index.html}}
}{}
\ifdefstring{\processor}{RP2040}{
% TODO: parameterize this (when we eventually port to the bare-metal Arduino toolchain, and to the Pico SDK)
    \subsubsection{Constraints on MBED OS}

    You may not use any functions, macros, types or constants from MBED that are not part of the C standard.\footnote{\url{https://os.mbed.com/docs/mbed-os/v6.16/introduction/index.html}}
}{}

\subsubsection{Constraints on the CowPi library}

You may use any functions provided by the CowPi\footnote{
    \url{https://cow-pi.readthedocs.io/en/latest/library.html}
}
and the CowPi\_stdio\footnote{
    \url{https://cow-pi.readthedocs.io/en/latest/stdio.html}
} libraries,
and you may use any data structures\footnote{
    \url{https://cow-pi.readthedocs.io/en/latest/microcontroller.html}
} provided by the CowPi library.

\subsubsection{Constraints on other libraries}

You may \textit{not} use any libraries beyond those explicitly identified here.


    \section{Getting Started}                                           Download the zip file or tarball from \filesource.
Once downloaded, unpackage the file and open the project in your IDE\@.

\subsection{Description of RangeFinder Files}

\subsubsection{combolock.c, interrupt\_support.h, interrupt\_support.cpp, display.h, display.cpp}

Do not edit \textit{combolock.c}, \textit{interrupt\_support.h}, \textit{interrupt\_support.cpp}, \textit{display.}, or \textit{display.cpp}.

These files contain code to simplify interrupt management, and functions to place text on the display module.

\subsubsection{rotary-encoder.h \& rotary-encoder.c}

Do not edit \textit{rotary-encoder.h}

The \textit{rotary-encoder.c} file is where you will process inputs from the rotary encoder.

\subsubsection{servomotor.h \& servomotor.c}

Do not edit \textit{servomotor.h}

The \textit{servomotor.c} file is where you will control the servomotor.

\subsection{lock-controller.h \& lock-controller.c}

Do not edit \textit{lock-controller.h}

The \textit{lock-controller.c} file is where you will implement the logic for the combination lock.

%\subsubsection{shared\_variables.h}
%
%The \textit{shared\_variables.h} header file is where you will place any types that you define and where you will externalize any global variables that need to be used by more than one \textit{.c} file.
%
%It also contains a structure that you may use to access an analog-digital converter's registers,
%and it contains meaningfully-named constants to refer to specific pins you will use in this assignment.


\subsection{Assemble the Hardware}

\textcolor{red}{\textbf{BEFORE YOU PROCEED FURTHER:}}
\begin{description}
    \checkoffitem{Add the new hardware to your Cow~Pi as described in Appendix~\ref{sec:hardwareMods-mk4b}.}
\end{description}


    \section{Constants and Queries} \label{sec:constantsAndQueries}     \subsection{Constants}

There are seven named constants in \textit{fpu.c}.

\begin{description}
    \checkoffitem{Assign the appropriate bit vectors to \lstinline{SIGN_BIT_MASK}, \lstinline{EXPONENT_BITS_MASK}, and \lstinline{FRACTION_BITS_MASK} so that you can use them to mask-off the sign bit, the exponent bits, and the fraction bits, respectively, in a \lstinline{ieee754_t} floating point value.}
    \checkoffitem{Assign the single-precision exponent bias to \lstinline{EXPONENT_BIAS} and assign to \lstinline{NUMBER_OF_FRACTION_BITS} the number of bits used for the fraction bit field in a single-precision floating point number.}
    \checkoffitem{Assign to \lstinline{NAN} and \lstinline{INFINITY} suitable bit vectors for single-precision Infinity and Not-a-Number.}
\end{description}

These constants will not be graded directly;
they exist solely to make your code more readable.
You may define additional named constants as needed.

\subsection{Query Functions}

There are three functions to identify whether an \lstinline{ieee754_t} floating point value is neither normal nor subnormal.
The remaining query function determins whether an \lstinline{ieee754_t} floating point value is negative.
\begin{description}
    \checkoffitem{Implement \function{is_nan()} to detect whether a value is Not-a-Number without regard to the value's sign.}
    \begin{itemize}
        \item Note that \function{is_nan()} must return \lstinline{true} for \textit{all} valid NaN bit vectors and not just those that match your \lstinline{NAN} constant.
    \end{itemize}
    \checkoffitem{Implement \function{is_nan()} to detect whether a value is infinity without regard to the value's sign.}
    \checkoffitem{Implement \function{is_nan()} to detect whether a value is zero without regard to the value's sign.}
    \checkoffitem{Implement \function{is_negative()} to detect whether a value is negative.}
\end{description}


    \section{Examining IEEE 754-Compliant Values}                       \subsection{Extracting the Integer, Fraction, and Exponent} \label{subsec:printing}

There are three functions that extract the components of a value's magnitude.
These three functions all assume that the value is a finite number:
the value might be a normal number;
it might be a subnormal number;
it might be zero.

\begin{description}
    \checkoffitem{Implement \function{get_754_integer()} to produce the implicit integer portion of the significand.}
    \checkoffitem{Implement \function{get_754_fraction()} to return the fraction bits exactly as they appear in the \lstinline{number}.}
    \checkoffitem{Implement \function{get_754_exponent()} to produce the two's complement representation of the exponent that you get after removing the bias from the \lstinline{number}'s \texttt{E} field.}
\end{description}

You will use \function{ieee754_to_string()} to test your code.


\paragraph*{Check Your Work}

\begin{description}
    \checkoffitem{Compile and run \texttt{\textbf{\textit{./floatlab}}}, trying a few values, starting with values whose IEEE~754 representation are easy to check.}
\end{description}
For example:

\begin{verbatim}
        Enter a value, a two-operand arithmetic expression,
            "decode <value>", "recode <value>",
            or "quit": 1
        0x3f800000	+1.0000,0000,0000,0000,0000,000_{2} x 2^{0}

        Enter a value, ... or "quit": .25
        0x3e800000	+1.0000,0000,0000,0000,0000,000_{2} x 2^{-2}

        Enter a value, ... or "quit": 15.625
        0x417a0000	+1.1111,0100,0000,0000,0000,000_{2} x 2^{3}
\end{verbatim}

\begin{description}
    \checkoffitem{Try a few more on your own.}
    \checkoffitem{Create some bit vectors to see that the sign, significand, and exponent are all what you expect, based on your understanding of the IEEE~754 format.}
\end{description}
For example:

\begin{verbatim}
        Enter a value, ... or "quit": 0xF22AAAAA
        0xf22aaaaa	-1.0101,0101,0101,0101,0101,010_{2} x 2^{101}
\end{verbatim}

Don't forget to check subnormal numbers, too.
For example:

\begin{verbatim}
        Enter a value, ... or "quit": 5e-40
        0x000571cc	+0.0000,1010,1110,0011,1001,100_{2} x 2^{-126}
\end{verbatim}

\subsection{\texttt{unnormal\_t} and Its Functions} \label{subsec:unnormal}

The \lstinline{unnormal_t} data type is a structure with fields for the sign bit, the integer portion, the fractional portion, and the exponent.
It also has flags for Not-a-Number and for Infinity.
Unlike the IEEE~754 format, an \lstinline{unnormal_t} value can have more than one bit in its integer portion.

%\begin{center}
\hspace{-2cm}\colorbox{red!25}{The \lstinline{unnormal_t} type does not correspond to any floating point type described in the IEEE~754 standard!}
%\end{center}

You should \textit{not} access the \lstinline{unnormal_t} fields directly.
Instead, use the \function{unnormal()} macro to create an \lstinline{unnormal_t} value, the accessor functions that we provide to retrieve the fields, and the modifier functions that we provide to make adjustments in a value-preserving manner.

\textit{\textbf{NOTE:}} while you typically will see non-primitive variables passed by reference to C functions, the \lstinline{unnormal_t} functions use pass-by-value.
This is considerably less efficient in both time and memory (and there are other downsides beyond the scope of what you've learned so far);
however, passing the \lstinline{unnormal_t} variables by value ensures that there will be no aliased variables and also that memory management will be handled by the program stack.
\textcolor{red}{This means that the variables passed as arguments to functions will be unchanged, and if an \lstinline{unnormal_t} variable is returned then it will be a modified copy of the original.
A consequence of this is that \textit{you must make sure that you always use the value returned by a function if you expect to use the effect of the function.}}

You can, and should, look at the functions' signatures in \textit{unnormal.h}.
\textcolor{red}{The shift functions and the alignment functions are \textit{value-preserving}: changes to the significand create corresponding changes to the exponent, and vice-versa.}
We summarize the functions that you are likely to use here (there are more functions than we list here, and you may use \textit{any} function that is declared in \textit{unnormal.h}):

\begin{itemize}
    \item Creating and printing an \lstinline{unnormal_t} value (all necessary calls to these functions are in the starter code)
    \begin{description}
        \item[unnormal()] returns an \lstinline{unnormal_t} value (\textit{not} a pointer) based on the arguments provided.
            The argument list is a series of dot-prefixed named fields (such as \lstinline{.sign=0, .infinity=1}) whose meanings are the obvious ones from the class discussion about floating point numbers.
            \colorbox{green}{\textit{\textbf{Note: }} the \lstinline{.exponent} argument, if} \\ \colorbox{green}{included, is expected to be a two's complement value.}
            \colorbox{yellow}{\textit{\textbf{Note: }} the \lstinline{.fraction}} \\ \colorbox{yellow}{argument, if included, is expected to be the numerator of $\frac{.fraction\ argument}{2^{64}}$ (\textit{i.e.},} \\ \colorbox{yellow}{the $2^{-1}$ bit is $bit_{63}$)\@.}
        \item[unnormal\_to\_string()] returns a string representation of the value.
            Because of the number of bits available in the \lstinline{unnormal_t} fields, the significand is represented in base-16, though the exponent is the exponent of 2.
    \end{description}
    \item Accessors
    \begin{description}
        \item[get\_unnormal\_sign()] returns 0 if the value is positive, 1 if the value is negative
        \item[get\_unnormal\_integer()] returns a \lstinline{uint64_t} storing the unsigned integer representation of the value's integer portion
        \item[get\_unnormal\_fraction()] returns a \lstinline{uint64_t} storing the numerator of $\frac{get\_fraction()}{2^{64}}$ (\textit{i.e.}, the $2^{-1}$ bit is $bit_{63}$)
        \item[get\_unnormal\_exponent()] returns a \lstinline{int16_t} storing the two's complement representation of the exponent
        \item[is\_infinite()] returns 0 if the value is finite (or NaN), 1 if the value is $\pm\infty$
        \item[is\_not\_a\_number()] returns 0 if the value is a valid number, 1 if the value is not a number
    \end{description}
    \item Bit shifts (some of these functions are equivalent to each other, to support whichever mental model works for you)
    \begin{description}
        \item[shift\_left\_once()] aka \function{decrement_exponent()}, aka \function{move_binary_point_to_the_right()} -- shifts the significand's bits to the left by one position, having the effect of moving the binary point to the right and decreasing the exponent by one
        \item[shift\_right\_once()] aka \function{increment_exponent()}, aka \function{move_binary_point_to_the_left()} -- shifts the significand's bits to the right by one position, having the effect of moving the binary point to the left and increasing the exponent by one
%        \item[shift\_left()] shifts the significand's bits to the left by a specified amount
%        \item[shift\_right()] shifts the significand's bits to the right by a specified amount
    \end{description}
    \item Alignment functions (shifts bits by specifying the desired result instead of the amount)
    \begin{description}
        \item[place\_all\_bits\_in\_integer()] aka \function{prepare_for_arithmetic()} shifts the significand such that the least-significant \lstinline{1} bit is in the $2^0$ position, with a corresponding change in the exponent (the fraction, of course, will be 0)
        \item[set\_integer()] if possible, shifts the significand such that the resulting integer portion is the specified value, with a corresponding changes in the fraction and exponent
        \item[set\_exponent()] shifts the significand, with a corresponding change in the exponent, such that the resulting exponent is the specified value
    \end{description}
    \item Static Warnings (warnings based on the current bit positions)
    \begin{description}
%        \item[multiplication\_is\_not\_recommended()] indicates that there are \lstinline{1} bits to the right of the binary point:
%                if there are \lstinline{1} bits in the fraction, then there will be extra bookkeeping that you will be responsible for;
%                we recommend that you attempt multiplication only when all \lstinline{1} bits are in the integer portion
%        \item[multiplication\_is\_unreliable()] indicates that there are \lstinline{1} bits far enough to the left of the binary point that multiplication could yield a product whose integer portion exceeds the available bits
        \item[addition\_is\_unreliable()] indicates that there are \lstinline{1} bits far enough to the left of the binary point that addition could yield a sum whose integer portion exceeds the available bits
    \end{description}
    \item Dynamic Warnings (warnings that result from the last function call)
    \begin{description}
%        \item[shift\_overflowed()] indicates that one or more \lstinline{1} bit shifted to the left beyond the available bits
%        \item[shift\_underflowed()] indicates that one or more \lstinline{1} bit shifted to the right beyond the available bits
%        \item[operation\_was\_not\_performed()] indicates that the previous function did not have the desired result, such as attempting to shift by a negative amount or attempt to set an integer value whose bits are not present in the significand
        \item[created\_number\_is\_improbable()] indicates that a call to \function{unnormal()} was made with all of the fraction's \lstinline{1} far enough from the binary point that it is unlikely to have been the intended value (because it is \textit{possible} that the requested fraction is also the intended fraction, an Unnormal value with the requested fraction was created)
    \end{description}
    \item Prediction Functions (warnings that indicate what will happen in the next operation)
    \begin{description}
%        \item[fraction\_will\_carry\_into\_integer\_on\_addition()] indicates that if the next operation is addition with the specified values, then adding the fractions will carry into the integer portion, requiring you to add 1 to the sum's integer
%        \item[fraction\_will\_borrow\_from\_integer\_on\_subtraction()] indicates that if the next operation is subtraction with the specified values, then subtracting the fractions will require ``borrowing'' from the integer portion, requiring you to subtract 1 to the difference's integer
%        \item[left\_shift\_will\_make\_multiplication\_unreliable()] indicates that if the next function is \function{left_shift_once()} (or one of its aliases) then after that function call, \function{multiplication_is_unreliable()} will return \lstinline{true}
        \item[left\_shift\_will\_make\_addition\_unreliable()] indicates that if the next function is \function{left_shift_once()} (or one of its aliases) then after that function call, \\ \function{addition_is_unreliable()} will return \lstinline{true}
%        \item[left\_shift\_will\_overflow()] indicates that if the next function is \function{left_shift_once()} (or one of its aliases) then after that function call, \function{shift_overflowed()} will return \lstinline{true}
%        \item[right\_shift\_will\_underflow()] indicates that if the next function is \function{right_shift_once()} (or one of its aliases) then after that function call, \function{shift_underflowed()} will return \lstinline{true}
    \end{description}
\end{itemize}

\subsection{Decoding Numbers from the IEEE 754 Format into \texttt{unnormal\_t}} \label{subsec:decoding}

The \function{decode()} function converts an IEEE~754-compliant \lstinline{ieee754_t} value into a format that facilitates arithmetic.
As described in Section~\ref{subsec:unnormal}, the \lstinline{unnormal_t} structure that is returned by \function{decode()} has fields for the sign bit, the integer portion, the fractional portion, and the exponent.
It also has flags for Not-a-Number and for Infinity.
Unlike the IEEE~754 format, an \lstinline{unnormal_t} value can have more than one bit in its integer portion.

You have already done most of the work to populate an \lstinline{unnormal_t} structure.
In \function{decode()} you will need to left-shift the \lstinline{fraction} returned by \function{get_754_fraction()} by several places such that the $2^{-1}$ bit is in $bit_{63}$ (the most-significant bit) of an \lstinline{uint64_t} bit vector, the $2^{-2}$ bit is in $bit_{62}$, and so on.
This is because the \function{unnormal()} function call expects the \lstinline{.fraction} named field to be the numerator of
\[\frac{.fraction\ argument}{2^{64}}\]


\subsubsection{A Visualization}

Envision the 64 bits in the \lstinline{.integer} field and the 64 bits in the \lstinline{.fraction} field:
\[ i_{63} i_{62} i_{61} i_{60} \cdots i_3 i_2 i_1 i_0\ \mathbf{.} \ f_{63} f_{62} f_{61} f_{60} \cdots f_3 f_2 f_1 f_0 \]

If we have the number $1\frac{3}{4}$, and if we were to place the \lstinline{ieee754_t}'s fraction bits directly into the \lstinline{unnormal_t}'s \lstinline{.fraction} field, then we would have:
\[ 0000 \cdots 0001\ \mathbf{.} \ 0000 \cdots 0110\ 0000 \cdots 0000 \]
giving us the value $1\frac{3 \times 2^{21}}{2^{64}} = 1\frac{3}{2^{43}}$, which not what we want.
On the other hand, if we were to shift the fraction bits by $(64 - NUMBER\_OF\_FRACTION\_BITS)$ places, then we would have:
\[ 0000 \cdots 0001\ \mathbf{.} \ 1100 \cdots 0000 \]
giving us the value $1\frac{3 \times 2^{62}}{2^{64}} = 1\frac{3}{2^2} = 1\frac{3}{4}$.

This is why the fraction bits need to be shifted so that the bit corresponding to the $2^{-1}$ place is in $bit_{63}$.

\subsubsection{Examples}

Suppose that the number is $68588.0_{10} = 1.0000,1011,1110,11_{2} \times 2^{16}$.
In the IEEE~754 normal form, this is 0x4785'F600 = 0b0\underline{100'0111'1}000'0101'1111'0110'0000'0000 (we have underlined the $E$ field to help you follow the discussion).

In the \lstinline{unnormal_t} data structure, the \lstinline{.integer} is 1, the \lstinline{.exponent} is 16, and the \lstinline{.fraction} is 0x0BEC'0000'0000'0000.
Mathematically, that is the \lstinline{.fraction} field because the fraction field needs to be the numerator of
\[\frac{BEC,0000,0000,0000_{16}}{1,0000,0000,0000,0000_{16}}\]
That is why you need to left-shift the \lstinline{fraction} by several places (so that the $2^{-1}$ bit is in the most-significant bit) before passing it to the \function{unnormal()} function.

To see this from a practical perspective, let us consider some of the functions defined for \lstinline{unnormal_t}.
In these examples, assume that all numbers are stored in \lstinline{unnormal_t} data structures.
\begin{itemize}
    \item \textbf{\texttt{shift\_left\_once($1.0000,1011,1110,11_{2} \times 2^{16}$)}} will return a \textit{copy} of the original \lstinline{unnormal_t} data structure, except that the significand's bits have been left-shifted by one, and the fraction has been decreased by one;
        specifically, it will return \\ $10.0001,0111,1101,1_{2} \times 2^{15}$.
        (The original data structure will be unchanged.)
    \item \textbf{\texttt{shift\_right($10.0001,0111,1101,1_{2} \times 2^{15}$, 4)}} will return \\
        $0.0010,0001,0111,1101,1_{2} \times 2^{19}$: the significand's bits have been right-shifted by four, and the fraction has increased by four.
    \item \textbf{\texttt{place\_all\_bits\_in\_integer($0.0010,0001,0111,1101,1_{2} \times 2^{19}$)}} will return \\
        $100,0010,1111,1011.0_{2} \times 2^{2}$.
    \item \textbf{\texttt{set\_exponent($100,0010,1111,1011.0_{2} \times 2^{2}$, 0)}} will return  \\
        $1,0000,1011,1110,1100.0_{2} \times 2^{0}$.
    \item \textbf{\texttt{set\_integer($1,0000,1011,1110,1100.0_{2} \times 2^{0}$, 1)}} will return  \\
        $1.0000,1011,1110,11_{2} \times 2^{16}$.
\end{itemize}


\subsubsection{Decoding Numbers}

The \lstinline{integer} and \lstinline{exponent} variables can obtain their correct values directly from the \function{get_754_integer()} and \function{get_754_exponent()} functions.
The \lstinline{fraction} variable, however, needs to shift the value obtained from \function{get_754_fraction()}.

\begin{description}
    \checkoffitem{Assign the appropriate value to \lstinline{fraction}.}
    \begin{itemize}
        \item Because the value obtained from \function{get_754_fraction()} has the fraction bits in the \lstinline{NUMBER_OF_FRACTION_BITS} least-significant bits,
            and because the most significant bit is $bit_{63}$, placing the $2^{-1}$ bit in the most significant bit can be accomplished by shifting the fraction bits by $(64 - NUMBER\_OF\_FRACTION\_BITS)$ places.
    \end{itemize}
\end{description}


\paragraph*{Check Your Work}

\begin{description}
    \checkoffitem{Compile and run \texttt{\textbf{\textit{./floatlab}}}, decoding a few values.}
\end{description}
We have provided a function that will print a \lstinline{unnormal_t} value.
When you run \texttt{\textbf{\textit{./floatlab}}}, you can specify that you want to decode a value, such as \texttt{\textbf{\textit{decode 12.375}}}.
The program will then print the value, based on the \lstinline{unnormal_t} fields.
Because there are 64 bits available for the integer portion and another 64 bits for the fractional portion, the \lstinline{unnormal_t} value will be printed in base-16:

\begin{verbatim}
        Enter a value, a two-operand arithmetic expression,
            "decode <value>", "recode <value>",
            or "quit": decode 12.375
        +0000000000000001.8c00000000000000_{16} x 2^{3}
\end{verbatim}

Because $12.375_{10} = 1.1000,11_{2} \times 2^3$, we can see that $1.8\mathrm{C}_{16} \times 2^3$ is correct.

%When denormalizing, you can optionally specify an amount to change the exponent.
%For example:
%
%\begin{verbatim}
%    Enter ... "decode <value> <change exponent amount>", ...
%        or "quit": decode 12.375 -4
%    +0000000000000018.c000000000000000_{16} x 2^{-1}
%
%    Enter ... "decode <value> <change exponent amount>", ...
%        or "quit": decode 12.375 3
%    +0000000000000000.3180000000000000_{16} x 2^{6}
%\end{verbatim}
%
%If your \function{decode()} function works correctly without the exponent adjustment, then it will work with the exponent adjustment.
%This feature is not useful to you \textit{now}, but you may find it useful if you need to debug your \function{encode()} function.

\begin{description}
    \checkoffitem{Try some other numbers that you can easily convert to base-two by hand, and see if your \function{decode()} function correctly unpacks their IEEE~754 representation and correctly sets the \lstinline{unnormal_t} fields.}
\end{description}

    \section{Encoding Numbers into the IEEE 754 Format}                 \subsection*{Normalize}

The \function{normalize()} function converts an \lstinline{unnormal_t} value into an IEEE~754-compliant format.
The function stub already handles zero and the flagged cases of Not-a-Number and Infinity.
The stub also handles the sign bit.

Your task is to handle:
\begin{itemize}
    \item Normal numbers, both those that already have exactly $1$ in the integer portion and those that need to be adjusted.
    \item Subnormal numbers, both those that can be directly converted and those that need to be adjusted.
    \item Cases that you will not be able to test until later:
    \begin{itemize}
        \item Numbers too great to be represented as normal numbers.
        \item Numbers too small to be represented as subnormal numbers.
        \item Rounding (when implemented, follow the IEEE~754 default of ``round-to-nearest-even.'')
    \end{itemize}
\end{itemize}

For the first two tasks, you will probably make use of \lstinline{unnormal_t}'s \function{set_integer()} and \function{set_exponent()} functions in addition to the functions that access the structure's fields.
Don't forget that the bit vector returned by \function{get_fraction()} is the numerator of $\frac{get\_fraction()}{2^{64}}$ and that \function{get_exponent()} returns the two's complement representation of the exponent.

\ifbool{offerdecompositionhints}{
    \paragraph{Hint}
    In the \function{normalize()} function, when your program reaches \\
    \lstinline{/* GENERATE THE APPROPRIATE BIT VECTOR AND PLACE IT IN RESULT */} \\
    then you know that \lstinline{number}'s ``true value'' is a non-zero, finite number.

    \begin{itemize}
        \item If the number is too great to be represented as a normal number, then it is indistinguishable from infinity.
            Suppose that the number \textit{can} be represented as a normal number -- what is the greatest exponent possible?
            Is \lstinline{number}'s exponent greater than that?
        \item If the number is too small to be represented as a normal number, then it might be representable as a subnormal number.
            Suppose that the number \textit{can} be represented as a normal number -- what is the least exponent possible?
            Is \lstinline{number}'s exponent less than that?
    \end{itemize}
}{}

\subsubsection*{Check Your Work}

When you run \texttt{\textbf{\textit{./floatlab}}}, you can specify that you want to renormalize a value, such as \texttt{\textbf{\textit{renormalize 12.375}}} and \texttt{\textbf{\textit{renormalize 12.375 6}}}.
The program will first denormalize the value.
It will then adjust the exponent by the specified amount (if an amount is specified).
Then it will send the result to \function{normalize()}.
Finally, it will print the original value and the \lstinline{ieee754_t} value returned by \function{normalize}.

For example:

\begin{verbatim}
Enter ... "renormalize <value> <change exponent amount>", ...
    or "quit": renormalize 12.375
expected: 12.3750000000_{10}	0x41460000	+1.1000,1100,0000,0000,0000,000_{2} x 2^{3}
actual:   12.3750000000_{10}	0x41460000	+1.1000,1100,0000,0000,0000,000_{2} x 2^{3}

Enter ... "renormalize <value> <change exponent amount>", ...
    or "quit": renormalize 12.375 6
expected: 12.3750000000_{10}	0x41460000	+1.1000,1100,0000,0000,0000,000_{2} x 2^{3}
actual:   12.3750000000_{10}	0x41460000	+1.1000,1100,0000,0000,0000,000_{2} x 2^{3}

Enter ... "renormalize <value> <change exponent amount>", ...
    or "quit": renormalize 0x00055000
expected: 0.0000000000_{10}	0x00055000	+0.0000,1010,1010,0000,0000,000_{2} x 2^{-126}
actual:   0.0000000000_{10}	0x00055000	+0.0000,1010,1010,0000,0000,000_{2} x 2^{-126}
\end{verbatim}

    \section{Addition and Subtraction}                                  As is the case for integer arithmetic, the heavy-lifting for addition and subtraction is handled solely by the \function{add()} function.
The \function{subtract()} function is already implemented in terms of \function{add} and the \function{negate()} function.

\subsection{Add}

The \function{add()} stub identifies a handful of special cases that you can easily handle.
After the guard clauses are handled, your task is to implement addition for two finite, non-zero numbers.

Recall that for any number base, $b$, adding two floating point values $m_1 \times b^{e_1} + m_2 \times b^{e_2}$ is simplified when $e_1 = e_2$.

Start by adjusting one of the denormalized operands so that the two denormalized operands have the same exponent.
It is acceptable for the least significant bit (or even several less significant bits) to be truncated;
however, \textit{you must take care that the most significant bit does not get truncated}!

You now have two options:
\begin{itemize}
    \item Use integer addition to add the fractional portions, use integer addition to add the integer portions, detect whether the fractional portion overflowed, and carry that 1 into the integer portion if the fractional portion did overflow.
    \item Adjust the two denormalized operands together until their significands are both fully in the integer portion or fully in the fractional portion.
        Make sure that, regardless of which field you chose, that that field's most significant bit is 0 for both operands, leaving room for a carry bit.
        Add the two operands together using integer addition.
\end{itemize}

\subsection{Negate}

The \function{negate()} function is simple: it only needs to change the number's sign bit.

\subsubsection*{Check Your Work}

When you run \texttt{\textbf{\textit{./floatlab}}}, you can enter a two-operand expression, such as addition and subtraction.
As before, if an operand is prepended with \texttt{\textbf{\textit{0x}}} then the parser will treat it as a bit vector;
otherwise, the parser will treat it as a floating point value.

\begin{verbatim}
Enter a value, a two-operand arithmetic expression,
    "denormalize <value> <change exponent amount>",
    "renormalize <value> <change exponent amount>",
    or "quit": 1 + 2
0x3f800000	+1.0000,0000,0000,0000,0000,000_{2} x 2^{0}
+
0x40000000	+1.0000,0000,0000,0000,0000,000_{2} x 2^{1}
expected: 3.0000000000_{10}	0x40400000	+1.1000,0000,0000,0000,0000,000_{2} x 2^{1}
actual:   3.0000000000_{10}	0x40400000	+1.1000,0000,0000,0000,0000,000_{2} x 2^{1}
\end{verbatim}

Be sure to check:
\begin{itemize}
    \item The identity and commutative properties
    \begin{itemize}
        \item[] \texttt{\textbf{\textit{5 + 0}}}
        \item[] \texttt{\textbf{\textit{2 + 3}}}
        \item[] \texttt{\textbf{\textit{3 + 2}}}
    \end{itemize}
    \item Cases in which the exponent will be greater than that of either operand
    \begin{itemize}
        \item[] \texttt{\textbf{\textit{3 + 3}}}
    \end{itemize}
    \item Fractional operands
    \begin{itemize}
        \item[] \texttt{\textbf{\textit{.3 + .03}}}
    \end{itemize}
    \item Not only the use of negative operands, but also subtraction
    \begin{itemize}
        \item[] \texttt{\textbf{\textit{-5 + 2}}}
        \item[] \texttt{\textbf{\textit{2 - 5}}}
        \item[] \texttt{\textbf{\textit{3 - -3}}}
    \end{itemize}
    \item Numbers both great and small
    \begin{itemize}
        \item[] \texttt{\textbf{\textit{1.65e35 + 2.39e29}}}
        \item[] \texttt{\textbf{\textit{1.65e-39 + 2.39e-33}}}
    \end{itemize}
    \item A sufficiently-large sum overflows to infinity
    \begin{itemize}
        \item[] \texttt{\textbf{\textit{0x7F7FFFFF + 0x73800000}}}
    \end{itemize}
    \item A sufficiently-small difference between normal numbers underflows to subnormal numbers
    \begin{itemize}
        \item[] \texttt{\textbf{\textit{0x01000000 - 0x00C00000}}}
    \end{itemize}
    \item A number subtracted from itself produces zero:
    \begin{itemize}
        \item[] \texttt{\textbf{\textit{0x00000001 - 0x00000001}}}
    \end{itemize}
    \item NaN and Infinity are ``sticky'' (except for $\infty - \infty$)
    \begin{itemize}
        \item[] \texttt{\textbf{\textit{nan + 1}}}
        \item[] \texttt{\textbf{\textit{inf - 0x7F7FFFFF}}}
        \item[] \texttt{\textbf{\textit{inf - inf}}}
    \end{itemize}
\end{itemize}

\textit{Note: For the \function{add()} function, we will not deduct points if you have the wrong sign for Zero or for Not-a-Number} because the appropriate sign is usually indeterminate
(There are two cases where the sign can be determined; can you discover which cases those are?)

\subsubsection*{Rounding}

You can now check the rounding code in your \function{normalize()} implementation.

\begin{itemize}
    \item When the rounded-off portion is less than halfway, you always round down
    \begin{itemize}
        \item[] \texttt{\textbf{\textit{0x40000000 + 0x33FFFFFF}}}
        \item[] \texttt{\textbf{\textit{0x40000001 + 0x33FFFFFF}}}
    \end{itemize}
    \item When the rounded-off portion is more than halfway, you always round up
    \begin{itemize}
        \item[] \texttt{\textbf{\textit{0x40000000 + 0x34000001}}}
        \item[] \texttt{\textbf{\textit{0x40000001 + 0x34000001}}}
    \end{itemize}
    \item When the rounded-off portion is exactly halfway, you round to the nearest-even
    \begin{itemize}
        \item[] \texttt{\textbf{\textit{0x40000000 + 0x34000000}}}
        \item[] \texttt{\textbf{\textit{0x40000001 + 0x34000000}}}
    \end{itemize}
    \item Sometimes rounding can carry all the way to the integer portion, causing the exponent to change
    \begin{itemize}
        \item[] \texttt{\textbf{\textit{0x407FFFFF + 0x34000000}}}
    \end{itemize}
\end{itemize}

    \section{Multiplication and Division}                               Recall that for any number base, $b$, multiplying two floating point values $(m_1 \times b^{e_1}) \cdot (m_2 \times b^{e_2})$ produces $m_1 \cdot m_2 \times b^{e_1 + e_2}$.
Similarly, dividing the two floating point values yields $\frac{m_1}{m_2} \times b^{e_1 - e_2}$.

\subsection{Multiply}

The \function{multiply()} stub identifies a handful of special cases that you can easily handle.
\begin{description}
    \checkoffitem{Produce the appropriate reutrn values for the guard clauses.}
    \begin{itemize}
        \item If it is easier for you, you \textit{may} change the compound conditionals in the guard clauses into separate guard clauses.
            For example, instead of handling \\ \lstinline{if (is_infinity(multiplier) || is_zero(multiplier))} as a single guard clause,
            you may handle \lstinline{if (is_infinity(multiplier)} as its own guard clause and then handle \lstinline{if (is_zero(multiplier))} as its own guard clause.
    \end{itemize}
\end{description}

After the guard clauses, the operands are guaranteed to both be finite, non-zero numbers.
These lines
\begin{lstlisting}
unnormal_t decoded_multiplicand = prepare_for_arithmetic(decode(multiplicand));
unnormal_t decoded_multiplier = prepare_for_arithmetic(decode(multiplier));
\end{lstlisting}
create \lstinline{unnormal_t} representations of the operands that have their significands fully in the integer portion.
Because you do not need to give the two operands the same exponent, you do not need to worry about further adjustment.

\begin{description}
    \checkoffitem{Multiply the two operands together using integer arithmetic.}
\end{description}

\textit{Note}: For the \function{multiply()} function, we will not deduct points if you have the wrong sign for Not-a-Number.
\textit{We will, however, deduct points if you have the wrong sign for Zero.}


\subsubsection*{Check Your Work}

\begin{description}
    \checkoffitem{Compile and run \texttt{\textbf{\textit{./floatlab}}}, multiplying a few values.}
\end{description}
Be sure to check:
\begin{itemize}
    \item The identity, zero, and commutative properties
    \begin{itemize}
        \item[] \texttt{\textbf{\textit{5 * 1}}}
        \item[] \texttt{\textbf{\textit{5 * 0}}}
        \item[] \texttt{\textbf{\textit{2 * 3}}}
        \item[] \texttt{\textbf{\textit{3 * 2}}}
    \end{itemize}
    \item Integer operands
    \begin{itemize}
        \item[] \texttt{\textbf{\textit{75 * 7}}}
        \item[] \texttt{\textbf{\textit{5 * 25}}}
    \end{itemize}
    \item Fractional operands
    \begin{itemize}
        \item[] \texttt{\textbf{\textit{.75 * 7}}}
        \item[] \texttt{\textbf{\textit{5 * .25}}}
    \end{itemize}
    \item Negative operands
    \begin{itemize}
        \item[] \texttt{\textbf{\textit{-5 * 2}}}
        \item[] \texttt{\textbf{\textit{5 * -2}}}
        \item[] \texttt{\textbf{\textit{-5 * -2}}}
        \item[] \texttt{\textbf{\textit{5 * -0}}}
    \end{itemize}
    \item Numbers both great and small
    \begin{itemize}
        \item[] \texttt{\textbf{\textit{1.65e25 * 2.39e11}}}
        \item[] \texttt{\textbf{\textit{1.65e-25 * 2.39e-11}}}
        \item[] \texttt{\textbf{\textit{1e-30 * 1e-8}}}
        \item[] \texttt{\textbf{\textit{2e30 * 2e-30}}}
    \end{itemize}
    \item A sufficiently-large product overflows to infinity
    \begin{itemize}
        \item[] \texttt{\textbf{\textit{0x7E000000 * 0x41000000}}}
    \end{itemize}
    \item A sufficiently-small product underflows to zero
    \begin{itemize}
        \item[] \texttt{\textbf{\textit{0x3C800000 * 0x00000020}}}
    \end{itemize}
    \item NaN and Infinity are ``sticky'' (except for $\infty \times 0$)
    \begin{itemize}
        \item[] \texttt{\textbf{\textit{nan * 1}}}
        \item[] \texttt{\textbf{\textit{inf * 2}}}
        \item[] \texttt{\textbf{\textit{inf * 0}}}
    \end{itemize}
\end{itemize}

\subsection{Divide}

Implementing the \function{divide()} function is very similar to implementing \function{multiply()} with three exceptions:

\begin{itemize}
    \item There are different special cases
    \item You subtract the exponents and divide the significands
    \item In general, integer division is insufficient
    \begin{itemize}
        \item We will limit our implementation of \function{divide()} to the special cases and to cases in which the dividend's significand is a multiple of the divisor's significand.
        \item This limitation guarantees that if the \lstinline{unnormal_t} operands' significands are fully in the \lstinline{.integer} field then the \lstinline{unnormal_t} quotient's significand will fully be in the \lstinline{.integer} field and will not spill over into the \lstinline{.fraction} field.
        \item In this limited implementation, integer division is sufficient.
    \end{itemize}
\end{itemize}

\begin{description}
    \checkoffitem{Produce the appropriate reutrn values for the guard clauses.}
    \checkoffitem{Divide $dividend \div divisor$ using integer arithmetic.}
\end{description}

\textit{Note}: For the \function{divide()} function, we will not deduct points if you have the wrong sign for Not-a-Number.
\textit{We will, however, deduct points if you have the wrong sign for Zero.}


\subsubsection*{Check Your Work}

\begin{description}
    \checkoffitem{Compile and run \texttt{\textbf{\textit{./floatlab}}}, dividing a few values.}
\end{description}

Be sure to check:
\begin{itemize}
    \item The identity property
    \begin{itemize}
        \item[] \texttt{\textbf{\textit{5 / 1}}}
    \end{itemize}
    \item Integer operands
    \begin{itemize}
        \item[] \texttt{\textbf{\textit{75 / 4}}}
    \end{itemize}
    \item Fractional operands
    \begin{itemize}
        \item[] \texttt{\textbf{\textit{.75 / 4}}}
        \item[] \texttt{\textbf{\textit{5 / .25}}}
        \item[] \texttt{\textbf{\textit{.75 / .25}}}
    \end{itemize}
    \item Negative operands
    \begin{itemize}
        \item[] \texttt{\textbf{\textit{-4 / 2}}}
        \item[] \texttt{\textbf{\textit{4 / -2}}}
        \item[] \texttt{\textbf{\textit{-4 / -2}}}
    \end{itemize}
    \item Divisors that require more than one 1 in the significand (but the dividend's significand is still a multiple of the divisor's significand)
    \begin{itemize}
        \item[] \texttt{\textbf{\textit{30 / 5}}}
        \item[] \texttt{\textbf{\textit{25 / 0x3F200000}}}
        \item[] \texttt{\textbf{\textit{0x3F480000 / 5}}}
    \end{itemize}
    \item The special cases
    \begin{itemize}
        \item[] \texttt{\textbf{\textit{nan / 2}}}
        \item[] \texttt{\textbf{\textit{2 / nan}}}
        \item[] \texttt{\textbf{\textit{inf / 2}}}
        \item[] \texttt{\textbf{\textit{0 / 2}}}
        \item[] \texttt{\textbf{\textit{2 / inf}}}
        \item[] \texttt{\textbf{\textit{2 / 0}}}
        \item[] \texttt{\textbf{\textit{inf / inf}}}
        \item[] \texttt{\textbf{\textit{0 / 0}}}
    \end{itemize}
\end{itemize}


    \section{Turn-in and Grading}                                       \filesubmission

\policyforcodethatdoesnotcompile

\latepolicy

\subsection*{Rubric}

This assignment is worth 20 points.
\begin{description}
    \rubricitem{4}{\textit{problem1.c} produces the specified output.}
    \rubricitem{4}{\function{iz_digit()} in \textit{problem2.c} determines whether
    or not a character is a digit.}
    \rubricitem{4}{\function{decapitalize()} in \textit{problem2.c} converts
    uppercase letters to lowercase and leaves other characters unchanged.}
    \rubricitem{4}{\function{is_even()} in \textit{problem3.c} determines whether
    a number is even or odd.}
    \item[\hspace{1cm}]\function{produce_multiple_of_ten()} in \textit{problem3.c}
    has the following:
    \begin{description}
        \rubricitem{1}{Code to assign the value 5 to the variable \lstinline{five}}
        \rubricitem{1}{Code to divide an even number by 2}
        \rubricitem{1}{Code to subtract 1 from an odd number}
        \rubricitem{1}{Correct functionality}
    \end{description}
    \item[Penalties]
    \penaltyitem{4}{for each solution that depends on a prohibited character.}
    \penaltyitem{4}{for each solution that hard-codes a return value instead of attempting to solve the specified problem}
    \softwareengineeringpenalties
\end{description}


    \section*{Epilogue}                                                 \GoingBackToTheZoo

    \textit{To be continued...}

\end{document}
