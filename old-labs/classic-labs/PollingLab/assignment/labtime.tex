During your lab period, the TAs will guide the class through the first modifications to the starter code that you must make, described in this section.
If you do not attend your lab period, then you must complete this section on your own.
\textbf{\textit{Except during lab time, you may }not\textit{ discuss the solutions for this section with other students.}}




\subsection{Populate the Keypad's Lookup Table} \label{subsec:populatekeypad}

Locate the \lstinline{keys} $4 \times 4$ nested array on lines 29--34 of \textit{io\_functions.c}.
This nested array serves as a lookup table to obtain the appropriate value when a key is pressed.
\textit{Even the starter code depends on this nested array being populated correctly.}
The element \lstinline{keys[0][0]} will correspond to the \texttt{1} key;
\lstinline{keys[0][3]} will correspond to the \texttt{A} key;
\lstinline{keys[3][0]} will correspond to the \texttt{*} key;
and \lstinline{keys[3][3]} will correspond to the \texttt{D} key.

We want the numerals \texttt{0}--\texttt{9} to produce their respective decimal (and hexadecimal) values.
We want \texttt{A}--\texttt{D} to produce their respective hexadecimal values.
We want \texttt{\#} to produce the hexadecimal value 0xE, and we want \texttt{*} to produce the hexadecimal value 0xF\@.

\begin{itemize}
    \item Populate \lstinline{keys}' nested array initializer so that the lookup table will produce the correct value for each row/column combination.
\end{itemize}

\subsubsection*{Test your code}

You can confirm that you correctly populated the array's initializer by running the test code.
Place the \textbf{left switch} in the \textit{right} position and upload your code.
(Or, if you have already uploaded your code, place the \textbf{left switch} in the \textit{right} position and press the RESET button in the middle of the \developmentboard.)
The test code will output on the display module.
The output will include the key that was pressed (if any), a hyphen if no key is being pressed, or a question mark if \function{get_keypress()} returns an invalid value.

\begin{lstlisting}[numberstyle=\color{gray}, numbers=left, firstnumber=351, basicstyle=\ttfamily\small]
    ...
    if (0x00 <= key_pressed && key_pressed <= 0x0F) {
        fprintf(display, "%2X%5c%2c%4c%2c\f",
                key_pressed,
                left_button_position ? 'D' : 'U', right_button_position ? 'D' : 'U',
                left_switch_position ? 'R' : 'L', right_switch_position ? 'R' : 'L');
    } else {
        fprintf(display, "%2c%5c%2c%4c%2c\f",
                key_pressed == 0xFF ? '-' : '?',
                left_button_position ? 'D' : 'U', right_button_position ? 'D' : 'U',
                left_switch_position ? 'R' : 'L', right_switch_position ? 'R' : 'L');
    }
    ...
\end{lstlisting}

Familiarize yourself with the test code's other outputs.
The output will include the positions of the left and right buttons (``U'' for up and ``D'' for down) and of the left and right switches (``L'' for left position and ``R'' for right position).

Finally, if both buttons are pressed then the left LED will illuminate, and if both switches are in the right position then the right LED will illuminate.

\begin{lstlisting}[numberstyle=\color{gray}, numbers=left, firstnumber=347]
    ...
    set_left_led(left_button_position && right_button_position);
    set_right_led(left_switch_position && right_switch_position);
    ...
\end{lstlisting}


\subsection{Determine the Base Addresses of Certain I/O Register Banks} \label{subsec:baseAddresses}

In Sections~\ref{subsec:simpleIO}--\ref{subsec:ScannedInput}, you will use an array of \lstinline{cowpi_ioport_t} structures to access the memory-mapped I/O registers for the \developmentboard's external pins.
In Section~\ref{subsec:DisplayModule}, you will use a single \lstinline{cowpi_i2c_t} structure to transmit data to the display module.

Read the section of the Cow~Pi datasheet that \href{https://cow-pi.readthedocs.io/en/latest/microcontroller.html#structure-for-memory-mapped-input-output}{discusses the \lstinline{cowpi_ioport_t} structure}.
(During the guided discussion in your lab period, the TAs may direct you to particular parts of that section of the datasheet,
but be sure to go back and read the full section covering \href{https://cow-pi.readthedocs.io/en/latest/microcontroller.html#external-pins-input-output}{input/output pins} later.)
After you have done so,
\begin{itemize}
    \item Uncomment the \lstinline{ioports = ...} line of the starter code's \function{initialize_io()} function, and
    \item Assign the appropriate address to the \lstinline{ioports} pointer on that line.
\end{itemize}

\begin{lstlisting}[numberstyle=\color{gray}, numbers=left, firstnumber=48]
    void initialize_io(void) {
        // Uncomment these lines and set the correct addresses during lab time
        // ioports = (cowpi_ioport_t *)(0x00);
        // i2c = (cowpi_i2c_t *)(0x00);

        // Uncomment this line when you are ready to implement the `send_halfbyte()` function
        // cowpi_hd44780_send_halfbyte = send_halfbyte;
    }
\end{lstlisting}

You can now use the \lstinline{ioports} pointer as an array of \lstinline{cowpi_ioport_t} structures, which you can index using the \lstinline{D0_D7}, \lstinline{D8_D13}, and \lstinline{D14_D19} named constants.

Read the section of the Cow~Pi datasheet that \href{https://cow-pi.readthedocs.io/en/latest/microcontroller.html#atmega328ptwistruct}{discusses the \lstinline{cowpi_i2c_t} structure}.
(You can wait to read the rest of the datasheet's Section that discusses the Inter-Integrated Circuit Protocol until you're working on the assignment's Section~\ref{subsec:DisplayModule}.)
After you have done so,
\begin{itemize}
    \item Uncomment the \lstinline{i2c = ...} line of the starter code's \function{initialize_io()} function, and
    \item Assign the appropriate address to the \lstinline{i2c} pointer on that line.
\end{itemize}

When you get to the assignment's Section~\ref{subsec:DisplayModule}, you will be able to use the \lstinline{i2c} pointer to access the memory-mapped registers to communicate with the display module.

\textbf{Do \textit{not} uncomment the assignment to \function{cowpi_hd44780_send_halfbyte} until you get to the assignment's Section~\ref{subsec:DisplayModule}!}


\subsection{Detect a Change on the Numeric Keypad} \label{subsec:detectKeyAction}

The next part of the group activity is detecting when someone presses (or releases) a key on the numeric keypad.
The \function{key_movement_detected()} function in the starter code is:

\begin{lstlisting}[numberstyle=\color{gray}, numbers=left, firstnumber=69, escapechar=`]
    static bool key_movement_detected(void) {
        static bool key_was_pressed = false;
        // this is four distinct reads to figure out if any of the column pins has a 0 on it -- can you do it in one read?
        bool key_is_pressed = !(digitalRead(14) && digitalRead(15) && digitalRead(16) && digitalRead(17));  `\label{code:readColumns}`
        bool debounced_key_is_pressed = cowpi_debounce_byte(key_is_pressed, KEYPAD);
        bool change_detected = (debounced_key_is_pressed != key_was_pressed);
        key_was_pressed = debounced_key_is_pressed;
        return change_detected;
    }
\end{lstlisting}

Line~\ref{code:readColumns} uses the Arduino function \function{digitalRead()} to read the values on the four pins attached to the keypad's columns.
If any of those pins has a 0 on it, then a key has been pressed (see Section~\ref{subsec:ScannedInput}).
The next line debounces that reading (see Section~\ref{subsec:debouncing}).

The remainder of the code compares the reading to the previous reading.
If no key was pressed and there is still no key being pressed, the function returns \lstinline{false}.
Similarly, if a key was pressed and is still being pressed, the function returns \lstinline{false}.
On the other hand, if no key was pressed and there is now a key being pressed, or if a key was being pressed and there is now no key being pressed, then the function returns \lstinline{true}.

One problem is that, as noted in Section~\ref{subsec:constraints}, the \function{digitalRead()} function is not one of the Arduino functions that can be used in this assignment.
The other problem is that the \function{digitalRead()} function can read only one pin at a time, and so there are four distinct reads.
Now that you have the base address for the \lstinline{ioports} array, you can use memory-mapped I/O to replace the call to \function{digitalRead()},
and you can do so in a manner that obtains the values on all four pins at the same time.

If you are currently in your lab period, then work with the other students in your lab period to determine the answers to these questions.
If you are not attending your lab period, then work individually to determine the answers to these questions.
Using the datasheet's \href{https://cow-pi.readthedocs.io/en/latest/microcontroller.html#id9}{Table~8}:
\begin{itemize}
    \item What should you use to index the \lstinline{ioports} array to read from pins~14--17, the pins connected to the keypad's columns?
    \item What field in that element should you use?
    \item What bitmask should you use?
    \item What should line~\ref{code:readColumns} look like?
\end{itemize}


\subsubsection*{Test your code}

You can confirm that you correctly detected a change on the keypad by running the test code.
Place the \textbf{left switch} in the \textit{right} position and upload your code.
(Or, if you have already uploaded your code, place the \textbf{left switch} in the \textit{right} position and press the RESET button in the middle of the \developmentboard.)
The test code will output on both the Serial Monitor and on the display module.
The output will include the key that was pressed (if any), a hyphen if no key is being pressed, or a question mark if \function{get_keypress()} returns an invalid value.


\subsection{Controlling an LED} \label{subsec:controlLED}

The final part of the group activity is illuminating and deluminating the built-in LED\@.
Locate the \function{set_left_led()} function.
In the starter code, this function is implemented by calling functions in the CowPi library.

\begin{itemize}
    \item Modify this function to replace the calls to CowPi library functions with code that uses the \lstinline{ioports} array to turn the LEDs on or off.
        \begin{itemize}
            \item What should you use to index the \lstinline{ioports} array to access the pin connected to that LED?
                What field in that element should you use?
        \end{itemize}
    \item Use the read-modify-write pattern to do so, so that you do not change any pins that you do not intend to change.\footnote{Even on ``input'' pins, changing the ``output'' register has an effect.}
    \begin{itemize}
        \item When turning the LED on, what bitmask should you use?
            What operation should you use?
        \item When turning the LED off, what bitmask should you use?
            What operation should you use?
    \end{itemize}
    \item No debouncing code is necessary because these functions do not read from mechanical switches.
\end{itemize}

\paragraph{Test your code}

Place the left switch in the right position and upload the program to your \developmentboard.
If both buttons are pressed then the left LED will illuminate.


\vspace{1cm}

You are now ready to complete the remainder of this assignment on your own.
Reminders:
\begin{itemize}
    \item You can complete Sections~\ref{sec:MemMapIO} and \ref{sec:SimpleSystem} in either order.
    \item \collaborationrules
\end{itemize}
