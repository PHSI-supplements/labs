You are not required to have your assignment checked-off by a TA or the professor.
If you do not do so, then we will perform a functional check ourselves.
We are, however, offering a small bonus if you complete your assignment early and get it checked-off by a TA or the professor during office hours.

\textcolor{red}{\textbf{\textit{You can also use this checklist yourself to make sure you've implemented everything correctly.}}}

\begin{enumerate}
%\precheckoffitem{Establish that the code you are demonstrating is the code
%    you submitted to to \filesubmission.}
%    \begin{itemize}
%    \item If you are getting checked-off during lab time, show the TA that the file was submitted before it was due.
%    \item Download the file into your number\_builder directory.
%        If necessary, rename it to \textit{number\_builder.ino}.
%    \end{itemize}


    \checkoffitem{Show the TA your \lstinline{keys} nested array and your \function{initialize_io()} function.} \\
        \textit{+1 The \lstinline{key} nested array is correctly populated} \\
        \textit{+\textonehalf\ The correct address is assigned to
            \ifdefstring{\processor}{ATmega328P}{\lstinline{ioports}}{}
            \ifdefstring{\processor}{RP2040}{\lstinline{ioport}}{}
            \\
        \textit{+\textonehalf\ The correct address is assigned to \lstinline{timer}} }
    \checkoffitem{Show the TA your implementations of the Input/Output functions.}
    \precheckoffitem{TA make note of which functions are implemented using memory-mapped I/O and which are not.
        Do not award credit for functions that do not use memory-mapped I/O.}

    \item [] \textbf{Memory-Mapped Input/Output}
    \precheckoffitem{Place both switches in the right position and upload your code to your Cow~Pi.}
    \checkoffitem{Demonstrate that the display shows that a key is pressed whenever any key is pressed, and that the display shows that no key is pressed whenever no keys are pressed. \\
        \textit{+2 The \function{key_is_pressed} function is correctly implemented} }
    \checkoffitem{Demonstrate that, as you press those keys, the display shows time advancing at the rate of one second per second. \\
        \textit{+2 The \function{get_microseconds} function is correctly implemented} } \vspace{5mm}

    \precheckoffitem{Place the right switch in the left position and RESET your Cow~Pi.}
    \checkoffitem{Demonstrate that both pushbuttons' positions are correctly detected. \\
        \textit{+1 The \function{xx_button_is_pressed()} functions are correctly implemented}}
    \checkoffitem{Demonstrate that both switches' positions are correctly detected. \\
        \textit{+1 The \function{xx_switch_is_in_right_position()} functions are correctly implemented}}
    \checkoffitem{Demonstrate that when and only when both pushbuttons are pressed, the left LED (the built-in LED) illuminates.
        Demonstrate that when and only when both switches are in the right position, the right LED (the external LED) illuminates. \\
        \textit{+2 The \function{set_left_led()}  function is correctly implemented} \\
        \textit{+1 The \function{set_right_led()} function is correctly implemented}}
    \checkoffitem{Demonstrate that each of the keys on the keypad is correctly detected and that the absence of a keypress is also reported correctly. \\
        \textit{+8 The \function{get_keypress()} function is correctly implemented}}

    \item [] \textbf{Number Builder}
    \precheckoffitem{Place the left switch in the left position and RESET button your Cow~Pi.}
    \checkoffitem{Place both switches in the left position (left justified mode, decimal number base).}
    \checkoffitem{Press 2, then 3. The right LED illuminates for \textonehalf\ second with each keypress. \\
        Left-justified on the display module is shown: \\
        \display{
            \colorbox{LightGreen}{2\phantom{xxxxxxxxxxxxxxx}} \vspace{-1mm}\\
            \colorbox{LightGreen}{0x2\phantom{xxxxxxxxxxxxx}}
        } \\
    and then: \\
    \display{
        \colorbox{LightGreen}{23\phantom{xxxxxxxxxxxxxx}} \vspace{-1mm}\\
        \colorbox{LightGreen}{0x17\phantom{xxxxxxxxxxxx}}
    } \\
        \textit{+1 Right LED illumination occurs as specified} \\
        \textit{+\textonehalf\ Left justified mode is implemented correctly} \\
        \textit{+1 The first digit is displayed in the correct position in this mode, making a blank display no longer blank.} \\
        \textit{+1 Subsequent digits update the number correctly in this mode}}
    \checkoffitem{Press B. The display is unchanged. \\
        \textit{+\textonehalf\ Decimal number base (positive values) is implemented correctly}}
    \checkoffitem{Press the right pushbutton. The display clears, and the left LED lluminates for \textonehalf\ second: \\
        \display{
            \colorbox{LightGreen}{\phantom{xxxxxxxxxxxxxxxx}} \vspace{-1mm}\\
            \colorbox{LightGreen}{\phantom{xxxxxxxxxxxxxxxx}}
        } \\
        \textit{+1 Left LED illumination occurs as specified} \\
        \textit{+1 The right pushbutton clears the display}}
    \checkoffitem{Press 0.The right LED illuminates for \textonehalf\ second with each keypress. \\ Left-justified on the display module is shown: \\
        \display{
            \colorbox{LightGreen}{0\phantom{xxxxxxxxxxxxxxx}} \vspace{-1mm}\\
            \colorbox{LightGreen}{0\phantom{xxxxxxxxxxxxxxx}}
        } \\
        \textit{+1 The right pushbutton sets the value to 0}}
    \checkoffitem{Press the right pushbutton. The display clears, and the left LED lluminates for \textonehalf\ second.}
    \checkoffitem{Place both switches in the right position (right justified mode, hexadecimal number base).}
    \checkoffitem{Press 2, then 3. The right LED illuminates for \textonehalf\ second with each keypress. \\ Right-justified on the display module is shown: \\
        \display{
            \colorbox{LightGreen}{\phantom{xxxxxxxxxxxxxxx}2} \vspace{-1mm}\\
            \colorbox{LightGreen}{\phantom{xxxxxxxxxxxxx}0x2}
        } \\
        and then: \\
        \display{
            \colorbox{LightGreen}{\phantom{xxxxxxxxxxxxxx}35} \vspace{-1mm}\\
            \colorbox{LightGreen}{\phantom{xxxxxxxxxxxx}0x23}
        } \\
        \textit{+\textonehalf\ Right justified mode is implemented correctly.} \\
        \textit{+1 The first digit is displayed in the correct position in this mode, making a blank display no longer blank.} \\
        \textit{+1 Subsequent digits update the number correctly in this mode.}}
    \checkoffitem{Press B. \\
        \display{
            \colorbox{LightGreen}{\phantom{xxxxxxxxxxxxx}571} \vspace{-1mm}\\
            \colorbox{LightGreen}{\phantom{xxxxxxxxxxx}0x23B}
        } \\
        \textit{+\textonehalf\ Hexadecimal number base (positive values) is implemented correctly.}}
    \checkoffitem{Press the left pushbutton. The left LED illuminates for \textonehalf\ second. The display shows: \\
        \display{
            \colorbox{LightGreen}{\phantom{xxxxxxxxxxxx}-571} \vspace{-1mm}\\
            \colorbox{LightGreen}{\phantom{xxxxxx}0xFFFFFDC5}
        } \\
        \textit{+\textonehalf\ The left pushbutton negates positive values in hexadecimal mode.}}
    \precheckoffitem{If hexadecimal negation does not work, then clear the number with the right pushbutton and then attempt to directly enter the negative value 0xFFFFFDC5.
        If a negative hex value cannot be produced through negation, and if a negative hex value cannot be directly entered,
        then there is no way to test how the number builder handles negative hex values, which will result in a further loss of points.}
    \checkoffitem{Press A. \\
        \display{
            \colorbox{LightGreen}{\phantom{xxxxxxxxxxx}-9126} \vspace{-1mm}\\
            \colorbox{LightGreen}{\phantom{xxxxx}0xFFFFFDC5A}
        } \\
        \textit{+\textonehalf\ Hexadecimal number base (negative values) is implemented correctly.}}
    \checkoffitem{Press the left pushbutton. \label{step:negateHex} \\
        \display{
            \colorbox{LightGreen}{\phantom{xxxxxxxxxxxx}9126} \vspace{-1mm}\\
            \colorbox{LightGreen}{\phantom{xxxxxxxxxx}0x23A6}
        } \\
        \textit{+\textonehalf\ The left pushbutton negates negative values in hexadecimal mode.}}
    \precheckoffitem{If hexadecimal negation does not work, then clear the number with the right pushbutton and then enter the positive value 0x23A6.}
    \checkoffitem{Press 7, 8, 9, C. \\
        \display{
            \colorbox{LightGreen}{\phantom{xxxxxxx}598112412} \vspace{-1mm}\\
            \colorbox{LightGreen}{\phantom{xxxxxx}0x23A6789C}
        }}
    \checkoffitem{Press D. The system displays: \\
        \display{
            \colorbox{LightGreen}{\phantom{xxxxxx}TOO\phantom{xxxxxx}} \vspace{-1mm}\\
            \colorbox{LightGreen}{\phantom{xxxxxx}BIG!\phantom{xxxxxx}}
        } \\
        \textit{+\textonequarter\ Detects too-big positive hexadecimal numbers.} \\
        \textit{+\textonehalf\ Displays the correct ``too big'' error message.}}
    \checkoffitem{Press the right pushbutton to clear the display and reset the value to 0.}
    \checkoffitem{Generate the value 0x98765432 by one of two means:}
        \begin{multicols}{2}
            Indirectly:
            First enter the value 0x6789ABCE \\
                \display{
                    \colorbox{LightGreen}{\phantom{xxxxxx}1737075662} \vspace{-1mm}\\
                    \colorbox{LightGreen}{\phantom{xxxxxx}0x6789ABCE}
                } \\
            Then negate it with the left pushbutton \\
                \display{
                    \colorbox{LightGreen}{\phantom{xxxxx}-1737075662} \vspace{-1mm}\\
                    \colorbox{LightGreen}{\phantom{xxxxxx}0x98765432}
                }

        \columnbreak

            Directly: enter the value 0x98765432 \\
            \display{
                \colorbox{LightGreen}{\phantom{xxxxx}-1737075662} \vspace{-1mm}\\
                \colorbox{LightGreen}{\phantom{xxxxxx}0x98765432}
            }
        \end{multicols}
    \checkoffitem{Press 1. \\
        \display{
            \colorbox{LightGreen}{\phantom{xxxxxx}TOO\phantom{xxxxxx}} \vspace{-1mm}\\
            \colorbox{LightGreen}{\phantom{xxxxxx}BIG!\phantom{xxxxxx}}
        } \\
        \textit{+\textonequarter\ Detects too-big negative hexadecimal numbers.} \\
        \textit{+\textonequarter\ No false ``too big'' detection of hexadecimal numbers.}}
    \checkoffitem{Press the right pushbutton to clear the display and reset the value to 0.}
    \checkoffitem{Place the right swtich in the left position (decimal number base).}
    \checkoffitem{Enter 1, 2, 3, 4, 5. \\
        \display{
            \colorbox{LightGreen}{\phantom{xxxxxxxxxxx}12345} \vspace{-1mm}\\
            \colorbox{LightGreen}{\phantom{xxxxxxxxxx}0x3039}
        }}
    \checkoffitem{Press the left pushbutton. \\
        \display{
            \colorbox{LightGreen}{\phantom{xxxxxxxxxx}-12345} \vspace{-1mm}\\
            \colorbox{LightGreen}{\phantom{xxxxxx}0xFFFFCFC7}
        } \\
        \textit{+\textonehalf\ The left pushbutton negates positive values in decimal mode.}}
    \checkoffitem{Enter 6, 7, 8, 9, 0. \\
        \display{
            \colorbox{LightGreen}{\phantom{xxxxx}-1234567890} \vspace{-1mm}\\
            \colorbox{LightGreen}{\phantom{xxxxxx}0xB669FD2E}
        } \\
        \textit{+\textonehalf\ Decimal number base (negative values) is implemented correctly.}}
    \checkoffitem{Press the left pushbutton. \\
        \display{
            \colorbox{LightGreen}{\phantom{xxxxxx}1234567890} \vspace{-1mm}\\
            \colorbox{LightGreen}{\phantom{xxxxxx}0x499602D2}
        } \\
        \textit{+\textonehalf\ The left pushbutton negates negative values in decimal mode.}}
    \checkoffitem{Press Press 1. \\
        \display{
            \colorbox{LightGreen}{\phantom{xxxxxx}TOO\phantom{xxxxxx}} \vspace{-1mm}\\
            \colorbox{LightGreen}{\phantom{xxxxxx}BIG!\phantom{xxxxxx}}
        } \\
        \textit{+\textonequarter\ Detects too-big positive decimal numbers.}}
    \checkoffitem{Clear the number with the right pushbutten and enter 1, 2, 3, 4, 5, 6, 7, 8, 9, 0. \\
        \display{
            \colorbox{LightGreen}{\phantom{xxxxxx}1234567890} \vspace{-1mm}\\
            \colorbox{LightGreen}{\phantom{xxxxxx}0x499602D2}
        }}
    \checkoffitem{Press the left pushbutton. \\
        \display{
            \colorbox{LightGreen}{\phantom{xxxxx}-1234567890} \vspace{-1mm}\\
            \colorbox{LightGreen}{\phantom{xxxxxx}0xB669FD2E}
        }}
    \checkoffitem{Press 0. \\
        \display{
            \colorbox{LightGreen}{\phantom{xxxxxx}TOO\phantom{xxxxxx}} \vspace{-1mm}\\
            \colorbox{LightGreen}{\phantom{xxxxxx}BIG!\phantom{xxxxxx}}
        } \\
        \textit{+\textonequarter\ Detects too-big negative decimal numbers.} \\
        \textit{+\textonequarter\ No false ``too big'' detection of decimal numbers.}}
    \checkoffitem{Clear the number with the right pushbutton, and \textbf{rapidly} (strictly less than \textonehalf~second between presses) enter 1, 2, 3, left pushbutton, left pushbutton, left pushbutton, 4, 5, 6. \\
        The display updates as fast as you can press the keys and buttons.
        The right LED stays illuminated between keypresses until \textonehalf~second after the 3 is pressed, and again \textonehalf~second after the 6 is pressed.
        The left LED stays illuminated between button presses until \textonehalf~second after the third button press.
        At the end, the display shows: \\
        \display{
            \colorbox{LightGreen}{\phantom{xxxxxxxxx}-123456} \vspace{-1mm}\\
            \colorbox{LightGreen}{\phantom{xxxxxx}0xFFFE1DC0}
        } \\
        \textit{+1 The system is responsive and does not block after an input.}}
    \item [] \textbf{Overall} \\
    \textit{+\textonehalf\ A single button press is treated as a single press.} \\
    \textit{+\textonehalf\ A single key press is treated as a single press.}
\end{enumerate}

This concludes the demonstration of your system's functionality.
The TAs will later examine your code for violations of the assignment's constraints.
If your code looks like it is tailored for this checklist, the TAs may re-grade using a different checklist.
