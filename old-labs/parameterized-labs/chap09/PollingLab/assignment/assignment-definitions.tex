%%
%% labs/common/assignment.tex
%% (c) 2021-23 Christopher A. Bohn
%%
%% Licensed under the Apache License, Version 2.0 (the "License");
%% you may not use this file except in compliance with the License.
%% You may obtain a copy of the License at
%%     http://www.apache.org/licenses/LICENSE-2.0
%% Unless required by applicable law or agreed to in writing, software
%% distributed under the License is distributed on an "AS IS" BASIS,
%% WITHOUT WARRANTIES OR CONDITIONS OF ANY KIND, either express or implied.
%% See the License for the specific language governing permissions and
%% limitations under the License.
%%

\usepackage{addfont}
\usepackage{amsmath}
\usepackage{amssymb}
\usepackage{animate}
\usepackage{bold-extra}
\usepackage{cancel}
\usepackage{caption}
\usepackage{ccicons}
\usepackage{enumitem}
\usepackage{etoolbox}
\usepackage{fancyhdr}
\usepackage{fullpage}
\usepackage{graphicx}
\usepackage{hyperref}
\usepackage[utf8]{inputenc}
\usepackage[procnames]{listings}
%\usepackage{media9}
\usepackage{multicol}
\usepackage{subfig}
\usepackage{textcomp}
\usepackage{tikz}
\usepackage[americanresistor]{circuitikz}
%\usepackage{tikz-uml}
\usetikzlibrary{automata,positioning,arrows}
\usepackage[normalem]{ulem}
\usepackage{wrapfig}
\usepackage{xcolor}
%\usepackage[dvipsnames]{xcolor}
\definecolor{LightGreen}{rgb}{0.88,1,0.88}

\lstset{language=C, tabsize=4, upquote=true, basicstyle=\ttfamily}
\newcommand{\function}[1]{\textbf{\lstinline{#1}}}
\setlength{\headsep}{0.7cm}
\hypersetup{colorlinks=true}

%% CREDIT FOR MARKERLESSFOOTNOTE WHERE CREDIT IS DUE: https://tex.stackexchange.com/questions/30720/footnote-without-a-marker?answertab=scoredesc#tab-top
\newcommand\markerlessfootnote[1]{%
    \begingroup
    \renewcommand\thefootnote{}\footnote{#1}%
    \addtocounter{footnote}{-1}%
    \endgroup
}

\newcommand{\pagelayout}{
    \pagestyle{fancy}
    \fancyhf{}
    \lhead{\coursenumber}
    \chead{\ Lab \labnumber: \labname}
    \rhead{\courseterm}
    \cfoot{\shortlabname-\thepage}
}

\newcommand{\labidentifier}{
    \title{\ Lab \labnumber}
    \author{\labname}
    \date{Due: \duedate}
    \maketitle

    \textit{\collaborationrules}
    \markerlessfootnote{\tiny{\ifdefstring{\instructorname}{Christopher A. Bohn}{Assignment}{Original assignment} and starter code \copyright\ Christopher A. Bohn, licensed under the Creative Commons Attribution 4.0 International License~\ccby\ and under the Apache License Version 2.0, respectively.}}
    \ifdefstring{\instructorname}{Christopher A. Bohn}{}{\markerlessfootnote{\tiny{Configured for \coursenumber\ at \institutename\ by \instructorname.}}}

    \begin{description}
        \item[Obtaining the starter code] \filesource
        \item[Runtime environment] We will grade this assignment by compiling and running it on \runtimeenvironment;
            you should compile and test your code on \runtimeenvironment\ before turning it in.
        \item[Submitting your work] \filesubmission
    \end{description}
}

% display module fonts for hardware kit
% use with the Capital baseball "matrix printer" font collection (https://www.ctan.org/tex-archive/fonts/capbas/)
% Identifying the specific font in the assignment sheet is deprecated
%   -- instead, set the `usedisplayfont` boolean and the `displaymodule` variable in semester.tex,
%      and \display{...} in the assignment sheet

\addfont{OT1}{d7seg}{\dviiseg}
\addfont{OT1}{deseg}{\deseg}
\addfont{OT1}{necker}{\necker}

\providebool{usedisplayfont}

\newcommand{\display}[1]{
    \ifboolexpe{bool{usedisplayfont}}{
        \ifdefstring{\displaymodule}{MAX7219digits}{{\dviiseg #1}}{}
        \ifdefstring{\displaymodule}{MAX7219matrix}{{\deseg #1}}{}
        \ifdefstring{\displaymodule}{LCD1602}{{\necker #1}}{}
        % We don't yet have a Cow Pi configuration with 14-segment displays, so no \deseg yet
    }{
        \texttt{#1}
    }
}

\newcommand{\rubricitem}[2]{\item[\underline{\hspace{1cm}} +#1] #2}
\newcommand{\bonusitem}[2]{\item[\underline{\hspace{1cm}} Bonus +#1] #2}
\newcommand{\penaltyitem}[2]{\item[\underline{\hspace{1cm}} -#1] #2}
\newcommand{\checkoffitem}[1]{\item (\phantom{xxx}) #1}
\newcommand{\precheckoffitem}[1]{\item [] (\phantom{xxx}) #1}

\newcommand{\institutename}{University of Nebraska-Lincoln}
\newcommand{\instructorname}{Christopher A. Bohn}
\newcommand{\coursenumber}{CSCE~231}
\newcommand{\cstwo}{CSCE~156, RAIK~184H, or SOFT~161}
\newcommand{\courseterm}{Spring 2025}
\newcommand{\labnumber}{10}
\newcommand{\labname}{Using Polling with Memory-Mapped Input/Output}
\newcommand{\shortlabname}{PollingLab}
\newcommand{\duedate}{Week of April 14, before the start of your lab section}
\newcommand{\pathone}{chap09/io/memory-mapped}
\newcommand{\pathtwo}{chap09/system-descriptions/number-builder}
\newcommand{\filesource}{Download \startercode\ from Canvas, or copy \startercode\ from {\footnotesize$\sim$}cbohn2/csce231 on \textit{nuros.unl.edu}}
\newcommand{\filesubmission}{When you have completed this assignment, submit \requiredfiles\ to the assignment in Canvas.}
\newcommand{\runtimeenvironment}{a Cow Pi development board with a \processor\ microcontroller}
\newcommand{\startercode}{pollinglab.zip or pollinglab.tar}
\newcommand{\requiredfiles}{\textit{io\_functions.c} and \textit{number\_builder.c}}
\newcommand{\buildsystem}{platformio}
\newcommand{\processor}{RP2040}
\newcommand{\lowercaseprocessor}{rp2040}
\newcommand{\collaborationrules}{During your scheduled lab time, you may, \textbf{but are not required to}, form a partner group of 2 students.
    When necessary, there may be a group of 3 students.
    During your scheduled lab time, and until the end of your lab day, you may discuss problem decomposition and solution design with your lab partner.
    After your scheduled lab day, you may discuss concepts and syntax with other students, but you may discuss solutions only with the professor and the TAs.
    Sharing code with or copying code from another student or the internet is prohibited.
}
\newcommand{\policyforcodethatdoesnotcompile}{\subsection*{No Credit for Uncompilable Code}
    If the TA cannot create an executable from your code, then your code will be assumed to have no functionality.\footnote{
        At the TA's discretion, if they can make your code compile with \textit{one} edit (such as introducing a missing semicolon) then they may do so and then assess a 10\% penalty on the resulting score.
        The TA is under no obligation to do so, and you should not rely on the TA's willingness to edit your code for grading.
        If there are multiple options for a single edit that would make your code compile, there is no guarantee that the TA will select the option that would maximize your score.
    }
    Before turning in your code, be sure to compile and test your code on \runtimeenvironment\ with the original driver code, the original header file(s), and the original \textit{Makefile}.}
\newcommand{\latepolicy}{\subsection*{Late Submissions}
    This assignment is due before the start of your lab section.
     After you have exhausted your grace days, we will accept late turn-ins up to one hour late, assessing a 10\% penalty on these late submissions.
    After you have exhausted your grace days, any work turned in more than one hour late will not be graded.}
\newcommand{\softwareengineeringfrontmatter}{\section*{No Spaghetti Code Allowed}
        In the interest of keeping your code readable, you may \textit{not} use
        any \lstinline{goto} statements, nor may you use any
        \lstinline{continue} statements, nor may you use any \lstinline{break}
        statements to exit from a loop, nor may you have any functions
        \lstinline{return} from within a loop.}
\newcommand{\softwareengineeringpenalties}{\penaltyitem{1}{for each \lstinline{goto} statement,
            \lstinline{continue} statement, \lstinline{break} statement used to
            exit from a loop, or \lstinline{return} statement that occurs within
            a loop.}}
\newcommand{\scenariointroduction}{    Archie walks up to you, along with Herb Bee from Eclectic Electronics.
    Herb is holding a tangled mess of electronics.
    Archie explains, ``Herb here has developed a prototype of a device that he thinks will be useful for our physical security needs, as well as a few other applications around here. He calls it the \textit{Cow Pi}.''

    You look at the device in Herb's hands and see the Raspberry Pi Pico central to the circuit.
    ``I suppose the `\textit{-Pi}' suffix is because it uses a Raspberry Pi Pico?''
+        
    Herb replies, ``Um, yeah, sure.
 We considered using an Arduino Nano, but \textit{Cowduino} isn't very punny, is it?''

    Archie chimes in, ``Maybe with the right emphasis: \textit{Cow-DOO-ino}.''

    ``That's kind of subtle, don't you think? How will people know to put the emPHAsis on that sylLAble?''
    
    ``I think we're getting off topic here,'' you point out.
    ``How can I help?''

    ``Oh, right,'' Herb says, ``We'd like you to kick its proverbial tires.
    Let's start off with something simple, like a number builder tool.''}
\newcommand{\scenariowrapup}{Herb looks over your work.
    ``Hmm, yes. I think this is coming along nicely.
    Let's run a few more tests.''

    Archie storms into the room.
    ``We have \textit{got} to do something about security!
    How's that doodad coming along? 
    Because there's now a half-man/half-fly in the labs going on-and-on about Chaos Theory and how if we just give him a MacBook and a spaceship then he'll be able to get the Lord of Thunder to travel across the 8th Dimension.
    Is that thing just about ready?''

    Herb shakes his head, ``No, not quite yet. It should be ready in about a week.''}
