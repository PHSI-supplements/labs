The links in this appendix are also provided in the regular sections of the assignment sheet.
However, as a convenience, this appendix summarizes the links to the Cow~Pi datasheet that you may find useful for this assignment.

\hspace{1cm}

\begin{footnotesize}\begin{tabular}[h]{p{3cm}ll}
    Datasheet Section                                                           & URL (https://cow-pi.readthedocs.io/en/latest/\dots)                                                                                                                                                                           & Useful when working on\dots                                                               \\ \hline\hline
    Matrix Keypad                                                               & \href{https://cow-pi.readthedocs.io/en/latest/hardware/inputs.html#matrix-keypad}{hardware/inputs.html\#matrix-keypad}                                                                                                        & Sections~\ref{subsec:populatekeypad} \& \ref{subsec:ScannedInput}                         \\ \hline
%    \raggedright{HD44780-driven LCD Character Display}                          & \href{https://cow-pi.readthedocs.io/en/latest/hardware/outputs.html#hd44780-driven-lcd-character-display}{hardware/outputs.html\#hd44780-driven-lcd-character-display}                                                        & Section~\ref{subsec:baseAddresses} \& \ref{subsec:DisplayModule}                          \\ \hline
    \raggedright{Structure for data pin registers}                              & \href{https://cow-pi.readthedocs.io/en/latest/CowPi_\lowercaseprocessor/io_registers.html#structure-for-memory-mapped-input-output}{CowPi\_\lowercaseprocessor/io\_registers.html\#structure-for-memory-mapped-input-output}  & Sections~\ref{subsec:baseAddresses}--\ref{subsec:controlLED} \& \ref{subsec:simpleIO}      \\ \hline
    \raggedright{Structure for timer registers}                                 & \href{https://cow-pi.readthedocs.io/en/latest/CowPi_\lowercaseprocessor/io_registers.html#timers}{CowPi\_\lowercaseprocessor/io\_registers.html\#timers}                                                                      & Sections~\ref{subsec:baseAddresses} \& \ref{subsec:timer}                                 \\ \hline
    Table \mappingtablenumber                                                   & \href{https://cow-pi.readthedocs.io/en/latest/CowPi_\lowercaseprocessor/io_registers.html#\mappingtablelink}{CowPi\_\lowercaseprocessor/io\_registers.html\#\mappingtablelink}                                                & Sections~\ref{subsec:detectKeyAction}--\ref{subsec:controlLED} \& \ref{subsec:simpleIO}    \\ \hline
%    \raggedright{Structure for Memory-Mapped Input/Output (for I$^2$C)}         & \href{https://cow-pi.readthedocs.io/en/latest/CowPi_\lowercaseprocessor/io_registers.html#atmega328ptwistruct}{microcontroller.html\#atmega328ptwistruct}                                                                     & Section~\ref{subsec:DisplayModule}                                                        \\ \hline
%    \raggedright{I$^2$C Controller Transmitter Sequence}                        & \href{https://cow-pi.readthedocs.io/en/latest/CowPi_\lowercaseprocessor/io_registers.html#controller-transmitter-sequence}{microcontroller.html\#controller-transmitter-sequence}                                             & Section~\ref{subsec:DisplayModule}                                                        \\ \hline
%    \raggedright{Custom Transmission Function (for display module)}             & \href{https://cow-pi.readthedocs.io/en/latest/CowPi_\lowercaseprocessor/io_registers.html#custom-transmission-function}{CowPi\_stdio/lcd\_character.html\#custom-transmission-function}                                       & Section~\ref{subsec:DisplayModule}                                                        \\ \hline
%    \raggedright{ASCII Control Characters (for display module)}                 & \href{https://cow-pi.readthedocs.io/en/latest/CowPi_\lowercaseprocessor/io_registers.html#ascii-control-characters}{CowPi\_stdio/lcd\_character.html\#ascii-control-characters}                                               & Section~\ref{subsec:numberBuilderOutput}                                                  \\ \hline
\end{tabular}\end{footnotesize}