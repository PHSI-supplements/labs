Read the Datasheet section that discusses \href{https://cow-pi.readthedocs.io/en/latest/CowPi_\lowercaseprocessor/io_registers.html#interrupts}{Interrupts}, and review the \href{https://cow-pi.readthedocs.io/en/latest/CowPi_\lowercaseprocessor/io_registers.html#timers}{Timers} section.

%You can skip over the section that covers registering external interrupt handlers using \function{attachInterrupt()}, as we will use pin change interrupts.
%You can skip over the section that covers the timers ``Normal'' mode, as we will use ``Clear Timer on Compare'' (CTC) mode.
%You can skip over the section that covers configuring TIMER0, as we will use only TIMER1 and TIMER2.

You can skip over the section that discusses the use of \function{attachInterrupt()}, as we will use \function{register_pin_ISR()}.
The discussion of \function{cowpi_register_pin_ISR()} is relevant, as \function{register_pin_ISR()} in \textit{interrupt\_support.cpp} has the same signature and exhibits the same behavior.

Review the header comments in \textit{interrupt\_support.h}.
\ifdefstring{\processor}{ATmega328P}{
    We will use the \function{configure_timer()} function to configure two timers to fire periodic interrupts,
    and we will use the \function{register_timer_ISR()} function to register interrupt handlers for those interupts.
    \textit{You will} not \textit{need to explicitly set the timer registers' bits as described in the datasheet};
    however, familiarizing yourself with the timers' limitations might be helpful.
}{}
\ifdefstring{\processor}{RP2040}{
    We will use the \function{register_timer_ISR()} function to configure two timers to fire periodic interrupts
    and to register interrupt handlers for those interupts.
    \textit{You will} not \textit{need to explicitly set the timer registers' bits,
        nor will you need to directly invoke any MBED~OS functions.}
}{}
