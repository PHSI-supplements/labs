%%
%% labs/common/assignment.tex
%% (c) 2021-23 Christopher A. Bohn
%%
%% Licensed under the Apache License, Version 2.0 (the "License");
%% you may not use this file except in compliance with the License.
%% You may obtain a copy of the License at
%%     http://www.apache.org/licenses/LICENSE-2.0
%% Unless required by applicable law or agreed to in writing, software
%% distributed under the License is distributed on an "AS IS" BASIS,
%% WITHOUT WARRANTIES OR CONDITIONS OF ANY KIND, either express or implied.
%% See the License for the specific language governing permissions and
%% limitations under the License.
%%

\usepackage{addfont}
\usepackage{amsmath}
\usepackage{amssymb}
\usepackage{animate}
\usepackage{bold-extra}
\usepackage{cancel}
\usepackage{caption}
\usepackage{ccicons}
\usepackage{enumitem}
\usepackage{etoolbox}
\usepackage{fancyhdr}
\usepackage{fullpage}
\usepackage{graphicx}
\usepackage{hyperref}
\usepackage[utf8]{inputenc}
\usepackage[procnames]{listings}
%\usepackage{media9}
\usepackage{multicol}
\usepackage{subfig}
\usepackage{textcomp}
\usepackage{tikz}
\usepackage[americanresistor]{circuitikz}
%\usepackage{tikz-uml}
\usetikzlibrary{automata,positioning,arrows}
\usepackage[normalem]{ulem}
\usepackage{wrapfig}
\usepackage{xcolor}
%\usepackage[dvipsnames]{xcolor}
\definecolor{LightGreen}{rgb}{0.88,1,0.88}

\lstset{language=C, tabsize=4, upquote=true, basicstyle=\ttfamily}
\newcommand{\function}[1]{\textbf{\lstinline{#1}}}
\setlength{\headsep}{0.7cm}
\hypersetup{colorlinks=true}

%% CREDIT FOR MARKERLESSFOOTNOTE WHERE CREDIT IS DUE: https://tex.stackexchange.com/questions/30720/footnote-without-a-marker?answertab=scoredesc#tab-top
\newcommand\markerlessfootnote[1]{%
    \begingroup
    \renewcommand\thefootnote{}\footnote{#1}%
    \addtocounter{footnote}{-1}%
    \endgroup
}

\newcommand{\pagelayout}{
    \pagestyle{fancy}
    \fancyhf{}
    \lhead{\coursenumber}
    \chead{\ Lab \labnumber: \labname}
    \rhead{\courseterm}
    \cfoot{\shortlabname-\thepage}
}

\newcommand{\labidentifier}{
    \title{\ Lab \labnumber}
    \author{\labname}
    \date{Due: \duedate}
    \maketitle

    \textit{\collaborationrules}
    \markerlessfootnote{\tiny{\ifdefstring{\instructorname}{Christopher A. Bohn}{Assignment}{Original assignment} and starter code \copyright\ Christopher A. Bohn, licensed under the Creative Commons Attribution 4.0 International License~\ccby\ and under the Apache License Version 2.0, respectively.}}
    \ifdefstring{\instructorname}{Christopher A. Bohn}{}{\markerlessfootnote{\tiny{Configured for \coursenumber\ at \institutename\ by \instructorname.}}}

    \begin{description}
        \item[Obtaining the starter code] \filesource
        \item[Runtime environment] We will grade this assignment by compiling and running it on \runtimeenvironment;
            you should compile and test your code on \runtimeenvironment\ before turning it in.
        \item[Submitting your work] \filesubmission
    \end{description}
}

% display module fonts for hardware kit
% use with the Capital baseball "matrix printer" font collection (https://www.ctan.org/tex-archive/fonts/capbas/)
% Identifying the specific font in the assignment sheet is deprecated
%   -- instead, set the `usedisplayfont` boolean and the `displaymodule` variable in semester.tex,
%      and \display{...} in the assignment sheet

\addfont{OT1}{d7seg}{\dviiseg}
\addfont{OT1}{deseg}{\deseg}
\addfont{OT1}{necker}{\necker}

\providebool{usedisplayfont}

\newcommand{\display}[1]{
    \ifboolexpe{bool{usedisplayfont}}{
        \ifdefstring{\displaymodule}{MAX7219digits}{{\dviiseg #1}}{}
        \ifdefstring{\displaymodule}{MAX7219matrix}{{\deseg #1}}{}
        \ifdefstring{\displaymodule}{LCD1602}{{\necker #1}}{}
        % We don't yet have a Cow Pi configuration with 14-segment displays, so no \deseg yet
    }{
        \texttt{#1}
    }
}

\newcommand{\rubricitem}[2]{\item[\underline{\hspace{1cm}} +#1] #2}
\newcommand{\bonusitem}[2]{\item[\underline{\hspace{1cm}} Bonus +#1] #2}
\newcommand{\penaltyitem}[2]{\item[\underline{\hspace{1cm}} -#1] #2}
\newcommand{\checkoffitem}[1]{\item (\phantom{xxx}) #1}
\newcommand{\precheckoffitem}[1]{\item [] (\phantom{xxx}) #1}

\newcommand{\institutename}{University of Nebraska-Lincoln}
\newcommand{\instructorname}{Christopher A. Bohn}
\newcommand{\coursenumber}{CSCE~231}
\newcommand{\cstwo}{CSCE~156, RAIK~184H, or SOFT~161}
\newcommand{\courseterm}{Spring 2025}
\newcommand{\labnumber}{11}
\newcommand{\labname}{Using Interrupt-Driven Input/Output}
\newcommand{\shortlabname}{InterruptLab}
\newcommand{\duedate}{Week of April 21, before the start of your lab section}
\newcommand{\pathone}{chap09/io/interrupts}
\newcommand{\pathtwo}{chap09/system-descriptions/number-pad-texter}
\newcommand{\filesource}{Download \startercode\ from Canvas, or copy \startercode\ from {\footnotesize$\sim$}cbohn2/csce231 on \textit{nuros.unl.edu}}
\newcommand{\filesubmission}{When you have completed this assignment, submit \requiredfiles\ to the assignment in Canvas.}
\newcommand{\runtimeenvironment}{a Cow Pi development board with a \processor\ microcontroller}
\newcommand{\startercode}{interruptlab.zip or interruptlab.tar}
\newcommand{\requiredfiles}{\textit{inputs.c}, \textit{character\_selector.c}, and \textit{message\_editor.c}}
\newcommand{\buildsystem}{platformio}
\newcommand{\processor}{RP2040}
\newcommand{\lowercaseprocessor}{rp2040}
\newcommand{\collaborationrules}{During your scheduled lab time, you may, \textbf{but are not required to}, form a partner group of 2 students.
    When necessary, there may be a group of 3 students.
    During your scheduled lab time, and until the end of your lab day, you may discuss problem decomposition and solution design with your lab partner.
    After your scheduled lab day, you may discuss concepts and syntax with other students, but you may discuss solutions only with the professor and the TAs.
    Sharing code with or copying code from another student or the internet is prohibited.
}
\newcommand{\policyforcodethatdoesnotcompile}{\subsection*{No Credit for Uncompilable Code}
    If the TA cannot create an executable from your code, then your code will be assumed to have no functionality.\footnote{
        At the TA's discretion, if they can make your code compile with \textit{one} edit (such as introducing a missing semicolon) then they may do so and then assess a 10\% penalty on the resulting score.
        The TA is under no obligation to do so, and you should not rely on the TA's willingness to edit your code for grading.
        If there are multiple options for a single edit that would make your code compile, there is no guarantee that the TA will select the option that would maximize your score.
    }
    Before turning in your code, be sure to compile and test your code on \runtimeenvironment\ with the original driver code, the original header file(s), and the original \textit{Makefile}.}
\newcommand{\latepolicy}{\subsection*{Late Submissions}
    \textcolor{red}{This assignment does \textit{not} have a 48-hour grace period.}
    This assignment is due before the start of your lab section.
    The due date in Canvas is five minutes after that, which is ample time for you to arrive to lab and then discover that you'd forgotten to turn in your work without Canvas reporting your work as having been turned in late.
    We will accept late turn-ins up to one hour late, assessing a 10\% penalty on these late submissions.
    Any work turned in more than one hour late will not be graded.}
\newcommand{\softwareengineeringfrontmatter}{\section*{No Spaghetti Code Allowed}
        In the interest of keeping your code readable, you may \textit{not} use
        any \lstinline{goto} statements, nor may you use any
        \lstinline{continue} statements, nor may you use any \lstinline{break}
        statements to exit from a loop, nor may you have any functions
        \lstinline{return} from within a loop.}
\newcommand{\softwareengineeringpenalties}{\penaltyitem{1}{for each \lstinline{goto} statement,
            \lstinline{continue} statement, \lstinline{break} statement used to
            exit from a loop, or \lstinline{return} statement that occurs within
            a loop.}}
\newcommand{\scenariointroduction}{    Smoke wafts from Herb's soldering iron as he looks up when you approach.
    Cleaning the iron's tip, he notes:
    ``Somebody once said that one of the most dangerous things in the world is a programmer with a soldering iron.''\footnote{
        ``The three most dangerous things in the world are a programmer with a soldering iron, a hardware engineer with a software patch, and a user with an idea.'' -- Rick Cook, \textit{The Wizardry Consulted}, 1995.
    }$^{\mathrm{,}}$\footnote{
        The notion of being wary of programmers wielding screwdrivers or soldering irons long pre-dated this quote, as there are apocryphal tales of people who found it easier to modify the hardware to suit the software rather than the other way around.
    }
        About this time, you hear Archie rumbling in the distance.
    A short time later, he enters Herb's lab and declares,
    ``We need a way to communicate over distances.''

    You ask the obvious question, ``You mean like a phone or a walkie-talkie?''

    Archie shakes his head, ``No, I think it needs to be text-based, so we can send messages quietly without disturbing the specimens.''

    Thinking about smartphones, you hesitatingly suggest, ``You want texting?''

    Archie beamingly smiles, ``Say! That's a great name for it!''

    Herb jumps in, ``Yes, we'll do it.''
    Herb looks at you and adds, ``It'll give us a chance to test the Cow Pi's timer and whether we can take inputs without constantly polling the input devices.''
}
\newcommand{\scenariowrapup}{    You and Herb look for Archie in the Pleistocene Petting Zoo's labs to give him the good news, and you find a blond woman wearing cargo shorts, butchering a Gilbert and Sullivan song\dots \\ \\
    \textmusicalnote\ I am the very model of a modern vice admiral \textmusicalnote \\
    \textmusicalnote\ I've information about all things paleobotanical \textmusicalnote \\
    \textmusicalnote\ And I've been up to my armpits in problems scatological \textmusicalnote \\
    \textmusicalnote\ During the regency I had experience matriarchical \textmusicalnote \\
    \textmusicalnote\ I plot space travel, normal and superluminal \textmusicalnote \\
    \textmusicalnote\ (Even if I challenge the Pauli exclusion principle) \textmusicalnote \\

    ``I don't know how these people keep getting into our labs.
    \textit{Please} tell me that you have good news,'' pleads Archie.

    ``Yes, the Cow Pi is ready for whatever you need: calculators, security systems, parking meters -- you name it,'' Herb cheerfully responds.

    ``Excellent.''
    Archie turns to you.
    ``I'd like you and H.Awk... no, \textit{not} H.Awk~Aye.
    I'd like you and someone else on the staff to get started right away.
    Here's what I'd like to have built first.''
}
