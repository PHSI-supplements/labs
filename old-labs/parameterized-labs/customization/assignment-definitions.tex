%%
%% labs/common/assignment.tex
%% (c) 2021-23 Christopher A. Bohn
%%
%% Licensed under the Apache License, Version 2.0 (the "License");
%% you may not use this file except in compliance with the License.
%% You may obtain a copy of the License at
%%     http://www.apache.org/licenses/LICENSE-2.0
%% Unless required by applicable law or agreed to in writing, software
%% distributed under the License is distributed on an "AS IS" BASIS,
%% WITHOUT WARRANTIES OR CONDITIONS OF ANY KIND, either express or implied.
%% See the License for the specific language governing permissions and
%% limitations under the License.
%%

\usepackage{addfont}
\usepackage{amsmath}
\usepackage{amssymb}
\usepackage{animate}
\usepackage{bold-extra}
\usepackage{cancel}
\usepackage{caption}
\usepackage{ccicons}
\usepackage{enumitem}
\usepackage{etoolbox}
\usepackage{fancyhdr}
\usepackage{fullpage}
\usepackage{graphicx}
\usepackage{hyperref}
\usepackage[utf8]{inputenc}
\usepackage[procnames]{listings}
%\usepackage{media9}
\usepackage{multicol}
\usepackage{subfig}
\usepackage{textcomp}
\usepackage{tikz}
\usepackage[americanresistor]{circuitikz}
%\usepackage{tikz-uml}
\usetikzlibrary{automata,positioning,arrows}
\usepackage[normalem]{ulem}
\usepackage{wrapfig}
\usepackage{xcolor}
%\usepackage[dvipsnames]{xcolor}
\definecolor{LightGreen}{rgb}{0.88,1,0.88}

\lstset{language=C, tabsize=4, upquote=true, basicstyle=\ttfamily}
\newcommand{\function}[1]{\textbf{\lstinline{#1}}}
\setlength{\headsep}{0.7cm}
\hypersetup{colorlinks=true}

%% CREDIT FOR MARKERLESSFOOTNOTE WHERE CREDIT IS DUE: https://tex.stackexchange.com/questions/30720/footnote-without-a-marker?answertab=scoredesc#tab-top
\newcommand\markerlessfootnote[1]{%
    \begingroup
    \renewcommand\thefootnote{}\footnote{#1}%
    \addtocounter{footnote}{-1}%
    \endgroup
}

\newcommand{\pagelayout}{
    \pagestyle{fancy}
    \fancyhf{}
    \lhead{\coursenumber}
    \chead{\ Lab \labnumber: \labname}
    \rhead{\courseterm}
    \cfoot{\shortlabname-\thepage}
}

\newcommand{\labidentifier}{
    \title{\ Lab \labnumber}
    \author{\labname}
    \date{Due: \duedate}
    \maketitle

    \textit{\collaborationrules}
    \markerlessfootnote{\tiny{\ifdefstring{\instructorname}{Christopher A. Bohn}{Assignment}{Original assignment} and starter code \copyright\ Christopher A. Bohn, licensed under the Creative Commons Attribution 4.0 International License~\ccby\ and under the Apache License Version 2.0, respectively.}}
    \ifdefstring{\instructorname}{Christopher A. Bohn}{}{\markerlessfootnote{\tiny{Configured for \coursenumber\ at \institutename\ by \instructorname.}}}

    \begin{description}
        \item[Obtaining the starter code] \filesource
        \item[Runtime environment] We will grade this assignment by compiling and running it on \runtimeenvironment;
            you should compile and test your code on \runtimeenvironment\ before turning it in.
        \item[Submitting your work] \filesubmission
    \end{description}
}

% display module fonts for hardware kit
% use with the Capital baseball "matrix printer" font collection (https://www.ctan.org/tex-archive/fonts/capbas/)
% Identifying the specific font in the assignment sheet is deprecated
%   -- instead, set the `usedisplayfont` boolean and the `displaymodule` variable in semester.tex,
%      and \display{...} in the assignment sheet

\addfont{OT1}{d7seg}{\dviiseg}
\addfont{OT1}{deseg}{\deseg}
\addfont{OT1}{necker}{\necker}

\providebool{usedisplayfont}

\newcommand{\display}[1]{
    \ifboolexpe{bool{usedisplayfont}}{
        \ifdefstring{\displaymodule}{MAX7219digits}{{\dviiseg #1}}{}
        \ifdefstring{\displaymodule}{MAX7219matrix}{{\deseg #1}}{}
        \ifdefstring{\displaymodule}{LCD1602}{{\necker #1}}{}
        % We don't yet have a Cow Pi configuration with 14-segment displays, so no \deseg yet
    }{
        \texttt{#1}
    }
}

\newcommand{\rubricitem}[2]{\item[\underline{\hspace{1cm}} +#1] #2}
\newcommand{\bonusitem}[2]{\item[\underline{\hspace{1cm}} Bonus +#1] #2}
\newcommand{\penaltyitem}[2]{\item[\underline{\hspace{1cm}} -#1] #2}
\newcommand{\checkoffitem}[1]{\item (\phantom{xxx}) #1}
\newcommand{\precheckoffitem}[1]{\item [] (\phantom{xxx}) #1}
