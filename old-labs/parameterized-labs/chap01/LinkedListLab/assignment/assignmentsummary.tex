This assignment is principally about getting comfortable when explicitly working with memory.
Being able to think about a value and a reference to that value distinctly will improve your programming skills in any language.

Before you do so, in Section~\ref{sec:archiesCode} you will examine Archie's code.
Parts of Archie's programs use code that the C standard explicitly states will result in undefined behavior.
By understanding the mistakes that Archie made, we hope that you can avoid them in your own code.

In Section~\ref{sec:challengeResponse}, you will build and use a linked list.
This will require you to allocate space for the list's nodes and manipulate pointers that connect the nodes to each other.

\subsection{Constraints}

There are no particular restrictions on uses of C's features in this assignment other than those common to most lab assignments in this course.
    You can check whether you're using a \lstinline{goto} or \lstinline{continue} statement, or whether you're using \lstinline{break} or \lstinline{return} to exit a loop, by running the constraint-checking Python script:
    \texttt{python constraint-check.py linkedlistlab.json}

Some operations on a list have ``undefined behavior'' when performed using an invalid iterator.
When the behavior is undefined, any result is acceptable \textit{except} crashing the program.
Specifically, you may not dereference a NULL pointer.
\texttt{constraint-check.py} will not check for this; however, you can test for it using the code provided to you.
