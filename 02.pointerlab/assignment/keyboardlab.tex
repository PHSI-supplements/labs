%%
%% KeyboardLab (c) 2021 Christopher A. Bohn
%%

%%
%% labs/common/assignment.tex
%% (c) 2021-22 Christopher A. Bohn
%%
%% Licensed under the Apache License, Version 2.0 (the "License");
%% you may not use this file except in compliance with the License.
%% You may obtain a copy of the License at
%%     http://www.apache.org/licenses/LICENSE-2.0
%% Unless required by applicable law or agreed to in writing, software
%% distributed under the License is distributed on an "AS IS" BASIS,
%% WITHOUT WARRANTIES OR CONDITIONS OF ANY KIND, either express or implied.
%% See the License for the specific language governing permissions and
%% limitations under the License.
%%

\documentclass[12pt]{article}

\usepackage{fullpage}
\usepackage{fancyhdr}
\usepackage[procnames]{listings}
\usepackage{hyperref}
\usepackage{textcomp}
\usepackage{bold-extra}
\usepackage[dvipsnames]{xcolor}
\usepackage{etoolbox}

% These are placeholder commands and will be renewed in each lab

\newcommand{\labnumber}{}
\newcommand{\labname}{Lab \labnumber\ Assignment}
\newcommand{\shortlabname}{}
\newcommand{\duedate}{}

% Individual or team effort

\newcommand{\individualeffort}{This is an individual-effort project. You may
    discuss concepts and syntax with other students, but you may discuss
    solutions only with the professor and the TAs. Sharing code with or copying
    code from another student or the internet is prohibited.}
\newcommand{\teameffort}{This is a team-effort project. You may discuss concepts
    and syntax with other students, but you may discuss solutions only with your
    assigned partner(s), the professor, and the TAs. Sharing code with or
    copying code from a student who is not on your team, or from the internet,
    is prohibited.}
\newcommand{\freecollaboration}{In addition to the professor and the TAs, you
    may freely seek help on this assignment from other students.}
\newcommand{\collaborationrules}{}

% Software engineering (if you care about that)

\providebool{allowspaghetticode}

\newcommand{\softwareengineeringfrontmatter}{
    \ifboolexpe{not bool{allowspaghetticode}}{
        \section*{No Spaghetti Code Allowed}
        In the interest of keeping your code readable, you may \textit{not} use
        any \lstinline{goto} statements, nor may you use any
        \lstinline{continue} statements, nor may you use any \lstinline{break}
        statements to exit from a loop, nor may you have any functions
        \lstinline{return} from within a loop.
    }{}
}

\newcommand{\spaghetticodepenalties}[1]{
    \ifboolexpe{not bool{allowspaghetticode}}{
        \penaltyitem{1}{for each \lstinline{goto} statement,
            \lstinline{continue} statement, \lstinline{break} statement used to
            exit from a loop, or \lstinline{return} statement that occurs within
            a loop.}
    }{}
}

% You shouldn't need to customize these,
% but you can if you like

\lstset{language=C, tabsize=4, upquote=true, basicstyle=\ttfamily}
\newcommand{\function}[1]{\textbf{\lstinline{#1}}}
\setlength{\headsep}{0.7cm}
\hypersetup{colorlinks=true}

\newcommand{\pagelayout}{
    \pagestyle{fancy}
    \fancyhf{}
    \lhead{\coursenumber}
    \chead{\ Lab \labnumber: \labname}
    \rhead{\courseterm}
    \cfoot{\shortlabname-\thepage}
}

\newcommand{\labidentifier}{
    \title{\ Lab \labnumber}
    \author{\labname}
    \date{Due: \duedate}
    \maketitle

    \textit{\collaborationrules}
}

% deprecated
\newcommand{\startdocument}{
    \pagelayout
	\begin{document}
	\labidentifier
}

\newcommand{\rubricitem}[2]{\item[\underline{\hspace{1cm}} +#1] #2}
\newcommand{\bonusitem}[2]{\item[\underline{\hspace{1cm}} Bonus +#1] #2}
\newcommand{\penaltyitem}[2]{\item[\underline{\hspace{1cm}} -#1] #2}
\newcommand{\checkoffitem}[1]{\item (\phantom{xxx}) #1}
\newcommand{\precheckoffitem}[1]{\item [] (\phantom{xxx}) #1}

\usepackage{graphicx}
\usepackage{wrapfig}
\usepackage{ulem}
\usepackage{cancel}
\usepackage{multicol}
\lstset{language=c, numbers=left, showstringspaces=false,
    moredelim = [s][\ttfamily]{/*}{*/} % I shouldn't need this parameter!
    }

\renewcommand{\labnumber}{2}
\renewcommand{\labname}{Bits and Pointers Manipulation Lab}
\renewcommand{\shortlabname}{keyboardlab}
\renewcommand{\collaborationrules}{\individualeffort}
%\renewcommand{\duedate}{See \filesubmission\ for the due date}
\renewcommand{\duedate}{Week of January 31, before the start of your lab section}

\newcommand{\tab}{\ensuremath{\longrightarrow}}
\newcommand{\nl}{\ensuremath{\hookleftarrow}}

\startdocument
% \begin{document}



The purpose of this assignment is to give you more confidence in C programming
and to begin your exposure to the underlying bit-level representation of data.
You will also gain practice with pointers and with file input/output.

The instructions are written assuming you will edit and run the code on
\runtimeenvironment. If you wish, you may edit and run the code
in a different environment; be sure that your compiler suppresses no warnings,
and that if you are using an IDE that it is configured for C and not C++.


\section*{Scenario}

\begin{wrapfigure}{r}{0.33\textwidth}
    \centering
    \includegraphics[scale=.33]{some-expenses-spared}
    % \caption{Some expenses were spared.}
\end{wrapfigure}

You've recently been hired to help get the Pleistocene Petting Zoo get started.
Your new employer, Archie, is surprisingly honest: he admits to you that some
expenses were spared. Archie cheerfully points out that any challenge is also
an opportunity to succeed. You suspect your job will offer plenty of
``opportunities to succeed.''

\section{Using the ASCII table}

You soon discover your first
$\cancelto{\mathrm{opportunity}}{\mathrm{challenge}}$. Archie purchased your
workstation from government surplus. Your keyboard is left over from early
2001 and is missing the letter \textit{\texttt{W}}!\footnote{In January 2001,
when President Bill Clinton's staff left the White House so that President
George~W. Bush's staff could move in, they removed the \textit{\texttt{W}} key
from several keyboards as a prank.} You decide to write an email requesting a
new keyboard: \\
\\
\texttt{TO\tab Archie\nl} \\
\texttt{RE\tab I Need a Working Keyboard\nl} \\
\nl \\
\texttt{Please order a new keyboard for me. This one is broken.\nl } \\ \\
(Note: here, the \tab\ symbol represents the \texttt{TAB} character which is
needed by the email program, and the \nl\ symbol represents a \texttt{NEWLINE}
character.)

You quickly realize that you can't type this directly into your mail program
because of the missing \textit{\texttt{W}} key. So you decide to write a short
program that will output the text that you want to send. The code you would
like to write is:

\begin{lstlisting}
#include <stdio.h>

int main() {
    printf("TO\tArchie\n");
    printf("RE\tI Need a Working Keyboard\n\n");
    printf("Please order a new keyboard for me. This one is broken.\n");
    return 0;
}
\end{lstlisting}

Of course, the \texttt{W} and the \texttt{w} are still a problem, but you
realize you can insert those characters by using their ASCII
values.\footnote{Use the ASCII table in the textbook or type \texttt{man ascii}
in a terminal window.} For example,

\lstinline{printf("Hello World!\n");} \\
can be replaced with

\lstinline{printf("%s%c%s\n", "Hello ", ..., "orld!");} \\
replacing \dots\ with the ASCII value for \texttt{W}. Recall that the first
argument for \function{printf()} is a \textit{format string}: \texttt{\%s}
specifies that a string should be placed at that position in the output, and
\texttt{\%c} specifies that a character should be placed at that position in
the output.

As you open your editor, the \textit{\texttt{\textbackslash}} key falls off the
keyboard, preventing you from typing \texttt{\textbackslash t} and
\texttt{\textbackslash n}.

Edit \texttt{problem1.c} so that it produces the specified output without using
the W key or the backslash key. Build the executable with the command:
\texttt{make keyboardlab1} -- be sure to fix both errors and warnings.

You can double-check that you aren't using the W key or the backslash key by
running the constraint-checking shell script: \\
\texttt{./constraint-check.sh} \\
(If you get a ``\texttt{Permission denied}'' error message, then run the
command \texttt{chmod +x constraint-checking.sh} and then run the shell script.)

You can check that your program has the correct output with this command: \\
\texttt{./keyboardlab1 | diff - problem1oracle}


\section{Treating Characters as Numbers}

Archie replies to your email, assuring you that a new keyboard has been
ordered. Meanwhile, he needs you to write some code that will convert uppercase
letters to lowercase letters and to indicate whether or not a character is a
decimal digit. You realize this is easy work since those actual functions are
part of the standard C library with their prototypes in \texttt{ctype.h}. As
you get ready to impress your boss with how fast you can ``write'' this code by
calling those standard functions, the \textit{\texttt{3}} key (which is also
used for \textit{\texttt{\#}}) falls off of your keyboard, preventing you from
typing \lstinline{#include <ctype.h>}. Several other number keys fall off soon
thereafter (only \textit{\texttt{0}}, \textit{\texttt{7}}, and
\textit{\texttt{9}} remain), along with the \textit{\texttt{s}} key. The
\textit{\texttt{f}} key is looking fragile, so you decide that you had better
not type too many \lstinline{if} statements (and without the
\textit{\texttt{s}} key, you can't use a \lstinline{switch} statement at all).

Edit \texttt{problem2.c} so that
\begin{itemize}
\item \function{iz_digit()} returns 1 if the character is a decimal digit
    (\textquotesingle 0\textquotesingle, \textquotesingle 1\textquotesingle,
    \textquotesingle 2\textquotesingle, \dots) and 0 otherwise
\item \function{decapitalize()} will return the lowercase version of an
    uppercase letter (\textquotesingle A\textquotesingle, \textquotesingle
    B\textquotesingle, \textquotesingle C\textquotesingle, \dots) but will
    return the original character if it is not an uppercase letter
\end{itemize}
You may not \lstinline{#include} any headers, you may not use any number keys
other than the 0, 9, and 7 (which is also used for \textbf{\texttt{\&}}) keys,
you may not use \lstinline{switch} statements, and you may use at most one
\lstinline{if} statement in each function.

Build the executable with the command: \texttt{make keyboardlab2} -- be sure to
fix both errors and warnings.

You can double-check that you aren't disallowed keys by running the
constraint-checking shell script: \\
\texttt{./constraint-check.sh} \\
Because you are allowed at most one \lstinline{if} statement in each function,
you may see two lines with \lstinline{if} statements reported when the script
checks for \textquotesingle f\textquotesingle. You may also see some comments
reported when the script checks for \textquotesingle *\textquotesingle.


\section{Using Bitmasks and Shifting Bits}

Your keyboard was mistakenly delivered to the Plywood Scenery Cutting Studio
instead of the Pleistocene Petting Zoo. While that gets sorted out, you
``borrow'' some tar from the La~Brea Tar Pits diorama and use the tar to
re-attach your keyboard's missing keys. As you fasten a Scrabble tile in place
for the \textit{\texttt{W}}, more keys fall off, denying you the use of
\textit{\texttt{+}}, \textit{\texttt{-}}, \textit{\texttt{/}},
\textit{\texttt{\%}}, \textit{\texttt{5}}, and \textit{\texttt{b}}. You cannot
get any more tar from the diorama, so you sit down to your next programming
tasks without those keys.

Edit \texttt{problem3.c} so that
\begin{itemize}
\item \function{is_even()} returns 1 if the number is even (that is,
    divisible by 2) and 0 if the number is odd
\item \function{produce_multiple_of_ten()} will always output a multiple of
    10 following a specific formula: if a number is even then divide it by
    2; otherwise, subtract 1 from the number and multiply the difference
    by 5 (for example, an input of 7 yields 30 because $(7-1) \times 5 = 30$);
    repeat until the last decimal digit is 0.
\end{itemize}
These numbers are guaranteed to be non-negative. You may not use addition (+),
subtraction (-), division (/), nor modulo (\%). You also may not use the
number 5 nor the letter b. (Exceptions: you \textit{may} use the forward-slash
(/) for comments, and the percent-sign (\%) that is already present in the
\function{sprintf()} calls' format strings is allowed)

Hints:
\begin{itemize}
\item Following the weighted-sum technique to convert between binary and
    decimal (or by examining the textbook's Table~2.1), you will note that all
    even numbers have a 0 as their least significant bit, and all odd numbers
    have a 1 as their least significant bit
\item Less obvious is that you can subtract 1 from an odd number by changing
    its least significant bit to a 0
\item As we will cover in Chapter~3, you can halve a number by shifting its
    bits to the right by one
\item You can create an integer by producing its bit pattern through a series
    of bit operations
\end{itemize}

Build the executable with the command: \texttt{make keyboardlab3} -- be sure to
fix both errors and warnings.

You can double-check that you aren't disallowed keys by running the
constraint-checking shell script: \\
\texttt{./constraint-check.sh} \\
You may see some comments reported when the script checks for
\textquotesingle *\textquotesingle.


\section{Stray Values on the Stack}

Great news! Archie brings you your new keyboard. He also brings you a problem
of his own. Because you were held up with the broken keyboard, Archie decided
to try some programming on his own. He shows you his code:

\begin{lstlisting}
/***********************************************************************
 * This program will output
 **         Welcome to the
 **    Pleistocene Petting Zoo!
 **
 ** Get ready for hands-on excitement on the count of three! 1.. 2.. 3..
 ** Have fun!
 * With brief pauses during the "Get ready" line.
 ***********************************************************************/

#include <stdio.h>
#include <unistd.h>

void splash_screen() {
    const char *first_line = "\t     Welcome to the\n";
    const char *second_line = "\tPleistocene Petting Zoo!\n";
    printf("%s%s\n", first_line, second_line);
}

void count() {
    int i;
    sleep(1);
    printf("Get ready for hands-on excitement on the count of three! ");
    while (i < 3) {
        fflush(stdout);
        sleep(1);
        i++;
        printf("%d.. ", i);
    }
    printf("\nHave fun!\n");
}

int main() {
    splash_screen();
    count();
    return 0;
}
\end{lstlisting}

Sometimes the output was what he expected:
\begin{verbatim}
         Welcome to the
    Pleistocene Petting Zoo!

 Get ready for hands-on excitement on the count of three! 1.. 2.. 3..
 Have fun!
\end{verbatim}

But sometimes the output was missing the
``\texttt{1.. 2.. 3..}'':
\begin{verbatim}
         Welcome to the
    Pleistocene Petting Zoo!

 Get ready for hands-on excitement on the count of three!
 Have fun!
\end{verbatim}

What mistake did Archie make? What change to \textit{one} line will fix
Archie's bug? Place your answers in \textit{answers4.txt}.


\section{Challenge and Response}

You plug in your shiny, new keyboard, tune your satellite radio to the
Greatest Hits of the 1920s, and settle in to solving a more interesting problem.

To protect against corporate espionage, you are responsible for writing code
for a challenge-and-response system. Anybody can challenge anyone else in the
Pleistocene Petting Zoo's non-public areas by providing the name of a book and
a word contained within the book, and the person being challenged must respond
with another word from that book, based on certain rules:
\begin{itemize}
\item All of the book's words are sorted alphabetically without regard to
    capitalization (for example, ``hello'' occurs after ``Hear'' and before
    ``HELP'')
\item The challenge word occurs \textit{occurrences} times in the book
\item If \textit{occurrences} is an even number then the response word is the
    word \textit{occurrences} places \textbf{before} the challenge word in the
    alphabetized list; if the challenge word is less than \textit{occurrences}
    places from the start of the list then the response word is the first word
    in the list
\item If \textit{occurrences} is an odd number then the response word is the
    word \textit{occurrences} places \textbf{after} the challenge word in the
    alphabetized list; if the challenge word is less than \textit{occurrences}
    places from the end of the list then the response word is the last word
    in the list
\end{itemize}

Here is a simple example. Suppose the words in the specified book are:

\begin{center}
\begin{tabular}{cc}
\textit{word} & \textit{occurrences} \\ \hline
apple       & 7 \\
banana      & 4 \\
carrot      & 15 \\
date        & 3 \\
eggplant    & 2 \\
fig         & 6 \\
granola     & 9 \\
horseradish & 9 \\
ice         & 6 \\
jelly       & 3 \\
kale        & 1 \\
lemon       & 2 \\
mango       & 8 \\
naan        & 7 \\
orange      & 5 \\
pineapple   & 1 \\
quinoa      & 11 \\
raisin      & 4 \\
spaghetti   & 10 \\
tomato      & 12 \\
\end{tabular}
\end{center}
If the challenge word is ``horseradish'' then because horseradish occurs 9
times in the book, the response word is ``quinoa,'' which is 9 places in the
list after ``horseradish.'' If the challenge word is ``eggplant'' then the
response is ``carrot,'' which is 2 places earlier in the list than
``eggplant.'' If the challenge word is ``banana'' then the response word is
``apple,'' which is the first word in the list. If the challenge word is
``quinoa'' then the response word is ``tomato,'' the last word in the list.

You break the problem down into four sub-problems:

\subsection{Doubly-Linked List}

A \textit{doubly-linked list} is a linked list with the property that each node
maintains a link not only to the \lstinline{next} node but also a link to the
\lstinline{previous} node. In C, these links are pointers.
\begin{center}
\includegraphics[scale=0.5]{doubly-linked-list}
\end{center}
Inserting new node, $C$ between adjacent nodes $A$ and $B$ (where $B = A.next$
and $A = B.previous$) requires connecting $C.previous$ to $A$ and $C.next$ to
$B$, and re-assigning $A.next$ to $C$ and $B.prevous$ to $C$.
\begin{center}
\includegraphics[scale=0.5]{list-insertion}
\end{center}

Design a \lstinline{struct} for a doubly-linked list's node. Besides the
\lstinline{next} and \lstinline{previous} pointers, the node will have a
payload: a string for a word and an integer for the number of times that word
occurs in the book. In \textit{problem5.c}, write the code to create a node
and the code to build a doubly-linked list.

The starter code also includes a \function{print_list()} function to help with
debugging. This function traverses a linked list, printing each node's payload
and its \lstinline{previous} and \lstinline{next} pointers.

\subsection{Alphabetize Words}

In Problem~2, you wrote code to convert uppercase letters to lowercase letters.
In \textit{problem5.c}, write code that calls that function to convert all
letters in a word to lowercase letters (do not copy the
\function{to_lowercase()} function into \textit{problem5.c}; we will link to
the function in \textit{problem2.c}).

The starter code includes a function to compare two words (you do not need to
write this function) but it assumes that both words are completely lowercase.

\subsection{Insertion Sort}

The \textit{insertion sort} algorithm reads an input and then traverses a
sorted list to find the proper location in the sorted list for the input. The
input is then inserted into the list at that location.

For this problem, the user will be prompted to enter the name of a book, which
will be the filename of a file that contains all of the book's words. All
punctuation has already been removed from the files, and each line in the
file contains exactly one word. In \textit{problem5.c}, write the code to read
the file one line at a time.\footnote{See \S7.5 of Kernighan \& Ritchie's
\textit{The C Programming Language}, 2nd ed. for \function{fopen()} and
\function{fclose()}, and \S7.7 for \function{fgets()}.} For each word,
convert it to lowercase, and then traverse the list to find the appropriate
place for the word. (Note that there will not be a list to traverse when your
code reads the first word!) If the word is not in the list then create a node
for that word and insert it into the list at the correct location. If there is
already a node containing that word, then increment that node's variable that
tracks the number of occurrences.

If your program requires more than a few seconds to build the list, there is a
bug in your code.

\subsection{Respond to a Challenge}

After the word list is complete (after you have inserted all words in the
file), the user will be prompted to enter the challenge word. In
\textit{problem5.c}, write the code to traverse the word list to find that
word. (Do not copy the \function{is_even()} function into \textit{problem5.c};
we will link to the function in \textit{problem3.c}.) If the word is not
present in the list, output ``(word) is not present!'' where ``(word)'' is the
challenge word. If the word is in the list then use the number of occurrences
recorded in that word's node to find the response word as described in the
challenge-and-response rules, and output that word.

If your program does not provide the response word nearly instantaneously,
there is a bug in your code.

\subsection*{Building the Executable}

Build the executable with the command: \texttt{make keyboardlab5} -- be sure to
fix both errors and warnings.

\subsection*{The Books}

As a small ``book'' to work with as you write your code, we have provided
Abraham Lincoln's Gettysburg Address (filename ``GettysburgAddress''). This
assignment's appendix has a table of the words in the Gettysburg Address and
the corresponding response word. We hope that this table will be helpful when
you test your code.

For grading, we will use one of these five books:\footnote{For each book, this
statement applies: ``This eBook is for the use of anyone anywhere in the United
States and most other parts of the world at no cost and with almost no
restrictions whatsoever. You may copy it, give it away or re-use it under the
terms of the \href{http://www.gutenberg.org/policy/license}{Project Gutenberg
License} included with this eBook or online at \url{www.gutenberg.org}. If you
are not located in the United States, you'll have to check the laws of the
country where you are located before using this ebook.'' A copy of the Project
Gutenberg license is included with the starter code (filename
``ProjectGutenbergLicense'').}
\begin{itemize}
\item Lewis Carroll's \textit{Alice's Adventures in Wonderland} (filename
    ``AliceInWonderland'') \url{https://www.gutenberg.org/ebooks/11}
\item Mary Shelly's \textit{Frankenstein; Or, The Modern Promethius} (filename
    ``Frankenstein'') \url{https://www.gutenberg.org/ebooks/84}
\item Arthur Conan Doyle's \textit{The Lost World} (filename ``TheLostWorld'')
    \url{https://www.gutenberg.org/ebooks/139}
\item Jonathan Swift's \textit{Gulliver's Travels into Several Remote Nations
    of the World} (filename ``GulliversTravels'')
    \url{https://www.gutenberg.org/ebooks/829}
\item Oscar Wilde's \textit{The Importance of Being Earnest: A Trivial Comedy
    for Serious People} (filename ``TheImportanceOfBeingEarnest'')
    \url{https://www.gutenberg.org/ebooks/844}
\end{itemize}

\section*{Turn-in and Grading}

When you have completed this assignment, upload \textit{problem1.c},
\textit{problem2.c}, \textit{problem3.c}, \textit{answers4.txt}, and
\textit{problem5.c} to \filesubmission.

This assignment is worth 45 points.
\begin{description}
\rubricitem{4}{\textit{problem1.c} produces the specified output without the
    use of \texttt{w}, \texttt{W}, \texttt{\textbackslash{}n}, or
    \texttt{\textbackslash{}t}}.
\rubricitem{4}{\function{iz_digit()} in \textit{problem2.c} determines whether
    or not a character is a digit, without \lstinline{#include}ing any files,
    without using any numbers other than 0 and 9, without using a
    \lstinline{switch} statment, and using at most one \lstinline{if}
    statement.}
\rubricitem{4}{\function{decapitalize()} in \textit{problem2.c} converts
    uppercase letters to lowercase and leaves other characters unchanged,
    without \lstinline{#include}ing any files, without using a
    \lstinline{switch} statment, and using at most one \lstinline{if}
    statement.}
\rubricitem{4}{\function{is_even()} in \textit{problem3.c} determines whether
    a number is even or odd, without using arithmetic.}
\rubricitem{4}{\function{produce_multiple_of_ten()} in \textit{problem3.c}
    has the following, without using arithmetic other than multiplication, and
    without using the values 5, 0x5, 05, or 0b101:}
    \begin{itemize}
    \item Code to assign the value 5 to the variable \lstinline{five}
    \item Code to divide an even number by 2
    \item Code to subtract 1 from an odd number
    \item Correct functionality
    \end{itemize}
\rubricitem{3}{Student's answers in \textit{answers4.txt} demonstrate an
    understanding of the bug in the code and how to correct it.}
\rubricitem{5}{Implemented a doubly-linked list in \textit{problem5.c}:
    \lstinline{struct node}, \function{create_node()},
    \function{insert_after()}, \function{insert_before()}.}
\rubricitem{2}{Implemented \function{word_to_lowercase()} in
    \textit{problem5.c}.}
\rubricitem{5}{Implemented insertion sort in \textit{problem5.c}:
    \function{insert_word()}.}
\rubricitem{5}{\function{build_list()} in \textit{problem5.c} opens a file for
    reading, reads one line at a time, passes the word that is read to
    \function{insert_word()}, and closes the file after the last line has been
    read.}
\rubricitem{5}{\function{respond()} in \textit{problem5.c} produces the correct response word in accordance with the specified rules.}
\end{description}

\newpage

\section*{Appendix: Gettysburg Address Challenge-and-Response Table}

\begin{multicols}{2}
\scriptsize
\begin{tabular}{ccc}
\textit{challenge word} & \textit{occurrences} & \textit{response word} \\ \hline
a & 7 & and \\
above & 1 & add \\
add & 1 & advanced \\
advanced & 1 & ago \\
ago & 1 & all \\
all & 1 & altogether \\
altogether & 1 & and \\
and & 6 & above \\
any & 1 & are \\
are & 3 & be \\
as & 1 & battlefield \\
battlefield & 1 & be \\
be & 2 & as \\
before & 1 & birth \\
birth & 1 & brave \\
brave & 1 & brought \\
brought & 1 & but \\
but & 2 & brave \\
by & 1 & can \\
can & 2 & but \\
cannot & 3 & come \\
cause & 1 & civil \\
civil & 1 & come \\
come & 1 & conceived \\
conceived & 2 & civil \\
consecrate & 1 & consecrated \\
consecrated & 1 & continent \\
continent & 1 & created \\
created & 1 & dead \\
dead & 3 & detract \\
dedicate & 2 & created \\
dedicated & 4 & continent \\
detract & 1 & devotion \\
devotion & 2 & dedicated \\
did & 1 & died \\
died & 1 & do \\
do & 1 & earth \\
earth & 1 & endure \\
endure & 1 & engaged \\
engaged & 1 & equal \\
equal & 1 & far \\
far & 2 & engaged \\
fathers & 1 & field \\
field & 1 & final \\
final & 1 & fitting \\
fitting & 1 & for \\
for & 5 & freedom \\
forget & 1 & forth \\
forth & 1 & fought \\
fought & 1 & four \\
four & 1 & freedom \\
freedom & 1 & from \\
from & 2 & four \\
full & 1 & gave \\
\end{tabular}

\columnbreak

\begin{tabular}{ccc}
\textit{challenge word} & \textit{occurrences} & \textit{response word} \\ \hline
gave & 2 & from \\
god & 1 & government \\
government & 1 & great \\
great & 3 & have \\
ground & 1 & hallow \\
hallow & 1 & have \\
have & 5 & increased \\
here & 8 & full \\
highly & 1 & honored \\
honored & 1 & in \\
in & 4 & have \\
increased & 1 & is \\
is & 3 & last \\
it & 5 & live \\
larger & 1 & last \\
last & 1 & liberty \\
liberty & 1 & little \\
little & 1 & live \\
live & 1 & lives \\
lives & 1 & living \\
living & 2 & live \\
long & 2 & lives \\
measure & 1 & men \\
men & 2 & long \\
met & 1 & might \\
might & 1 & nation \\
nation & 5 & not \\
never & 1 & new \\
new & 2 & nation \\
nobly & 1 & nor \\
nor & 1 & not \\
not & 2 & nobly \\
note & 1 & now \\
now & 1 & of \\
of & 5 & perish \\
on & 1 & or \\
or & 2 & of \\
our & 2 & on \\
people & 3 & poor \\
perish & 1 & place \\
place & 1 & poor \\
poor & 1 & portion \\
portion & 1 & power \\
power & 1 & proper \\
proper & 1 & proposition \\
proposition & 1 & rather \\
rather & 2 & proper \\
remaining & 1 & remember \\
remember & 1 & resolve \\
resolve & 1 & resting \\
resting & 1 & say \\
\end{tabular}

\columnbreak

\begin{tabular}{ccc}
\textit{challenge word} & \textit{occurrences} & \textit{response word} \\ \hline
say & 1 & score \\
score & 1 & sense \\
sense & 1 & seven \\
seven & 1 & shall \\
shall & 3 & struggled \\
should & 1 & so \\
so & 3 & task \\
struggled & 1 & take \\
take & 1 & task \\
task & 1 & testing \\
testing & 1 & that \\
that & 12 & resting \\
the & 10 & sense \\
their & 1 & these \\
these & 2 & the \\
they & 3 & thus \\
this & 6 & testing \\
those & 1 & thus \\
thus & 1 & to \\
to & 8 & that \\
under & 1 & unfinished \\
unfinished & 1 & upon \\
upon & 1 & us \\
us & 3 & we \\
vain & 1 & war \\
war & 2 & us \\
we & 10 & this \\
what & 2 & war \\
whether & 1 & which \\
which & 2 & what \\
who & 3 & world \\
will & 1 & work \\
work & 1 & world \\
world & 1 & years \\
years & 1 & years \\
\end{tabular}
\end{multicols}

\end{document}
