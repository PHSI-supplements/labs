The purpose of this assignment is to give you more confidence in C programming and to begin your exposure to pointers and to file input/output.

The instructions are written assuming you will edit and run the code on \runtimeenvironment.
If you wish, you may develop your code in a different environment;
be sure that your compiler suppresses no warnings, and that if you are using an IDE that it is configured for C and not C++.
Before you submit your code to \filesubmission, be sure that it compiles and runs on \runtimeenvironment.

\section*{Learning Objectives}

After successful completion of this assignment, students will be able to:
\begin{itemize}
    \item Recognize the hazards of indeterminate values.
    \item Use C's string functions from \lstinline{string.h}\footnote{\label{note:stringFunctions}See \S7.8.1 and \S{}B.3 of Kernighan \& Ritchie's \textit{The C Programming Language}, 2nd ed.}.
    \item Use C's file I/O functions from \lstinline{stdio.h}\footnote{See \S7.5, \S7.7, and \S{}B1.1 \textit{The C Programming Language}, 2nd ed.}.
    \item Alias and reassign pointers.
    \item Create and traverse a linked list.
\end{itemize}

\subsection*{Continuing Forward}

Being able to understand the mistakes in Sections~\ref{subsec:uninitializedvariables} and~\ref{subsec:danglingPointers} will help you avoid them in future labs.
Being able to work with pointers -- that is, with variables that hold addresses -- will help you specifically in future labs that use pointers but more generally in future labs that require you to think about accessing memory.

\section*{During Lab Time}

During your lab period, the TAs will provide a refresher on linked lists and will describe Insertion Sort.
The TAs will also describe some string functions and some I/O functions from C's standard library.
During the remaining time, the TAs will be available to answer questions.

During your lab period (and only during your lab period), you may discuss problem decomposition and solution design with your lab partner.
\textit{Be sure to add your name and your partner's name to the top of your source code files.}
To receive full credit for the work you and your partner do during lab time, you must be an active participate in the partnership.
In accordance with the School of Computing's Academic Integrity Policy, we reserve the right to adjust your calculated grade if you merely ``tag along'' and let your partner do all the thinking.

If you worked with a partner, then \textcolor{red}{before leaving lab, upload \textit{answers.txt} and \textit{challenge-response.c} to \filesubmission\ to create a record of what you and your partner worked on during lab time.}
After you have finished the full assignment, you will upload your completed work as well.
If you did not work with a partner, you do not need to upload your files until you have completed the assignment.