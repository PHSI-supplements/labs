\subsection{Pleistocene Petting Zoo Marquee} \label{subsec:uninitializedvariables}

Archie shows you his first program, which he hoped would be used to greet guests, but it doesn't always work right:

\begin{lstlisting}
/***********************************************************************
 * This program will output
 **         Welcome to the
 **    Pleistocene Petting Zoo!
 **
 ** Get ready for hands-on excitement on the count of three! 1.. 2.. 3..
 ** Have fun!
 * With brief pauses during the "Get ready" line.
 ***********************************************************************/

#include <stdio.h>
#include <unistd.h>

void splash_screen() {
    const char *first_line = "\t     Welcome to the\n";
    const char *second_line = "\tPleistocene Petting Zoo!\n";
    printf("%s%s\n", first_line, second_line);
}

void count() {
    int i;
    sleep(1);
    printf("Get ready for hands-on excitement on the count of three! ");
    while (i < 3) {
        fflush(stdout);
        sleep(1);
        i++;
        printf("%d.. ", i);
    }
    printf("\nHave fun!\n");
}

int main() {
    splash_screen();
    count();
    return 0;
}
\end{lstlisting}

Sometimes the output was what he expected:
\begin{verbatim}
         Welcome to the
    Pleistocene Petting Zoo!

 Get ready for hands-on excitement on the count of three! 1.. 2.. 3..
 Have fun!
\end{verbatim}

But sometimes the output was missing the
``\texttt{1.. 2.. 3..}'':
\begin{verbatim}
         Welcome to the
    Pleistocene Petting Zoo!

 Get ready for hands-on excitement on the count of three!
 Have fun!
\end{verbatim}

What mistake did Archie make?
What change to \textit{one} line will fix Archie's bug?
Place your answers in \textit{answers.txt}.

\subsection{Math Doesn't Work Right \dots Or Does It?} \label{subsec:danglingPointers}

``Wow,'' Archie says.
``I can't believe that I missed that.
Maybe you can help me with some other code that I wrote for one of my other start-ups.
Here it is.'' Archie shows you the code:

\begin{lstlisting}
/***********************************************************************
 * This program will add two numbers and then it will multiply two other
 * numbers. Finally, it will subtract the second result from the first
 * result.
 ***********************************************************************/

#include <stdio.h>

int *add(int a, int b) {
    int result = a + b;
    return &result;
}

int *multiply(int p, int q) {
    int result = p * q;
    return &result;
}

int main() {
    int *sum = add(4, 5);
    printf("sum = %d\n", *sum);
    int *product = multiply(2, 3);
    printf("product = %d\n", *product);
    printf("sum - product = %d - %d = %d\n",
                        *sum, *product, *sum - *product);
    return 0;
}
\end{lstlisting}

Archie explains that when he compiles the program with the \textbf{clang} compiler and then runs it, he gets this output:

\begin{verbatim}
sum = 9
product = 6
sum - product = 6 - 6 = 0
\end{verbatim}

And when he compiles the program with the \textbf{gcc} compiler and then runs it, the program terminates with a segmentation fault.

``I see that one compiler is giving me an incorrect answer, and the other compiler is telling me that I'm using memory in an unsafe way -- but what am I doing wrong, and why does it produce an incorrect answer?''

What mistake did Archie make?
Why does \lstinline{*sum} have the value 6 on line 25?
Why does \lstinline{*sum - *product} produce the value 0?
Place your answer in \textit{answers.txt}.
