If you added a \function{main()} function to \textit{card.c}, remove it so that there is only one \function{main()} function when you compile the full program.

In \textit{poker.c}, the first thing you'll want to do is write the code for \function{populate_deck()}.
Using \function{create_card()} from \textit{card.c}, create cards corresponding to the 52 standard playing cards and add them to the \lstinline{deck[]} array.
You might put code in \function{main()} to print out all 52 cards in \lstinline{deck[]} using \function{display_card()}, to confirm that you wrote \function{populate_deck()} correctly.

\subsection{Types of Poker Hands} \label{subsec:typesofpokerhands}

In the game of poker, hands are characterized by the similarities of the cards within.
Traditionally, you characterize the hand by the ``best'' characterization (that is, the one that is least likely to occur);
for example, a hand that is a three of a kind also contains a pair, but you would only characterize the hand as a three of a kind.
The \function{is...()} functions in \textit{poker.c} are intentionally simple;
they do not (and should not) check whether there is a better way to characterize the hand.
The types of hands (from most desirable to least desirable) are:

\begin{description}
    \item[Royal Flush] This is an Ace, a King, a Queen, a Jack, and a 10, all the same suit.
    There is no function in the starter code for a royal flush, nor do you need to write one, since a royal flush is essentially the best-possible straight flush.
    (Note also that a Royal Flush is not possible for this lab, based on our re-definition of a Straight, below.)
    \item[Straight Flush] This is five cards in a sequence, all the same suit;
    that is, five cards that are both a straight and a flush.
    This characterization is checked by the function \function{is_straight_flush()}.
    \item[Four of a Kind] Four cards all have the same value.
    This characterization is checked by the function \function{is_four_of_kind()}.
    \item[Full House] The hand contains a three of a kind and also contains a pair with a different value than that of the first three cards.
    This characterization is checked by the function \function{is_full_house()}.
    \item[Flush] Five cards all the same suit.
    This characterization is checked by the function \function{is_flush()}.
    In the interest of simplicity, for this assignment we changed the definition of a flush to ``all cards are of the same suit'' (this distinction only matters if the number of cards in the hand is not five).
    \item[Straight] Five cards in a sequence.
    This characterization is checked by the function \function{is_straight()}.
    In the interest of simplicity, for this assignment we changed the definition of a straight to ``all cards are in a sequence'' (this distinction only matters if the number of cards in the hand is not five).
    We further re-defined an Ace to be adjacent only to 2 (in traditional poker, an Ace can be adjacent to 2 or to King but not both at the same time).
    \item[Three of a Kind] Three cards all have the same value.
    This characterization is checked by the function \function{is_three_of_kind()}.
    \item[Two Pair] The hand holds two different pairs.
    This characterization is checked by the function \function{is_two_pair()}.
    \item[Pair] Two cards with the same value.This characterization is checked by
    the function \function{is_pair()}.
    \item[High Card] If the hand cannot be better characterized, it is characterized by the greatest-value card in the hand.
    The starter code does not have a function to check for this since this is the characterization if all the other functions return a \textbf{0}.
\end{description}

\subsection{Study the Code} \label{subsec:studythecode}

Look at the code for \function{is_pair()}.
Notice that the parameter \lstinline{hand}'s type is \lstinline{card*}; that is, \lstinline{}{hand} is a pointer to a \lstinline{}{card}.
In the code, though, we treat \lstinline{}{hand} as though it were an array.
This is because in C, arrays are pointers, and we can treat pointers as arrays.
Now look at the rest of the code in \function{is_pair()}.
Why does this return a \textbf{1} when the hand contains at least one pair?
Why does it return a \textbf{0} when the hand contains no pairs?
If you can't determine this on your own, you may talk it over with other students or the TA. Type your answer in \textit{answers.txt}.

Look at the code for \function{is_flush()}.
Why does this return a \textbf{1} when all cards in the hand have the same suit?
Why does it return a \textbf{0} when at least two cards have different suits than each other?
If you can't determine this on your own, you may talk it over with other students or the TA. Type your answer in \textit{answers.txt}.

Look at the code for \function{is_straight()}.
This is a little more challenging to understand than \function{is_pair()} and \function{is_flush()}.
Why does it return a \textbf{1} when all cards in the hand are in sequence?
Why does it return a \textbf{0} when they are not in sequence?
If you can't determine this on your own, you may talk it over with other students or the TA. Type your answer in \textit{answers.txt}.

Look at the code for \function{is_two_pair()}.
Recall that in C, arrays are pointers.
The assignment \lstinline{partial_hand = hand + i} makes use of \textit{pointer arithmetic}.
If the assignment were \lstinline{partial_hand = hand} then it would assign \lstinline{hand}'s base address to \lstinline{partial_hand}, and so \lstinline{partial_hand} would point to the $0^{th}$ element of \lstinline{hand}.
The expression $hand+i$ generates the address for the $i^{th}$ element of \lstinline{}{hand}, and so \lstinline{partial_hand = hand + i} assigns to \lstinline{partial_hand} the address of the $i^{th}$ element of \lstinline{hand}.
This effectively makes \lstinline{partial_hand} an array such that $\forall j : \mathtt{partial\_hand[j] = hand[i+j]}$.

Examine the remaining starter code in \textit{poker.c} to make sure you understand it.

\subsection{Complete the code} \label{subsec:completepoker}

Write the code in \textit{poker.c}'s \function{main()} function to generate a hand of five cards by calling \function{get_hand()}.\footnote{When you test the other functions you need to write, you might want to temporarily bypass \function{get_hand()} and explicitly assign specific cards to an array of five \lstinline{card}s.}
(Uncomment the \lstinline{int size_of_hand = 5} line if you previously commented it.)
Then have the program print out the five cards in the hand.
Finally, by calling the \function{is...()} functions, determine the best-possible characterization of the hand and print out that information.

Now write the code for \function{is_three_of_kind()}, \function{is_full_house()}, and \function{is_four_of_kind()}.