%! suppress = MissingImport
%%
%% I/O Prelab (c) 2021-22 Christopher A. Bohn
%%
%% Licensed under the Apache License, Version 2.0 (the "License");
%% you may not use this file except in compliance with the License.
%% You may obtain a copy of the License at
%%     http://www.apache.org/licenses/LICENSE-2.0
%% Unless required by applicable law or agreed to in writing, software
%% distributed under the License is distributed on an "AS IS" BASIS,
%% WITHOUT WARRANTIES OR CONDITIONS OF ANY KIND, either express or implied.
%% See the License for the specific language governing permissions and
%% limitations under the License.
%%

%%
%% labs/common/assignment.tex
%% (c) 2021-22 Christopher A. Bohn
%%
%% Licensed under the Apache License, Version 2.0 (the "License");
%% you may not use this file except in compliance with the License.
%% You may obtain a copy of the License at
%%     http://www.apache.org/licenses/LICENSE-2.0
%% Unless required by applicable law or agreed to in writing, software
%% distributed under the License is distributed on an "AS IS" BASIS,
%% WITHOUT WARRANTIES OR CONDITIONS OF ANY KIND, either express or implied.
%% See the License for the specific language governing permissions and
%% limitations under the License.
%%

\documentclass[12pt]{article}

\usepackage{fullpage}
\usepackage{fancyhdr}
\usepackage[procnames]{listings}
\usepackage{hyperref}
\usepackage{textcomp}
\usepackage{bold-extra}
\usepackage[dvipsnames]{xcolor}
\usepackage{etoolbox}

% These are placeholder commands and will be renewed in each lab

\newcommand{\labnumber}{}
\newcommand{\labname}{Lab \labnumber\ Assignment}
\newcommand{\shortlabname}{}
\newcommand{\duedate}{}

% Individual or team effort

\newcommand{\individualeffort}{This is an individual-effort project. You may
    discuss concepts and syntax with other students, but you may discuss
    solutions only with the professor and the TAs. Sharing code with or copying
    code from another student or the internet is prohibited.}
\newcommand{\teameffort}{This is a team-effort project. You may discuss concepts
    and syntax with other students, but you may discuss solutions only with your
    assigned partner(s), the professor, and the TAs. Sharing code with or
    copying code from a student who is not on your team, or from the internet,
    is prohibited.}
\newcommand{\freecollaboration}{In addition to the professor and the TAs, you
    may freely seek help on this assignment from other students.}
\newcommand{\collaborationrules}{}

% Software engineering (if you care about that)

\providebool{allowspaghetticode}

\newcommand{\softwareengineeringfrontmatter}{
    \ifboolexpe{not bool{allowspaghetticode}}{
        \section*{No Spaghetti Code Allowed}
        In the interest of keeping your code readable, you may \textit{not} use
        any \lstinline{goto} statements, nor may you use any
        \lstinline{continue} statements, nor may you use any \lstinline{break}
        statements to exit from a loop, nor may you have any functions
        \lstinline{return} from within a loop.
    }{}
}

\newcommand{\spaghetticodepenalties}[1]{
    \ifboolexpe{not bool{allowspaghetticode}}{
        \penaltyitem{1}{for each \lstinline{goto} statement,
            \lstinline{continue} statement, \lstinline{break} statement used to
            exit from a loop, or \lstinline{return} statement that occurs within
            a loop.}
    }{}
}

% You shouldn't need to customize these,
% but you can if you like

\lstset{language=C, tabsize=4, upquote=true, basicstyle=\ttfamily}
\newcommand{\function}[1]{\textbf{\lstinline{#1}}}
\setlength{\headsep}{0.7cm}
\hypersetup{colorlinks=true}

\newcommand{\pagelayout}{
    \pagestyle{fancy}
    \fancyhf{}
    \lhead{\coursenumber}
    \chead{\ Lab \labnumber: \labname}
    \rhead{\courseterm}
    \cfoot{\shortlabname-\thepage}
}

\newcommand{\labidentifier}{
    \title{\ Lab \labnumber}
    \author{\labname}
    \date{Due: \duedate}
    \maketitle

    \textit{\collaborationrules}
}

% deprecated
\newcommand{\startdocument}{
    \pagelayout
	\begin{document}
	\labidentifier
}

\newcommand{\rubricitem}[2]{\item[\underline{\hspace{1cm}} +#1] #2}
\newcommand{\bonusitem}[2]{\item[\underline{\hspace{1cm}} Bonus +#1] #2}
\newcommand{\penaltyitem}[2]{\item[\underline{\hspace{1cm}} -#1] #2}
\newcommand{\checkoffitem}[1]{\item (\phantom{xxx}) #1}
\newcommand{\precheckoffitem}[1]{\item [] (\phantom{xxx}) #1}

%%
%% labs/common/semester.tex
%% (c) 2021-22 Christopher A. Bohn
%%
%% Licensed under the Apache License, Version 2.0 (the "License");
%% you may not use this file except in compliance with the License.
%% You may obtain a copy of the License at
%%     http://www.apache.org/licenses/LICENSE-2.0
%% Unless required by applicable law or agreed to in writing, software
%% distributed under the License is distributed on an "AS IS" BASIS,
%% WITHOUT WARRANTIES OR CONDITIONS OF ANY KIND, either express or implied.
%% See the License for the specific language governing permissions and
%% limitations under the License.
%%


% Customize the semester (or quarter) and the course number

\newcommand{\courseterm}{Spring 2022}
\newcommand{\coursenumber}{CSCE 231}

% Customize how a typical lab will be managed;
% you can always use \renewcommand for one-offs

\newcommand{\runtimeenvironment}{your account on the \textit{csce.unl.edu} Linux server}
\newcommand{\filesource}{Canvas or {\footnotesize$\sim$}cse231 on \textit{csce.unl.edu}}
\newcommand{\filesubmission}{Canvas}

% Customize for the I/O lab hardware

\newcommand{\developmentboard}{Arduino Nano}

%\newcommand{\serialprotocol}{SPI}
\newcommand{\serialprotocol}{I2C}

%\newcommand{\displaymodule}{MAX7219digits}
%\newcommand{\displaymodule}{MAX7219matrix}
\newcommand{\displaymodule}{LCD1602}

\setbool{usedisplayfont}{true}

\newcommand{\obtaininghardware}{
    The EE Shop has prepared ``class kits'' for CSCE 231; your class kit costs \$20. The EE Shop is located at 122 Scott
    Engineering Center and is open M-F 7am-4pm. You do not need an appointment. You may pay at the window with cash,
    with a personal check, or with your NCard. If you wish to pay by credit card, you must make the purchase from
    \url{https://marketplace.unl.edu/ees/engineering-class-kits/csce231-kit.html} the day before you visit the EE
    Shop.\footnote{The price listed on the website is \$18.65; after sales tax is added, your total will be \$20.}
}

% Update to reflect the CS2 course(s) at your institute

\newcommand{\cstwo}{CSCE~156, RAIK~184H, or SOFT~161}

% Do you care about software engineering?

\setbool{allowspaghetticode}{false}

% Which assignments are you using this semester, and when are they due?

\newcommand{\pokerlabnumber}{1}
\newcommand{\pokerlabcollaboration}{Except as noted in Section~\ref{StudyTheCode}, \individualeffort}
\newcommand{\pokerlabdue}{Week of January 24, before the start of your lab section}

\newcommand{\keyboardlabnumber}{2}
\newcommand{\keyboardlabcollaboration}{\individualeffort}
\newcommand{\keyboardlabdue}{Week of January 31, before the start of your lab section}

\newcommand{\pointerlabnumber}{3}
\newcommand{\pointerlabcollaboration}{\individualeffort}
\newcommand{\pointerlabdue}{Week of February 7, before the start of your lab section}

\newcommand{\integerlabnumber}{4}
\newcommand{\integerlabcollaboration}{\individualeffort}
\newcommand{\integerlabdue}{Week of February 14, before the start of your lab section}

\newcommand{\floatlabnumber}{5}
\newcommand{\floatlabcollaboration}{\individualeffort}
\newcommand{\floatlabdue}{soon}

\newcommand{\addressinglabnumber}{6}
\newcommand{\addressinglabcollaboration}{\individualeffort}
\newcommand{\addressinglabdue}{Week of February 28, before the start of your lab section}

%bomblab was 7
%attacklab was 8

\newcommand{\pollinglabnumber}{9}
\newcommand{\pollinglabcollaboration}{\individualeffort}
\newcommand{\pollinglabdue}{Week of April 11, before the start of your lab section}
\newcommand{\pollinglabenvironment}{your \developmentboard-based class hardware kit}

\newcommand{\ioprelabnumber}{\pollinglabnumber-prelab}
\newcommand{\ioprelabcollaboration}{\freecollaboration}
\newcommand{\ioprelabdue}{Before the start of your lab section on April 5 or 6}

\newcommand{\interruptlabnumber}{10}
\newcommand{\interruptlabcollaboration}{\individualeffort}
\newcommand{\interruptlabdue}{Week of April 18, before the start of your lab section}
\newcommand{\interruptlabenvironment}{your \developmentboard-based class hardware kit}

\newcommand{\capstonelab}{ComboLock}    % this will come into play when we generalize capstonelab
\newcommand{\capstonelabnumber}{11}
\newcommand{\capstonelabcollaboration}{\teameffort}
\newcommand{\capstonelabdue}{Week of May 2, Before the start of your lab section\footnote{See Piazza for the due dates of teams with students from different lab sections.}}
\newcommand{\capstonelabenvironment}{your \developmentboard-based class hardware kit}

\newcommand{\memorylabnumber}{12}
\newcommand{\memorylabcollaboration}{This is an individual-effort project. You may discuss the nature of memory technologies and of memory hierarchies with classmates, but you must draw your own conclusions.}
\newcommand{\memorylabdue}{Week of May 2, at the end of your lab section}
\newcommand{\memorylabenvironment}{your \developmentboard-based class hardware kit and your account on the \textit{csce.unl.edu} Linux server}

% Labs not used this semester

\newcommand{\concurrencylabnumber}{XX}
\newcommand{\concurrencylabcollaboration}{\individualeffort}
\newcommand{\concurrencylabdue}{not this semester}

\newcommand{\ssbcwarmupnumber}{XX}
\newcommand{\ssbcwarmupcollaboration}{\freecollaboration}
\newcommand{\ssbcwarmupdue}{not this semester}

\newcommand{\ssbcpollingnumber}{XX}
\newcommand{\ssbcpollingcollaboration}{\individualeffort}
\newcommand{\ssbcpollingdue}{not this semester}

\newcommand{\ssbcinterruptnumber}{XX}
\newcommand{\ssbcinterruptcollaboration}{\individualeffort}
\newcommand{\ssbcinterruptdue}{not this semester}

%! suppress = NonMatchingIf

% Text expansion

\newcommand{\power}{power~({\color{red}\textbf{+}}) rail}
\newcommand{\ground}{ground~({\color{blue}\textbf{--}}) rail}
\newcounter{checkpoint}
\newcommand{\checkpoint}[1]{
    \stepcounter{checkpoint}\vspace{1cm}
    \textbf{\textsc{CheckPoint}~\thecheckpoint:}
    Before proceeding further, have a TA or a classmate verify that you have correctly #1.
    Update \textit{checkpoints.txt} file to indicate who checked your work and when they did so.
    \vspace{1cm}
}
\newcommand{\disconnect}{\textbf{Before proceeding further, disconnect the USB cable from the \developmentboard.}}
\newcommand{\rainbow}{male-to-male rainbow cable}


% Customization variables

\newcommand{\microcontroller}{}
\newcommand{\formfactor}{}
\providebool{fivevolt}

\ifdefstring{\developmentboard}{Arduino Nano}{
    \renewcommand{\microcontroller}{ATmega328P}
    \renewcommand{\formfactor}{nano}
}{}

\ifdefstring{\microcontroller}{ATmega328P}{
    \setbool{fivevolt}{true}
}{}

%! suppress = NonMatchingIf
\newcommand{\namerows}[1]{
    \draw (#1,1)  node {\tiny{a}};
    \draw (#1,2)  node {\tiny{b}};
    \draw (#1,3)  node {\tiny{c}};
    \draw (#1,4)  node {\tiny{d}};
    \draw (#1,5)  node {\tiny{e}};
    \draw (#1,8)  node {\tiny{f}};
    \draw (#1,9)  node {\tiny{g}};
    \draw (#1,10) node {\tiny{h}};
    \draw (#1,11) node {\tiny{i}};
    \draw (#1,12) node {\tiny{j}};
}

\newcommand{\drawic}[4]{
    \draw[gray,very thin] (#1,#2) +(-.4,-.4) rectangle +(#3-.6,3.4);
    \draw[gray,very thin]  (#1,#2) +(-.4,1) arc [start angle=-90, end angle=90, radius=.5];
    \draw[black] (#1,#2) +(-.2,-.2) rectangle +(#3-.8,.2);
    \draw[black] (#1,#2) +(-.2,2.8) rectangle +(#3-.8,3.2);
    \draw (#1,#2) +(#3*.5-.5,1.5) node {\centering \tiny{#4}};
}

\newcommand{\drawtarget}[2]{\draw[black] (#1,#2) circle (2pt);}

%! suppress = MissingImport
\newcommand{\drawlabelledtarget}[4]{
    \drawtarget{#1}{#2}
    \ifnumequal{#3}{1}{\draw (#1,#2) +( 4pt, 4pt) node {\rotatebox{ 45}{\scalebox{.4}{#4}}};}{}
    \ifnumequal{#3}{2}{\draw (#1,#2) +(-4pt, 4pt) node {\rotatebox{-45}{\scalebox{.4}{#4}}};}{}
    \ifnumequal{#3}{3}{\draw (#1,#2) +(-4pt,-4pt) node {\rotatebox{ 45}{\scalebox{.4}{#4}}};}{}
    \ifnumequal{#3}{4}{\draw (#1,#2) +( 4pt,-4pt) node {\rotatebox{-45}{\scalebox{.4}{#4}}};}{}
}

\newcommand{\drawx}[2]{
    \draw (#1,#2) +(-.2,-.2) -- +(.2,.2);
    \draw (#1,#2) +(-.2,.2) -- +(.2,-.2);
}

\newcommand{\drawnanopower}[1]{
    \drawtarget{#1+11}{12}
    \draw (#1+11,12) -- (#1+11,20) -- (9,20) node[ground] {\tiny{upper ground rail}};
    \drawlabelledtarget{#1+13}{1}{1}{a24} % direct connection to NAND ground
%    \draw (#1+13,1) -- (#1+13,-2) node[ground] {\tiny{lower ground rail}};
    \ifbool{fivevolt}{
        \drawtarget{#1+11}{1}
        \draw (#1+11,1) -- (#1+11,-3)
    }{
        \drawtarget{#1+1}{1}
        \draw (#1+1,1) -- (#1+1,-3)
    } -- (-2,-3) -- (-2,17) node[vcc] {\tiny{upper power rail}};
}

\newcommand{\drawnandpower}[1]{
    \foreach \x in {0,1,2} {
        \drawtarget{#1+\x}{12}
        \draw[rounded corners] (#1,12) +(\x,0) -- +(\x,5) -- (#1-.5,17);
    }
    \draw (#1,17) -- (16.5,17);
    \draw (16.5,17) arc [start angle=0, end angle=180, radius=.5];
    \draw (15.5,17) -- (12.5,17);
    \draw (12.5,17) arc [start angle=0, end angle=180, radius=.5];
    \draw (11.5,17) -- (-2,17);
    \drawlabelledtarget{#1+6}{1}{1}{a14} % direct connection to Nano lower ground
%    \draw (#1+6,1) -- (#1+6,-2) -- (14,-2);
}

%! suppress = Ellipsis
\newcommand{\drawbreadboard}{
    \foreach \x in {1,2,...,63} {
        \foreach \y in {1,2,...,5} {
            \draw[white,fill=gray] (\x,\y) circle (1pt);
        }
        \foreach \y in {8,9,...,12} {
            \draw[white,fill=gray] (\x,\y) circle (1pt);
        }
        \draw (\x,-.5)  node {\rotatebox{45}{\tiny{$\x$}}};
        \draw (\x,13.5) node {\rotatebox{45}{\tiny{$\x$}}};
    }
    \foreach \x in {5,10,...,60} {
        \draw[gray,very thin] (\x,-.3) -- (\x,.5);
    }
    \drawx{1}{1}
    \drawx{1}{12}
    \drawx{63}{1}
    \drawx{63}{12}
    \namerows{0}
    \namerows{64}
    \namerows{16.5}
    \namerows{44.5}
}

\newcommand{\drawnano}[3] {
    \draw[gray,very thin] (#1,#2) +(-1.4,-.4) rectangle +(15.4,6.4);
    \draw[black] (#1,#2) +(-.2,-.2)  rectangle +(14.2,.2);
    \draw[black] (#1,#2) +(-.2,5.8)  rectangle +(14.2,6.2);
    \draw (#1,#2) +(7,3.5) node {\centering \tiny{#3}};
    \draw (#1,#2) +(-1.8,3) node {\rotatebox{90}{\centering \tiny{USB}}};
    \drawnanopower{#1}
}

\newcommand{\drawnand}[2]{
    \drawic{#1}{#2}{7}{\ifbool{fivevolt}{74LS20}{74HC20}}
    \drawnandpower{#1}
}

\newcommand{\drawswitch}[2]{
    \draw[gray,very thin] (#1,#2) +(-.4,-.4) rectangle +(2.4,.4);
    \draw[black] (#1,#2) +(-.2,-.2) rectangle +(2.2,.2);
    \draw (#1,#2) +(1,-.6) node {\centering \tiny{switch}};
}

\newcommand{\drawbutton}[3]{
    \ifnumequal{#3}{2}{
        \draw[gray,very thin] (#1,#2) +(-.25,-1.25) rectangle +(2.25,1.25);
        \draw[black] (#1,#2) +(-.2,-.2) rectangle +(.2,.2);
        \draw[black] (#1,#2) +(1.8,-.2) rectangle +(2.2,.2);
        \draw (#1,#2) +(1,-.6) node {\centering \tiny{button}};
    }{
        \draw[gray,very thin] (#1,#2) +(-.25,.2) rectangle +(2.25,2.8);
        \draw[black] (#1,#2) +(-.2,-.2) rectangle +(.2,.2);
        \draw[black] (#1,#2) +(1.8,-.2) rectangle +(2.2,.2);
        \draw[black] (#1,#2) +(-.2,2.8) rectangle +(.2,3.2);
        \draw[black] (#1,#2) +(1.8,2.8) rectangle +(2.2,3.2);
        \draw (#1,#2) +(1,.6) node {\centering \tiny{button}};
    }
}

\newcommand{\drawkeypad}[2]{
    \draw[gray,very thin] (#1,#2) +(-.4,-.4) rectangle +(7.4,.4);
    \draw[black] (#1,#2) +(-.2,-.2) rectangle +(7.2,.2);
    \draw (#1,#2) +(1.5,.5) node {\centering \tiny{rows}};
    \draw (#1,#2) +(5.5,.6) node {\centering \tiny{cols}};
    \draw (#1,#2) +(3.5,-.6) node {\centering \tiny{keypad}};
}

\newcommand{\drawiiclcd}[2]{
    \draw[gray,very thin] (#1,#2) +(-.4,-.4) rectangle +(15.4,6.4);
    \draw[black] (#1,#2) +(-.2,-.2) rectangle +(15.2,.2);
    \foreach \y in {1.5,2.5,3.5,4.5} {
        \draw[gray,very thin] (#1,#2) +(-2.5,\y) -- +(.5,\y);
    }
    \draw[gray!50,fill=gray!50] (#1,#2) +(15.5,2.4) rectangle +(17,3.6);
    \draw[gray] (#1,#2) +(14.5,2.5) -- +(17,2.5);
    \draw[gray] (#1,#2) +(14.5,3.5) -- +(17,3.5);
    \draw (#1,#2) +(7.5,5.5) node {\centering \tiny{LCD Serial Adapter}};
}

\newcommand{\drawlcdisplay}[2]{
    \draw[gray,very thin] (#1,#2) +(-3,.4) rectangle +(28.4,-13.4);
    \draw[gray,very thin] (#1,#2) +(-1,-1.8) rectangle +(26,-11.4);
    \draw[black] (#1,#2) +(-.2,-.2) rectangle +(15.2,.2);
    \draw (#1,#2) +(8,-12.5) node {\centering \tiny{LCD 2x16-character Display Module}};
}

\newcommand{\drawledcircuit}{
    \ctikzset{bipoles/resistor/height=0.1}
    \ctikzset{bipoles/resistor/width=0.3}
    \ctikzset{bipoles/diode/height=0.2}
    \ctikzset{bipoles/diode/width=0.2}
    \draw (.8,10.8)   rectangle (1.2,11.2);
    \draw (15.8,10.8) rectangle (16.2,11.2);
    \draw (15.8,11.8) rectangle (16.2,12.2);
    \draw (1,11) -- (1.5,10.5) to[R] (15.5,10.5) -- (16,11);
    \draw (16,12) -- (16,14) to[led] (16,17) -- (16,20) -- (9,20);
}

\newcommand{\drawswitches}[1]{
    \foreach \x in {0,5} {
        \drawswitch{#1+\x}{1}
        \drawtarget{#1+\x}{5}
        \draw[rounded corners] (#1+\x,5) -- ++(1,2-\x/10) -- ++(6-.8*\x,0) -- ++(.5,1+\x/10) -- +(0,12) -- (9,20);
        \drawtarget{#1+\x+1}{5}
    }
    \drawtarget{3}{12}
    \drawtarget{2}{12}
}

\newcommand{\drawbuttons}[2]{
    \foreach \x in {0,4} {
        \drawbutton{#1+\x}{1}{#2}
        \drawtarget{#1+\x}{5}
        \drawtarget{#1+\x+2}{5}
        \draw[rounded corners] (#1+\x+2,5) -- +(.5,1) -- +(.5,15) -- (9,20);
    }
    \drawtarget{22}{10} \drawtarget{22}{11} \drawtarget{5}{12}
    \drawtarget{23}{10} \drawtarget{23}{11} \drawtarget{4}{12}
    \drawtarget{24}{11} \drawtarget{11}{12}
}

%! suppress = Ellipsis
\newcommand{\drawkeypadandtargets}[1]{
    \drawkeypad{#1}{12}
    \foreach \x in {0,1,...,7} {
        \drawtarget{#1+\x}{10}
    }
    \foreach \x in {6,7,8,9,10} {
        \drawtarget{\x}{12}
    }
    \drawtarget{23}{3}
    \foreach \x in {18,19,21,22} {
        \drawtarget{\x}{2}
        \drawtarget{\x}{3}
    }
    \foreach \x in {4,5,6,7} {
        \drawtarget{\x}{1}
    }
}

\newcommand{\drawdisplay}[1]{
    \drawiiclcd{#1}{11}
    \drawlcdisplay{#1}{9}
    \drawtarget{8}{1}
    \drawtarget{9}{1}
}


\renewcommand{\labnumber}{\ioprelabnumber}
\renewcommand{\labname}{Physical Assembly of Hardware for I/O Labs}
\renewcommand{\shortlabname}{i/o-prelab}
\renewcommand{\collaborationrules}{\ioprelabcollaboration}
\renewcommand{\duedate}{\ioprelabdue}


\pagelayout
\begin{document}
\labidentifier

In the I/O labs, we will use a microcontroller board with some peripherals.
In this prelab, you will assemble the hardware for the I/O labs.

\section{Obtaining the Hardware} \obtaininghardware

\section{Inventorying the Hardware} \label{sec:inventory} %! suppress = NonMatchingIf
\newcommand{\devboardimage}{}
\newcommand{\nandchipitem}{}
\newcommand{\displaymoduleitem}{}
\newcommand{\fmcableitem}{}

\ifdefstring{\developmentboard}{Arduino Nano}{
    \renewcommand{\devboardimage}{\includegraphics[height=2cm]{inventory/nano}}
    \renewcommand{\nandchipitem}{\item One (1) 74LS20\footnote{Any 74x20 or 54x20 integrated circuit is acceptable.} dual 4-input NAND integrated circuit \\ \includegraphics[height=2cm]{inventory/nand}}
}{}
\ifdefstring{\displaymodule}{MAX7219digits}{
    \renewcommand{\displaymoduleitem}{\item One (1) 8-digit 7-segment display module \\ \includegraphics[height=2cm]{inventory/max7219digits}}
    \renewcommand{\fmcableitem}{\item One (1) 5-conductor 20cm ``rainbow'' cable (female-to-male) \\ \includegraphics[height=2cm]{fm-5cable}}
}{}
\ifdefstring{\displaymodule}{MAX7219matrix}{
    \renewcommand{\displaymoduleitem}{\item One (1) $8 \times 8$ LED matrix display module \\ \textbf{\textit{Add LED matrix image here}}} %\includegraphics[height=2cm]{inventory/max7219matrix}}
}{}
%! suppress = MissingImport
\ifdefstring{\displaymodule}{LCD1602}{
    \renewcommand{\displaymoduleitem}{
        \item One (1) $2 \times 16$ character LCD display module \\ \includegraphics[height=2cm]{inventory/lcd1602}
        \ifdefstring{\serialprotocol}{I2C}{ \item One (1) I$^2$C-LCD Serial Interface (might be attached to display module) \\ \includegraphics[height=2cm]{inventory/lcd-adapter} \hspace{1cm} or \hspace{1cm} \includegraphics[height=2cm]{inventory/adafruit-lcd-adapter} \hspace{1cm} or \hspace{1cm} \includegraphics[height=2cm]{inventory/piggyback-lcd-adapter} }{}
        \ifdefstring{\serialprotocol}{SPI}{\item One (1) 74HC595 8-bit shift register \\ \textbf{\textit{Add shift register image here}}}{} %\includegraphics[height=2cm]{inventory/shiftregister}}{}
    }
    \ifdefstring{\serialprotocol}{I2C}{\renewcommand{\fmcableitem}{\item One (1) 4-conductor 20cm ``rainbow'' cable (female-to-male) \\ \includegraphics[height=2cm]{inventory/fm-4cable}}}{}
}{}

%%%%%%%%%%

Examine the contents of your class kit.
It contains:

%! suppress = MissingImport
\begin{itemize}
    \item One (1) full-sized solderless breadboard \\
        \includegraphics[height=2cm]{inventory/breadboard}
    \item One (1) \developmentboard\ (or clone) microcontroller board \\
        \devboardimage
    \item One (1) USB cable (mini-USB shown;
        yours may be different) \\
        \includegraphics[height=2cm]{inventory/usb}
    \nandchipitem
    \item One (1) $4 \times 4$ matrix keypad \\
        \includegraphics[height=2cm]{inventory/keypad}
    \item One (1) 8-pin male-male header strip (might already be inserted into keypad's female connectors;
        might have more than 8 pins) \\
        \includegraphics[height=2cm]{inventory/keypad-header-in-connector} \hspace{1cm} or
        \hspace{1cm} \includegraphics[height=2cm]{inventory/keypad-header-without-connector}
    \item Two (2) breadboard-mount momentary pushbuttons, aka tactile switches;
        these might have two leads (which might or might not be attached to cardboard strip), or they might have 4 prongs \\
        \includegraphics[height=2cm]{inventory/buttons-2pin} \hspace{1cm} or
        \hspace{1cm} \includegraphics[height=2cm]{inventory/buttons-4pin}
    \item Two (2) breadboard-mount slide switches. \\
%    \item Two (2) breadboard-mount slide switches;
%        these might have three pins spaced 0.1in (2.54mm) apart, or they might have two pins spaced 0.3in (7.62mm) apart. \\
        \includegraphics[height=2cm]{inventory/sliders-spdt}% \hspace{1cm} or
        %\hspace{1cm} \textbf{\textit{Add dip switch image here}} %\includegraphics[height=2cm]{inventory/sliders-dip1}
    \displaymoduleitem
    \item One (1) Light Emitting Diode (LED) (color may be different from shown) \\
        \includegraphics[height=1cm]{inventory/led}
    \item One (1) 1k$\Omega$ resistor \\
        \includegraphics[height=1cm]{inventory/resistor}
    \item One (1) 40-conductor 10cm ``rainbow'' cable (male-to-male), \textit{or} One (1) 20-conductor 10cm ``rainbow'' cable (male-to-male) and one (1) 20-conductor 20cm ``rainbow'' cable (male-to-male) \\
        \includegraphics[height=2cm]{inventory/mm-cable}
    \fmcableitem
\end{itemize}

There may be other items in the class kit.
Set these aside;
you will not need them for this prelab, though they may be used in a specific lab.


\vspace{0.5cm}

\section*{Assembling the Class Kit}

    You will assemble the hardware in the following steps.
    \textbf{At various checkpoints, you should pause to have a TA or classmate double-check your work.}
    When you do so, update the \textit{checkpoints.txt} file to indicate who checked your work and when they did so.

    You may want to store your partially- and fully-completed kit in a plastic food container or some other container to prevent jumper wires from being pulled out while in your backpack.

    \textit{
        \textbf{Note:} The following pages include both diagrams and photographs of the assembly.
        The wire colors in the diagrams do not match the wire colors in the assembly.
        The wire colors in the diagrams are coded by the purpose they serve, whereas the wire colors in the photographs are the colors of wires removed from the \rainbow.
    }

\section{Microcontroller and IDE} %! suppress = MissingImport
\newcommand{\microcontrollerreference}{}
\newcommand{\pindescription}{}
\newcommand{\icspdescription}{}
\newcommand{\usartdescription}{}
\newcommand{\digitalpindescription}{}
\newcommand{\analogpindescription}{}
\newcommand{\regulatedvoltagedescription}{}
\newcommand{\unregulatedvoltagedescription}{}
\newcommand{\resetdescription}{Finally, the \texttt{RESET} pins will reset the \developmentboard\ if grounded (pressing the button in the middle of the \developmentboard\ will also reset it).}
\newcommand{\microcontrollerprocessorandmemory}{}
\newcommand{\microcontrollerintegertiming}{}
\newcommand{\microcontrollerdivisionandfloats}{There is no hardware divider, and there is no floating point hardware, so integer division (to include the modulo operation) and all floating point operations are performed in software, requiring hundreds of clock cycles.}
\newcommand{\memorymodeldescription}{If you have already read the first half of Chapter 8, the \microcontroller\ has separate instruction and data memory, similar to the simple processor design described in the first half of Chapter 8.}
\newcommand{\pipelinedescription}{If you have already read the second half of Chapter 8, the \microcontroller\ has a 2-stage pipeline (with \textit{Fetch} and \textit{Execute} stages).}

%! suppress = NonMatchingIf
\ifdefstring{\microcontroller}{ATmega328P}{
    \renewcommand{\microcontrollerreference}{Atmel ATmega328P\footnote{\url{https://ww1.microchip.com/downloads/en/DeviceDoc/Atmel-7810-Automotive-Microcontrollers-ATmega328P_Datasheet.pdf}}}
    \renewcommand{\icspdescription}{The six upward-pointing pins are used to program the \developmentboard\ without using a host computer; we will not use these.}
    \renewcommand{\usartdescription}{\texttt{RX0} and \texttt{TX1} are used for asynchronous serial communication; as the USB interface also uses the same corresponding pins on the \microcontroller, we will not use these two pins (you will notice that when the \developmentboard\ communicates with the host computer, the \texttt{RX} and \texttt{TX} LEDs will illuminate).}
    \renewcommand{\microcontrollerprocessorandmemory}{an 8-bit processor with 32KB of flash memory for the program and 2KB of RAM for data}
    \renewcommand{\microcontrollerintegertiming}{While 8-bit logical operations, as well as 8-bit addition and subtraction, can be completed in one clock cycle, multiplication requires two clock cycles (16-bit operations require additional clock cycles).}
}{}

%! suppress = NonMatchingIf
\ifdefstring{\formfactor}{nano}{
    \renewcommand{\analogpindescription}{Pins \texttt{A0}-\texttt{A7} are analog input pins; however, \texttt{A0}-\texttt{A5} can also be used as digital input/output pins \texttt{D14}-\texttt{D19}. \texttt{AREF} (analog reference) is used to provide a reference voltage for the ADC (we will not use this pin).}
    \renewcommand{\pindescription}{It has thirty downward-pointing pins.}
    \renewcommand{\digitalpindescription}{Pins \texttt{D2}-\texttt{D13} are digital input/output pins.}
    \renewcommand{\unregulatedvoltagedescription}{\texttt{VIN} can be used to power the \developmentboard\ if connected to an unregulated power supply, such as a 9V battery; the \developmentboard's onboard voltage regulator will then provide regulated voltages needed.}
}{}

%! suppress = NonMatchingIf
\ifbool{fivevolt}{
    \renewcommand{\regulatedvoltagedescription}{Pins \texttt{3V3} and \texttt{5V} provide regulated 3.3 volts and 5 volts for external circuitry; \texttt{5V} can also be used to power the \developmentboard\ if connected to a regulated 5V power supply.}
}{
    \renewcommand{\regulatedvoltagedescription}{The \texttt{3V3} pin provides regulated 3.3 volts for external circuitry; it can also be used to power the \developmentboard\ if connected to a regulated 3.3V power supply. The \texttt{VUSB} pin can provide regulated 5 volts only if the \texttt{VUSB} jumper pads on the \developmentboard's underside are soldered together.}
}

%%%%%%%%%%

A microcontroller, such as the \microcontrollerreference\ on the \developmentboard, is a very simple processor when compared to a microprocessor designed for general-purpose computing.
At the same time, a microcontroller has some features not present on a microprocessor, such as built-in analog-to-digital converters (ADCs).\footnote{We will not use the ADCs in the I/O labs.} A microcontroller board, such as the \developmentboard, combines the microcontroller with other components\footnote{Typically, a voltage regulator, a crystal oscillator, and a USB interface.} in a form factor convenient for experimentation.

The \developmentboard\ has a USB port to connect to a computer and/or to provide power to the \developmentboard.
\icspdescription\
\pindescription\
\usartdescription\
\digitalpindescription\
\analogpindescription\
\regulatedvoltagedescription\
\unregulatedvoltagedescription\
The \texttt{GND} pins are for the common ground;
the ground portions of external circuitry and of external power supplies must be electrically connected to the \developmentboard's ground.
\resetdescription
Note that, unlike a general-purpose computer, when a microcontroller is reset it will restart its program when the reset is released.

The \microcontroller\ microcontroller on the \developmentboard\ is \microcontrollerprocessorandmemory.
\microcontrollerintegertiming\
\microcontrollerdivisionandfloats\

\memorymodeldescription\
\pipelinedescription\
If you have already read Chapter 10, the \microcontroller\ does not have cache memory;
however, the data memory is SRAM, the same memory technology used in microprocessors' memory caches.
If you have already read Chapter 10, the \microcontroller\ does not have a memory management unit for virtual memory;
instead, the \microcontroller\ uses only physical addressing.

\subsection{Breadboard Terminology}

    If you are not familiar with solderless breadboards, read the
    \href{https://learn.adafruit.com/breadboards-for-beginners?view=all}{Breadboards for Beginners} Guide at adafruit.com.

    Even though breadboards are often viewed in ``landscape'' orientation (such as in the photo in Section~\ref{sec:inventory} and as seen in the diagram figures) instead of ``portrait'' orientation, the numbered sections are called rows and the lettered sections are called columns.
    In the interest of preserving common usage, we will use this terminology.
    We will refer to specific contact points using the letter-number combination.

\subsection{Optional: Breadboard Templates}                    %! suppress = MissingImport
Figure~\ref{fig:templates} is a set of two %four
templates for the Cow Pi circuit %(one for each combination of pushbuttons (2-lead or 4-prong) and slide-switches (2-pin or 3-pin).
(one with 2-lead pushbuttons and one with 4-prong pushbuttons)
Each dot (\tikz{\draw[white,fill=gray] (0,0) +(0,3pt) circle (1pt);}) represents a breadboard contact point.
Each dot with a circle (\tikz{\drawtarget{0}{3pt} \draw[white,fill=gray] (0,0) +(0,3pt) circle (1pt);}) is a contact point in which you will insert a jumper lead.
Attached to most of these circles is the contact point for the other end of the jumper wire (\tikz{\drawlabelledtarget{0}{0}{1}{k64} \draw[white,fill=gray] (0,0) circle (1pt);}).
The footprints of several components are shown as light-gray outlines;
resistors (\tikz{\ctikzset{bipoles/resistor/height=0.2}\ctikzset{bipoles/resistor/width=0.3}\draw (0,0) to[R] (1,0)}) and LEDs (\tikz{\ctikzset{bipoles/diode/height=0.2}\ctikzset{bipoles/diode/width=0.2}\draw (0,0) to[led] (1,0)}) are shown using their conventional symbols.
Squares (\tikz{\draw[white,fill=gray] (0,0) +(2pt,3pt) circle (1pt); \draw (0,0) +(0,1pt) rectangle +(4pt,5pt);}) are where you'll insert component individual pins,
and rectangles (\tikz{\draw (0,0) +(0,1pt) rectangle +(11pt,5pt); \draw[white,fill=gray] (0,0) +(2pt,3pt) circle (1pt); \draw[white,fill=gray] (0,0) +(5.5pt,3pt) circle (1pt); \draw[white,fill=gray] (0,0) +(9pt,3pt) circle (1pt);}) are where you'll insert components' in-line pins.
Finally, the four corners (\tikz{\draw[white,fill=gray] (0,0) +(2pt,3pt) circle (1pt); \draw (0,1pt) -- (4pt,5pt); \draw (0,5pt) -- (4pt,1pt);}) are used to align the template on your solderless breadboard.

%TODO: add dip1 switch subfigures
\begin{figure}[p]
    \subfloat[Template that uses 3-pin slide-switches and 2-lead pushbuttons.]{
        \hspace{-.5in}
        \begin{tikzpicture}[x=.1in, y=.1in]
            \drawbreadboard
            \drawnano{1}{3}{Arduino Nano}
            \drawledcircuit
            \drawnand{18}{5}
            \drawswitches{\controlsstartat+1}
            \drawbuttons{\controlsstartat+11}{2}
            \drawkeypadandtargets{\controlsstartat}
            \drawdisplay{48}
        \end{tikzpicture}
    }
    \vspace{.5in}
    \subfloat[Template that uses 3-pin slide-switches 4-prong pushbuttons.]{
        \hspace{-.5in}
        \begin{tikzpicture}[x=.1in, y=.1in]
            \drawbreadboard
            \drawnano{\mcux}{\mcuy}{Arduino Nano}
            \drawledcircuit
            \drawnand{\nandx}{\nandy}
            \drawswitches{\controlsstartat+1}
            \drawbuttons{\controlsstartat+11}{4}
            \drawkeypadandtargets{\controlsstartat}
            \drawdisplay{48}
        \end{tikzpicture}
    }
    \caption{Templates to improve accuracy when constructing the Cow Pi circuit on a solderless breadboard.}\label{fig:templates}%\addcontentsline{toc}{section}{Breadboard Templates}
\end{figure}

\textit{Optionally}, print the page that has the template appropriate to your particular switches and pushbuttons.
When (if) you do so, be sure to select ``Actual size'' (see Figure~\ref{fig:printmenu}).
If you mistakenly select a different option, the template will not line up properly with your breadboard: even a tiny scaling factor will add-up over the length of the breadboard.
Using the lead from a jumper wire, punch holes into the four $\times$s at the corners (contact points a1, j1, a63, and j32); see Figure~\ref{fig:punchingholes}.
Place the lead from a jumper wire into each of the four holes, and insert the leads into the corresponding contact points on the breadboard, pinning the template to the breadboard.
Confirm that the four jumpers are aligning the template to the breadboard by visually checking that the four leads are in the breadboard's contact points a1, j1, a63, and j32 (see Figure~\ref{fig:confirmalignment}).

\begin{figure}
    \centering
    \includegraphics[width=0.5\textwidth]{breadboard-guides/print-menu}
    \caption{When printing a breadboard template, be sure to select ``Actual size''.} \label{fig:printmenu}
\end{figure}

\begin{figure}
    \centering
    \subfloat[Punching alignment holes in breadboard template.]{
        \includegraphics[height=3cm]{breadboard-guides/punching-holes}
        \label{fig:punchingholes}
    }
    \hfil
    \subfloat[Confirming that the corners of the template are aligned with the corners of the breadboard.]{
        \includegraphics[height=3cm]{breadboard-guides/confirm-alignment}
        \label{fig:confirmalignment}
    }
    \caption{Preparing a breadboard template.}
\end{figure}



%! suppress = MissingImport
\subsection{Install the \developmentboard onto the Breadboard}

Orient the breadboard in front of you so that row 1 is on your left and row 63 is on your right;
column a should be at the bottom, and column j should be at the top.

Remove the anti-static foam from the \developmentboard's pins.
You will place the \developmentboard\ on the left side of the breadboard with the USB connector on the left (that is, facing away from the breadboard).
Position the upper row of pins on contact points g1-g15 and the lower row of pins on contact points c1-c15.
The left side of the \developmentboard\ will obscure the labels for columns c-g.
The right side of the \developmentboard\ will cover contact points c16-g16 but won't use them.
Double-check that:
\begin{itemize}
    \item the pin labeled \texttt{D12} is in the upper-left, on contact point g1
    \item the pin labeled \texttt{D13} is in the lower-left, on contact point c1
    \item the pin labeled \texttt{VIN} is in the lower-right, on contact point c15
    \item the pin labeled \texttt{TX1} is in the upper-right, on contact point g15
\end{itemize}

Gently press on both ends of the \developmentboard\ to insert the pins into the contact points, using a slight rocking motion if necessary (Figure~\ref{fig:inserting-mcu}).
Press the \developmentboard\ into the breadboard until it physically cannot be inserted any deeper (Figure~\ref{fig:mcu-inserted}).

%! suppress = NonMatchingIf
\begin{figure}
    \centering
    \subfloat[Press gently on both ends of the \developmentboard.] {
        \ifdefstring{\developmentboard}{Arduino Nano}{\includegraphics[height=3cm]{microcontroller/breadboard/inserting-nano}}{}
        \label{fig:inserting-mcu}
    }
    \hfil
    \subfloat[The \developmentboard\ fully inserted.] {
        \includegraphics[height=3cm]{microcontroller/breadboard/nano-fully-inserted}
        \label{fig:mcu-inserted}
    }
    \caption{Inserting the \developmentboard\ into the breadboard.}
\end{figure}

\checkpoint{inserted the \developmentboard\ into the breadboard}


\subsection{Install Arduino IDE}

    The Arduino IDE is installed on the lab computers.
    If you choose to install the Arduino IDE on your personal laptop, you can download it from
    \url{https://www.arduino.cc/en/software}.
    Alternatively, you can install a browser plugin to use the
    \href{https://create.arduino.cc/projecthub/Arduino_Genuino/getting-started-with-arduino-web-editor-on-various-platforms-4b3e4a}{Arduino Web Editor}.
    There are third-party plugins for many other IDEs; however, using these may limit our ability to help you if your have difficulties.

\subsubsection*{About Arduino Programs}

    An Arduino program is called a \textit{sketch} for historical reasons.\footnote{The Arduino language is based off of the Wiring language, which in turn is based off of the Processing language, which was designed to make computing accessible to artists.}
    For all intents and purposes, you can think of it as a C++ program\footnote{Your code in the I/O labs will be C code.} in which you write two functions, \function{setup} and \function{loop}, along with any helper code that you need.
    The file extension for sketches is \textbf{\textit{.ino}} (as in, Ardu\textbf{\textit{ino}}).
    The Arduino IDE will compile your sketch and link it to a \function{main} function that looks something like:
    \begin{lstlisting}
    int main() {
        setup();
        while(1) {
            loop();
        }
    }
    \end{lstlisting}
    (The actual \function{main} function\footnote{\url{https://github.com/arduino/ArduinoCore-avr/blob/master/cores/arduino/main.cpp}} also calls a few other functions from the Arduino core library.)

\subsection{Connect to the \developmentboard}

    Connect one end of the mini-USB cable to a lab computer or to your personal laptop.\footnote{You can connect it to a ``wall wart'' USB power supply to run the \developmentboard, but you need to connect it to a computer to upload a new sketch to the \developmentboard.}
    Connect the mini-USB end of the cable to your \developmentboard.
    The \texttt{PWR} LED will light up, and you may see the \texttt{L} LED repeatedly blink on-and-off.
    The \texttt{L} LED is connected to the \developmentboard's pin D13, and Arduino microcontroller boards typically leave the factory with \textit{Blink.ino} loaded, but it does not matter if yours does not have \textit{Blink.ino} pre-loaded.

    \begin{lstlisting}[basicstyle=\ttfamily\footnotesize]
    // the setup function runs once when you press reset or power the board
    void setup() {
      // initialize digital pin LED_BUILTIN as an output.
      pinMode(LED_BUILTIN, OUTPUT);
    }

    // the loop function runs over and over again forever
    void loop() {
      digitalWrite(LED_BUILTIN, HIGH);   // turn the LED on (HIGH is the voltage level)
      delay(1000);                       // wait for a second
      digitalWrite(LED_BUILTIN, LOW);    // turn the LED off by making the voltage LOW
      delay(1000);                       // wait for a second
    }
    \end{lstlisting}

    Open the Arduino IDE on the computer that your \developmentboard\ is connected to.
    Connect the Arduino IDE to the \developmentboard.
    %! suppress = MissingImport
%If you are using Arduino IDE 1.8, see this \href{https://www.arduino.cc/en/Guide/ArduinoNano#select-your-board-type-and-port}{Tutorial} for selecting the \developmentboard\ board, processor, and COM port (or this \href{https://www.arduino.cc/en/Guide/ArduinoUno#select-your-board-type-and-port}{Tutorial for the Arduino Uno}, which has more detail on selecting the COM port).
If you are using Arduino IDE 1.8, see the Quickstart Guide on the \href{https://docs.arduino.cc/hardware/nano}{\developmentboard's documentation page} for selecting the \developmentboard\ board, processor, and COM port.
If you are using Arduino IDE 2.0 or the Arduino Web Editor, COM port should be automatically detected.
You will still need to select the board and processor;
see Figure~\ref{fig:selecting-mcu} and the discussion in Section~\ref{subsubsec:processor-selection}.

\begin{figure}
    \centering
    \subfloat[Selecting the board with Arduino IDE 1.8.] {
        \includegraphics[width=7cm]{microcontroller/ide/selecting-nano-1_8}
%        \label{fig:selecting-board-1}
    }
    \hfil
    \subfloat[Selecting the board with Arduino IDE 2.0.] {
    % \includegraphics[width=7cm]{selecting-nano-from-menu}
        \includegraphics[height=4.1cm]{microcontroller/ide/selecting-nano}
%        \label{fig:selecting-board-2}
    }

    \subfloat[Selecting the processor after selecting the board.] {
    % \vtop{\vskip-4.1cm\hbox{\includegraphics[width=5cm]{selecting-nano-processor}}}
        \includegraphics[width=7cm]{microcontroller/ide/selecting-nano-processor}
        \label{fig:selecting-processor}
    }
    \caption{Selecting board and processor in the Arduino IDE.
    \label{fig:selecting-mcu}}
\end{figure}

\subsubsection{Selecting the Correct ``Processor''}\label{subsubsec:processor-selection}

There are \textit{three} choices for the \developmentboard{}'s processor, two of which specify the ATmega328P processor.
Even though the difference is a USB interface issue, it is resolved through the Arduino IDE's ``Processor'' selection:

\begin{itemize}
    \item Official \developmentboard{}s use the FT232RL USB interface chip.
    Under the ``Tools'' menu, when choosing ``Processor'', select ``ATmega328P''.
    \item \textit{Most} \developmentboard\ clones use the CH340 USB interface chip.
    Under the ``Tools'' menu, when choosing ``Processor'', select ``ATmega328P (Old Bootloader)''.
    (If you are using the Arduino IDE 1.8.4 and earlier, which don't have the ``(Old Bootloader)'' option, simply select ``ATmega328P'').
    \item If you have an older \developmentboard\ that the ATmega168 processor, replace it with one that has an ATmega328P processor.
\end{itemize}

\subsubsection{Updating USB Driver if Necessary}

We have seen some Windows computers without the CH340 USB driver.
If you encounter this problem and the Device Manager shows you the warning in Figure~\ref{fig:usb-warning}, then the first thing to try is updating the driver.
Right-click on USB2.0-Seri! (Figure~\ref{fig:update-driver}) and choose ``Update driver''.
Then choose ``Search automatically for updated driver software''.

\begin{figure}
    \centering
    \subfloat[] {
        \includegraphics[width=.4\textwidth]{microcontroller/usb-drivers/device-manager-warning}
        \label{fig:usb-warning}
    }
    \hfil
    \subfloat[] {
        \includegraphics[width=.4\textwidth]{microcontroller/usb-drivers/update-driver}
        \label{fig:update-driver}
    }
    \caption{Selecting board and processor in the Arduino IDE.}
\end{figure}

If Windows reports that ``Windows has successfully updated your drivers'' then you should now be able to connect to the \developmentboard.
On the other hand, if Windows reports that ``Windows was unable to install your USB2.0-Ser!'', then the \href{https://learn.sparkfun.com/tutorials/how-to-install-ch340-drivers/}{How to Install CH340 Drivers} page at sparkfun.com will guide you through manually downloading the driver and installing it.

Sparkfun's \href{https://learn.sparkfun.com/tutorials/how-to-install-ch340-drivers/}{How to Install CH340 Drivers} page also has instructions for installing the driver on MacOS and on Linux;
however, we are not aware of any students needing to manually install the CH340 driver on MacOS\@.

\subsubsection{No Driver Warning but Cannot Connect}

Probably what happened is that your computer has the driver, but you're telling the IDE to connect to the wrong virtual COM port.
The typical way to handle this is to disconnect the \developmentboard\ from your computer, go to the part of the menu where you connect to the COM port, connect the \developmentboard\ to your computer, and select whichever COM port appears after plugging in the \developmentboard.


\subsection{Upload a New Sketch}

    From the Arduino IDE's File menu, open the \textit{Blink.ino} example: \\
    \textit{File} $\rightarrow$ \textit{Examples} $\rightarrow$ \textit{01.Basics} $\rightarrow$ \textit{Blink} \\
    Select \textit{Save As...} and save the project as \textit{MyBlink}.

    Edit the values in the \function{delay} calls to change the delays between the LED turning on, off, and on again.
    Select values that will visibly have a difference, such as 250 or 2000.
    Compile the program using the ``Verify'' checkmark in the IDE's toolbar and make corrections if the program doesn't compile.
    Upload the program to your \developmentboard\ using the ``Upload'' arrow in the IDE's toolbar.
    (If you forget to compile first, the IDE will compile your program before uploading, but I find it useful to find compile-time mistakes before attempting to upload the program.)

    If you successfully uploaded \textit{MyBlink.ino} then you will see the following in the IDE's \textit{Output} window:
    \begin{quote}
    \dots (elided configuration data)\dots
    \begin{verbatim}
    avrdude: AVR device initialized and ready to accept instructions

    Reading | ################################################## | 100% 0.01s

    avrdude: Device signature = 0x1e950f (probably m328p)
    avrdude: reading input file "/var/folders/p7/lx4gt70d0_34cpy8r0j3c95c0000gp/T/arduino-sketch-11A4823C54657006C9F78B0812B621A8/MyBlink.ino.hex"
    avrdude: writing flash (932 bytes):

    Writing | ################################################## | 100% 0.33s

    avrdude: 932 bytes of flash written

    avrdude done.  Thank you.


    --------------------------
    upload complete.
    \end{verbatim}\end{quote}
    and then the LED's on-off pattern will change, reflecting the \function{delay} values you assigned (Figure~\ref{fig:myblink}).

    \subsubsection*{Handling Errors}

        If you get an error when attempting to upload a sketch, try these corrective measures:

        \begin{enumerate}
            \item Double-check that you have ``ATmega328P (Old Bootloader)'' selected (see Figure~\ref{fig:selecting-processor}).
            \item Try uploading again (if you attempt to upload a sketch too soon after connecting your \developmentboard\ to your computer, the USB interface won't have finished its handshake).
            \item The \href{https://support.arduino.cc/hc/en-us/articles/4401874331410--Error-avrdude-when-uploading}{Troubleshooting Guide} recommends disconnecting your \developmentboard\ and reconnecting it, then selecting whichever COM port appears.
        \end{enumerate}

        If, instead of an error, your IDE ``hangs'' while collecting configuration data, try this corrective measure:

        \begin{itemize}
        \item Press the \texttt{RESET} button in the middle of the \developmentboard; the IDE should begin uploading the sketch after you release the button.
        \end{itemize}

    \begin{figure}
        \centering
        \animategraphics[autoplay,loop,height=4cm]{8}{microcontroller/animations/myblink-}{0}{6}
        \caption{\textit{MyBlink.ino} has a different on-off pattern than \textit{Blink.ino}.\label{fig:myblink}}
    \end{figure}

\checkpoint{uploaded new code to the \developmentboard}


\section{Direct Input/Output Devices} \subsection{Connect Power and Ground to Power Bus Strips} %! suppress = MissingImport
The columns marked with red and blue stripes are the power bus strips, also known as the power rails.
You will now provide power to the bus strips so that the other components can use power.

\disconnect\

Take the \rainbow, and peel off two wires.
Insert one end of a wire into contact point \mcufivevoltcontactpoint\ (notice that contact point \mcufivevoltcontactpoint\ is electrically connected to the \developmentboard's \texttt{5V} pin, which is in contact point \mcufivevolt).
Insert the other end of the \texttt{5V} wire into the upper \power\ marked with a red stripe.
Now insert one end of the other wire into contact point \mcuuppergroundcontactpoint\ (notice that contact point \mcuuppergroundcontactpoint\ is electrically connected to one of the \developmentboard's \texttt{GND} pins, which is in contact point \mcuupperground).
Insert the other end of the \texttt{GND} wire into the upper \ground\ marked with a blue stripe.
See Figure~\ref{fig:power-without-template}.
If you are using a breadboard template, you will need to fold back the paper (or cut away some paper) to access the power rails;
see Figure~\ref{fig:power-with-template}.

\begin{figure}
    \centering
    \subfloat[Tapping power and ground from the \developmentboard.]{
        \includegraphics[width=0.4\textwidth]{direct/power/power-without-template}
        \label{fig:power-without-template}
    }
    \hfil
    \subfloat[Tapping power and ground from the \developmentboard\ when using a breadboard template.]{
        \includegraphics[width=0.4\textwidth]{direct/power/power-with-template}
        \label{fig:power-with-template}
    }
    \caption{Providing power and ground to power busses.}
\end{figure}

\checkpoint{connected the \developmentboard\ to the upper \power\ and the upper \ground}


\subsection{Light Emitting Diode}                               %! suppress = MissingImport
You will now connect an external LED. An LED is a \textit{light emitting diode}, and like all diodes it allows current to flow only in one direction.
As shown in Figure~\ref{fig:led-annotated}, one lead on the LED is longer than the other, and this tells us which direction current will flow.
When we insert the LED into the circuit, power will flow from one of the \developmentboard's pins through the LED to ground.
Most LEDs have so little internal resistance that, unless current is otherwise limited, enough current will flow through the LED to destroy the semiconductor material.
The typical solution, which we will use, is to employ a \textit{current-limiting resistor}.
(If you look very closely at your \developmentboard, you will see a tiny surface-mount resistor next to each built-in LED.)

Figure~\ref{fig:led-diagram} shows a diagram of the components you will install for the LED output.

\begin{figure}
    \centering
    \includegraphics[height=2cm]{direct/led/led-annotated}
    \caption{The LED's longer lead connects to power; the shorter lead connects to ground.\label{fig:led-annotated}}
\end{figure}

\begin{figure}[p]
    \centering
    \includegraphics[width=0.9\textwidth]{fritzing_diagrams/led}
    \caption{Diagram of component assembly for LED output. \label{fig:led-diagram}}
\end{figure}

\begin{figure}[p]
    \centering
    \includegraphics[height=2cm]{direct/led/resistor-bent}
    \caption{Bend the resistor's leads about 1cm from the ends.\label{fig:resistor-bent}}
\end{figure}

\begin{figure}[p]
    \centering
    \subfloat[The resistor run between contact points \resistorcontactpointone\ and \resistorcontactpointtwo.]{
        \includegraphics[width=0.65\textwidth]{direct/led/resistor-inserted}
        \label{fig:resistor-inserted}
    }
    \hfil
    \subfloat[The LED's longer lead is in contact point \ledanodecontactpoint, and its shorter lead is in the \ground.]{
        \includegraphics[width=0.25\textwidth]{direct/led/led-inserted}
        \label{fig:led-inserted}
    }
    \caption{Constructing the LED assembly.}
\end{figure}

\begin{description}
    \checkoffitem{Take the 1k$\Omega$ resistor and place a right-angle bend in each lead about 0.4in (1cm) from the ends (we want the remaining length to be about 1.5in (3.8cm) -- you do not need to be exact;\footnote{If you want to try to be exact, you can use the breadboard's contact points to measure: they are 0.1in (2.54mm) apart.}
        the leads are flexible enough that you only need to be approximate) -- see Figure~\ref{fig:resistor-bent}.}
    \checkoffitem{Insert one of the resistor's leads into contact point \resistorcontactpointone\ (electrically connected to the \developmentboard's \ledpin\ pin in g1) and the other into contact point \resistorcontactpointtwo.}
    \checkoffitem{Gently press along the length of the resistor, causing the leads to deform slightly, until the resistor's height above the breadboard is about the same as the \developmentboard's printed circuit board.
        See Figure~\ref{fig:resistor-inserted}.}

    \checkoffitem{Take the LED and spread the leads apart slightly.}
    \checkoffitem{Insert the longer lead (the anode) in contact point \ledanodecontactpoint, and the shorter lead (the cathode) in the upper \ground.
        See Figure~\ref{fig:led-inserted}.}
\end{description}

When you have finished installing the external LED, there should be the electrical connections described in Table~\ref{tab:led}.
Read each of this and subsequent tables' rows as describing which electrical components are connected to which other components.
For example, the LED's anode is connected to the resistor's right lead;
the LED's cathode is connected to ground;
and the resistor's left lead is connected to the Arduino's ``\ledpin'' pin.

\begin{table}
    \begin{center}\begin{tabular}{||c|c|c|c||} \hline\hline
    LED lead    & Resistor lead & \developmentboard\ pin    & Pulled High/Low \\ \hline
    Anode       & Right         &           & \\
    Cathode     &               &           & \ground\ \\
                & Left          & \ledpin\  & \\ \hline\hline
    \end{tabular}\end{center}
    \caption{Electrical Connections for External LED.\label{tab:led}}
\end{table}

\checkpoint{installed the LED and its current-limiting resistor}

\begin{description}
    \checkoffitem{In the Arduino IDE, load \textit{MyBlink.ino}.}
%TODO: figure out how to generalize MyBlink
    \checkoffitem{In the \function{pinMode} and the two \function{digitalWrite} calls, replace the \lstinline{LED_BUILTIN} argument with \lstinline{12}:}
        \begin{lstlisting}
        void setup() {
          pinMode(12, OUTPUT);
        }

        void loop() {
          digitalWrite(12, HIGH);
          delay(250);   // or whatever value you used
          digitalWrite(12, LOW);
          delay(1500);  // or whatever value you used
        }
        \end{lstlisting}
    \checkoffitem{Re-connect the USB cable to your \developmentboard.}
    \checkoffitem{Compile the sketch and upload it to your \developmentboard.}
\end{description}
Now, instead of the built-in LED, the external LED that you installed will blink (Figure~\ref{fig:revisedblink}).

\begin{figure}
    \centering
    \animategraphics[autoplay,loop,height=4cm]{8}{direct/animations/revisedblink-}{0}{6}
    \caption{The revised \textit{MyBlink.ino} causes the external LED to blink.\label{fig:revisedblink}}
\end{figure}



\subsection{NAND Gates}\label{subsec:nand}                      %! suppress = MissingImport
The 74LS20 ``chip'' is an integrated circuit (IC) that holds two 4-input NAND gates.
It is in a \textit{dual inline package} (DIP), and solderless breadboards are designed for the DIP form factor to straddle the breadboard's center divider.
A notch on the left side of the DIP helps us orient the IC; the pins are numbered counter-clockwise from the lower-left to the upper-left (Figure~\ref{fig:nand-annotated}).
The relationship between the 74LS20's pins and the NAND gates' inputs and outputs is shown in Figure~\ref{fig:nand-pinout}.

Figure~\ref{fig:nand-diagram} shows the wiring to install the 74LS20.

\begin{figure}
    \centering
    \subfloat[Pin numbers for the 74LS20.]{
        \includegraphics[width=6cm]{direct/nand/nand-annotated}
        \label{fig:nand-annotated}
    }
    \hfil
    \subfloat[Connection diagram showing the 74LS20's pinout.]{
        \includegraphics[width=6cm]{direct/nand/nand-pinout}
        \label{fig:nand-pinout}
    }
    \caption{Pin information for the 74LS20.}
\end{figure}

\begin{figure}
    \centering
    \includegraphics[width=0.9\textwidth]{fritzing_diagrams/nand}
    \caption{Diagram of the 74LS20's installation. \label{fig:nand-diagram}}
\end{figure}

\disconnect\

\prepunch{\nandupperrow\ and \nandlowerrow}

Remove the anti-static foam from the 74LS20's pins.
As described in the \href{https://learn.adafruit.com/breadboards-for-beginners/breadboard-usage}{guide at adafruit.com}, gently press the 74LS20's pins against a tabletop until they're approximately square to the IC's case.
With its notch to the left, place the 74LS20 on the breadboard straddling the center divider on rows 18--24.
Double-check that the 74LS20's pins 1--7 are on contact points \nandlowerrow\ and that pins 8--14 are on contact points \nandupperrow\ (Figure~\ref{fig:nand-ready} shows that the IC's pins are not splayed outside the contact points nor are folded under the IC's case).
Gently press on the 74LS20 to insert the pins into the contact points, using a slight rocking motion if necessary.
As shown in Figure~\ref{fig:nand-inserted}, the IC is fully inserted when its case is flush with the breadboard.

\begin{figure}
    \centering
    \subfloat[Integrated circuit ready to be inserted.]{
        \includegraphics[height=4cm]{direct/nand/nand-ready-to-insert}
        \label{fig:nand-ready}
    }
    \hfil
    \subfloat[Integrated circuit fully inserted.]{
        \includegraphics[height=4cm]{direct/nand/nand-fully-inserted}
        \label{fig:nand-inserted}
    }
    \caption{Inserting the 74LS20.}
\end{figure}

Peel one wire from the \rainbow;
use this wire to connect contact point \nandvcc\ (electrically connected to the 74LS20's $\mathtt{V_{CC}}$, pin 14) to the upper \power.
Peel off another wire from the \rainbow;
use this wire to connect contact point \nandground\ (electrically connected to the 74LS20's \texttt{GND}, pin 7) to the contact point \mculowergroundcontactpoint (electrically connected to one of the \developmentboard's \texttt{GND} pins).
See Figure~\ref{fig:nand-power}.

Peel four more wires from the \rainbow.
Use two wires to connect contact points \nandupperc\ and \nandupperd\ to the upper \power.
Use another wire to connect contact point \nanduppery (electrically connected to the 74LS20's \texttt{Y2}, pin 8) to contact point \mcubuttonnandpoint\ (electrically connected to the \developmentboard's \mcubuttonnand\ pin).
These three wires configured the 74LS20's upper 4-input NAND gate to act as a 2-input NAND gate;
you will provide the inputs in Section~\ref{subsec:pushbuttons}.
Use the fourth wire to connect contact point \nandlowery\ (electrically connected to the 74LS20's \texttt{Y1}, pin 6) to contact point \mcukeypadnandpoint\ (electrically connected to the \developmentboard's \mcukeypadnand\ pin);
you will  provide the inputs for the lower 4-input NAND gate in Section~\ref{sec:keypad}.
See Figure~\ref{fig:nand-outputs}.

\begin{figure}
    \centering
    \subfloat[The 74LS20 connected to power and ground.]{
        \includegraphics[width=.4\textwidth]{direct/nand/nand-power}
        \label{fig:nand-power}
    }
    \hfil
    \subfloat[The 74LS20's outputs connected to the \developmentboard.]{
        \includegraphics[width=.5\textwidth]{direct/nand/nand-outputs}
        \label{fig:nand-outputs}
    }
    \caption{Wiring the 74LS20.}
\end{figure}

When you have finished wiring the 74LS20, there should be the electrical connections described in Table~\ref{tab:nand}.

\begin{table}
    \begin{center}\begin{tabular}{||c|c|c||} \hline\hline
    74LS20 Pin  & \developmentboard\ pin    & Pulled High/Low \\ \hline
    6           & \texttt{D3}   & \\
    7           &               & Pulled Low \\
    8           & \texttt{D2}   & \\
    12          &               & Pulled High \\
    13          &               & Pulled High \\
    14          &               & Pulled High \\ \hline\hline
    \end{tabular}\end{center}
    \caption{Initial Electrical Connections for NAND Gates.\label{tab:nand}}
\end{table}

\checkpoint{inserted and wired the 74LS20}


\subsection{Install the CowPi Library}

\subsection{Slider Switches}                                    %! suppress = MissingImport
In this section, you will install the ``slider'' switches that toggle between their two positions, holding their position until toggled again.
We will wire them such that when a switch is toggled to the left, it will produce a 0, and when it is toggled to the right, it will produce a 1.
Figure~\ref{fig:switch-diagram} shows a diagram of the wiring for the slider switches.

%! suppress = NonMatchingIf
\begin{figure}[p]
    \centering
    \ifdefstring{\serialprotocol}{SPI}{
        \subfloat[3-pin switches]{
            \includegraphics[width=0.9\textwidth]{fritzing_diagrams/switch-spi-spdt}
        }
%        \hfil
%        \subfloat[2-pin switches]{
%            \includegraphics[width=0.9\textwidth]{fritzing_diagrams/switch-spi-dip1}
%        }
    }{}
    \ifdefstring{\serialprotocol}{I2C}{
        \subfloat[3-pin switches]{
            \includegraphics[width=0.9\textwidth]{fritzing_diagrams/switch-i2c-spdt}
        }
%        \hfil
%        \subfloat[2-pin switches]{
%            \includegraphics[width=0.9\textwidth]{fritzing_diagrams/switch-i2c-dip1}
%        }
    }{}
    \caption{Diagram of wiring associated with toggle switch input.
        \label{fig:switch-diagram}}
\end{figure}

\disconnect\

%\prepunch{a\controlrow{1} and a\controlrow{6}}
%If you have 3-pin switches, then pre-punch holes into contact points a\controlrow{2}, a\controlrow{3}, a\controlrow{7} and a\controlrow{8}.
%If you have 2-pin switches, then pre-punch holes into contact points a\controlrow{4} and a\controlrow{9}.
\begin{description}
    \checkoffitem{\prepunch{a\controlrow{1}--a\controlrow{3} and a\controlrow{6}--a\controlrow{8}}}

    \checkoffitem{Insert one slider switch into contact points a\controlrow{1}--a\controlrow{3}.} % (for 3-pin switches) or contact points a\controlrow{1}--a\controlrow{4} (for 2-pin switches).
    \checkoffitem{Place the other slider switch into contact points a\controlrow{6}--a\controlrow{8}.} % (for 3-pin switches) or contact points a\controlrow{6}--a\controlrow{9} (for 2-pin switches).
\end{description}
For the two wires that will connect the switches to the \developmentboard, you can use 10cm jumpers (especially if that is all that you have);
however, if you use 20cm jumpers, then in Section~\ref{sec:keypad} we will show how to keep some wires away from the controls.
\begin{description}
    \checkoffitem{Peel off one wire from the \rainbow\ and use it to connect contact point e\controlrow{1} (electrically connected to the left switch's left pin) to contact point \mculeftswitchpoint\ (electrically connected to the \developmentboard's \mculeftswitch\ pin).}
    \checkoffitem{Peel off another wire from the \rainbow\ and use it to connect contact point e\controlrow{6} (electrically connected to the right switch's left pin) to contact point \mcurightswitchpoint\ (electrically connected to the \developmentboard's \mcurightswitch\ pin).}

    \checkoffitem{Peel off two more wires from the \rainbow.}
\end{description}
You will use these to connect the switches center pins %(for 3-pin switches) or right pins (for 2-pin switches)
to the upper \ground.
Specifically,
\begin{description}
    \checkoffitem{Place the end of one wire into contact point e\controlrow{2}.} %(for 3-pin switches) or e\controlrow{4} (for 2-pin switches);
    \checkoffitem{Place the other end of that wire into the upper \ground.}
    \checkoffitem{Now place the end of the other wire into contact point e\controlrow{7}.} %(for 3-pin switches) or e\controlrow{9} (for 2-pin switches);
    \checkoffitem{Place the other end of that wire into the upper \ground.}
\end{description}
The switches' right pins will not be electrically connected to anything.

%If you have 3-pin switches, the switches' right pins will not be electrically connected to anything.

%TODO: parameterize for I2C vs SPI; add spdt vs dip1 close-ups
\begin{figure}
    \centering
    \includegraphics[width=0.6\textwidth]{direct/switches/switch-spdt-i2c}
    \caption{The slider switches, each with one pin grounded, one pin connected to the \developmentboard, and one pin floating.
        \label{fig:switch-spdt}}
\end{figure}

When you have finished setting up the switches' wiring, there should be the
electrical connections described in Table~\ref{tab:switch-spdt}.% or \ref{tab:switch-dip1}.

\begin{table}
    \begin{center}\begin{tabular}{||c|c|c||} \hline\hline
    Switch                      & \developmentboard\ pin    & Pulled High/Low \\ \hline
    Left switch's left pin      & \mculeftswitch    & \\
    Left switch's center pin    &                   & \ground\ \\
    Left switch's right pin     & \multicolumn{2}{c||}{not connected / floating} \\
    Right switch's left pin     & \mcurightswitch   & \\
    Right switch's center pin   &                   & \ground \\
    Right switch's right pin    & \multicolumn{2}{c||}{not connected / floating} \\ \hline\hline
    \end{tabular}\end{center}
    \caption{Electrical Connections for 3-pin Slider Switches.
        \label{tab:switch-spdt}}
\end{table}

%\begin{table}
%    \begin{center}\begin{tabular}{||c|c|c||} \hline\hline
%    Switch                      & \developmentboard\ pin    & Pulled High/Low \\ \hline
%    Left switch's left pin      & \mculeftswitch    & \\
%    Left switch's right pin     &                   & Pulled Low \\
%    Right switch's left pin     & \mcurightswitch   & \\
%    Right switch's right pin    &                   & Pulled Low \\ \hline\hline
%    \end{tabular}\end{center}
%    \caption{Electrical Connections for 2-pin Slider Switches.
%        \label{tab:switch-dip1}}
%\end{table}

\checkpoint{inserted and wired the slider switches}

Connect your \developmentboard\ to the computer.
In the IDE's Serial Monitor, notice that Left~switch is LEFT when the left switch is toggled to the left, and it is RIGHT when the left switch is toggled to the right.
Similarly, Right~switch is LEFT or RIGHT, depending on whether the right switch is toggled to the left or right.

\subsection{Momentary Pushbuttons} \label{subsec:pushbuttons}   %! suppress = MissingImport
If your momentary pushbuttons are attached to a cardboard strip with tape, remove them from the cardboard strip.
If your momentary pushbuttons' leads have metal tabs at the end (Figure~\ref{fig:pushbutton-tabs}), you will need to snip off the tabs before inserting the pushbutton leads into the breadboard;
ordinary scissors will suffice for this task.
Regardless of whether the leads have metal tabs at the end, you may optionally trim the leads to be about $\frac{1}{4}$in (6.4mm) long -- you can use the exposed lead from a jumper wire as a reference -- so that the pushbuttons sit flush on the breadboard.
It is not necessary that they sit flush;
this is simply to keep the buttons from wiggling under your fingers.
\textit{Do not cut the leads shorter than $\mathit{\frac{1}{8}}$in (3.2mm)!}
\textbf{Be sure to use eye protection in case the leads' ends fly off when you snip them.}

These are ``normally open'' momentary ``switches'' that close when pressed and re-open when released.
We will wire the pushbuttons such that they normally produce a 1, and when pressed will produce a 0.
Figure~\ref{fig:pushbutton-diagram} shows a diagram of the wiring for the pushbuttons.

\begin{figure}
    \centering
    \includegraphics[height=2cm]{direct/buttons/pushbutton-tabs}
    \caption{Some momentary pushbuttons have metal tabs on their leads.\label{fig:pushbutton-tabs}}
\end{figure}

\begin{figure}
    \centering
    \includegraphics[width=0.9\textwidth]{fritzing_diagrams/pushbutton}
    \caption{Diagram of wiring associated with momentary pushbutton input.
        \textit{Note: connection between the 74LS20's pin 6 and the \developmentboard's
        \texttt{D2} pin was previously installed in Section~\ref{subsec:nand}.}
        \label{fig:pushbutton-diagram}}
\end{figure}

\disconnect\

Insert the leads of one pushbutton into contact points a39 and a41.
Peel off one wire from the \rainbow, and use it to connect contact point e41 to the upper \ground.
Insert the leads of the other pushbutton into contact points a43 and a45.
Peel off one wire from the \rainbow, and use it to connect contact point e45 to the upper \ground.
See Figure~\ref{fig:pushbutton-grounded}.

Peel off a 2-conductor cable from the \rainbow, and use it to connect contact points a21 and a22 (electrically connected to the 74LS20's \texttt{C1} and \texttt{D1}, pins 4 and 5) to a \power.
Peel off another 2-conductor cable and use it to connect contact points d39 and d43 (electrically connected to the ungrounded sides of the pushbuttons) to the 74LS20: contact point d39 should be connected to contact point b19 (electrically connected to the 74LS20's \texttt{B1}, pin 2), and contact point d43 should be connected to contact point b18 (electrically connected to the 74LS20's \texttt{A1}, pin 1).
See Figure~\ref{fig:pushbutton-nand}.

Peel off another 2-conductor cable from the \rainbow, and use it to connect contact points j4 and j5 (electrically connected to the \developmentboard's \texttt{D9} and \texttt{D8} pins) to contact points c18 and c19 (electrically connected to the 74LS20's \texttt{A1} and \texttt{B1}, and to the pushbuttons through the cable you installed in the previous paragraph), respectively.
See Figure~\ref{fig:pushbutton-nano}.

\begin{figure}\begin{multicols}{2}
    \centering
    \subfloat[The momentary pushbuttons, each with one lead grounded.]{
        \includegraphics[width=0.4\textwidth]{direct/buttons/pushbutton-grounded}
        \label{fig:pushbutton-grounded}
    }
    \columnbreak

    \subfloat[Connections between the momentary pushbuttons and the 74LS20.]{
        \includegraphics[width=0.55\textwidth]{direct/buttons/pushbutton-nand}
        \label{fig:pushbutton-nand}
    }

    \subfloat[Connections between the \developmentboard\ and wiring to the momentary
        pushbuttons.]{
        \includegraphics[width=.55\textwidth]{direct/buttons/pushbutton-nano}
        \label{fig:pushbutton-nano}
    }
    \end{multicols}
    \caption{Wiring the Momentary Pushbuttons.}
\end{figure}

When you have finished setting up the pushbuttons' wiring, there should be the electrical paths described in Table~\ref{tab:pushbutton}.

\begin{table}
    \begin{center}\begin{tabular}{||c|c|c|c||} \hline\hline
    Pushbutton                      & 74LS20            & \developmentboard\ pin    & Pulled High/Low \\ \hline
    Left button's grounded lead     &                   &               & Pulled Low \\
    Left button's ungrounded lead   & Lower NAND Input  & \texttt{D8}   & \\
    Right button's grounded lead    &                   &               & Pulled Low \\
    Right button's ungrounded lead  & Lower NAND Input  & \texttt{D9}   & \\
                                    & Lower NAND Input  &               & Pulled High \\
                                    & Lower NAND Input  &               & Pulled High \\
                                    & Lower NAND Output & \texttt{D2}   & \\ \hline\hline
    \end{tabular}\end{center}
    \caption{Electrical Paths for Momentary Pushbuttons.\label{tab:pushbutton}}
\end{table}

\checkpoint{inserted and wired the momentary pushbuttons}

Connect your \developmentboard\ to the computer.
In the IDE's Serial Monitor, notice that the LEFT~BUTTON is always 1, the RIGHT~BUTTON is always 1, and the BUTTON~NAND is always 0.
Notice that when you press a button, the Serial Monitor shows that its value becomes 0, and when you release a button, its value becomes 1 again.
If either button is pressed, BUTTON~NAND becomes 1, and it is 0 only when both buttons are not pressed.

\section{Matrix Keypad} %! suppress = MissingImport
%\begin{wrapfigure}{r}{0.3\textwidth}
%    \centering
%    \includegraphics[width=0.25\textwidth]{keypad/keypad-annotated}
%    \caption{The numeric keypad's header has four row pins and four column pins. \label{fig:keypad-annotated}}
%\end{wrapfigure}

Observe that the matrix keypad has sixteen buttons has eight pins in its female connector.
As shown in Figure~\ref{fig:keypad-annotated}, when the keypad is face-up and oriented for reading, the four pins on the left are the \textit{row} pins, and the four pins on the right are the \textit{column} pins.
From left-to-right, we will name these pins \texttt{row1}, \texttt{row4}, \texttt{row7}, \texttt{row*}, \texttt{column1}, \texttt{column2}, \texttt{column3}, \texttt{columnA}.
Figure~\ref{fig:keypad-matrix} shows the membrane contacts and which \developmentboard\ pin will be connected to each keypad pin.

%\begin{wrapfigure}{r}{0.3\textwidth}
%    \centering
%    \includegraphics[width=0.25\textwidth]{keypad/keypad-matrix}
%    \caption{The numeric keypad's underlying contact matrix. \label{fig:keypad-matrix}}
%\end{wrapfigure}

\begin{figure}
    \centering
    \subfloat[Front of matrix keypad.] {
        \includegraphics[height=7cm]{keypad/keypad-annotated}
        \label{fig:keypad-annotated}
    }
    \hfil
    \subfloat[Keypad's underlying contact matrix.] {
        \includegraphics[height=7cm]{keypad/keypad-matrix}
        \label{fig:keypad-matrix}
    }
    \caption{The numeric keypad's header has four row pins and four column pins.}
\end{figure}

Figure~\ref{fig:keypad-diagram} shows a diagram of the wiring for the matrix keypad.

\begin{figure}%[p]
    \centering
    \includegraphics[width=0.9\textwidth]{fritzing_diagrams/keypad}
    \caption{Diagram of wiring associated with matrix keyboard input.
        \textit{Note: connection between the 74LS20's pin 6 and the \developmentboard's
        \texttt{D3} pin was previously installed in Section~\ref{subsec:nand}.}
        \label{fig:keypad-diagram}}
\end{figure}

\disconnect\

\begin{description}
    \checkoffitem{\prepunch{j\controlrow{0}--j\controlrow{7}}
        If your male-male header strip has more than eight pins, then pre-punch additional holes to the right (j\controlrow{8}, j\controlrow{9}, \dots) as needed.}

    \checkoffitem{If your 8-pin male-male header strip is not already inserted into the keypad's female connectors, insert it into the female connectors now.
        If your male-male header strip has more than eight pins, position the excess pins to the right of the column pins.}
    \checkoffitem{Connect your keypad to your breadboard such that \texttt{row1} is in contact point j\controlrow{0}, and \texttt{columnA} is in contact point j\controlrow{7} (and any unused pins on the male-male header are in contact points j\controlrow{8}, j\controlrow{9}, etc.).}
    \item \textbf{NOTE}: if you used 20cm wires to connect your slide-switches and/or pushbuttons to the \developmentboard, then you can use the matrix keypad's ribbon cable to pull these wires away from the circuit, reducing clutter near the controls (Figure~\ref{fig:keypad-pullingwires}).

\begin{figure}
    \centering
    \includegraphics{keypad/keypad-pullingwires}
    \caption{The keypad's ribbon cable can be used to pull long wires out of the way. \label{fig:keypad-pullingwires}}
\end{figure}

    \checkoffitem{Peel off three 4-conductor cables from the \rainbow (Figure~\ref{fig:keypad-cables}).}
    \item While you \textit{can} use individual wires, having these 4-conductor cables will simplify keeping track of the wires.

\begin{figure}
    \centering
    \includegraphics{keypad/keypad-cables}
    \caption{Three 4-conductor cables. \label{fig:keypad-cables}}
\end{figure}

    \checkoffitem{Insert one end of one of the 4-conductor cables in contact points h\controlrow{0}--h\controlrow{3}, in the same breadboard rows as the keypad's row pins.}
    \checkoffitem{Insert the other end of the cable in contact points \mcukeypadrowonepoint--\mcukeypadrowstarpoint.}
    \item You want the \developmentboard's \mcukeypadrowone\ pin to connect to the keypad's \texttt{row1} pin, \mcukeypadrowfour\ to \texttt{row4}, \mcukeypadrowseven\ to \texttt{row7}, and \mcukeypadrowstar\ to \texttt{row*};
        you can use the wires' colors to make sure that you do so.

    \checkoffitem{Insert one end of another 4-conductor cable in contact points h\controlrow{4}--h\controlrow{7}, in the same breadboard rows as the keypad's column pins.}
    \checkoffitem{Insert the other end of the cable in contact points \nandlowerain, \nandlowerbin, \nandlowercin, and \nandlowerdin\ (electrically connected to the 74LS20's \texttt{A1}, \texttt{B1}, \texttt{C1}, and \texttt{D1}: pins 1, 2, 4, and 5).}
    \item \textit{Notice that there are no wires in \nandlowernc} because, as you can see in Figure~\ref{fig:nand-pinout}, the 74LS20's pin 3 is not connected (``NC'') to anything.

    \checkoffitem{Insert one end of the remaining 4-conductor cable in contact points \nandloweraout, \nandlowerbout, \nandlowercout, and \nandlowerdout.}
    \checkoffitem{Insert the other end in contact points \mcukeypadcolonepoint--\mcukeypadcolApoint\ (electrically connected to the \developmentboard's \mcukeypadcolone--\mcukeypadcolA\ pins).}
    \item You want the 74LS20's pin 1 to connect the \developmentboard's \mcukeypadcolone\ pin and the keypad's \texttt{column1} pin, the 74LS20's pin 2 to connect \mcukeypadcoltwo\ and \texttt{column2}, the 74LS20's pin 4 to connect \mcukeypadcolthree\ and \texttt{column3}, and the 74LS20's pin 5 to connect \mcukeypadcolA\ and \texttt{columnA};
        you can use the wires' colors to make sure that you do so.
\end{description}

%\begin{figure}
%    \centering
%    \subfloat[The matrix keypad's connector, with the male-male header strip, just before being inserted into the breadboard.
%        Note the excess header strip's excess pin to the right of the column pins, in contact point j34.]{
%        \includegraphics[width=.5\textwidth]{keypad/keypad-header}
%        \label{fig:keypad-header}
%    }
%    \hfil
%    \subfloat[One end of the ``rows'' cable electrically connected to the keypad's row pins.]{
%        \includegraphics[width=.4\textwidth]{keypad/keypad-row-cable}
%        \label{fig:keypad-row-cable}
%    }
%
%    \subfloat[The other end of the ``rows'' cable electrically connected to the \developmentboard's \texttt{D4}-\texttt{D7} pins.]{
%        \includegraphics[width=.4\textwidth]{keypad/keypad-row-nano}
%        \label{fig:keypad-row-nano}
%    }
%    \hfil
%    \subfloat[Connection between the keypad's row pins and the \developmentboard's
%        \texttt{D4}-\texttt{D7} pins.]{
%        \includegraphics[width=.5\textwidth]{keypad/keypad-row-full}
%        \label{fig:keypad-row-full}
%    }
%
%    \subfloat[Connection between the keypad's column pins and the 74LS20.]{
%        \includegraphics[width=.4\textwidth]{keypad/keypad-col-nand}
%        \label{fig:keypad-col-nand}
%    }
%    \hfil
%    \subfloat[Connection between the the 74LS20 and the \developmentboard's \texttt{A0}-\texttt{A3} pins.]{
%        \includegraphics[width=.5\textwidth]{keypad/keypad-col-nano}
%        \label{fig:keypad-col-nano}
%    }
%    \caption{Wiring the Matrix Keypad.}
%\end{figure}

When you have finished setting up the keypad wiring, there should be the electrical paths described in Table~\ref{tab:keypad}.

\begin{table}
    \begin{center}\begin{tabular}{||c|c|c||} \hline\hline
    Keypad pin          & 74LS20            & \developmentboard\ pin \\ \hline
    \texttt{row1}       &                   & \mcukeypadrowone      \\
    \texttt{row4}       &                   & \mcukeypadrowfour     \\
    \texttt{row7}       &                   & \mcukeypadrowseven    \\
    \texttt{row*}       &                   & \mcukeypadrowstar     \\
    \texttt{column1}    & Lower NAND Input  & \mcukeypadcolone      \\
    \texttt{column2}    & Lower NAND Input  & \mcukeypadcoltwo      \\
    \texttt{column3}    & Lower NAND Input  & \mcukeypadcolthree    \\
    \texttt{columnA}    & Lower NAND Input  & \mcukeypadcolA        \\
                        & Lower NAND Output & \mcukeypadnand        \\ \hline
                        & pin 3 (\nandlowernc) & not connected / floating \\ \hline\hline
    \end{tabular}\end{center}
    \caption{Electrical Paths for Matrix Keypad.\label{tab:keypad}}
\end{table}

\checkpoint{inserted and wired the matrix keypad}

\textbf{NOTE:} Do not press more than one key on the matrix keypad at a time.
There are certain combinations of keys that could result in a short-circuit from power to ground, possibly damaging your \developmentboard.
Your \developmentboard\ has some safety measures to prevent damage in that situation, but it would be better for you not to test those safety measures.

Connect your \developmentboard\ to the computer.
In the IDE's Serial Monitor, notice that there is normally no character after \texttt{Keypad:}, that Column~pins is normally 1111, and that Keypad~NAND is normally 0.
Press the 5 key on the matrix keypad.
Notice that the first line of the message from the \developmentboard\ is now
\begin{verbatim}
    Keypad:      5        Column pins:  1011    Keypad NAND: 1
\end{verbatim}
In general, when you press a key on the keypad, the corresponding character will be displayed after \texttt{Keypad:}, and Keypad~NAND will become 1.
When you press 1, 4, 7, or *, Column~pins becomes 0111;
similarly, pressing a key in the 2$^{nd}$ column causes Column~pins to become 1011;
in the 3$^{rd}$ column, 1101;
and in the A$^{th}$ column, 1110.
Be sure to test all 16 keys.


\section{Display Module} %! suppress = MissingImport
%\subsection{Seven-Segment Display Module}

\newcommand{\numberofserialpins}{zero}
\newcommand{\numeralofserialpins}{0}
\newcommand{\serialpins}{\dots\dots\dots}
\newcommand{\displaytest}{\dots\dots\dots}
%! suppress = NonMatchingIf
\ifdefstring{\serialprotocol}{SPI}{
    \renewcommand{\numberofserialpins}{five}
    \renewcommand{\numeralofserialpins}{5}
    \renewcommand{\serialpins}{\texttt{DIN} (data in), \texttt{CS} (chip select), and \texttt{CLK} (clock)}
    \renewcommand{\displaytest}{max7219\_7segment\_hello\_world}
    Examine the 7-segment display module.
}{}
%! suppress = NonMatchingIf
\ifdefstring{\serialprotocol}{I2C}{
    \renewcommand{\numberofserialpins}{four}
    \renewcommand{\numeralofserialpins}{4}
    \renewcommand{\serialpins}{\texttt{SDA} (serial data), and \texttt{SCL} (serial clock)}
    \renewcommand{\displaytest}{lcd1602\_hello\_world}
    Examine the I$^2$C-LCD serial interface.
}{}
Notice that the header has \numberofserialpins pins (Figure~\ref{fig:display-module-header}): \texttt{VCC} (common collector voltage), \texttt{GND} (ground), \serialpins.
When the display module is oriented for viewing, these header pins will be on the left.

Figure~\ref{fig:display-diagram} shows a diagram of the wiring to connect the display module to the breadboard.

%TODO: parameterize the files (module header, display, display module female connectors)
\begin{figure}
    \centering
    \includegraphics[width=5cm]{display/spi-module-header}
    \caption{The display module's header has \numberofserialpins\ pins.
    \label{fig:display-module-header}}
\end{figure}

\begin{figure}
    \centering
    \includegraphics[width=0.9\textwidth]{fritzing_diagrams/display-max7219}
    \caption{Diagram of display module's connections to the breadboard.
    \label{fig:display-diagram}}
\end{figure}

\disconnect\

%! suppress = NonMatchingIf
\ifdefstring{\serialprotocol}{I2C}{
    \textbf{If your I$^2$C-LCD serial interface is \textit{not} attached to the LCD display module}, then you will use the breadboard to provide the electrical connections between the serial interface and the display module.
    \prepunch{i\controlrow{22}--i\controlrow{37} and g\controlrow{22}--g\controlrow{37}}
    If you are using a breadboard template then you can now remove the jumper wires from contact points a63 and j63.
    Insert the LCD display module's sixteen pins into contact points g\controlrow{22}--g\controlrow{37}.
    With the four header pins pointing to the left, insert the I$^2$C-LCD serial interface's sixteen downward-pointing pins into contact points i\controlrow{22}--i\controlrow{37}.

    \textbf{If your I$^2$C-LCD serial interface \textit{is} attached to the LCD display module}, then the sixteen pins connecting the serial adapter to the display module do not need to be inserted into the breadboard.
}{}
\textit{If you are using a breadboard template} then you can now remove the jumper wires from contact points a63 and j63, but you do not need to do so
(you might use a jumper wire looped from a63 to j63 to prevent the display module from sliding around).

Take the \numeralofserialpins-conductor female-to-male rainbow cable and attach the \numberofserialpins\ female connectors to the display module's \numberofserialpins\ header pins.

%! suppress = NonMatchingIf
\ifdefstring{\serialprotocol}{SPI}{
    As you insert the male connectors into the breadboard, you may have to partially separate the wires at the male end.
    In the interest of keeping track of which wires are used for which purposes, do not fully separate the wires.
    Identify the wire that is connected to the display module's \texttt{CLK} pin;
    insert the male end of this wire in contact point \mcusckpoint (electrically connected to the \developmentboard's \mcusck pin).
    Insert the male end of the \texttt{CS} wire into contact point \mcucspoint (electrically connected to the \developmentboard's \mcucs pin).
    Insert the male end of the \texttt{DIN} wire into contact point \mcucopipoint (electrically connected to the \developmentboard's \mcucopi pin).
}{}
%! suppress = NonMatchingIf
\ifdefstring{\serialprotocol}{I2C}{
    Identify the wire that is connected to the display module's \texttt{SCL} pin;
    insert the male end of this wire in contact point \mcusclpoint\ (electrically connected to the \developmentboard's \mcuscl pin).
    Insert the male end of the \texttt{SDA} wire into contact point \mcusdapoint\ (electrically connected to the \developmentboard's \mcusda pin).
}{}
Insert the \texttt{GND} wire into the upper \ground, and the \texttt{VCC} wire into the upper \power.

%\begin{figure}
%    \centering
%    \subfloat[Connect the female ends of the \numeralofserialpins-conductor cable to the display module's header pins (your colors may be different).] {
%        \includegraphics[width=0.27\textwidth]{display/display-module-female-connectors}
%        \label{fig:display-module-female-connectors}
%    }
%    \hfil
%    \subfloat[The display module's clock will be driven by \texttt{D13}.] {
%        \includegraphics[width=0.27\textwidth]{display/display-module-CLK}
%        \label{fig:display-module-CLK}
%    }
%    \hfil
%    \subfloat[The display module's chip-select will be driven by
%    \texttt{D10}.] {
%        \includegraphics[width=0.27\textwidth]{display/display-module-CS}
%        \label{fig:display-module-CS}
%    }
%
%    \subfloat[The display module's data-in will be driven by \texttt{D11}] {
%        \includegraphics[width=0.27\textwidth]{display/display-module-DIN}
%        \label{fig:display-module-DIN}
%    }
%    \hfil
%    \subfloat[The display module will be powered by the breadboard's power bus] {
%        \includegraphics[width=0.27\textwidth]{display/display-module-power}
%        \label{fig:display-module-power}
%    }
%    \caption{Connecting the Display Module.}
%\end{figure}

When you have finished connecting the display module, there should be the electrical connections described in Table~\ref{tab:display}.

%! suppress = NonMatchingIf
\begin{table}
    \ifdefstring{\serialprotocol}{SPI}{
        \begin{center}\begin{tabular}{||c|c|c||} \hline\hline
            Display Module Pin  & \developmentboard\ pin    & Pulled High/Low \\ \hline
            \texttt{CLK}        & \mcusck   &               \\
            \texttt{CS}         & \mcucs    &               \\
            \texttt{DIN}        & \mcucopi  &               \\
            \texttt{GND}        &           & Pulled Low    \\
            \texttt{VCC}        &           & Pulled High   \\ \hline\hline
        \end{tabular}\end{center}
    }{}
    \ifdefstring{\serialprotocol}{I2C}{
        \begin{center}\begin{tabular}{||c|c|c||} \hline\hline
            Display Module Pin  & \developmentboard\ pin    & Pulled High/Low \\ \hline
            \texttt{SCL}        & \mcuscl   &               \\
            \texttt{SDA}        & \mcusda   &               \\
            \texttt{GND}        &           & Pulled Low    \\
            \texttt{VCC}        &           & Pulled High   \\ \hline\hline
        \end{tabular}\end{center}
    }{}
\caption{Electrical Connections for External LED.\label{tab:display}}
\end{table}

\checkpoint{connected the display module to the breadboard}

In the Arduino IDE, open the
\textit{File} $\rightarrow$ \textit{Examples} $\rightarrow$ \textit{CowPi} $\rightarrow$ \textit{\displaytest}
example.
Find these lines in the \function{setup()} function:

\begin{lstlisting}[numbers=left, firstnumber=9]
//    protocol = SPI;
    protocol = I2C;
\end{lstlisting}

%! suppress = NonMatchingIf
\ifdefstring{\serialprotocol}{I2C}{
    Make sure that the \lstinline{protocol = I2C} line is uncommented and that the \lstinline{protocol = SPI} line is commented-out.
}{}
%! suppress = NonMatchingIf
\ifdefstring{\serialprotocol}{SPI}{
    Comment-out the \lstinline{protocol = I2C} line, and uncomment the \lstinline{protocol = SPI} line.
}{}

Compile the program and upload it to your \developmentboard.

%! suppress = NonMatchingIf
\ifdefstring{\serialprotocol}{I2C}{
    You should see the display module's backlight blink on and off.
    If so, then you have correctly connected the display module and serial adapter even if you don't see a message on the display module.

    Using a screwdriver, turn the trim potentiometer on the serial adapter until you can see the message: \\
    \display{
        Hello, world! \\
        i2c address=0x27
    }
}{}
%! suppress = NonMatchingIf
\ifdefstring{\serialprotocol}{SPI}{
    \dots\dots\dots \\
    Old message: \\
    You should see {\dviiseg 8.} move back and forth across the display (Figure~\ref{fig:display-test}).
    You may be able to see the \developmentboard's built-in LED blink rapidly: recall that it's connected to pin 13, which is used as the clock signal when the \developmentboard\ sends data to the display module.
}{}

\begin{figure}
    \centering
    \animategraphics[autoplay,palindrome,height=5cm]{10}{display/animations/displaytest-}{0}{7}
    \caption{\textit{DisplayTest.ino} illuminates all segments on a digit, one digit at a time. \label{fig:display-test}}
\end{figure}


\section*{Kit Assembly is Complete}

    You have now finished assembling the class kit (Figure~\ref{fig:complete}).
    In the upcoming I/O labs, you will use the kit to learn about memory-mapped I/O and about handling low-level interrupts.

    \begin{figure}
        \centering
        \subfloat[]{
            \includegraphics[height=0.55\textheight]{fritzing_diagrams/complete}
%            \label{fig:complete-diagram}
        }

        \subfloat[]{
            \includegraphics[height=0.35\textheight]{completed-kit}
%            \label{fig:complete-kit}
        }
        \caption{The fully-assembled class kit.\label{fig:complete}}
    \end{figure}

\section{Turn-in and Grading}

    When you have completed this assignment, upload \textit{checkpoints.txt} to \filesubmission.

    This assignment is worth 1 point, with up to 5 bonus points for helping other students. \\

    Rubric:
    \begin{description}
        \rubricitem{0.5}{You bring your fully-assembled class kit to your lab section the week it is due.}
        \rubricitem{0.5}{You upload the completed \textit{checkpoints.txt} file to \filesubmission.}
        \bonusitem{0.1}{For each checkpoint you verify for fellow students, as reported in their \textit{checkpoints.txt} files;
            maximum of 5.0 bonus points.}
    \end{description}

\section*{Appendix: Breadboard Templates}

    Figure~\ref{fig:templates} is a pair of templates for the Cow Pi circuit (one if your kit has pushbuttons with two leads, and the other if your kit has pushbuttons with four prongs).
    Each dot (\tikz{\draw[white,fill=gray] (0,0) +(0,3pt) circle (1pt);}) represents a breadboard contact point.
    Each dot with a circle (\tikz{\drawtarget{0}{3pt} \draw[white,fill=gray] (0,0) +(0,3pt) circle (1pt);}) is a contact point in which you will insert a jumper lead.
    Attached to most of these circles is the contact point for the other end of the jumper wire (\tikz{\drawlabelledtarget{0}{0}{1}{k64} \draw[white,fill=gray] (0,0) circle (1pt);}).
    The footprints of several components are shown as light-gray outlines;
    resistors (\tikz{\ctikzset{bipoles/resistor/height=0.2}\ctikzset{bipoles/resistor/width=0.3}\draw (0,0) to[R] (1,0)}) and LEDs (\tikz{\ctikzset{bipoles/diode/height=0.2}\ctikzset{bipoles/diode/width=0.2}\draw (0,0) to[led] (1,0)}) are shown using their conventional symbols.
    Squares (\tikz{\draw[white,fill=gray] (0,0) +(2pt,3pt) circle (1pt); \draw (0,0) +(0,1pt) rectangle +(4pt,5pt);}) are where you'll insert component individual pins,
        and rectangles (\tikz{\draw (0,0) +(0,1pt) rectangle +(11pt,5pt); \draw[white,fill=gray] (0,0) +(2pt,3pt) circle (1pt); \draw[white,fill=gray] (0,0) +(5.5pt,3pt) circle (1pt); \draw[white,fill=gray] (0,0) +(9pt,3pt) circle (1pt);}) are where you'll insert components' in-line pins.
    Finally, the four corners (\tikz{\draw[white,fill=gray] (0,0) +(2pt,3pt) circle (1pt); \draw (0,1pt) -- (4pt,5pt); \draw (0,5pt) -- (4pt,1pt);}) are used to align the template on your solderless breadboard.

    \begin{figure}[p]
        \subfloat[Template that uses 2-lead pushbuttons.]{
            \hspace{-.5in}
%            \begin{tikzpicture}[x=.1081in, y=.1088in] % TODO: determine which to use
            \begin{tikzpicture}[x=.1in, y=.1in]
                \drawbreadboard
                \drawnano{1}{3}{Arduino Nano}
                \drawnand{18}{5}
                \drawswitches{27}
                \drawbuttons{37}{2}
                \drawkeypadandtargets{26}
                \drawdisplay{48}
                \drawledcircuit
            \end{tikzpicture}
        }
        \vspace{.5in}
        \subfloat[Template that uses 4-prong pushbuttons.]{
            \hspace{-.5in}
            \begin{tikzpicture}[x=.108in, y=.1081in]  % TODO: determine which to use
                \drawbreadboard
                \drawnano{1}{3}{Arduino Nano}
                \drawnand{18}{5}
                \drawswitches{27}
                \drawbuttons{37}{4}
                \drawkeypadandtargets{26}
                \drawdisplay{48}
                \drawledcircuit
            \end{tikzpicture}
        }
        \caption{Templates to improve accuracy when constructing the Cow Pi circuit on a solderless breadboard.}\label{fig:templates}\addcontentsline{toc}{section}{Breadboard Templates}
    \end{figure}

\end{document}
