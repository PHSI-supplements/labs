%! suppress = MissingImport
This paragraph applies only to 2-lead pushbuttons:
\begin{quote}
    If your momentary pushbuttons are attached to a cardboard strip with tape, remove them from the cardboard strip.
    If your momentary pushbuttons' leads have metal tabs at the end (Figure~\ref{fig:pushbutton-tabs}), you will need to snip off the tabs before inserting the pushbutton leads into the breadboard;
    ordinary scissors will suffice for this task.
    Regardless of whether the leads have metal tabs at the end, you may optionally trim the leads to be about $\frac{1}{4}$in (6.4mm) long -- you can use the exposed lead from a jumper wire as a reference -- so that the pushbuttons sit flush on the breadboard.
    It is not necessary that they sit flush;
    this is simply to keep the buttons from wiggling under your fingers.
    \textit{Do not cut the leads shorter than $\mathit{\frac{1}{8}}$in (3.2mm)!}
    \textbf{Be sure to use eye protection in case the leads' ends fly off when you snip them.}
\end{quote}

This paragraph applies only to 4-prong pushbuttons:
\begin{quote}
    The four prongs on a momentary pushbutton are electrically connected as two pairs.
    If you attach the pushbuttons in the wrong orientation, it will appear to the \developmentboard as though they are always pressed.
    Fortunately, there is only one orientation that will place the prongs in the specified contact points.
    As long as the prongs are in the specified contact points, your pushbuttons will work fine.
    %TODO: add annotated photo of 4-prong electrical connections
\end{quote}

These are ``normally open'' momentary ``switches'' that close when pressed and re-open when released.
We will wire the pushbuttons such that they normally produce a 1, and when pressed will produce a 0.
Figure~\ref{fig:pushbutton-diagram} shows a diagram of the wiring for the pushbuttons.

\begin{figure}
    \centering
    \includegraphics[height=2cm]{direct/buttons/pushbutton-tabs}
    \caption{Some momentary pushbuttons have metal tabs on their leads.\label{fig:pushbutton-tabs}}
\end{figure}

\begin{figure}
    \centering
    \subfloat[2-lead pushbuttons]{
        \includegraphics[width=0.9\textwidth]{fritzing_diagrams/pushbutton-2lead}
    }
    \hfil
    \subfloat[4-prong pushbuttons]{
        \includegraphics[width=0.9\textwidth]{fritzing_diagrams/pushbutton-4prong}
    }
    \caption{Diagram of wiring associated with momentary pushbutton input.
        \textit{Note: connection between the 74LS20's pin 8 and the \developmentboard's
        \mcubuttonnand\ pin was previously installed in Section~\ref{subsec:nand}.}
        \label{fig:pushbutton-diagram}}
\end{figure}

\disconnect\

\begin{description}
    \checkoffitem{\prepunch{a\controlrow{11}, a\controlrow{13}, a\controlrow{15}, and a\controlrow{17}}
    If you have 4-prong pushbuttons, then also pre-punch holes into contact points d\controlrow{11}, d\controlrow{13}, d\controlrow{15}, and d\controlrow{17}.}

    \checkoffitem{If you have 2-lead pushbuttons:
        \begin{quote}
            Insert the leads of one pushbutton into contact points a\controlrow{11} and a\controlrow{13}.
            Insert the leads of the other pushbutton into contact points a\controlrow{15} and a\controlrow{17}.
        \end{quote}}

    \checkoffitem{If you have 4-prong pushbuttons:
        \begin{quote}
            Insert the prongs of one pushbutton into contact points a\controlrow{11}, d\controlrow{11}, a\controlrow{13} and d\controlrow{13}.
            Insert the prongs of the other pushbutton into contact points a\controlrow{15}, d\controlrow{15}, a\controlrow{17} and d\controlrow{17}.
        \end{quote}}

    \checkoffitem{Peel off one wire from the \rainbow, and use it to connect contact point e\controlrow{13} to the upper \ground.}
    \checkoffitem{Peel off one wire from the \rainbow, and use it to connect contact point e\controlrow{17} to the upper \ground.}

    \checkoffitem{Use two wires from the \rainbow to connect the ungrounded side of the left pushbutton to the 74LS20 and to the \developmentboard:
        connect e\controlrow{11} to \leftbuttontarget, and connect \mculeftbuttonfrom\ to \mculeftbuttonpoint.}
    \checkoffitem{Now use another two wires from the \rainbow to connect the ungrounded side of the right pushbutton to the 74LS20 and to the \developmentboard:
        connect e\controlrow{15} to \rightbuttontarget, and connect \mcurightbuttonfrom\ to \mcurightbuttonpoint.}
\end{description}
\textit{Notice that there are no wires in \nanduppernc} because, as you can see in Figure~\ref{fig:nand-pinout}, the 74LS20's pin 11 is not connected (``NC'') to anything.

See Figure~\ref{fig:pushbutton-wired}.

\begin{figure}%\begin{multicols}{2}
    \centering
    \subfloat[Pushbuttons with two leads.] {
        \includegraphics[height=4cm]{direct/buttons/pushbutton-2lead}
    }
    \hfil
    \subfloat[Pushbuttons with four prongs.] {
        \includegraphics[height=4cm]{direct/buttons/pushbutton-4prong}
    }
    \\
    \subfloat[The momentary pushbuttons, wired to the 74LS20 and the \developmentboard.]{
        \includegraphics[height=4cm]{direct/buttons/pushbutton-wired}
    }
%    \columnbreak
%    \end{multicols}
    \caption{Wiring the Momentary Pushbuttons.
        \label{fig:pushbutton-wired}}
\end{figure}

When you have finished setting up the pushbuttons' wiring, there should be the electrical paths described in Table~\ref{tab:pushbutton}.

\begin{table}
    \begin{center}\begin{tabular}{||c|c|c|c||} \hline\hline
    Pushbutton                      & 74LS20                & \developmentboard\ pin    & Pulled High/Low \\ \hline
    Left button's grounded lead     &                       &                   & \ground\    \\
    Left button's ungrounded lead   & Upper NAND Input      & \mculeftbutton    &               \\
    Right button's grounded lead    &                       &                   & \ground\    \\
    Right button's ungrounded lead  & Upper NAND Input      & \mcurightbutton   &               \\
                                    & Upper NAND Input      &                   & \ground\   \\
                                    & Upper NAND Input      &                   & \ground\   \\
                                    & Upper NAND Output     & \mcubuttonnand    &               \\ \hline
                                    & pin 11 (\nanduppernc) & \multicolumn{2}{c||}{not connected / floating} \\ \hline\hline
    \end{tabular}\end{center}
    \caption{Electrical Paths for Momentary Pushbuttons.\label{tab:pushbutton}}
\end{table}

\checkpoint{inserted and wired the momentary pushbuttons}

Connect your \developmentboard\ to the computer.
In the IDE's Serial Monitor, notice that Left~button is normally UP, but it becomes DOWN when you press the left button.
Similarly, Right~button is normally UP, but it becomes DOWN when you press the right button.
Notice also that Button~NAND is normally 0, but it becomes 1 whenever you press either button (or both).

Notice that both Left~LED and Right~LED are normally OFF\@.
Position both switches to the right.
Notice than when (and only when) the left button is to the right and you're also pressing the left button, Left~LED becomes ON, and the LED labelled ``L'' on the \developmentboard\ illuminates.
Similarly, when (and only when) the right button is to the right and you're also pressing the right button, Right~LED becomes ON, and the LED that you installed illuminates.
(There may be a delay of about a half-second between you pressing a button and the LED illuminating, and between you releasing a button and the LED deluminating.
We will look at why this happens and how to prevent it in an upcoming lab.)