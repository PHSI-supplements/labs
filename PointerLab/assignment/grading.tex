
When you have completed this assignment, upload \textit{answers.txt} and
\textit{challenge-response.c} to \filesubmission.

This assignment is worth 30 points.
\begin{description}
    \rubricitem{2}{Student's answers in \textit{answers.txt} demonstrate an
    understanding of the bug in Section~\ref{subsec:uninitializedvariables}'s code
    and how to correct it.}
    \rubricitem{2}{Student's answer in \textit{answers.txt} demonstrate an
    understanding of the bug in Section~\ref{subsec:localaddresses}'s code.}
    \rubricitem{1}{\function{create_node} creates and initializes a
    \lstinline{struct node} as specified.} % Depending on the insert_word
    % code, 0 or 1 is a reasonable initial value for the occurrences field
    \item[\hspace{1cm}]\function{insert_after} correctly places a new node in a
    list by updating:
    \begin{description}
        \rubricitem{2}{\lstinline{next} pointers (singly- and doubly-linked lists).}
        \rubricitem{2}{\lstinline{previous} pointers (doubly-linked lists).}
    \end{description}
    \rubricitem{2}{\function{insert_before} correctly places a new node in a
    doubly-linked list.}
    \rubricitem{2}{\function{word_to_lowercase} returns a copy of the input string
    with uppercase letters replaced with lowercase letters}
    \item[\hspace{1cm}]\function{insert_word}:
    \begin{description}
        \rubricitem{2}{creates a new node at the end of the list when the word
        properly belongs at the end of the list and is not already present in
        the list.}
        \rubricitem{2}{does not create a new node but instead updates the number of
        occurrences, when the word is present at the end of the list.}
        \rubricitem{2}{creates a new node at the appropriate location in the list,
            regardless of where its appropriate location is, when it is not already
            present in the list.}
        \rubricitem{2}{does not create a new node but instead updates the number of
        occurrences when the word is present, regardless of  its location in
        the list.}
    \end{description}
    \item[\hspace{1cm}]\function{build_list} opens a file for reading, builds a
    list by reading one line at a time and the word that is read to
    \function{insert_word()}, and closes the file after the last line has been
    read, when:
    \begin{description}
        \rubricitem{2}{the words in the file are pre-sorted, and the
        \lstinline{next} pointers are updated (\textit{i.e.}, part 1 of the
        sub-problems are complete).}
        \rubricitem{1}{the words in the file are not pre-sorted, and the
        \lstinline{next} pointers are updated (\textit{i.e.},
            Section~\ref{subsec:insertionsort} is complete).}
        \rubricitem{1}{the words in the file are pre-sorted, and the
        \lstinline{previous} pointers are updated (\textit{i.e.},
            Section~\ref{subsec:doublylinkedlist} is complete).}
        \rubricitem{1}{the words in the file are not pre-sorted, and the
        \lstinline{previous} pointers are updated (\textit{i.e.}, part 2 of the
        sub-problems are complete).}
    \end{description}
    \item[\hspace{1cm}]\function{respond} produces the correct response word in accordance with the specified rules when:
    \begin{description}
        \rubricitem{1}{the words in the file are pre-sorted, and the
        \lstinline{next} pointers are updated (\textit{i.e.}, part 1 of the
        sub-problems are complete).}
        \rubricitem{1}{the words in the file are not pre-sorted, and the
        \lstinline{next} pointers are updated (\textit{i.e.},
            Section~\ref{subsec:insertionsort} is complete).}
        \rubricitem{1}{the words in the file are pre-sorted, and the
        \lstinline{previous} pointers are updated (\textit{i.e.},
            Section~\ref{subsec:doublylinkedlist} is complete).}
        \rubricitem{1}{the words in the file are pre-sorted, and the
        \lstinline{previous} pointers are updated (\textit{i.e.}, part 2 of the
        sub-problems are complete).}
    \end{description}
    \item[Penalties]
    \penaltyitem{1}Newline characters are included in the word strings when
    building a list.
    \spaghetticodepenalties{1}
\end{description}
