\filesubmission.

\policyforcodethatdoesnotcompile

\latepolicy

\subsection*{Rubric}

This assignment is worth 35 points.
\begin{description}
    \rubricitem{1}{\function{is_nan()} correctly reports whether or not its argument is a number}
    \rubricitem{1}{\function{is_zero()} correctly reports whether or not its argument is zero}
    \rubricitem{1}{\function{is_infinity()} correctly reports whether or not its argument is infinite}
    \rubricitem{1}{\function{is_negative()} correctly reports whether or not its argument is negative}
    \rubricitem{1}{\function{get_754_integer()} correctly extracts the significand's implicit integer}
    \rubricitem{1}{\function{get_754_fraction()} correctly extracts the significand's fraction bits}
    \rubricitem{1}{\function{get_754_exponent()} correctly extracts the exponent}
    \rubricitem{1}{\function{decode()} correctly converts an \lstinline{ieee754_t} value into a \lstinline{unnormal_t} structure}
    \rubricitem{1}{\function{negate()} correctly changes its argument's sign}
    \rubricitem{5}{\function{add()} can add integers \& fractions, positive \& negative values, and ``large'' \& ``small'' numbers}
    \rubricitem{1}{The identity and commutative properties hold for \function{add()}}
    \rubricitem{1}{\function{add()} provides correct answers for its special cases}
    \rubricitem{5}{\function{multiply()} can multiply integers \& fractions, positive \& negative values, and ``large'' \& ``small'' numbers}
    \rubricitem{2}{The identity, zero, and commutative properties hold for \function{multiply()}}
    \rubricitem{1}{\function{multiply()} provides correct answers for its special cases}
    \rubricitem{1}{\function{divide()} provides correct answers for its special cases}
    \rubricitem{1}{\function{divide()} can divide when the divisor is of the form $\pm 2^n, -126 \le n \le 127$}
    \rubricitem{1}{\function{divide()} can divide when the dividend's significand is a multiple of the divisor's significand}
    \rubricitem{1}{\function{add()} demonstrates that \function{encode()} rounds down when the truncated part of the significand is less than halfway between representable values}
    \rubricitem{1}{\function{add()} demonstrates that \function{encode()} rounds up when the truncated part of the significand is more than halfway between representable values}
    \rubricitem{2}{\function{add()} demonstrates that \function{encode()} rounds to the nearest-even when the truncated part of the significand is exactly halfway between representable values}
    \rubricitem{1}{Rounding can carry into the exponent}
    \rubricitem{1}{\function{add()} and/or \function{multiply()} demonstrate that \function{encode()} overflows to infinity}
    \rubricitem{1}{\function{add()}, \function{multiply()}, and/or \function{divide()} demonstrate that \function{encode()} gracefully underflows through subnormal numbers}
    \rubricitem{1}{\function{multiply()} and/or \function{divide()} demonstrate that \function{encode() underflows to zero}}
    \bonusitem{2}{\function{divide()} can divide arbitrary values}
\end{description}

\textbf{Penalties}
\begin{description}
    \softwareengineeringpenalties
    \item[no credit] for functions that use \lstinline{float} or \lstinline{double} variables or constants, use \lstinline{union} variables, use C's floating point operations, and/or a function you did not write
    \item[no credit] for arithmetic functions, if \function{decode()} and/or \function{encode()}  use \lstinline{float} or \lstinline{double} variables or constants, use \lstinline{union} variables, use C's floating point operations, and/or a function you did not write
\end{description}
