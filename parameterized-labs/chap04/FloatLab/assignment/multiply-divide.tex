Recall that for any number base, $b$, multiplying two floating point values $(m_1 \times b^{e_1}) \cdot (m_2 \times b^{e_2})$ produces $m_1 \cdot m_2 \times b^{e_1 + e_2}$.
Similarly, dividing the two floating point values yields $\frac{m_1}{m_2} \times b^{e_1 - e_2}$.

\subsection{Multiply}

The \function{multiply()} stub identifies a handful of special cases that you can easily handle.
\begin{description}
    \checkoffitem{Produce the appropriate return values for the guard clauses.}
    \begin{itemize}
        \item If it is easier for you, you \textit{may} change the compound conditionals in the guard clauses into separate guard clauses.
            For example, instead of handling \\ \lstinline{if (is_infinity(multiplier) || is_zero(multiplier))} as a single guard clause,
            you may handle \lstinline{if (is_infinity(multiplier)} as its own guard clause and then handle \lstinline{if (is_zero(multiplier))} as its own guard clause.
    \end{itemize}
\end{description}

After the guard clauses, the operands are guaranteed to both be finite, non-zero numbers.
These lines
\begin{lstlisting}
unnormal_t decoded_multiplicand = prepare_for_arithmetic(decode(multiplicand));
unnormal_t decoded_multiplier = prepare_for_arithmetic(decode(multiplier));
\end{lstlisting}
create \lstinline{unnormal_t} representations of the operands that have their significands fully in the integer portion.
Because you do not need to give the two operands the same exponent, you do not need to worry about further adjustment.

\begin{description}
    \checkoffitem{Multiply the two operands together using integer arithmetic.}
\end{description}

\textit{Note}: For the \function{multiply()} function, we will not deduct points if you have the wrong sign for Not-a-Number.
\textit{We will, however, deduct points if you have the wrong sign for Zero.}


\subsubsection*{Check Your Work}

\begin{description}
    \checkoffitem{Compile and run \texttt{\textbf{\textit{./floatlab}}}, multiplying a few values.}
\end{description}
Be sure to check:
\begin{itemize}
    \item The identity, zero, and commutative properties
    \begin{itemize}
        \item[] \texttt{\textbf{\textit{5 * 1}}}
        \item[] \texttt{\textbf{\textit{5 * 0}}}
        \item[] \texttt{\textbf{\textit{2 * 3}}}
        \item[] \texttt{\textbf{\textit{3 * 2}}}
    \end{itemize}
    \item Integer operands
    \begin{itemize}
        \item[] \texttt{\textbf{\textit{75 * 7}}}
        \item[] \texttt{\textbf{\textit{5 * 25}}}
    \end{itemize}
    \item Fractional operands
    \begin{itemize}
        \item[] \texttt{\textbf{\textit{.75 * 7}}}
        \item[] \texttt{\textbf{\textit{5 * .25}}}
    \end{itemize}
    \item Negative operands
    \begin{itemize}
        \item[] \texttt{\textbf{\textit{-5 * 2}}}
        \item[] \texttt{\textbf{\textit{5 * -2}}}
        \item[] \texttt{\textbf{\textit{-5 * -2}}}
        \item[] \texttt{\textbf{\textit{5 * -0}}}
    \end{itemize}
    \item Numbers both great and small
    \begin{itemize}
        \item[] \texttt{\textbf{\textit{1.65e25 * 2.39e11}}}
        \item[] \texttt{\textbf{\textit{1.65e-25 * 2.39e-11}}}
        \item[] \texttt{\textbf{\textit{1e-30 * 1e-8}}}
        \item[] \texttt{\textbf{\textit{2e30 * 2e-30}}}
    \end{itemize}
    \item A sufficiently-large product overflows to infinity
    \begin{itemize}
        \item[] \texttt{\textbf{\textit{0x7E000000 * 0x41000000}}}
    \end{itemize}
    \item A sufficiently-small product underflows to zero
    \begin{itemize}
        \item[] \texttt{\textbf{\textit{0x3C800000 * 0x00000020}}}
    \end{itemize}
    \item NaN and Infinity are ``sticky'' (except for $\infty \times 0$)
    \begin{itemize}
        \item[] \texttt{\textbf{\textit{nan * 1}}}
        \item[] \texttt{\textbf{\textit{inf * 2}}}
        \item[] \texttt{\textbf{\textit{inf * 0}}}
    \end{itemize}
\end{itemize}

\subsection{Divide}

Implementing the \function{divide()} function is very similar to implementing \function{multiply()} with three exceptions:

\begin{itemize}
    \item There are different special cases
    \item You subtract the exponents and divide the significands
    \item In general, integer division is insufficient
    \begin{itemize}
        \item We will limit our implementation of \function{divide()} to the special cases and to cases in which the dividend's significand is a multiple of the divisor's significand.
        \item This limitation guarantees that if the \lstinline{unnormal_t} operands' significands are fully in the \lstinline{.integer} field then the \lstinline{unnormal_t} quotient's significand will fully be in the \lstinline{.integer} field and will not spill over into the \lstinline{.fraction} field.
        \item In this limited implementation, integer division is sufficient.
    \end{itemize}
\end{itemize}

\begin{description}
    \checkoffitem{Produce the appropriate return values for the guard clauses.}
    \checkoffitem{Divide $dividend \div divisor$ using integer arithmetic.}
\end{description}

\textit{Note}: For the \function{divide()} function, we will not deduct points if you have the wrong sign for Not-a-Number.
\textit{We will, however, deduct points if you have the wrong sign for Zero.}


\subsubsection*{Check Your Work}

\begin{description}
    \checkoffitem{Compile and run \texttt{\textbf{\textit{./floatlab}}}, dividing a few values.}
\end{description}

Be sure to check:
\begin{itemize}
    \item The identity property
    \begin{itemize}
        \item[] \texttt{\textbf{\textit{5 / 1}}}
    \end{itemize}
    \item Integer operands
    \begin{itemize}
        \item[] \texttt{\textbf{\textit{75 / 4}}}
    \end{itemize}
    \item Fractional operands
    \begin{itemize}
        \item[] \texttt{\textbf{\textit{.75 / 4}}}
        \item[] \texttt{\textbf{\textit{5 / .25}}}
        \item[] \texttt{\textbf{\textit{.75 / .25}}}
    \end{itemize}
    \item Negative operands
    \begin{itemize}
        \item[] \texttt{\textbf{\textit{-4 / 2}}}
        \item[] \texttt{\textbf{\textit{4 / -2}}}
        \item[] \texttt{\textbf{\textit{-4 / -2}}}
    \end{itemize}
    \item Divisors that require more than one 1 in the significand (but the dividend's significand is still a multiple of the divisor's significand)
    \begin{itemize}
        \item[] \texttt{\textbf{\textit{30 / 5}}}
        \item[] \texttt{\textbf{\textit{25 / 0x3F200000}}}
        \item[] \texttt{\textbf{\textit{0x3F480000 / 5}}}
    \end{itemize}
    \item The special cases
    \begin{itemize}
        \item[] \texttt{\textbf{\textit{nan / 2}}}
        \item[] \texttt{\textbf{\textit{2 / nan}}}
        \item[] \texttt{\textbf{\textit{inf / 2}}}
        \item[] \texttt{\textbf{\textit{0 / 2}}}
        \item[] \texttt{\textbf{\textit{2 / inf}}}
        \item[] \texttt{\textbf{\textit{2 / 0}}}
        \item[] \texttt{\textbf{\textit{inf / inf}}}
        \item[] \texttt{\textbf{\textit{0 / 0}}}
    \end{itemize}
\end{itemize}
