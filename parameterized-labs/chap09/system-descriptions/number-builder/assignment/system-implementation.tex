In this section you will implement a ``number building tool'' that turns a sequence of inputs into a number.
A notional use of this number builder would be to create the operands for a calculator.
Section~\ref{subsec:functionalspecification} has the tool's specification.
You should design your system using the I/O functions in \textit{io\_functions.c} -- either the implementations from the starter code, or the implementations you wrote (or will write) from Section~\ref{sec:MemMapIO}.
All logic for your ``number building tool'' should go in \textit{number\_builder.c}.
After the specification are a few suggestions for making good use of those functions.

\textit{Remember that to start your simple embedded system, the \textbf{left switch} must be in the \underline{left} position when the Cow~pi restarts.}

You can complete this part of the assignment before or after implementing the I/O code from Section~\ref{sec:MemMapIO}.
You must, however, complete Section~\ref{subsec:populatekeypad} first.
If you have already implemented the I/O code from Section~\ref{sec:MemMapIO}, you may want to make a backup copy of your program now.


\subsection{Number Builder Specification} \label{subsec:functionalspecification}

%! suppress = LabelConvention
\begin{enumerate}
    \item The tool shall have two alignment modes, \textit{left justified mode} and \textit{right justified mode}. The \textbf{left switch} controls the alignment mode.
        When the \textbf{left switch} is toggled to the left, the tool shall be in left justified mode.
        When the \textbf{left switch} is toggled to the right, the tool shall be in right justified mode.
    \item If the user toggles the \textbf{left switch} (\textit{i.e.}, changes alignment mode):
        if no number is being displayed, then the system shall seamlessly transition between alignment modes;
        otherwise, the system's behavior is undefined.
    \item The tool shall have two number bases, \textit{decimal} and \textit{hexadecimal}.
        While the number shall be displayed in both number bases (requirement~\ref{spec:displayFormat}),
        the \textbf{right switch} controls which number base is used to build the number.
    \item \label{spec:decimalExplained} When the \textbf{right switch} is toggled to the left, the number builder will use the decimal number base for inputs.
        When the user presses a button on the \textbf{matrix keypad} with a decimal numeral (\textit{0-9}) then the number builder shall take the appropriate action as specified in requirement~\ref{spec:initialKeyPress} or \ref{spec:BuildingValue}.
        Any other buttons on the keypad shall be ignored.
    \item \label{spec:hexadecimalExplained} When the \textbf{right switch} is toggled to the right, the number builder will use the hexadecimal number base for inputs.
        When the user presses a button on the \textbf{matrix keypad} then the number builder shall take the appropriate action as specified in requirement~\ref{spec:initialKeyPress} or \ref{spec:BuildingValue}.
        Buttons with decimal numerals (\textit{0-9}) or alphabetic letters (\textit{A-D}) shall be interpreted as having the corresponding hexadecimal numeral;
        the button with the octothorp (\#) shall be interpreted as having the hexadecimal numeral \textit{E};
        and the button with the asterisk (*) shall be interpreted as having the hexadecimal numeral \textit{F}.
    \item If the user toggles the \textbf{right switch} (\textit{i.e.}, changes the number base):
        if no number is displayed, or if the number 0 is displayed, then the system shall seamlessly transition between number bases;
        if a non-zero number is being built then the system's behavior is undefined.
    \item Initially, the \textbf{display module} shall be blank.
    \item \label{spec:displayFormat} When the \textbf{display module} is non-blank (\textit{i.e.}, a number is being built):
        the number's decimal representation shall be on the \textbf{display module}'s first row,
        and the number's hexadecimal representation shall be on the \textbf{display module}'s second row.
        These representations shall be aligned to the left side or to the right side of the \textbf{display module}, in accordance with the tool's alignment mode.
    \item \label{spec:illuminateLED} Whenever the user presses a key on the \textbf{matrix keypad}, the \textbf{right LED} (aka, the external LED) shall illuminate for approximately one-half of a second.
        Whenever the user presses one of the \textbf{pushbuttons}, the \textbf{left LED} (aka the internal LED) shall illuminate for approximately one-half of a second.
    \item \label{spec:LEDoffWhenOtherwise} The \textbf{LEDs} shall not illuminate except as specified in requirement~\ref{spec:illuminateLED}.
    \item The system shall build numbers in response to user input:
        \begin{enumerate}
        \item \label{spec:32bits} The numbers' values shall be representable as 32-bit signed integers.
        \item \label{spec:initialKeyPress} If the display is blank (\textit{i.e.}, there is not yet a number being built), then whenever the user presses a button on the \textbf{matrix keypad}, the corresponding numeral shall be displayed on the \textbf{display module}, either in the far-left column or the far-right column, depending on the alignment mode.
            If decimal input mode is used, then both the decimal and hexadecimal representations shall display the numeral.
            If hexadecimal input mode is used, then the hexadecimal representation shall display the numeral,
            and the decimal representation shall display the corresponding decimal value.
            The numeral displayed shall follow the interpretations specified in requirements~\ref{spec:decimalExplained} and \ref{spec:hexadecimalExplained}.
            \begin{itemize}
                \item \textit{Note:} In accordance with requirement~\ref{spec:displayFormat}, the number shall be displayed in both decimal and in hexadecimal.
                    If the tool is receiving inputs in the hexadecimal number base, and if the first digit is in the inclusive range $A-F$, then the decimal representation will necessarily require more than one decimal digit.
            \end{itemize}
        \item \label{spec:BuildingValue} If the display is not blank (\textit{i.e.}, there is a number being built), then whenever the user presses a button on the \textbf{matrix keypad}, then the number's value shall change:
            \begin{itemize}
                \item The digits in the representation corresponding to the input number base shall increase in significance by 1, and the newly-input digit shall become that number base's least significant digit.
                \item The representation in the other number base shall also update to reflect the new value.
            \end{itemize}
            For example, if the display shows \\
            \display{
                \colorbox{LightGreen}{\phantom{xxxxxxxxxxxxx}234} \vspace{-1mm}\\
                \colorbox{LightGreen}{\phantom{xxxxxxxxxxxx}0xEA}
            } \\
            and the user presses \texttt{5} while the input number base is decimal, then the display shall update to \\
            \display{
                \colorbox{LightGreen}{\phantom{xxxxxxxxxxxx}2345} \vspace{-1mm}\\
                \colorbox{LightGreen}{\phantom{xxxxxxxxxxx}0x929}
            } \\
            Similarly, if the display shows \\
            \display{
                \colorbox{LightGreen}{\phantom{xxxxxxxxxxxxx}564} \vspace{-1mm}\\
                \colorbox{LightGreen}{\phantom{xxxxxxxxxxx}0x234}
            } \\
            and the user presses \texttt{5} while the input number base is hexadecimal, then the display shall update to \\
            \display{
                \colorbox{LightGreen}{\phantom{xxxxxxxxxxxx}9029} \vspace{-1mm}\\
                \colorbox{LightGreen}{\phantom{xxxxxxxxxx}0x2345}
            } \\
            The numeral displayed shall follow the interpretations specified in requirements~\ref{spec:decimalExplained} and \ref{spec:hexadecimalExplained}.
            \begin{enumerate}
                \item There shall be no noticeable lag in updating the display.
                \item The alignment mode shall be preserved.
                    If in left justified mode, then the most-significant digit shall remain in the leftmost column.
                    If in right justified mode, then the new least-significant digit shall be in the rightmost column.
                % \item \label{spec:printValueToConsole} The new value shall be printed
                %     to the Serial Monitor in decimal or hexadecimal, depending on the
                %     system's current number base.
            \end{enumerate}
        \item ``0x'' or ``0X'' shall be displayed as part of a hexadecimal value (``0x'' is optional when the value is 0).
        % \item ``0x'' shall not be displayed as part of a hexadecimal value.
        \item Whenever the user presses the \textbf{left pushbutton}, the value being displayed shall be negated.
            \begin{enumerate}
                \item In the decimal representation, the presence or absence of negative sign shall indicate whether or not a value is negative.
                \item In the hexadecimal representation, 32-bit two's complement shall be used.
            \end{enumerate}
        \item A positive sign shall not be displayed as part of a positive value.
        \item When a negative decimal value is displayed, the negative sign shall be displayed immediately to the left of the most-significant digit being displayed.
            If in left justified mode, the negative sign shall be in the leftmost column, and the most-significant digit shall be in the column immediately to the right of that.
            For example, these two outputs are correctly displayed: \\
            \display{\colorbox{LightGreen}{\phantom{xxxxxxxxxxxx}-456}} \hspace{5mm} \display{\colorbox{LightGreen}{-456\phantom{xxxxxxxxxxxx}}} \\
            but these three are not correctly displayed: \\
            \display{\colorbox{LightGreen}{\phantom{xxxxxxxx}-\phantom{xxxx}456}} \hspace{5mm} \display{\colorbox{LightGreen}{-\phantom{xxxx}456\phantom{xxxxxxxx}}} \hspace{5mm} \display{\colorbox{LightGreen}{-\phantom{xxxxxxxxxxxx}456}}
        \item Leading 0s may be displayed but are discouraged.
            For example, \display{\colorbox{LightGreen}{782}} is preferred, but \display{\colorbox{LightGreen}{00000782}} is allowed.
        \item Digit separators are permitted but not required.
            For example, both \display{\colorbox{LightGreen}{12345}} and \display{\colorbox{LightGreen}{12,345}} are acceptable, as are both \display{\colorbox{LightGreen}{0x6789A}} and \display{\colorbox{LightGreen}{0x6'789A}}.
        \item \label{spec:clearNumber} Whenever the user presses the \textbf{right pushbutton} while a number is being built, the number that has been built shall reset to 0, and the \textbf{display module} shall be cleared.
%        \item \label{spec:clearNumber} Whenever the user presses the \textbf{right pushbutton} while a number is being built, the number that has been built shall be displayed on the bottom row, aligned according to the alignment mode, for approximately one second.
%            After approximately one second, the number that has been built shall reset to 0, and the \textbf{display module} shall be cleared.
        % \item In no case shall the tool allow the user to input a value too
        %     great to be displayed on the \textbf{display module}. If
        %     the user attempts to enter a value greater than $0x7FFF,FFFF$ in
        %     hexadecimal, less than $0x8000,0000$ in hexadecimal, greater than
        %     $99,999,999$ in decimal, or less than $-9,999,999$ in decimal, then the
        %     \textbf{display module} shall display \display{too big}.
        \item \label{spec:tooBig} In no case shall the tool allow the user to input a value too great to be stored as a 32-bit signed integer.
            If the user attempts to enter a value greater than $0x7FFF,FFFF$ in hexadecimal, less than $0x8000,0000$ in hexadecimal, greater than $2,147,483,647$ in decimal, or less than $-2,147,483,648$ in decimal, then the \textbf{display module} shall display \\
            \display{
                \colorbox{LightGreen}{\phantom{xxxxxx}TOO\phantom{xxxxxx}} \vspace{-1mm}\\
                \colorbox{LightGreen}{\phantom{xxxxxx}BIG!\phantom{xxxxx}}
            } \\
        \begin{itemize}
            \item \textit{n.b.}, in a hexadecimal 32-bit negative integer, an \texttt{F} in the most-significant hex-digit might only be a sign extension;
                in this scenario, if the user inputs another hex-digit then it is still a valid number because it would not be less than $0x8000,0000$. \\ \\
                For example, adding another digit to the 32-bit integer $0xF865,4321$ would \textit{not} produce a too-great value because the 36-bit integer $0xF,8654,3210 = -2,041,302,512_{10}$, and also the 32-bit integer $0x8654,3210 = -2,041,302,512_{10}$.
                Because $0x8000,0000 = -2,147,483,648_{10} \leq -2,041,302,512_{10}$, adding the \texttt{0} digit did not produce a ``too big'' value. \\ \\
                On the other hand, adding another digit to the 32-bit integer $0xF765,4321$ \textit{would} produce a too-great value because the 36-bit integer $0xF,7654,3210 = -2,309,737,968_{10}$, but the 32-bit integer $0x8654,3210 = 1,985,229,328_{10}$.
                Because $0x8000,0000 = -2,147,483,648_{10} > -2,309,737,968_{10}$, adding the \texttt{0} digit did produce a ``too big'' value.
            \end{itemize}
        \end{enumerate}
    %\item \label{spec:singleKeypress} When the user presses a button for less than
    %    approximately one-half of a second, regardless of whether it is one of the
    %    \textbf{pushbuttons} or a key on the \textbf{matrix keypad}, it shall be treated
    %    as a single press.
    \item \label{spec:singleKeypress} When the user presses a button, regardless of whether it is one of the \textbf{pushbuttons} or a key on the \textbf{matrix keypad}, and regardless of the duration of the press, it shall be treated as a single press.
    \item \label{spec:responsive} The system shall always be responsive to user input.
        \begin{itemize}
            \item There shall be no noticeable lag in updating the display or illuminating/deluminating the LEDs.
            \item When the user is using an input device, the system shall also respond to other input devices.
                \begin{itemize}
                    \item When the user is pressing a button, the system shall respond to the user pressing the other button and/or to pressing a key on the keypad.
                    \item When the user is pressing a key, the system shall respond to the user pressing one or both of the buttons.
                    \item During the half-second that an LED is illuminated for Requirement~\ref{spec:illuminateLED}, the system shall respond to inputs (restarting the half-second if appropriate).
                \end{itemize}
            \item In general, the system shall not block except as necessary for an accurate scan of the keypad and for sending data to the display module.
        \end{itemize}
        Exception to Requirement~\ref{spec:responsive}:
        \begin{itemize}
            \item There is no requirement to detect that the user is simultaneously pressing multiple keys on the keypad.
%            \item During the second that a number is displayed on the bottom row before the display clears (Requirement~\ref{spec:clearNumber}), if the user presses a button or key then the appropriate LED shall illuminate in accordance with Requirement~\ref{spec:illuminateLED}, but no other action is required.
        \end{itemize}
\end{enumerate}


\subsection{Simplified State Diagram}

The state diagram in Figure~\ref{fig:numberBuilderStateDiagram} doesn't fully capture all of the system's requirements,
but it characterizes the basic handling of inputs.

\begin{figure}[h]
    \centering
    \begin{tikzpicture}
        \node[state,initial](blank){Blank Display};
        \node[state] at (-4, -3)  (decimalUpdate){Update Number};
        \node[state] at (4, -3)   (hexadecimalUpdate){Update Number};
        \node[state] at (-3, -7) (decimalDisplay){Display Number};
        \node[state] at (3, -7)  (hexadecimalDisplay){Display Number};
        \draw   (blank) edge[-Latex, left=1] node{\textcolor{blue}{decimal mode; 0--9 pressed}} (decimalUpdate)
        (blank) edge[-Latex, right=1] node{\textcolor{blue}{hexadecimal mode; 0--F pressed}} (hexadecimalUpdate)
        (decimalUpdate) edge[-Latex, bend left, below] node{} (decimalDisplay)
        (hexadecimalUpdate) edge[-Latex, bend right, below] node{} (hexadecimalDisplay)
        (decimalDisplay) edge[-Latex, bend left, above] node{\textcolor{blue}{negate or 0--9 pressed\phantom{xxxxx}}} (decimalUpdate)
        (hexadecimalDisplay) edge[-Latex, bend right, above] node{\textcolor{blue}{\phantom{xxxxx}negate or 0--F pressed}} (hexadecimalUpdate)
        (decimalDisplay) edge[-Latex, bend right] node{\textcolor{blue}{clear pressed}} (blank)
        (hexadecimalDisplay) edge[-Latex, bend left] node{} (blank);
    \end{tikzpicture}
    \caption{Simplified state diagram for building numbers. \label{fig:numberBuilderStateDiagram}}
\end{figure}

%\begin{figure}[h]
%    \centering
%    \begin{tikzpicture}
%        \umlstateinitial[y=3, name=initial]
%        \umlbasicstate[name=blank]{Blank Display}
%        \begin{umlstate}[x=-4, y=-5, name=decimalInput, do={$number = 10 \times number + digit$}]{Update Number}\end{umlstate}
%        \begin{umlstate}[x=4, y=-5, name=hexadecimalInput, do={$number = 16 \times number + digit$}]{Update Number}\end{umlstate}
%        \umlbasicstate[x=-4, y=-8, name=decimalDisplay]{Display Number}
%        \umlbasicstate[x=4, y=-8, name=hexadecimalDisplay]{Display Number}
%        \umltrans{initial}{blank}
%        \umltrans[arg={decimal mode}, mult={0--9 pressed}, align=left]{blank}{decimalInput}
%        \umltrans[arg={hexadecimal mode}, mult={0--F pressed}, align=left]{blank}{hexadecimalInput}
%        \umlHVHtrans[arm1=-4, arg={0--9 or - pressed}]{decimalDisplay}{decimalInput}
%        \umlHVHtrans[arm1=4, arg={0--F or - pressed}]{hexadecimalDisplay}{hexadecimalInput}
%        \umltrans{decimalInput}{decimalDisplay}
%        \umltrans{hexadecimalInput}{hexadecimalDisplay}
%        \umlHVtrans{decimalDisplay}{blank}
%        \umlHVtrans[arg={clear pressed}]{hexadecimalDisplay}{blank}
%    \end{tikzpicture}
%    \caption{Simplified state diagram for building numbers.}
%\end{figure}


\subsection{Using I/O Functions for Polling}

Polling, which you encountered in DuplicatorLab, is simply repeatedly checking whether some condition holds, and performing some action if it does.
In this assignment, you will poll the pushbuttons, slide-switches, and numeric keypad.
The I/O functions in \textit{io\_functions.c} access these devices to determine what buttons are pressed and what the slides' positions are.
Thus, you can poll the inputs by repeatedly calling those functions.
The specification in Section~\ref{subsec:functionalspecification} tells you what actions should be taken based on those inputs.

Recall that \function{build_number()} is called in each iteration of a \lstinline{while(true)} loop.
This means that you can repeatedly call the I/O functions -- you can poll the inputs -- simply by calling the functions once in your \function{build_number()} function's body.

The header comments for the I/O functions in the starter code amply describe the functions.
Updating the LEDs and reading the switches' positions require no special handling;
however, polling the momentary pushbuttons and the keypad keys requires some extra care.


\subsection{Detecting Key and Button Actions}

Requirement~\ref{spec:singleKeypress} requires that each button press or key press is treated as exactly one press.
Debouncing addresses the problem of the mechanical components bouncing back and forth between ``open'' and ``closed''.
The other problem that needs to be addressed is that the pushbuttons and keys are polled with each iteration of the control loop,
and the control loop iterates much faster than a human can remove their finger -- this resuls in a single press of a button or key being detected many, many times.

%You will overcome this with the \function{is_changed_input()} function.
%The idea is that after you poll a pushbutton or the keypad,
%pass the value to \function{is_changed_input()} along with the input type (\lstinline{KEYPAD}, \lstinline{LEFT_BUTTON_DOWN}, or \lstinline{RIGHT_BUTTON_DOWN}).\footnote{
%    These are defined in the CowPi library and are convenient for our use here.
%}
%The function will compare that value with the value that was passed to it in the previous iteration to determine whether the input has changed.
%The first few lines of \function{build_number()} demonstrate the use of \function{is_changed_input()}, using the slide-switches.
%\textit{The number builder specification does not require you to detect the exact moment that a slide-switch is toggled.}
%
%Locate the \function{is_changed_input()} function.
%\begin{description}
%    \checkoffitem{Use the \lstinline{input_source} argument to index the \lstinline{values_from_last_iteration} array to obtain the input source's previous value.}
%    \checkoffitem{Compare the previous value to the \lstinline{current_value}.}
%    \checkoffitem{Store the current value in the array, to be compared in the next iteration.}
%    \checkoffitem{Return \lstinline{true} if the previous value and the current value are different, or \lstinline{false} if there is no change.}
%\end{description}
%
%Locate the \function{build_number()} function.
%\begin{description}
%    \checkoffitem{Add code to determine which (if any) key is being pressed, and code to determine whether each pushbutton is being pressed.}
%    \checkoffitem{Add an \lstinline{if} block that will execute if a key was just pressed.}
%    \checkoffitem{Add an \lstinline{if} block that will execute if the left pushbutton was just pressed.}
%    \checkoffitem{Add an \lstinline{if} block that will execute if the right pushbutton was just pressed.}
%\end{description}

You will overcome this by implementing three state machines: one for each pushbutton and one for the keypad.
The idea is that each of these inputs is in one of four possible states (Figure~\ref{fig:buttonStateMachine}).
\begin{itemize}
    \item If the input's state machine is in the \textit{Not Pressed} state, it stays in that state until the input is pressed.
        When the input is pressed, the input's state machine transitions to the \textit{Respond to Press} state.
    \item If the input's state machine is in the \textit{Respond to Press} state, then the system takes whatever one-time action needs to be taken in response to the input being pressed.
        After this action is complete, the input's state machine transitions to the \textit{Pressed} state.
    \item If the input's state machine is in the \textit{Pressed} state, it stays in that state until the input is released.
        When the input is released, the input's state machine transitions to the \textit{Respond to Release} state.
    \item If the input's state machine is in the \textit{Respond to Release} state, then the system takes whatever one-time action (if any) needs to be taken in response to the input being released.
        After this action is complete, the input's state machine transitions to the \textit{Not Pressed} state.
\end{itemize}

\begin{figure}%[h]
    \centering
    \begin{tikzpicture}
        \node[state, initial] at (-4, -3)  (notPressed){Not Pressed};
        \node[state]          at (4, -3)   (respondingToPress){Respond to Press};
        \node[state]          at (3, -7)   (pressed){Pressed};
        \node[state]          at (-3, -7)  (respondingToRelease){Respond to Release};
        \draw   (notPressed)          edge[-Latex, bend left, above] node{\textcolor{blue}{button/keypad is pressed}} (respondingToPress)
                (respondingToPress)   edge[-Latex, bend left] (pressed)
                (pressed)             edge[-Latex, bend left, below] node{\textcolor{blue}{button/keypad is released}} (respondingToRelease)
                (respondingToRelease) edge[-Latex, bend left] (notPressed);
    \end{tikzpicture}
    \caption{State machine that responds to presses and releases, but not to continuously pressing (or not pressing) a button. \label{fig:buttonStateMachine}}
\end{figure}

Locate the \function{build_number()} function.
\begin{description}
    \checkoffitem{Add code to determine which (if any) key is being pressed,\footnote{Reminder: \function{get_keypress()} returns \lstinline{0xFF} when no key is being pressed.} and code to determine whether each pushbutton is being pressed.}
    \checkoffitem{Add either chained \lstinline{if}/\lstinline{else if} blocks, or a \lstinline{switch} statement that will transition \lstinline{keypad_state} between states according the the input state machine just described.}
    \checkoffitem{Add either chained \lstinline{if}/\lstinline{else if} blocks, or a \lstinline{switch} statement that will transition \lstinline{left_button_state} between states according the the input state machine just described.}
    \checkoffitem{Add either chained \lstinline{if}/\lstinline{else if} blocks, or a \lstinline{switch} statement that will transition \lstinline{right_button_state} between states according the the input state machine just described.}
\end{description}


\subsection{Measuring the Passage of Time}

Requirement~\ref{spec:illuminateLED} requires that LEDs be illuminated for 500ms (or $500,000\mu\mathrm{s}$) after button presses or key presses.
You can use the \function{get_microseconds()} function to determine how long, in microseconds, your program has been running.
The busy-wait that you used in \function{get_keypress()} is \textit{blocking} code -- the program cannot progress until the busy-wait is finished.
You need a \textit{non-block} solution to satisfy the responsiveness requirement (Requirement~\ref{spec:responsive}).

In the case of Requirement~\ref{spec:illuminateLED}, you can call \function{get_microseconds()} in each execution of \function{build_number()} to determine the system time.
Using that ``now'' value, you can keep track of when an LED was required to illuminate.
If you compare the current time to the time that an LED was required to illuminate, you know whether an LED should be illuminated or not.\footnote{
    \textit{Technically}, this will also cause the LED to illuminate after 72 minutes of inactivity, violating Requirement~\ref{spec:LEDoffWhenOtherwise}, but we promise not to test for that particular scenario.
}
You do not even need to check whether the LED is already illuminated:
calling \lstinline{set_xx_led(true)} will illuminate that LED if it is not illuminated and will leave it illuminated if it already is.
Similarly, calling \lstinline{set_xx_led(false)} will deluminate that LED if it is illuminated and will leave it deluminated if it is already off.

\begin{description}
    \checkoffitem{Add variables to \function{build_number()} to track the times of the last key press and the last button press.
        \textit{These variables need to ``remember'' their values between calls to \function{build_number()}.}}
    \checkoffitem{In each of the aforementioned state machines, add code to note the time that a press occurred.}
    \checkoffitem{Add code to \function{build_number()} to update each LED's illumination state based on how much time has elapsed since the last key/button press.}
\end{description}

\subsubsection*{Test your code}
\begin{description}
    \checkoffitem{Place the \textbf{left switch} in the \textit{right position} and upload the program to your Cow~Pi.}
    \checkoffitem{Check to see if the appropriate LED lights up when you press a key or a button and go out a half-second later.}
    \checkoffitem{Can you make both LEDs illuminate by pressing a key \textit{and} a button at about the same time?}
    \checkoffitem{Does the time that an LED is illuminated continue to be a half-second after the last key/button press?
        \textit{e.g.}, if you press a button, release it, and press it again about a quarter-second after the first press, does the LED illuminate for about three-quarters of a second?}
\end{description}


\subsection{Outputting to the Display Module} \label{subsec:numberBuilderOutput}

The \textit{display.h} file declares a \function{display_string()} function.
The first argument is which row (0 or 1) that you want to display a string,
and the second argument is the string to display.
The display can show up to 21 characters per line.
%The display can show up to 16 characters per line.

%We recommend that you create a 17-byte buffer,
We recommend that you create a 22-byte buffer,
use \function{sprintf()} to ``print'' to the buffer,
and then pass the buffer to \function{display_string()}.
You might want to review the \function{printf}/\function{fprintf}/\function{sprintf} conversion specifiers in Section~\ref{subsec:conversionSpecifiers}.

Look at the end of the \function{test_io()} function in \textit{io\_functions.c} for examples.

\ifdefstring{\processor}{RP2040}{
    \textcolor{red}{
%        If a Windows driver issue prevents you from printing to the Serial Monitor,
%        then you can instead send short debug messages to the display module.
%        The display module can display up to 8 rows of text, 16 characters per row;
%        however, the assignment only makes use of rows~0 \& 1.
%        This leaves rows~2--7 available for short (16 character) debug messages.
        You can send short debug messages to the display module.
        The display module can display up to 8 rows of text, 21 characters per row;
        however, the assignment only makes use of rows~0 \& 1.
        This leaves rows~2--7 available for short (21 character) debug messages.
        You might find \function{count_visits}, declared in \textit{display.h} to be useful.
    }
}{}

Locate the \function{print_number()} function.
\begin{description}
    \checkoffitem{Add code to send the appropriate strings to the display module, considering whether the output should be left- or right-justified.}
\end{description}

%You will not call \function{send_halfbyte()} directly.
%Instead, the CowPi\_stdio library's functions that work with the display module make use of \function{send_halfbyte()} after you have implemented and uncommented the assignment to \lstinline{cowpi_hd44780_send_halfbyte}.
%
%The \function{fprintf()}\footnote{\url{https://pubs.opengroup.org/onlinepubs/7908799/xsh/fprintf.html}} function is essentially a version of \function{printf()} that ``prints'' to an arbitrary file stream,
%and nearly everything outside of the program is considered to be a file.
%Indeed, \function{printf(...)} is equivalent to \function{fprintf(stdout, ...)} because \lstinline{stdout} is the file stream to the standard output.
%The first argument is a pointer to the file stream; the second argument is the format string (which would be \function{printf()}'s first argument), and any subsequent arguments are the variables that populate the placeholders in the format string.
%
%You can print to the display module using \function{fprintf()} with the \lstinline{display} file stream;
%look at the end of the \function{test_io()} function in \textit{io\_functions.c} for examples of printing to the display module.

%We recommend that you review the \function{printf}/\function{fprintf}/\function{sprintf} conversion specifiers in Section~\ref{subsec:conversionSpecifiers}.
%and also the \href{https://cow-pi.readthedocs.io/en/latest/CowPi_stdio/lcd_character.html#ascii-control-characters}{behavior associated with ASCII control characters} for the display module.
%(Besides the description of the control codes' behaviors, you may want to scroll up on that webpage to look at the gif demonstrating the behaviors.)


\subsection{Building 32-Bit Numbers}

Some students in the past had difficulty implementing Requirement~\ref{spec:BuildingValue}.

Let us consider the case where you're providing decimal inputs.
Viewed as a weighted sum, an $n$-digit number $d_{n-1}d_{n-2}{\dots}d_2d_1d_0$ has the value
\[number = \sum_{i=0}^{n-1}d_i \times 10^i\]

Requirement~\ref{spec:BuildingValue} states than when you introduce a new digit, $d_{new}$ where $0 \leq d_{new} < 10$, it will be the least-significant digit (that is, its weight will be ``times 1''), and all pre-existing digits will increase in significance by one order of magnitude.
So the new value is
\begin{align*}
    new\_number = \left(\sum_{i=0}^{n-1}d_i \times 10^{i+1}\right) + \left(d_{new} \times 1\right)
        & = \left(\sum_{i=0}^{n-1}d_i \times 10^i \times 10\right) + d_{new} \\
        & = \left(10 \times \sum_{i=0}^{n-1}d_i \times 10^i\right) + d_{new} \\
        & = 10 \times number + d_{new}
\end{align*}

So, if you're building a positive decimal number, you update the number by multiplying the old value by 10 and adding the new digit.

The idea is similar when you're building a negative number.

When you're building a hexadecimal number, since you're specifying the actual bits,
it's probably conceptually easier to think of shifting the number's bits to make room for the new hex-digit,
and then placing the new hex-digit in the number's lower four bits.

Locate the \function{process_digit()} function.
\begin{description}
    \checkoffitem{Add code to determine whether the system is in decimal-input or hexadecimal-input mode.}
    \checkoffitem{Add code to compute the new number, considering:
        \begin{itemize}
            \item The input's number base
            \item Whether the digit is a valid digit for that number base
            \item If decimal, whether the old number is negative or non-negative
        \end{itemize}
    }
    \checkoffitem{Return the new number.}
\end{description}
In \function{build_number()}:
\begin{description}
    \checkoffitem{Add code to track the number being built.}
    \checkoffitem{Add code to update the number as digits are entered.
        \begin{itemize}
            \item Be sure to respond only to key \textit{presses} and not key releases.
        \end{itemize}
    }
\end{description}

\subsubsection*{Test your code}
\begin{description}
    \checkoffitem{Place the \textbf{left switch} in the \textit{right position} and upload the program to your Cow~Pi.}
    \checkoffitem{Test that you can ``build'' a number by pressing number keys.
        \begin{itemize}
            \item You cannot yet negate a number.
            \item You cannot yet detect a ``too bit'' number.
            \item So far, you have no way to clear a number except by pressing the RESET button.
        \end{itemize}
    }
    \checkoffitem{Check that you handle both decimal and hexadecimal inptuts.}
    \checkoffitem{Check both left justification and right justification.}
\end{description}


\subsection{Negating and Clearing Numbers}

In \function{build_number()}:
\begin{description}
    \checkoffitem{Add code that resets the number to 0 and clears the display module when the ``clear'' button has been pressed.}
    \checkoffitem{Add code that negates the number when the ``negate'' button has been pressed.}
\end{description}

\subsubsection*{Test your code}
\begin{description}
    \checkoffitem{Place the \textbf{left switch} in the \textit{right position} and upload the program to your Cow~Pi.}
    \checkoffitem{Test that you can ``build'' a number, clear it, and build another number.}
    \checkoffitem{Test that:}
        \begin{description}
            \checkoffitem{You can negate a positive number.}
            \checkoffitem{You can negate a negative number.}
            \checkoffitem{If you negate a number, and negate it again, then the result is the original number.}
        \end{description}
\end{description}



\subsection{Too-Big Numbers}

Requirement~\ref{spec:tooBig} is really about signed integer overflow.
Requirement~\ref{spec:32bits} says that the number has to be representable as a 32-bit signed integer;
if the number cannot fit in 32 bits then it is too big.

\subsubsection{Due to Processing a Digit}

In Chapter~3, you learned that you can detect signed integer overflow due to addition by examining the sign bits.
Unfortunately, here you are performing two arithmetic operations, multiplication and addition, either one of which could overflow, and together might defeat the ``examine the sign bits'' technique.
For example, when limited to 32 bits, \[-2,147,483,648_{10} \times 10_{10} - 1_{10} = -1_{10}\].
(Go ahead, try to build the number $-21,474,836,481$. I'll wait.)

Fortunately, because of the nature of the problem, we know that processing a new digit should increase the magnitude of the number being built:
a negative number should become more negative, and a positive number should become more positive.
If a negative number becomes a value \textit{greater than} its original value, or if a positive number becomes a value \textit{less than} its original value, then overflow occurred.

There is an exception to this rule.

When building a negative number with hexadecimal input, the user has two equally-valid options:

\begin{itemize}
    \item Build the number directly.
        For example, if we wanted to build $-126_{10} = 0xFFFF'FF82$ with hexadecimal input, then we could press the \texttt{F} key six times, followed by \texttt{8} and \texttt{2}.
        This is a special case because after entering $0xFFF'FFF8 = 268,435,448_{10}$, it's a positive number.
        But when we press the \texttt{2} to create $0xFFFF'FF82 = -126_{10}$ then it's a negative number, which of course is less than $268,435,448_{10}$.
        \begin{description}
            \item[Special case] If the upper four bits are initially 0, then overflow is not possible
        \end{description}
    \item For most negative numbers, build the number as a positive number and negate it,
        \textit{i.e.}, press \texttt{7}, \texttt{E}, \texttt{negate} will produce $0xFFFF'FF82 = -126_{10}$.
        This is not a special case that needs to be handled.
\end{itemize}

Locate the \function{overflow_occurred()} function.
\begin{description}
    \checkoffitem{Add code to handle the special case.}
    \checkoffitem{Add code to detect if overflow occurred for the regular cases.}
\end{description}

In \function{build_number()}:
\begin{description}
    \checkoffitem{Add code that detects if overflow occurred as a result of the new digit, and print the ``Too Big!'' message if so.}
    \checkoffitem{Add code that prevents further processing after an overflow until the ``too big'' number has been cleared.}
\end{description}


%You can determine that ``too big'' should be displayed either by predicting that overflow \textit{will} happen, or by determining that overflow \textit{has} happened.
%
%\paragraph{Detecting that Overflow Has Happened}
%
%Because multiplication is involved, detecting overflow after-the-fact isn't as simple as checking ``if the result is less than the operands\dots''
%The surest way is to use more bits.
%Using 64-bit arithmetic on an 8-bit microcontroller doesn't sound particularly appealing for the time budget and memory budget, but it is tremendously simple, and for this particular system you should have enough responsiveness and memory.\footnote{
%    In our sample solution using 64-bit arithmetic to detect that overflow has happened added about half of a kilobyte and took twice as long ($40\mu s$) as predicting overflow and using 32-bit arithmetic.
%}
%You can cast the 32-bit value to \lstinline{int64_t}, perform the arithmetic, cast it back to \lstinline{int32_t} or \lstinline{long}\footnote{
%    Remember: on an 8-bit processor, a \lstinline{long} only has 32 bits.
%} and see if the value is different than the 64-bit value.
%
%%    If the combination of \function{sprintf()} and 64-bit arithmetic introduces no noticeable lag, then this is fine.
%
%\paragraph{Predicting that Overflow Will Happen}
%
%Except for a couple of edge cases, predicting overflow isn't too challenging and will avoid all of the messy casting mentioned above.
%
%\textit{Hexadecimal}: What is in bits~31..28 (the most-significant half-byte) of the ``old value'' (before updating it with the new digit)?
%If it's anything other than \lstinline{0x0} or \lstinline{0xF}, overflow will happen.
%If it's \lstinline{0x0} then overflow won't happen.
%If it's \lstinline{0xF} then overflow might happen, depending on the value of bit~27.
%
%\textit{Decimal}: If you know what $\frac{1}{10}$ of \lstinline{INT32_MAX} and $\frac{1}{10}$ of \lstinline{INT32_MIN} (found in \lstinline{<limits.h>}) are, then you can compare the ``old value'' to those, to determine if multiplying by 10 will overflow.
%If multiplication won't overflow, then multiply the old value by 10 and compare that product with \lstinline{INT32_MAX} and \lstinline{INT32_MIN} to determine if the difference is less than the value of the new digit.

\subsubsection{Due to Negation}

Normally, negation doesn't overflow.
But what happens when you negate \lstinline{INT32_MIN}?

In \function{build_number()}:
\begin{description}
    \checkoffitem{Add code that handles the one case in which negation can produce a ``too big'' number.}
\end{description}


\subsubsection*{Test your code}
\begin{description}
    \checkoffitem{Place the \textbf{left switch} in the \textit{right position} and upload the program to your Cow~Pi.}
    \checkoffitem{Test that:}
    \begin{description}
        \checkoffitem{You can overflow a positive number.}
        \checkoffitem{You can overflow a negative number.}
        \checkoffitem{Overflow is not reported prematurely.}
        \checkoffitem{The ``clear'' button clears the ``too big'' condition.}
    \end{description}
\end{description}
