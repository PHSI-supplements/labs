In this assignment, you will write functions to use the input and output devices on \runtimeenvironment.
You will then use those functions to implement the ability to build a number, similar to entering a value on a calculator, using polling.

\begin{figure}[h]
    \centering
    \includegraphics[width=10cm]{AreWeThereYet}
    \caption{Polling. \tiny Image by 20th Century Fox Television}
\end{figure}

The instructions are written assuming you will develop the code using PlatformIO    % TODO - parameterize this?
and run the code on \runtimeenvironment.

\tableofcontents

\section*{Learning Objectives}

After successful completion of this assignment, students will be able to:
\begin{itemize}
    \item Use memory-mapped I/O to obtain inputs from peripheral devices
    \item Use memory-mapped I/O to send outputs to peripheral devices
    \item Poll an input to determine when it has taken on a value of interest
    \item Scan a matrix keypad
\end{itemize}

\section*{During Lab Time}

During your lab period, the TAs will provide a refresher on bitmasks, both to read inputs and to use the read-modify-write pattern for outputs.
They will also discuss some subtle semantics problems that students have encountered in the past.
Finally, \textbf{the TAs will guide the class through the first modifications to the starter code that you must make.}
During the remaining time, the TAs will be available to answer questions.

%During your lab period (and only during your lab period), you may discuss problem decomposition and solution design with your lab partner.
%\textit{Be sure to add your name and your partner's name to the top of your source code files.}
%To receive full credit for the work you and your partner do during lab time, you must be an active participate in the partnership.
%In accordance with the School of Computing's Academic Integrity Policy, we reserve the right to adjust your calculated grade if you merely ``tag along'' and let your partner do all the thinking.
%
%If you worked with a partner, then before leaving lab, upload \requiredfiles\ to Canvas to create a record of what you and your partner worked on during lab time.
%After you have finished the full assignment, you will upload your completed work as well.
%If you did not work with a partner, you do not need to upload your files until you have completed the assignment.
