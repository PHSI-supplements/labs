There are three parts to this assignment.

\subsection{Initial Changes to the Code}

The first part, \textbf{\textcolor{red}{which is intended to be completed during lab time under the guidance of a TA}},
is to populate the keypad's lookup table (Section~\ref{subsec:populatekeypad}),
determine the base addresses for the memory-mapped I/O registers used in this assignment (Section~\ref{subsec:baseAddresses}),
and introduce code to detect keypresses on the numeric keypad (Section~\ref{subsec:detectKeyAction}).

The remaining two parts of this assignment can be completed in either order.

\subsection{Memory-Mapped Input/Output}

The starter code contains functions that provide access to the buttons, switches, keypad, LEDs, and display module.
Initially, these functions make use of functions available in the CowPi library.
In Section~\ref{sec:MemMapIO}, you will re-implement these functions using memory-mapped I/O\@.

% TODO: parameterize this
As you complete this assignment, maintain a separation of concerns,
and place all of the code that accesses the memory-mapped I/O registers in \textit{io\_functions.c}.

\subsection{Implementing a Simple System using Polling}

In Section~\ref{sec:SimpleSystem}, you will implement a simple system that makes use of the development board's buttons, switches, keypad, LEDs, and display module.
You will implement this system by polling inputs.

% TODO: parameterize this
As you complete this assignment, maintain a separation of concerns,
and  place all of your system's logic in \textit{number\_builder.c}.


\subsection{Constraints} \label{subsec:constraints}

You may use any features that are part of the C standard if they are supported by the compiler.
You may use the constants and functions provided in the starter code
(to receive credit for the memory-mapped I/O portion of this lab, you will need to re-implement the I/O functions).

You must detect inputs using polling;
you may not use interrupts.

% TODO: parameterize these
All of your memory-mapped I/O code must go in \textit{io\_functions.c}.
All of your system's logic code must go in \textit{number\_builder.c}.

% TODO: parameterize this (when we eventually port to the Pico SDK)
\subsubsection{Constraints on the Arduino core}

You may not use any libraries, functions, macros, types, or constants from the Arduino core.\footnote{\url{https://www.arduino.cc/reference/en/}}

\ifdefstring{\processor}{ATmega328P}{
    \subsubsection{Constraints on AVR-libc}

    You may not use any AVR-specific functions, macros, types, or constants of avr-libc.\footnote{\url{https://www.nongnu.org/avr-libc/user-manual/index.html}}
}{}
\ifdefstring{\processor}{RP2040}{
    % TODO: parameterize this (when we eventually port to the bare-metal Arduino toolchain, and to the Pico SDK)
    \subsubsection{Constraints on MBED OS}

    You may not use any functions, macros, types or constants from MBED that are not part of the C standard.\footnote{\url{https://os.mbed.com/docs/mbed-os/v6.16/introduction/index.html}}
}{}

\subsubsection{Constraints on the CowPi library}

To receive credit for the memory-mapped I/O portion of this lab, all input and output must be accomplished using the memory-mapped I/O registers.
The starter code calls \function{cowpi_setup()} for you.
You also may use \href{https://cow-pi.readthedocs.io/en/latest/CowPi/inputs.html#debouncing}{\function{cowpi_debounce_byte()}}.
You may not use any other functions described in the CowPi Library\footnote{\url{https://cow-pi.readthedocs.io/en/latest/library.html}} portion of the Cow~Pi datasheet.

\subsubsection{Constraints on other libraries}

You may not use any libraries beyond those explicitly identified here.
