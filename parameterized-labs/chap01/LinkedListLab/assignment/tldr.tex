Sections~\ref{subsec:tldrBusinessRules}--\ref{subsec:tldrInsertionSort} are concise versions of Sections~\ref{subsec:BusinessRules}--\ref{subsec:TestingChallengeResponse};
if you need more-detailed instructions, see the appropriate subsection in Section~\ref{sec:challengeResponse}.
Sections~\ref{subsec:BuildingLinkedList}--\ref{subsec:MergingNodes} already offer concise instructions to implement the linked list;
we shall not unnecessarily duplicate them here.

\subsection{The Books} \label{subsec:tldrBusinessRules}

    The starter code includes six files that you can use as inputs.
    Three are pre-sorted, and three aren't.
    Two are short, only 7 words, which can be useful for debugging.
    Two are moderate-sized, 74-125 words, to give you confidence in the correctness of your solution.
    Two are large, in excess of 74,000 words, which are useful to reveal whether you have any memory leaks in your code.

    Each book file, ``\textit{file}'', has a corresponding ``\textit{file}-table.md'' that contains a Markdown-formatted table of the challenge words, the number of occurrences for each challenge word, and the corresponding response word.
    You may use these files to confirm the correctness of your solution.

%    The challenge-response app has rules to define how to find the appropriate response word.
%    The starter code already implements these rules, but that implementation will only work if your code builds an alphabetically-sorted list of words that occur in a file and the number of occurrences of each word.

    Note that alphabetization is case-insensitive for this application.


\subsection{Word Entries}

    The \lstinline{word_entry_t} type is defined in \textit{word\_entry.h}:

    \lstinputlisting[linerange=28-31, firstnumber=28]{../starter-code/word_entry.h}

    There are a handful of function prototypes in \textit{word\_entry.h} that encapsulate this datatype.

    \subsubsection{Creating and Destroying Word Entries}

        The \function{create_word_entry()} function allocates space for a \lstinline{word_entry_t}, and the \function{delete_word_entry()} function releases that memory.
        The \function{create_word_entry()} function, however, does not yet initialize the word entry.

        \begin{description}
            \checkoffitem{In \function{create_word_entry()}'s \lstinline{else} block, copy \lstinline{word} into \lstinline{word_entry}'s \lstinline{word} field.}
            \checkoffitem{In \function{create_word_entry()}'s \lstinline{else} block, set \lstinline{word_entry}'s \lstinline{occurrences} field to 0.}
        \end{description}

        You do not need to make any changes to \function{delete_word_entry()}.

    \subsubsection{Accessors and Mutators \\ \footnotesize{\textit{aka}, Getters and Setters}}

        \begin{description}
            \checkoffitem{In \function{increment_count()}, increase the word entry's number of occurrences by one.}
            \checkoffitem{In \function{get_count()}, return the number of occurrences.}
            \checkoffitem{In \function{get_word()}, return a pointer to the word entry's word. (Do \textit{not} make a copy of the word.)}
        \end{description}

    \paragraph{Testing Your Changes}

        You can build an executable that uses H.Awk's array-backed list with the command \\
        \verb+make arraylist+ \\
        or, if you want to limit each \function{malloc()} call to no more than 256 bytes, then use the command \\
        \verb+make arraylist OPTION="-DHOBBLE"+

        \begin{description}
            \checkoffitem{Build and run the executable.}
            \checkoffitem{Select task 1, ``Test word\_entry''.}
            \checkoffitem{Select function 1, ``create\_word\_entry()'', and enter a word when prompted.}
            \checkoffitem{Use fuctions 3 (``increment\_count()''), 4 (``get\_count()''), and 5 (``get\_word()'') to test your code.}
            \checkoffitem{Continue to test until you discover a bug or are satisfied that your implementations are correct.}
            \checkoffitem{When you have finished, use function 2 (``delete\_word\_entry()'') to release the word entry's memory, then select 0 to return to the main menu, and then select 0 to exit the program.}
            \begin{itemize}
                \item Note: whenever you return to the main menu, any existing list will be deleted.
                No memory references survive when moving between one main-menu-option and another.
            \end{itemize}
        \end{description}


\subsection{Alphabetical Functions}

    \subsubsection{Making a Lowercase Copy of a Word}

        \begin{description}
            \checkoffitem{Implement the \function{word_to_lowercase()} function.}
        \end{description}

    \subsubsection{Comparing Words}

        \begin{description}
            \checkoffitem{Implement the \function{words_are_equal()} function.}
            \checkoffitem{Implement the \function{word1_is_earlier_than_word2()} function.}
            \checkoffitem{Implement the \function{word1_is_later_than_word2()} function.}
        \end{description}

    \paragraph{Testing Your Changes}

        \begin{description}
            \checkoffitem{Build and run the executable.}
            \checkoffitem{Select task 3, ``Test alphabetical functions'', and enter words when prompted.}
            \checkoffitem{Continue to test until you discover a bug or are satisfied that your implementations are correct.}
            \checkoffitem{When you have finished, select 0 to exit the program.}
        \end{description}


\subsection{Preparing to Work with Lists}

    Like \lstinline{word_entry_t}, the \lstinline{list_t} and the \lstinline{iterator_t} datatypes have functions to encapsulate them.
    In the case of \lstinline{list_t} and \lstinline{iterator_t}, however, this encapsulation is essential because the code in \textit{sorted\_word\_entries.c} has access to the type declaration but not the type definition.

    \begin{description}
        \checkoffitem{Review the datatypes and functions declared in \textit{list.h}.}
    \end{description}

    A list is abstractly modeled as having a sequence of word entries and an iterator that points to the ``current'' word entry.
    The iterator can point to anywhere between the first word entry and the last word entry.

    \begin{description}
        \checkoffitem{Review the \function{test_list()} function in \textit{list-test.c} to see uses of the \lstinline{list_t} and \lstinline{iterator_t} functions.}
        \checkoffitem{Run the arraylist executable, selecting task 2 (``Test list'') to observe the behavior of the \lstinline{list_t} and \lstinline{iterator_t} functions.}
    \end{description}


\subsection{Inserting Words}

    \subsubsection{Limited Implementation}

        You will receive half of the credit for \function{insert_word()} if it works on pre-sorted books.
        If you choose to use this implementation:
        \begin{description}
            \checkoffitem{Implement \function{insert_word()} for pre-sorted books}
            \checkoffitem{Test this implementation and move on to implementing a linked list}
            \checkoffitem{Return to this sub-problem later to attempt a more-general implementation}
        \end{description}

    \subsubsection{Insertion Sort} \label{subsec:tldrInsertionSort}

        To receive full credit for \function{insert_word()}, it must work on books that are not pre-sorted.
        If you choose to use this implementation:
        \begin{description}
            \checkoffitem{Implement \function{insert_word()} such that:}
            \begin{itemize}
                \item The word is placed at the end of the list
                \item If the word is not in its proper location, then it is moved to is proper location
                \item Once in its proper location, if there is another word entry with the same word, then the two word entries are combined
            \end{itemize}
        \end{description}

    \paragraph{Testing Your Changes}

        \begin{description}
            \checkoffitem{Build and run the executable.}
            \checkoffitem{Use tasks 4--6 to test \function{insert_word()} until you discover a bug or are satisfied that your implementation is correct.}
            \checkoffitem{Use task 7 to test your code with the files of words.}
            \checkoffitem{When you have finished, select 0 to exit the program.}
        \end{description}


\subsection{Implementing a Linked List}

    Sections~\ref{subsec:BuildingLinkedList}--\ref{subsec:MergingNodes} are already fairly concise;
    we shall not duplicate them here.


