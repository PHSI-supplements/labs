\begin{description}
    \checkoffitem{\filesource\ and copy it to your account on \runtimeenvironment.}
    \checkoffitem{Once copied, extract the starter code from the archive file.}
    \checkoffitem{Place your name in the header comments of \textit{card.c}, \textit{poker.c}, and \textit{answers.txt}.} % TODO: parameterize to include partner(s) option
\end{description}

The five source code files (\textit{card.h}, \textit{card.c}, \textit{poker.h}, \textit{poker.c}, and \textit{pokerlab.c}) contain the starter code for this assignment, and the text file (\textit{answers.txt}) is where you'll provide some answers to demonstrate your ability to understand part of the starter code.
\textit{Makefile} encodes information that the \textbf{\texttt{make}} command needs to build the application.
In the \mbox{\textit{equivalent-java}} directory you will find \textit{Card.java} and \textit{Poker.java} that has Java code that is equivalent to the C code.
You do not need to use the Java files, but you may find them useful as a reference to help you understand some differences between Java and C\@.

The header file \textit{card.h} defines a ``card'' structure, specifies two functions that operate on cards, and defines some useful constants.
The source file \textbf{card.c} has the bodies for the specified functions, but some code is missing.
Finally, the source file \textit{poker.c} is supposed to generate a poker hand of five cards, print those five cards, and then print what kind of hand it is -- but much of its code is missing.
To compile the program, type:

\textbf{\texttt{make}}

When you compile the starter code, it will generate a warning:

\begin{verbatim}
card.c:63:30: warning: format string is empty [-Wformat-zero-length]
        sprintf(valueString, "", value);
                             ^~
\end{verbatim}

\begin{description}
    \checkoffitem{Before you make any other changes, you should edit \textit{card.c} so that the program compiles without generating any warnings or errors.}
    \begin{itemize}
        \item \textcolor{red}{You should get in the habit of correcting \textit{any} code that generates a warning!}
    \end{itemize}
\end{description}
If you look at the source code, you'll see a comment with instructions ``\texttt{THE SECOND ARGUMENT NEEDS THE CONVERSION SPECIFIER THAT YOU WOULD USE TO PRINT AN INTEGER}.''
The command \lstinline{sprintf()} is like \lstinline{printf()} and \lstinline{fprintf()} except that it ``prints'' to a string.
Some places you can find the conversion specifier to place in the format string:
\begin{itemize}
    \item Chapter 1 of the textbook, in the ``Printing'' section
    \item $\S7.2$ of \textit{The C Programming Language} on pages 153--155
    \item \url{https://en.cppreference.com/w/c/io/fprintf}
\end{itemize}


%% You will need to edit this file to tailor it to your  %%
%% particular environment.                               %%


\newcommand{\connectingrubricitem}{The screenshot shows that the student has connected to \runtimeenvironment\ and placed a copy of \textit{answers.txt} there.}


\subsection*{Demonstrate that you can connect to \runtimeenvironment}

\begin{description}
    \checkoffitem{By whatever means you use to place files on \runtimeenvironment, place \textit{answers.txt} on \runtimeenvironment.}
    \checkoffitem{Open a secure shell terminal and navigate to the directory in which \textit{answers.txt} is located.}
    \checkoffitem{Type these commands:
        \begin{description}
            \item \textbf{\texttt{cat /etc/hostname}} \\
            \textit{The response should be \texttt{nuros}.
            If it is any other name, then you connected to the wrong server.
            This will matter in some future lab assignments.}
            \item \textbf{\texttt{whoami}} \\
            \textit{This will print your UNL username.}
            \item \textbf{\texttt{ls answers.txt}} \\
            \textit{The response should be \texttt{answers.txt}. \\
            If it is \texttt{ls: cannot access $'$answers.txt$'$: No such file or directory} then you are not in the same directory as} answers.txt.
        \end{description}}
    \checkoffitem{Take a screenshot and save the screenshot to turn in with the rest of your work for this lab.}
\end{description}

You will receive half-credit for this part of the assignment if your screenshot shows that you connected to the wrong server.
You will receive half-credit for this part of the assignment if your screenshot shows that you connected to the correct server but doesn't show that you transferred a file to the server.
You will receive full credit for this part of the assignment only if your screenshot shows that you connected to and transferred a file to the correct server.

