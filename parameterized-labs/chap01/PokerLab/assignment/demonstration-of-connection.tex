%% You will need to edit this file to tailor it to your  %%
%% particular environment.                               %%


\newcommand{\connectingrubricitem}{The screenshot shows that the student has connected to \runtimeenvironment\ and placed a copy of \textit{answers.txt} there.}


\subsection*{Demonstrate that you can connect to \runtimeenvironment}

\begin{description}
    \checkoffitem{By whatever means you use to place files on \runtimeenvironment, place \textit{answers.txt} on \runtimeenvironment.}
    \checkoffitem{Open a secure shell terminal and navigate to the directory in which \textit{answers.txt} is located.}
    \checkoffitem{Type these commands:
        \begin{description}
            \item \textbf{\texttt{cat /etc/hostname}} \\
            \textit{The response should be \texttt{nuros}.
            If it is any other name, then you connected to the wrong server.
            This will matter in some future lab assignments.}
            \item \textbf{\texttt{whoami}} \\
            \textit{This will print your UNL username.}
            \item \textbf{\texttt{ls answers.txt}} \\
            \textit{The response should be \texttt{answers.txt}.
            If it is \texttt{ls: cannot access $'$answers.txt$'$: No such file or directory} then you are not in the same directory as} answers.txt.
        \end{description}}
    \checkoffitem{Take a screenshot and save the screenshot to submit to \filesubmission\ when you have completed the lab.}
\end{description}

You will receive half-credit for this part of the assignment if your screenshot shows that you connected to the wrong server.
You will receive half-credit for this part of the assignment if your screenshot shows that you connected to the correct server but doesn't show that you transferred a file to the server.
You will receive full credit for this part of the assignment only if your screenshot shows that you connected to and transferred a file to the correct server.
