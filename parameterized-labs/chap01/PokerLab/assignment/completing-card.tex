Look over the rest of the code in \textit{card.c} and work at understanding anything that you don't initially understand.
When you have done so:
\begin{description}
    \checkoffitem{Add the missing code to \function{create_card()} to populate a card's fields.}
    \checkoffitem{Change the first two lines of \function{card_to_string()} so that this function uses the fields from the card argument that is passed to the function.}
\end{description}

You will receive half-credit for \function{create_card()} and \function{card_to_string()} if you make a good-faith effort at implementing these functions as specified.
You will receive full credit for these functions if you implement them correctly.

\begin{description}
    \checkoffitem{Compile your code with the \textbf{\texttt{make}} command.}
    \checkoffitem{Run the program with the command \textbf{\texttt{./pokerlab}}}
    \checkoffitem{Test your changes to \textit{card.c} by selecting option 1. \\
        Catching errors now will be easier than trying to catch them after you've started the next task.}
\end{description}

Examine the remaining starter code in \textit{poker.c} to make sure you understand it.
