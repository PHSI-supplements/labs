The purpose of this assignment is to (re)familiarize you with some aspects of C that may not be intuitive to students who are new to C\@.
Even if you know C, work this assignment to re-familiarize yourself.

If you work faithfully at understanding the portions of code that you're instructed to study, and if you work faithfully at writing the code you're instructed to write, you will receive credit for this assignment.
The instructions are written assuming you will edit and run the code on \runtimeenvironment.
Except for demonstrating that you can connect to \runtimeenvironment, you may edit and run the code in a different environment if you wish;
be sure that your compiler suppresses no warnings, and that if you are using an IDE that it is configured for C and not C++.

\tableofcontents

\section*{Learning Objectives}

After successful completion of this assignment, students will be able to:
\begin{itemize}
    \item Connect to \runtimeenvironment.
    \item Edit and compile a C program.
    \item Understand the similarities between Java and C code.
    \item Adapt to some differences between Java and C code, specifically those
        associated with arrays, strings, and boolean values.
\end{itemize}

\section*{During Lab Time}

During your lab period, the TAs will help you connect to \runtimeenvironment, and they will show you how to edit and compile a C program.
During the remaining time, the TAs will be available to answer questions.

Before leaving lab, \textit{at a minimum} complete Sections~\ref{sec:connecting}--\ref{sec:gettingstarted}.
