Now that you've warmed up to bitwise operations and implementing operations without using C's built-in operations, let us turn your attention to arithmetic.
Before you can add two $n$-bit values, you must be able to add two 1-bit values.

\subsection{One Bit Full Adder} \label{subsec:one-bit-full-adder}

In the \function{one_bit_full_addition()} function, you will implement a 1-bit full adder;
that is, an adder that takes two operand bits and a carry-in bit, and it produces a sum bit and a carry-out bit.

The function takes one argument, a structure containing five fields.
As described in Section~\ref{subsubsec:alu.h}, these five fields are the operand bits \lstinline{a} and \lstinline{b}, the carry-in bit \lstinline{c_in}, the sum bit \lstinline{sum}, and the carry-out bit \lstinline{c_out}.
When the structure is passed in to the function, only \lstinline{a}, \lstinline{b}, and \lstinline{c_in} are populated.
Your task is to populate the \lstinline{sum} and \lstinline{c_out} fields, and return the structure.

\begin{description}
    \checkoffitem{Implement a 1-bit full adder using bitwise operations.}
\end{description}
Because the fields are guaranteed to be strictly 1 or 0, you do not need to apply any of the test functions to reduce them to 1 or 0.

\subsubsection*{Check your work}

\begin{description}
    \checkoffitem{Compile and run \texttt{\textbf{\textit{./integerlab}}}, trying all possible values.}
    \begin{itemize}
        \item Note that you will receive a warning for an unused variable in \function{ripple_carry_addition()};
            this is okay for now
    \end{itemize}
\end{description}
When you enter the inputs for your 1-bit full adder, only the least significant bit of each operand will be used.
For example:
\begin{verbatim}
    Enter ... "add1 <binary_value1> <binary_value2> <carry_in>" ...: add1 0 0 0
    expected: 0 + 0 + 0 = 0 carry 0
    actual:   0 + 0 + 0 = 0 carry 0

    Enter ... "add1 <binary_value1> <binary_value2> <carry_in>" ...: add1 0 0 1
    expected: 0 + 0 + 1 = 1 carry 0
    actual:   0 + 0 + 1 = 1 carry 0
\end{verbatim}

Check your code with all eight possible inputs, comparing your actual results with the expected results.


\subsection{Ripple-Carry Adder} \label{subsec:ripple-carry-adder}

\begin{description}
    \checkoffitem{Use your 1-bit full adder to implement a 32-bit ripple-carry adder.}
\end{description}
As a reminder, in a ripple-carry adder, the carry-out bit from bit position $n$ becomes the carry-in bit for bit position $n+1$.

Use whatever code that you need, that does not violate any of this assignment's constraints, to populate the input fields of a \lstinline{one_bit_adder_t} variable and pass that variable to \function{one_bit_full_addition()}.
Use the \lstinline{sum} bit to contribute to the 32-bit sum and the \lstinline{c_out} bit as the \lstinline{c_in} bit of the next bit position.
Repeatedly do this until you have added all 32-bit positions, resulting in the 32-bit sum.
(Discard the final carry-out.)

\subsubsection*{Check your work}

\begin{description}
    \checkoffitem{Compile and run \texttt{\textbf{\textit{./integerlab}}}, trying a few values.}
\end{description}
When you enter the inputs for your 32-bit adder, the operands will be interpreted as hexadecimal values even if you omit the leading ``0x'', and only the least-significant bit of the carry-in will be used.
For example:
\begin{verbatim}
    Enter ... "add32 <hex_value1> <hex_value2> <carry_in>" ...: add32 0x1a 0x22 0
    expected: 0x0000001A + 0x00000022 + 0 = 0x0000003C
    actual:   0x0000001A + 0x00000022 + 0 = 0x0000003C

    Enter ... "add32 <hex_value1> <hex_value2> <carry_in>" ...: add32 1a 22 1
    expected: 0x0000001A + 0x00000022 + 1 = 0x0000003D
    actual:   0x0000001A + 0x00000022 + 1 = 0x0000003D
\end{verbatim}

\begin{description}
    \checkoffitem{Check your code with other values, comparing your actual results with the expected results.}
    \checkoffitem{Run the constraint checker: \texttt{python constraint-check.py integerlab.json}}
\end{description}

\textcolor{red}{When you test your 32-bit adder, don't forget to test larger values, too}, such as:
\begin{verbatim}
    Enter ... "add32 <hex_value1> <hex_value2> <carry_in>" ...:
                                                add32 0x76543210 0x89ABCDEF 0
    expected: 0x76543210 + 0x89ABCDEF + 0 = 0xFFFFFFFF
    actual:   0x76543210 + 0x89ABCDEF + 0 = 0xFFFFFFFF

    Enter ... "add32 <hex_value1> <hex_value2> <carry_in>" ...:
                                                add32 0x76543210 0x89ABCDEF 1
    expected: 0x76543210 + 0x89ABCDEF + 1 = 0x00000000
    actual:   0x76543210 + 0x89ABCDEF + 1 = 0x00000000
\end{verbatim}


\subsection{16-Bit Addition}

The \function{add()} function, along with the other arithmetic functions, returns an \lstinline{alu_result_t} structure.

Having implemented a 32-bit adder, you can use it for your 16-bit addition function.
Following the convention that the most significant bit is $bit_{31}$, and the least significant bit is $bit_0$ then:
\begin{itemize}
    \item The 16-bit sum will be in $bits_{15..0}$, the lower 16 bits of the 32-bit adder's sum.
    \item Viewed from the perspective of 16-bit addition:
        \begin{itemize}
            \item The 32-bit sum's $bit_{15}$ is the 16-bit value's most significant bit, and $bit_0$ is the 16-bit value's least significant bit.
            \item The 32-bit sum's $bit_{16}$ is the final carry-out of 16-bit addition.
        \end{itemize}
\end{itemize}

\begin{description}
    \checkoffitem{Use the 32-bit adder to add \lstinline{addend} to \lstinline{augend} (\textit{i.e.}, calculate $augend + addend$).}
    \checkoffitem{Place the 16-bit sum in the \lstinline{alu_result_t} variable's \lstinline{result} field.}
    \checkoffitem{Assume that the operands are unsigned 16-bit integers and determine whether overflow occurred;
        set the \lstinline{alu_result_t} variable's \lstinline{unsigned_overflow} flag accordingly.}
    \checkoffitem{Assume that the operands are signed 16-bit integers and determine whether overflow occurred;
        set the \lstinline{alu_result_t} variable's \lstinline{signed_overflow} flag accordingly.}
\end{description}


\subsubsection*{Check your work}

\begin{description}
    \checkoffitem{Compile and run \texttt{\textbf{\textit{./integerlab}}}, trying a few values.}
\end{description}
For example:
\begin{verbatim}
    Enter ... a two-operand arithmetic expression... or "quit": 3 + 15
    UNSIGNED ADDITION
        expected result (hexadecimal): 0x0003 + 0x000F = 0x0012
        expected result (unsigned):    3 + 15 = 18	overflow: false
        actual result (hexadecimal):   0x0003 + 0x000F = 0x0012
        actual result (unsigned):      3 + 15 = 18	overflow: false
    SIGNED ADDITION
        expected result (hexadecimal): 0x0003 + 0x000F = 0x0012
        expected result (signed):      3 + 15 = 18	overflow: false
        actual result (hexadecimal):   0x0003 + 0x000F = 0x0012
        actual result (signed):        3 + 15 = 18	overflow: false

    Enter ... a two-operand arithmetic expression... or "quit": 0x6000 + 0x3000
    UNSIGNED ADDITION
        expected result (hexadecimal): 0x6000 + 0x3000 = 0x9000
        expected result (unsigned):    24576 + 12288 = 36864	overflow: false
        actual result (hexadecimal):   0x6000 + 0x3000 = 0x9000
        actual result (unsigned):      24576 + 12288 = 36864	overflow: false
    SIGNED ADDITION
        expected result (hexadecimal): 0x6000 + 0x3000 = 0x9000
        expected result (signed):      24576 + 12288 = -28672	overflow: true
        actual result (hexadecimal):   0x6000 + 0x3000 = 0x9000
        actual result (signed):        24576 + 12288 = -28672	overflow: true
\end{verbatim}

If you are performing this lab on \runtimeenvironment, then the expected overflow flags are obtained directly from flags set in the processor's ALU and are authoritative.\footnote{
    If you are not performing this lab on \runtimeenvironment\ and receive the compile-time warning ``Some of the code to determine the *expected* supplemental\_result and *expected* flags is not yet defined'' then the expected overflow flags should not be trusted.
}

\begin{description}
    \checkoffitem{Check your code with other values, comparing your actual results with the expected results.}
    \begin{itemize}
        \item Use positive and negative operands.
        \item Generate sums that produce signed overflow, sums that produce unsigned overflow, and sums that produce neither.
    \end{itemize}
    \checkoffitem{Run the constraint checker: \texttt{python constraint-check.py integerlab.json}}
\end{description}


\subsection{16-Bit Subtraction}

Having implemented a 32-bit adder, you can use it for your 16-bit subtraction function.

\begin{description}
    \checkoffitem{Use the 32-bit adder to subtract \lstinline{subtrahend} from \lstinline{menuend} (\textit{i.e.}, calculate $menuend - subtrahend$).}
        \begin{itemize}
            \item Use the adder using the technique discussed in Chapter~3 and in lecture
            \item \textcolor{red}{Apply a \texttt{0xFFFF} bitmask to your arguments when you call \function{ripple_carry_addition()} to make sure that only the 16-bit values are passed to \function{ripple_carry_addition()}!}\footnote{
                A subtle, normally-desirable, rule in the bitwise operator's semantics will cause 1s to be placed in $bits_{31..16}$.
                For our specific use, this is undesirable and so you need to force $bits_{31..16}$ to have 0s.
            }
        \end{itemize}
    \checkoffitem{Place the 16-bit difference in the \lstinline{alu_result_t} variable's \lstinline{result} field.}
    \checkoffitem{Assume that the operands are unsigned 16-bit integers and determine whether overflow occurred;
        set the \lstinline{alu_result_t} variable's \lstinline{unsigned_overflow} flag accordingly.}
    \checkoffitem{Assume that the operands are signed 16-bit integers and determine whether overflow occurred;
        set the \lstinline{alu_result_t} variable's \lstinline{signed_overflow} flag accordingly.}
\end{description}


\subsubsection*{Check your work}

\begin{description}
    \checkoffitem{Compile and run \texttt{\textbf{\textit{./integerlab}}}, trying a few values.}
\end{description}
For example:
\begin{verbatim}
    Enter ... a two-operand arithmetic expression... or "quit": 15 - 25
    UNSIGNED SUBTRACTION
        expected result (hexadecimal): 0x000F - 0x0019 = 0xFFF6
        expected result (unsigned):    15 - 25 = 65526	overflow: true
        actual result (hexadecimal):   0x000F - 0x0019 = 0xFFF6
        actual result (unsigned):      15 - 25 = 65526	overflow: true
    SIGNED SUBTRACTION
        expected result (hexadecimal): 0x000F - 0x0019 = 0xFFF6
        expected result (signed):      15 - 25 = -10	overflow: false
        actual result (hexadecimal):   0x000F - 0x0019 = 0xFFF6
        actual result (signed):        15 - 25 = -10	overflow: false

    Enter ... a two-operand arithmetic expression... or "quit": 0x100 - 0x7F
    UNSIGNED SUBTRACTION
        expected result (hexadecimal): 0x0100 - 0x007F = 0x0081
        expected result (unsigned):    256 - 127 = 129	overflow: false
        actual result (hexadecimal):   0x0100 - 0x007F = 0x0081
        actual result (unsigned):      256 - 127 = 129	overflow: false
    SIGNED SUBTRACTION
        expected result (hexadecimal): 0x0100 - 0x007F = 0x0081
        expected result (signed):      256 - 127 = 129	overflow: false
        actual result (hexadecimal):   0x0100 - 0x007F = 0x0081
        actual result (signed):        256 - 127 = 129	overflow: false
\end{verbatim}

As with addition, if you are performing this lab on \runtimeenvironment, then the expected overflow flags are obtained directly from flags set in the processor's ALU\@.

\begin{description}
    \checkoffitem{Check your code with other values, comparing your actual results with the expected results.}
    \begin{itemize}
        \item Use positive and negative operands.
        \item Generate differences that produce signed overflow, differences that produce unsigned overflow, and differences that produce neither.
    \end{itemize}
    \checkoffitem{Run the constraint checker: \texttt{python constraint-check.py integerlab.json}}
\end{description}
