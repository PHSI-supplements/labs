%%
%% PokerLab (c) 2018-21 Christopher A. Bohn
%%

%%
%% labs/common/assignment.tex
%% (c) 2021-22 Christopher A. Bohn
%%
%% Licensed under the Apache License, Version 2.0 (the "License");
%% you may not use this file except in compliance with the License.
%% You may obtain a copy of the License at
%%     http://www.apache.org/licenses/LICENSE-2.0
%% Unless required by applicable law or agreed to in writing, software
%% distributed under the License is distributed on an "AS IS" BASIS,
%% WITHOUT WARRANTIES OR CONDITIONS OF ANY KIND, either express or implied.
%% See the License for the specific language governing permissions and
%% limitations under the License.
%%

\documentclass[12pt]{article}

\usepackage{fullpage}
\usepackage{fancyhdr}
\usepackage[procnames]{listings}
\usepackage{hyperref}
\usepackage{textcomp}
\usepackage{bold-extra}
\usepackage[dvipsnames]{xcolor}
\usepackage{etoolbox}

% These are placeholder commands and will be renewed in each lab

\newcommand{\labnumber}{}
\newcommand{\labname}{Lab \labnumber\ Assignment}
\newcommand{\shortlabname}{}
\newcommand{\duedate}{}

% Individual or team effort

\newcommand{\individualeffort}{This is an individual-effort project. You may
    discuss concepts and syntax with other students, but you may discuss
    solutions only with the professor and the TAs. Sharing code with or copying
    code from another student or the internet is prohibited.}
\newcommand{\teameffort}{This is a team-effort project. You may discuss concepts
    and syntax with other students, but you may discuss solutions only with your
    assigned partner(s), the professor, and the TAs. Sharing code with or
    copying code from a student who is not on your team, or from the internet,
    is prohibited.}
\newcommand{\freecollaboration}{In addition to the professor and the TAs, you
    may freely seek help on this assignment from other students.}
\newcommand{\collaborationrules}{}

% Software engineering (if you care about that)

\providebool{allowspaghetticode}

\newcommand{\softwareengineeringfrontmatter}{
    \ifboolexpe{not bool{allowspaghetticode}}{
        \section*{No Spaghetti Code Allowed}
        In the interest of keeping your code readable, you may \textit{not} use
        any \lstinline{goto} statements, nor may you use any
        \lstinline{continue} statements, nor may you use any \lstinline{break}
        statements to exit from a loop, nor may you have any functions
        \lstinline{return} from within a loop.
    }{}
}

\newcommand{\spaghetticodepenalties}[1]{
    \ifboolexpe{not bool{allowspaghetticode}}{
        \penaltyitem{1}{for each \lstinline{goto} statement,
            \lstinline{continue} statement, \lstinline{break} statement used to
            exit from a loop, or \lstinline{return} statement that occurs within
            a loop.}
    }{}
}

% You shouldn't need to customize these,
% but you can if you like

\lstset{language=C, tabsize=4, upquote=true, basicstyle=\ttfamily}
\newcommand{\function}[1]{\textbf{\lstinline{#1}}}
\setlength{\headsep}{0.7cm}
\hypersetup{colorlinks=true}

\newcommand{\pagelayout}{
    \pagestyle{fancy}
    \fancyhf{}
    \lhead{\coursenumber}
    \chead{\ Lab \labnumber: \labname}
    \rhead{\courseterm}
    \cfoot{\shortlabname-\thepage}
}

\newcommand{\labidentifier}{
    \title{\ Lab \labnumber}
    \author{\labname}
    \date{Due: \duedate}
    \maketitle

    \textit{\collaborationrules}
}

% deprecated
\newcommand{\startdocument}{
    \pagelayout
	\begin{document}
	\labidentifier
}

\newcommand{\rubricitem}[2]{\item[\underline{\hspace{1cm}} +#1] #2}
\newcommand{\bonusitem}[2]{\item[\underline{\hspace{1cm}} Bonus +#1] #2}
\newcommand{\penaltyitem}[2]{\item[\underline{\hspace{1cm}} -#1] #2}
\newcommand{\checkoffitem}[1]{\item (\phantom{xxx}) #1}
\newcommand{\precheckoffitem}[1]{\item [] (\phantom{xxx}) #1}


\renewcommand{\labnumber}{1}
\renewcommand{\labname}{C Programming Familiarization Lab}
\renewcommand{\shortlabname}{pokerlab}
\renewcommand{\collaborationrules}{Except as noted in Section~\ref{StudyTheCode}, \individualeffort}
%\renewcommand{\duedate}{See \filesubmission\ for the due date}
\renewcommand{\duedate}{Week of January 24, before the start of your lab section}
\startdocument
% \begin{document}

The purpose of this assignment is to (re)familiarize you with some aspects of C
that may not be intuitive to students who are new to C. Even if you know C,
work this assignment to re-familiarize yourself.

If you work faithfully at understanding the portions of code that you're
instructed to study, and if you work faithfully at writing the code you're
instructed to write, you will receive credit for this assignment. The
instructions are written assuming you will edit and run the code on
\runtimeenvironment. Except for demonstrating that you can connect to
\runtimeenvironment, you may edit and run the code in a different environment
if you wish; be sure that your compiler suppresses no warnings, and that if you
are using an IDE that it is configured for C and not C++.

\section*{Learning Objectives}

After successful completion of this assignment, students will be able to:
\begin{itemize}
\item Connect to \runtimeenvironment.
\item Edit and compile a C program.
\item Understand the similarities between Java and C code.
\item Adapt to some differences between Java and C code, specifically those
    associated with arrays, strings, and boolean values.
\end{itemize}

\subsection*{Continuing Forward}

Even though the C code in this assignment can be edited, compiled, and run
anywhere, there will be future lab assignments that must be completed on
\runtimeenvironment. Having learned how to connect to \runtimeenvironment, you
will be able to complete future labs.

This lab is a first step in understanding the C language. Upcoming labs will
build upon this understanding, both to improve your familiarity with the C
language and also to apply that understanding to learning \coursenumber's
concepts.

\section*{During Lab Time}

During your lab period, the TAs will help you connect to \runtimeenvironment,
and they will show you how to edit and compile a C program. During the remaining
time, the TAs will be available to answer questions.

\section{Connecting to \runtimeenvironment}

%% You will need to edit this section to tailor it to your  %%
%% particular environment.                                  %%

You will need to be able to set up a secure shell terminal to
\runtimeenvironment. You will also need to be able to edit files either
directly on \runtimeenvironment, or on your personal computer (or a lab
computer) and to transfer files to and from \runtimeenvironment.

\subsection{Secure Shell Terminal}

You will need to run commands on \runtimeenvironment.

If your personal computer (or the lab computer you're using) is a Windows
machine, the most popular option is PuTTY. See
\url{https://computing.unl.edu/faq-section/working-remotely#node-29471} for
instructions. The principal difference is that instead of using
\textit{cse.unl.edu} has the Host Name, use \textit{csce.unl.edu}.

If your personal computer (or the lab computer you're using) is a Mac or a
Linux box, the simplest option is to open a terminal window on your computer
and type \texttt{ssh \textit{username}@csce.unl.edu}, where \textit{username}
is your School of Computing login ID. See
\url{https://computing.unl.edu/faq-section/working-remotely#node-300} for Mac,
or \url{https://computing.unl.edu/faq-section/working-remotely#node-30086} for
Linux.

For a broad variety of platforms, you can use NoMachine (see
\url{https://computing.unl.edu/faq-section/working-remotely#node-30855}). Note
that NoMachine will only connect to \textit{cse.unl.edu}. After you have
connected to \textit{cse.unl.edu} through NoMachine, you can open a terminal
window and \texttt{ssh} to \textit{csce.unl.edu} just as you would from any
other Linux system.

If you already have an IDE on your personal computer, that IDE may provide the
option of opening a secure shell terminal on a remote system. I do not
guarantee that the TAs or the School of Computing tech support team can help
you with connecting your IDE to \runtimeenvironment.

\subsection{Editing Files}

You might choose to edit files directly on \runtimeenvironment. If you do so,
your options are \texttt{vim}, an enhanced version of the classic \texttt{vi}
editor, GNU Emacs, or Pico (or its clone, GNU nano). Be aware that \texttt{vim}
and Emacs have non-trivial learning curves: knowing how to use them will pay
dividends in your future careers, but you may be frustrated if you're using to
using more conventional editors. Pico, an editor derived from the classic
\texttt{pine} email client, has a helpful list of available commands always
shown at the bottom of the terminal.

If you're using NoMachine, you can use Atom, a general-purpose text editor that
will have a more-familiar style user interface. Because \textit{cse.unl.edu}
and \textit{csce.unl.edu} share a file server, you can edit files on
\textit{cse.unl.edu} and use them on \textit{csce.unl.edu} without having to
take any action to transfer the files.

Many students choose to edit files on their personal computer. If you do so,
and if your personal computer is a Windows machine, try to configure your
editor to use ``Unix-style'' end-of-line characters. This won't matter for most
labs, but a couple of upcoming labs will need ``Unix-style'' line separators.
If you cannot change this setting, then get in the habit of running
\texttt{dos2unix \textit{filename}} on your files after transferring them to
\runtimeenvironment\ to convert ``DOS-style'' line separators to ``Unix-style''
line separators (this utility makes a few other changes, too, that have no
bearing on \coursenumber assignments).

Some IDEs allow you to edit files on a remote system. Bear in mind that these
IDEs may add metafiles to the remote system in sufficient quantity to exceed
your disk quota on the School of Computing's file server (VS Code is
particularly notorious for this). I do not guarantee that the TAs or the School
of Computing tech support team can help you with connecting your IDE to
\runtimeenvironment.

If you edit files on your personal computer but store files on your personal
computer (that is, you aren't editing remote files), then you will need to
transfer files between your personal computer and \runtimeenvironment.

\subsection{Transferring Files}

If you are editing your files on the School of Computing's file server, whether
from within a secure shell terminal, from within NoMachine, or by configuring
an IDE on your personal computer to do so, then you do not need to transfer
files between the file server and your local computer -- but you may wish to
set up the ability to do so anyway.

Similarly, if you are editing files on a lab computer, you do not need to
transfer files because the ``Z drive'' shares a file server with
\textit{cse.unl.edu} and \textit{csce.unl.edu}.

Otherwise, if you are editing local files on your personal computer, then you
will need to be able to transfer files. If you are using Windows, then the most
popular option is FileZilla -- see
\url{https://computing.unl.edu/faq-section/working-remotely#node-291}. It does
not matter whether you specify \textit{cse.unl.edu} or \textit{csce.unl.edu} as
the host because they share the same file server.

If you are using Mac or Linux, you have options. The School of Computing's FAQ
suggests Cyberduck for Mac -- see
\url{https://computing.unl.edu/faq-section/working-remotely#node-3459}, but
FileZilla works fine on Mac and Linux. Another option, since you're already
using terminal windows to \texttt{ssh} into \textit{csce.unl.edu} is to use
the \texttt{scp} command. Basic use of the \texttt{scp} command is very much
the same as basic use of the \texttt{cp} command, except that you specify the
remote host. Copying files from your computer to the server: \\
\texttt{scp \textit{file1} \textit{file2} \dots
username@csce.unl.edu:\textit{filepath}} copies the files from your local
computer to the \textit{filepath} on \runtimeenvironment, where
\textit{filepath} is relative to your home directory. For example, \\
\texttt{scp answers.txt username@csce.unl.edu:.} copies \textit{answers.txt}
to the top-level of your home directory. Or, working the other direction: \\
\texttt{scp username@csce.unl.edu:\textit{file} \textit{filepath}} copies
\textit{file} from the remote server to \textit{filepath} on your local
computer. Just as with \texttt{cp}, you can use the \texttt{-r} argument to
copy directories: \\
\texttt{scp -r pokerlab username@csce.unl.edu:.} \\
\texttt{scp -r username@csce.unl.edu:pokerlab .}

\softwareengineeringfrontmatter

\section*{Scenario}

You're relaxing at your favorite hangout when another customer catches your
attention. He's rather large (dare I say, \textit{mammoth}), a bit hairy, and
looking frustrated in front of his laptop. ``I'm Archie,'' he says, ``and I'm
trying to teach myself this card game called \textit{Poker}. I found this
source code that I thought I could use to understand Poker better, but the code
is incomplete, and I don't entirely understand what's there. Could you explain
the code to me, please?''

\section{Terminology}

The standard 52-card deck of ``French'' playing
cards\footnote{\url{https://en.wikipedia.org/wiki/Standard_52-card_deck}}
consists of 52 cards. The cards are divided into 4 ``suits,'' clubs
($\clubsuit$), diamonds ($\diamondsuit$), hearts ($\heartsuit$), and spades
($\spadesuit$). Each suit consists of 13 cards: the number cards 2-10, the
``face cards'' (Jack, Queen, King), and the Ace. In most card games (including
Poker), the Jack is greater in value than the 10, the Queen is greater in value
than the Jack, and the King is greater in value than the Queen. In some games,
the Ace is lesser in value than the 2, in other games it is greater in value
than the King, and in some games, it can be either.

Poker\footnote{\url{https://en.wikipedia.org/wiki/Poker}} is a game of chance
and skill played with a standard deck of 52 playing cards, in which players
attempt to construct the best ``hand'' they can. While there are many
variations of the game, they all have this in common. A hand is a set of five
cards, and it can be categorized into types of hands (described in
Section~\ref{TypesOfPokerHands}), which are ranked according to the statistical
likelihood of being able to construct such a hand. When completed, the code in
this assignment will generate a random hand and evaluate what type of hand it
is.

\section{Getting Started}

Download \textit{\shortlabname.zip} or \textit{\shortlabname.tar} from
\filesource\ and copy it to \runtimeenvironment. Once copied, unzip the file.
The three source code files (\textit{card.h}, \textit{card.c},
\textit{poker.c}) contain the starter code for this assignment, and the text
file (\textit{answers.txt}) is where you'll provide some answers to demonstrate
your ability to understand part of the starter code. In the
\mbox{\textit{equivalent-java}} directory you will find \textit{Card.java} and
\textit{Poker.java} that has Java code that is equivalent to the C code. You do
not need to use the Java files, but you may find them useful as a reference to
help you understand some of the differences between Java and C.

The header file \textit{card.h} defines a ``card'' structure and specifies two
functions that operate on cards. The source file \textbf{card.c} has the bodies
for the specified functions, but some code is missing. Finally, the source file
\textit{poker.c} is supposed to generate a poker hand of five cards, print
those five cards, and then print what kind of hand it is -- but much of its
code is missing. To compile the program, type:

\texttt{gcc -std=c99 -Wall -o poker poker.c card.c}

If you compile the starter code, it may generate a warning:

\begin{verbatim}
card.c:59:30: warning: format string is empty [-Wformat-zero-length]
        sprintf(valueString, "", value);
                             ^~
\end{verbatim}

Before you make any other changes, you should edit \textit{card.c} so that the
program compiles without generating any warnings or errors. If you look at the
source code, you'll see a comment with instructions ``\texttt{PLACE THE CONTROL
STRING IN THE SECOND ARGUMENT THAT YOU WOULD USE TO PRINT AN INTEGER}.'' The
command \lstinline{sprintf()} is like \lstinline{printf()} and
\lstinline{fprintf()} except that it ``prints'' to a string. See $\S7.2$ of
\textit{The C Programming Language} on pages 153-155 for a description of
\lstinline{printf()} and \lstinline{sprintf()}, including some of the format
specifiers you can put in the format string.

If you also get a warning for an unused variable
\begin{verbatim}
poker.c:154:9: warning: unused variable ‘size_of_hand’ [-Wunused-variable]
     int size_of_hand = 5;
         ^
\end{verbatim}
then you can temporarily fix this warning by commenting-out the line
\lstinline{int size_of_hand = 5} in \textit{poker.c}'s \function{main()}
function.

(Note that in future labs, we will provide a \textit{Makefile} that can be used
to build applications.)

\subsection*{Demonstrate that you can connect to \runtimeenvironment}

By whatever means you use to place files on \runtimeenvironment, place
\textit{answers.txt} on \runtimeenvironment. Open a secure shell terminal and
navigate to the directory in which \textit{answers.txt} is located. Type these
commands:
\begin{itemize}
\item[]\texttt{cat /etc/hostname} \\
\textit{The response should be \texttt{csce.cs.unl.edu}. If it is
\texttt{cse.unl.edu} then you connected to the wrong server. This will matter
in two future lab assignments.}
\item[]\texttt{whoami} \\
\textit{This will print your School of Computing login ID.}
\item[]\texttt{ls answers.txt} \\
\textit{The response should be \texttt{answers.txt}. If it is \texttt{ls:
cannot access $'$answers.txt$'$: No such file or directory} then you are not in
the same directory as} answers.txt.
\end{itemize}

Take a screenshot and save the screenshot to submit to \filesubmission\ when you
have completed the lab.

\section{Completing \textit{card.c}}

Look over the rest of the code in \textit{card.c} and work at understanding
anything that you don't initially understand. When you have done so, add the
missing code to \function{create_card()} to populate a card's fields.
Finally, change the first two lines of \function{display_card()} so
that this function uses the fields from the card argument that is passed to the
function.

You may want to add a \function{main()} function to \textit{card.c} and compile
only \textit{card.c} to check that you made the correct changes. Catching
errors now will be easier than trying to catch them after you've started the
next task.

Examine the remaining starter code in \textit{poker.c} to make sure you
understand it.

\section{Completing \textit{poker.c}}

If you added a \function{main()} function to \textit{card.c}, remove it so that
there is only one \function{main()} function when you compile the full program.

In \textit{poker.c}, the first thing you'll want to do is write the code for
\function{populate_deck()}. Using \function{create_card()} from \textit{card.c},
create cards corresponding to the 52 standard playing cards and add them to the
\lstinline{deck[]} array. You might put code in \function{main()} to print out
all 52 cards in \lstinline{deck[]} using \function{display_card()}, to confirm
that you wrote \function{populate_deck()} correctly.

\subsection{Types of Poker Hands}
\label{TypesOfPokerHands}

In the game of poker, hands are characterized by the similarities of the cards
within. Traditionally, you characterize the hand by the ``best''
characterization (that is, the one that is least likely to occur); for example,
a hand that is a three of a kind also contains a pair, but you would only
characterize the hand as a three of a kind. The \function{is...()} functions in
\textit{poker.c} are intentionally simple; they do not (and should not) check
whether there is a better way to characterize the hand. The types of hands
(from most desirable to least desirable) are:

\begin{description}
\item[Royal Flush] This is an Ace, a King, a Queen, a Jack, and a 10, all of
    the same suit. There is no function in the starter code for a royal flush,
    nor do you need to write one, since a royal flush is essentially the
    best- possible straight flush. (Note also that a Royal Flush is not possible
    for this lab, based on our re-definition of a Straight, below.)
\item[Straight Flush] This is five cards in a sequence, all of the same suit;
    that is, five cards that are both a straight and a flush. This
    characterization is checked by the function \function{is_straight_flush()}.
\item[Four of a Kind] Four cards all have the same value. This
    characterization is checked by the function \function{is_four_of_kind()}.
\item[Full House] The hand contains a three of a kind and also contains a pair
    with a different value than that of the first three cards. This
    characterization is checked by the function \function{is_full_house()}.
\item[Flush] Five cards all of the same suit. This characterization is checked
    by the function \function{is_flush()}. In the interest of simplicity, for
    this assignment we changed the definition of a flush to ``all cards are of
    the same suit'' (this distinction only matters if the number of cards in
    the hand is not five).
\item[Straight] Five cards in a sequence. This characterization is checked by
    the function \function{is_straight()}. In the interest of simplicity, for
    this assignment we changed the definition of a straight to ``all cards are
    in a sequence'' (this distinction only matters if the number of cards in
    the hand is not five). We further re-defined an Ace to be adjacent only to
    2 (in traditional poker, an Ace can be adjacent to 2 or to King but not
    both at the same time).
\item[Three of a Kind] Three cards all have the same value. This
    characterization is checked by the function \function{is_three_of_kind()}.
\item[Two Pair] The hand holds two different pairs. This characterization is
    checked by the function \function{is_two_pair()}.
\item[Pair] Two cards with the same value. This characterization is checked by
    the function \function{is_pair()}.
\item[High Card] If the hand cannot be better characterized, it is
    characterized by the greatest-value card in the hand. The starter code
    does not have a function to check for this since this is the
    characterization if all of the other functions return a \textbf{0}.
\end{description}

\subsection{Study the Code}
\label{StudyTheCode}

Look at the code for \function{is_pair()}. Notice that the parameter
\lstinline{hand}'s type is \lstinline{card*}; that is, \lstinline{}{hand} is a
pointer to a \lstinline{}{card}. In the code, though, we treat
\lstinline{}{hand} as though it were an array. This is because in C, arrays are
pointers and we can treat pointers as arrays. Now look at the rest of the code
in \function{is_pair()}. Why does this return a \textbf{1} when the hand
contains at least one pair?  Why does it return a \textbf{0} when the hand
contains no pairs?  If you can't determine this on your own, you may talk it
over with other students or the TA. Type your answer in \textit{answers.txt}.

Look at the code for \function{is_flush()}. Why does this return a \textbf{1}
when all cards in the hand have the same suit?  Why does it return a \textbf{0}
when at least two cards have different suits than each other?  If you can't
determine this on your own, you may talk it over with other students or the TA.
Type your answer in \textit{answers.txt}.

Look at the code for \function{is_straight()}. This is a little more challenging
to understand than \function{is_pair()} and \function{is_flush()}. Why does it
return a \textbf{1} when all cards in the hand are in sequence?  Why does it
return a \textbf{0} when they are not in sequence? If you can't determine this
on your own, you may talk it over with other students or the TA. Type your
answer in \textit{answers.txt}.

Look at the code for \function{is_two_pair()}. Recall that in C, arrays are
pointers. The assignment \lstinline{partial_hand = hand + i} makes use of
\textit{pointer arithmetic}. If the assignment were
\lstinline{partial_hand = hand} then it would assign \lstinline{hand}'s base
address to \lstinline{partial_hand}, and so \lstinline{partial_hand} would point
to the $0^{th}$ element of \lstinline{hand}. The expression $hand+i$ generates
the address for the $i^{th}$ element of \lstinline{}{hand}, and so
\lstinline{partial_hand = hand + i} assigns to \lstinline{partial_hand} the
address of the $i^{th}$ element of \lstinline{hand}. This effectively makes
\lstinline{partial_hand} an array such that
$\forall j : \mathtt{partial\_hand[j] = hand[i+j]}$.

Examine the remaining starter code in \textit{poker.c} to make sure you
understand it.

\subsection{Complete the code}

Write the code in \textit{poker.c}'s \function{main()} function to generate a
hand of five cards by calling \function{get_hand()}.\footnote{When you test the
other functions you need to write, you might want to temporarily bypass
\function{get_hand()} and explicitly assign specific cards to an array of five
\lstinline{card}s.}  (Uncomment the \lstinline{int size_of_hand = 5} line if you
previously commented it.)  Then have the program print out the five cards in
the hand. Finally, by calling the \function{is...()} functions, determine the
best-possible characterization of the hand and print out that information.

Now write the code for \function{is_three_of_kind()}, \function{is_full_house()}, and
\function{is_four_of_kind()}.

\section{Turn-in and Grading}

When you have completed this assignment, upload your screenshot,
\textit{card.c}, \textit{poker.c}, and \textit{answers.txt} to \filesubmission.

This assignment is worth 10 points.
\begin{description}
\rubricitem{1}{The screenshot shows that the student has connected to \runtimeenvironment\ and placed a copy of \textit{answers.txt} there.}
\rubricitem{1}{The student's answers in \textit{answers.txt} demonstrate an understanding of C's logical boolean operations.}
\rubricitem{1}{\function{create_card} populates a card's fields}
\rubricitem{1}{\function{display_card} generates the printable representation of a card}
\rubricitem{1}{\function{populate_deck} creates a deck of 52 cards}
\rubricitem{1}{\function{is_three_of_kind} determines whether a hand is a three of a kind}
\rubricitem{1}{\function{is_full_house} determines whether a hand is a full house}
\rubricitem{1}{\function{is_four_of_kind} determines whether a hand is a four of a kind}
\rubricitem{2}{\function{main} generates a hand of five cards, prints the hand, and determines the best-possible characterization of the hand}
\item[Penalties]
\spaghetticodepenalties{1}
\end{description}

\section*{Epilogue}

Archie's face lights up in a very big smile. ``Thanks!'' After pausing in
thought for a moment, he says, ``Say, I've got a new startup company that could
really use your help. Are you interested? It'll be exciting!''

\textit{To be continued...}

\end{document}
