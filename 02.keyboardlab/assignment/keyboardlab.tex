%%
%% KeyboardLab (c) 2021 Christopher A. Bohn
%%

%%
%% labs/common/assignment.tex
%% (c) 2021-22 Christopher A. Bohn
%%
%% Licensed under the Apache License, Version 2.0 (the "License");
%% you may not use this file except in compliance with the License.
%% You may obtain a copy of the License at
%%     http://www.apache.org/licenses/LICENSE-2.0
%% Unless required by applicable law or agreed to in writing, software
%% distributed under the License is distributed on an "AS IS" BASIS,
%% WITHOUT WARRANTIES OR CONDITIONS OF ANY KIND, either express or implied.
%% See the License for the specific language governing permissions and
%% limitations under the License.
%%

\documentclass[12pt]{article}

\usepackage{fullpage}
\usepackage{fancyhdr}
\usepackage[procnames]{listings}
\usepackage{hyperref}
\usepackage{textcomp}
\usepackage{bold-extra}
\usepackage[dvipsnames]{xcolor}
\usepackage{etoolbox}

% These are placeholder commands and will be renewed in each lab

\newcommand{\labnumber}{}
\newcommand{\labname}{Lab \labnumber\ Assignment}
\newcommand{\shortlabname}{}
\newcommand{\duedate}{}

% Individual or team effort

\newcommand{\individualeffort}{This is an individual-effort project. You may
    discuss concepts and syntax with other students, but you may discuss
    solutions only with the professor and the TAs. Sharing code with or copying
    code from another student or the internet is prohibited.}
\newcommand{\teameffort}{This is a team-effort project. You may discuss concepts
    and syntax with other students, but you may discuss solutions only with your
    assigned partner(s), the professor, and the TAs. Sharing code with or
    copying code from a student who is not on your team, or from the internet,
    is prohibited.}
\newcommand{\freecollaboration}{In addition to the professor and the TAs, you
    may freely seek help on this assignment from other students.}
\newcommand{\collaborationrules}{}

% Software engineering (if you care about that)

\providebool{allowspaghetticode}

\newcommand{\softwareengineeringfrontmatter}{
    \ifboolexpe{not bool{allowspaghetticode}}{
        \section*{No Spaghetti Code Allowed}
        In the interest of keeping your code readable, you may \textit{not} use
        any \lstinline{goto} statements, nor may you use any
        \lstinline{continue} statements, nor may you use any \lstinline{break}
        statements to exit from a loop, nor may you have any functions
        \lstinline{return} from within a loop.
    }{}
}

\newcommand{\spaghetticodepenalties}[1]{
    \ifboolexpe{not bool{allowspaghetticode}}{
        \penaltyitem{1}{for each \lstinline{goto} statement,
            \lstinline{continue} statement, \lstinline{break} statement used to
            exit from a loop, or \lstinline{return} statement that occurs within
            a loop.}
    }{}
}

% You shouldn't need to customize these,
% but you can if you like

\lstset{language=C, tabsize=4, upquote=true, basicstyle=\ttfamily}
\newcommand{\function}[1]{\textbf{\lstinline{#1}}}
\setlength{\headsep}{0.7cm}
\hypersetup{colorlinks=true}

\newcommand{\pagelayout}{
    \pagestyle{fancy}
    \fancyhf{}
    \lhead{\coursenumber}
    \chead{\ Lab \labnumber: \labname}
    \rhead{\courseterm}
    \cfoot{\shortlabname-\thepage}
}

\newcommand{\labidentifier}{
    \title{\ Lab \labnumber}
    \author{\labname}
    \date{Due: \duedate}
    \maketitle

    \textit{\collaborationrules}
}

% deprecated
\newcommand{\startdocument}{
    \pagelayout
	\begin{document}
	\labidentifier
}

\newcommand{\rubricitem}[2]{\item[\underline{\hspace{1cm}} +#1] #2}
\newcommand{\bonusitem}[2]{\item[\underline{\hspace{1cm}} Bonus +#1] #2}
\newcommand{\penaltyitem}[2]{\item[\underline{\hspace{1cm}} -#1] #2}
\newcommand{\checkoffitem}[1]{\item (\phantom{xxx}) #1}
\newcommand{\precheckoffitem}[1]{\item [] (\phantom{xxx}) #1}

\usepackage{graphicx}
\usepackage{wrapfig}
\usepackage{ulem}
\usepackage{cancel}
\usepackage{multicol}
\lstset{language=c, numbers=left, showstringspaces=false,
    moredelim = [s][\ttfamily]{/*}{*/} % I shouldn't need this parameter!
    }

\renewcommand{\labnumber}{2}
\renewcommand{\labname}{Bit Manipulation Lab}
\renewcommand{\shortlabname}{keyboardlab}
\renewcommand{\collaborationrules}{\individualeffort}
%\renewcommand{\duedate}{See \filesubmission\ for the due date}
\renewcommand{\duedate}{Week of January 31, before the start of your lab section}

\newcommand{\tab}{\ensuremath{\longrightarrow}}
\newcommand{\nl}{\ensuremath{\hookleftarrow}}

\startdocument
% \begin{document}

The purpose of this assignment is to give you more confidence in C programming
and to begin your exposure to the underlying bit-level representation of data.

The instructions are written assuming you will edit and run the code on
\runtimeenvironment. If you wish, you may edit and run the code
in a different environment; be sure that your compiler suppresses no warnings,
and that if you are using an IDE that it is configured for C and not C++.

\section*{Learning Objectives}

After successful completion of this assignment, students will be able to:
\begin{itemize}
\item Use the ASCII table to determine the corresponding integer values of C
    \lstinline{char} values.
\item Apply arithmetic operators and comparators to C \lstinline{}{char} values.
\item Construct and use a bitmask.
\item Use bitwise operators and bit shift operators to create and modify values.
\end{itemize}

\subsection*{Continuing Forward}

Your experience with viewing values as bit patterns will be applicable in
future labs, as will bit masks and bit operations. Some of the functions you
write in this lab will be used in the next lab.

\section*{During Lab Time}

During your lab period, the TAs will demonstrate how to read the ASCII table and
will provide a refresher on bitwise AND, bitwise OR, and left- and right-shifts.
During the remaining time, the TAs will be available to answer questions.

\softwareengineeringfrontmatter

\section*{Scenario}

\begin{wrapfigure}{r}{0.33\textwidth}
    \centering
    \includegraphics[scale=.33]{some-expenses-spared}
    \caption{Some expenses were spared.}
\end{wrapfigure}

You've recently been hired to help get the Pleistocene Petting Zoo get started.
Your new employer, Archie, is surprisingly honest: he admits to you that some
expenses were spared. Archie cheerfully points out that any challenge is also
an opportunity to succeed. You suspect your job will offer plenty of
``opportunities to succeed.''

\section{Using the ASCII table}

You soon discover your first
$\cancelto{\mathrm{opportunity}}{\mathrm{challenge}}$. Archie purchased your
workstation from government surplus. Your keyboard is left over from early
2001 and is missing the letter \textit{\texttt{W}}!\footnote{In January 2001,
when President Bill Clinton's staff left the White House so that President
George~W. Bush's staff could move in, they removed the \textit{\texttt{W}} key
from several keyboards as a prank.} You decide to write an email requesting a
new keyboard: \\
\\
\texttt{TO\tab Archie\nl} \\
\texttt{RE\tab I Need a Working Keyboard\nl} \\
\nl \\
\texttt{Please order a new keyboard for me. This one is broken.\nl } \\ \\
(Note: here, the \tab\ symbol represents the \texttt{TAB} character which is
needed by the email program, and the \nl\ symbol represents a \texttt{NEWLINE}
character.)

You quickly realize that you can't type this directly into your mail program
because of the missing \textit{\texttt{W}} key. So you decide to write a short
program that will output the text that you want to send. The code you would
like to write is:

\begin{lstlisting}
#include <stdio.h>

int main() {
    printf("TO\tArchie\n");
    printf("RE\tI Need a Working Keyboard\n\n");
    printf("Please order a new keyboard for me. This one is broken.\n");
    return 0;
}
\end{lstlisting}

Of course, the \texttt{W} and the \texttt{w} are still a problem, but you
realize you can insert those characters by using their ASCII
values.\footnote{Use the ASCII table in the textbook or type \texttt{man ascii}
in a terminal window.} For example,

\lstinline{printf("Hello World!\n");} \\
can be replaced with

\lstinline{printf("%s%c%s\n", "Hello ", ..., "orld!");} \\
replacing \dots\ with the ASCII value for \texttt{W}. Recall that the first
argument for \function{printf()} is a \textit{format string}: \texttt{\%s}
specifies that a string should be placed at that position in the output, and
\texttt{\%c} specifies that a character should be placed at that position in
the output.

As you open your editor, the \textit{\texttt{\textbackslash}} key falls off the
keyboard, preventing you from typing \texttt{\textbackslash t} and
\texttt{\textbackslash n}.

Edit \texttt{problem1.c} so that it produces the specified output without using
the W key or the backslash key. Build the executable with the command:
\texttt{make keyboardlab1} -- be sure to fix both errors and warnings.

You can double-check that you aren't using the W key or the backslash key by
running the constraint-checking shell script: \\
\texttt{./constraint-check.sh} \\
(If you get a ``\texttt{Permission denied}'' error message, then run the
command \texttt{chmod +x constraint-checking.sh} and then run the shell script.)

You can check that your program has the correct output with this command: \\
\texttt{./keyboardlab1 | diff - problem1oracle}


\section{Treating Characters as Numbers}

Archie replies to your email, assuring you that a new keyboard has been
ordered. Meanwhile, he needs you to write some code that will convert uppercase
letters to lowercase letters and to indicate whether or not a character is a
decimal digit. You realize this is easy work since those actual functions are
part of the standard C library with their prototypes in \texttt{ctype.h}. As
you get ready to impress your boss with how fast you can ``write'' this code by
calling those standard functions, the \textit{\texttt{3}} key (which is also
used for \textit{\texttt{\#}}) falls off of your keyboard, preventing you from
typing \lstinline{#include <ctype.h>}. Several other number keys fall off soon
thereafter (only \textit{\texttt{0}}, \textit{\texttt{7}}, and
\textit{\texttt{9}} remain), along with the \textit{\texttt{s}} key. The
\textit{\texttt{f}} key is looking fragile, so you decide that you had better
not type too many \lstinline{if} statements (and without the
\textit{\texttt{s}} key, you can't use a \lstinline{switch} statement at all).

Edit \texttt{problem2.c} so that
\begin{itemize}
\item \function{iz_digit()} returns 1 if the character is a decimal digit
    (\textquotesingle 0\textquotesingle, \textquotesingle 1\textquotesingle,
    \textquotesingle 2\textquotesingle, \dots) and 0 otherwise
\item \function{decapitalize()} will return the lowercase version of an
    uppercase letter (\textquotesingle A\textquotesingle, \textquotesingle
    B\textquotesingle, \textquotesingle C\textquotesingle, \dots) but will
    return the original character if it is not an uppercase letter
\end{itemize}
You may not \lstinline{#include} any headers, you may not use any number keys
other than the 0, 9, and 7 (which is also used for \textbf{\texttt{\&}}) keys,
you may not use \lstinline{switch} statements, and you may use at most one
\lstinline{if} statement in each function.

Build the executable with the command: \texttt{make keyboardlab2} -- be sure to
fix both errors and warnings.

You can double-check that you aren't disallowed keys by running the
constraint-checking shell script: \\
\texttt{./constraint-check.sh} \\
Because you are allowed at most one \lstinline{if} statement in each function,
you may see two lines with \lstinline{if} statements reported when the script
checks for \textquotesingle f\textquotesingle. You may also see some comments
reported when the script checks for \textquotesingle *\textquotesingle.


\section{Using Bitmasks and Shifting Bits}

Your keyboard was mistakenly delivered to the Plywood Scenery Cutting Studio
instead of the Pleistocene Petting Zoo. While that gets sorted out, you
``borrow'' some tar from the La~Brea Tar Pits diorama and use the tar to
re-attach your keyboard's missing keys. As you fasten a Scrabble tile in place
for the \textit{\texttt{W}}, more keys fall off, denying you the use of
\textit{\texttt{+}}, \textit{\texttt{-}}, \textit{\texttt{/}},
\textit{\texttt{\%}}, \textit{\texttt{5}}, and \textit{\texttt{b}}. You cannot
get any more tar from the diorama, so you sit down to your next programming
tasks without those keys.

Edit \texttt{problem3.c} so that
\begin{itemize}
\item \function{is_even()} returns 1 if the number is even (that is,
    divisible by 2) and 0 if the number is odd
\item \function{produce_multiple_of_ten()} will always output a multiple of
    10 following a specific formula: if a number is even then divide it by
    2; otherwise, subtract 1 from the number and multiply the difference
    by 5 (for example, an input of 7 yields 30 because $(7-1) \times 5 = 30$);
    repeat until the last decimal digit is 0.
\end{itemize}
These numbers are guaranteed to be non-negative. You may not use addition (+),
subtraction (-), division (/), nor modulo (\%). You also may not use the
number 5 nor the letter b. (Exceptions: you \textit{may} use the forward-slash
(/) for comments, and the percent-sign (\%) that is already present in the
\function{sprintf()} calls' format strings is allowed)

Hints:
\begin{itemize}
\item Following the weighted-sum technique to convert between binary and
    decimal (or by examining the textbook's Table~2.1), you will note that all
    even numbers have a 0 as their least significant bit, and all odd numbers
    have a 1 as their least significant bit
\item Less obvious is that you can subtract 1 from an odd number by changing
    its least significant bit to a 0
\item As we will cover in Chapter~3, you can halve a number by shifting its
    bits to the right by one
\item You can create an integer by producing its bit pattern through a series
    of bit operations
\end{itemize}

Build the executable with the command: \texttt{make keyboardlab3} -- be sure to
fix both errors and warnings.

You can double-check that you aren't disallowed keys by running the
constraint-checking shell script: \\
\texttt{./constraint-check.sh} \\
You may see some comments reported when the script checks for
\textquotesingle *\textquotesingle.


\section*{Turn-in and Grading}

When you have completed this assignment, upload \textit{problem1.c},
\textit{problem2.c}, and \textit{problem3.c} to \filesubmission.

This assignment is worth 20 points.
\begin{description}
\rubricitem{4}{\textit{problem1.c} produces the specified output.}
\rubricitem{4}{\function{iz_digit()} in \textit{problem2.c} determines whether
    or not a character is a digit.}
\rubricitem{4}{\function{decapitalize()} in \textit{problem2.c} converts
    uppercase letters to lowercase and leaves other characters unchanged.}
\rubricitem{4}{\function{is_even()} in \textit{problem3.c} determines whether
    a number is even or odd.}
\item[\hspace{1cm}]\function{produce_multiple_of_ten()} in \textit{problem3.c}
    has the following:
    \begin{description}
    \rubricitem{1}{Code to assign the value 5 to the variable \lstinline{five}}
    \rubricitem{1}{Code to divide an even number by 2}
    \rubricitem{1}{Code to subtract 1 from an odd number}
    \rubricitem{1}{Correct functionality}
    \end{description}
\item[Penalties]
\penaltyitem{4}{The solution to \textit{problem1.c} uses \texttt{w},
    \texttt{W}, \texttt{\textbackslash{}n}, or \texttt{\textbackslash{}t}.}
\penaltyitem{4}{\function{iz_digit())} uses a digit other than 0 and 9, uses a
    \lstinline{switch} statment, or uses more than one \lstinline{if}
    statement.}
\penaltyitem{4}{\function{decapitalize()} \lstinline{#include}s one or more
    files, uses a \lstinline{switch} statment, or uses more than one
    \lstinline{if} statement.}
\penaltyitem{4}{\function{is_even()} uses arithmetic.}
\penaltyitem{4}{\function{produce_multiple_of_ten()} uses addition,
    subtraction, division, or modulo; or \function{produce_multiple_of_ten()}
    uses the literal the value 5, 0x5, 05, or 0b101.}
\spaghetticodepenalties{1}
\end{description}

\section*{Epilogue}

Great news! Archie brings you your new keyboard. He also brings you a problem
of his own. Because you were held up with the broken keyboard, Archie decided
to try some programming on his own, and his code is behaving strangely.

\textit{To be continued...}

\end{document}
