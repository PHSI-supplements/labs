%%
%% SSBClab (c) 2020-21 Christopher A. Bohn
%%
%% Licensed under the Apache License, Version 2.0 (the "License");
%% you may not use this file except in compliance with the License.
%% You may obtain a copy of the License at
%%     http://www.apache.org/licenses/LICENSE-2.0
%% Unless required by applicable law or agreed to in writing, software
%% distributed under the License is distributed on an "AS IS" BASIS,
%% WITHOUT WARRANTIES OR CONDITIONS OF ANY KIND, either express or implied.
%% See the License for the specific language governing permissions and
%% limitations under the License.
%%

%%
%% (c) 2021 Christopher A. Bohn
%%

\documentclass[12pt]{article}

\usepackage{fullpage}
\usepackage{fancyhdr}
\usepackage[procnames]{listings}
\usepackage{hyperref}
\usepackage{textcomp}
\usepackage{bold-extra}
\usepackage[dvipsnames]{xcolor}
\usepackage{etoolbox}


% Customize the semester (or quarter) and the course number

\newcommand{\courseterm}{Spring 2022}
\newcommand{\coursenumber}{CSCE 231}

% Customize how a typical lab will be managed;
% you can always use \renewcommand for one-offs

\newcommand{\runtimeenvironment}{your account on the \textit{csce.unl.edu} Linux server}
\newcommand{\filesource}{Canvas or {\footnotesize$\sim$}cse231 on \textit{csce.unl.edu}}
\newcommand{\filesubmission}{Canvas}

% These are placeholder commands and will be renewed in each lab

\newcommand{\labnumber}{}
\newcommand{\labname}{Lab \labnumber\ Assignment}
\newcommand{\shortlabname}{}
\newcommand{\duedate}{}

% Individual or team effort

\newcommand{\individualeffort}{This is an individual-effort project. You may discuss concepts and syntax with other students, but you may discuss solutions only with the professor and the TAs. Sharing code with or copying code from another student or the internet is prohibited.}
\newcommand{\teameffort}{This is a team-effort project. You may discuss concepts and syntax with other students, but you may discuss solutions only with your assigned partner(s), the professor, and the TAs. Sharing code with or copying code from a student who is not on your team, or from the internet, is prohibited.}
\newcommand{\freecollaboration}{In addition to the professor and the TAs, you may freely seek help on this assignment from other students.}
\newcommand{\collaborationrules}{}

% Do you care about software engineering?

\providebool{allowspaghetticode}

\setbool{allowspaghetticode}{false}

\newcommand{\softwareengineeringfrontmatter}{
    \ifboolexpe{not bool{allowspaghetticode}}{
        \section*{No Spaghetti Code Allowed}
        In the interest of keeping your code readable, you may \textit{not} use
        any \lstinline{goto} statements, nor may you use any \lstinline{break}
        statements to exit from a loop, nor may you have any functions
        \lstinline{return} from within a loop.
    }{}
}

\newcommand{\spaghetticodepenalties}[1]{
    \ifboolexpe{not bool{allowspaghetticode}}{
        \penaltyitem{1}{for each \lstinline{goto} statement, \lstinline{break}
            statement used to exit from a loop, or \lstinline{return} statement
            that occurs within a loop.}
    }{}
}

% You shouldn't need to customize these,
% but you can if you like

\lstset{language=C, tabsize=4, upquote=true, basicstyle=\ttfamily}
\newcommand{\function}[1]{\textbf{\lstinline{#1}}}
\setlength{\headsep}{0.7cm}
\hypersetup{colorlinks=true}

\newcommand{\startdocument}{
    \pagestyle{fancy}
    \fancyhf{}
    \lhead{\coursenumber}
    \chead{\ Lab \labnumber: \labname}
    \rhead{\courseterm}
    \cfoot{\shortlabname-\thepage}

	\begin{document}
	\title{\ Lab \labnumber}
	\author{\labname}
	\date{Due: \duedate}
	\maketitle

    \textit{\collaborationrules}
}

\newcommand{\rubricitem}[2]{\item[\underline{\hspace{1cm}} +#1] #2}
\newcommand{\bonusitem}[2]{\item[\underline{\hspace{1cm}} Bonus +#1] #2}
\newcommand{\penaltyitem}[2]{\item[\underline{\hspace{1cm}} -#1] #2}

%%
%% labs/common/semester.tex
%% (c) 2021-22 Christopher A. Bohn
%%
%% Licensed under the Apache License, Version 2.0 (the "License");
%% you may not use this file except in compliance with the License.
%% You may obtain a copy of the License at
%%     http://www.apache.org/licenses/LICENSE-2.0
%% Unless required by applicable law or agreed to in writing, software
%% distributed under the License is distributed on an "AS IS" BASIS,
%% WITHOUT WARRANTIES OR CONDITIONS OF ANY KIND, either express or implied.
%% See the License for the specific language governing permissions and
%% limitations under the License.
%%


% Customize the semester (or quarter) and the course number

\newcommand{\courseterm}{Fall 2022}
\newcommand{\coursenumber}{CSCE 231}

% Customize how a typical lab will be managed;
% you can always use \renewcommand for one-offs

\newcommand{\runtimeenvironment}{your account on the \textit{csce.unl.edu} Linux server}
\newcommand{\filesource}{Canvas or {\footnotesize$\sim$}cse231 on \textit{csce.unl.edu}}
\newcommand{\filesubmission}{Canvas}

% Customize for the I/O lab hardware

\newcommand{\developmentboard}{Arduino Nano}
%\newcommand{\serialprotocol}{SPI}
\newcommand{\serialprotocol}{I2C}
%\newcommand{\displaymodule}{MAX7219digits}
%\newcommand{\displaymodule}{MAX7219matrix}
\newcommand{\displaymodule}{LCD1602}

\setbool{usedisplayfont}{true}

\newcommand{\obtaininghardware}{
    The EE Shop has prepared ``class kits'' for CSCE 231; your class kit costs \$30.
    The EE Shop is located at 122 Scott Engineering Center and is open M-F 7am-4pm. You do not need an appointment.
    You may pay at the window with cash, with a personal check, or with your NCard.
    The EE shop does \textit{not} accept credit cards.
}

% Update to reflect the CS2 course(s) at your institute

\newcommand{\cstwo}{CSCE~156, RAIK~184H, or SOFT~161}

% Do you care about software engineering?

\setbool{allowspaghetticode}{false}

% Which assignments are you using this semester, and when are they due?

\newcommand{\pokerlabnumber}{1}
\newcommand{\pokerlabcollaboration}{
    Sections~\ref{sec:connecting}, \ref{sec:terminology}, \ref{sec:gettingstarted}, \ref{subsec:typesofpokerhands}, and~\ref{subsec:studythecode}: \freecollaboration
    Sections~\ref{sec:completingcard} and~\ref{subsec:completepoker}: \individualeffort
}
\newcommand{\pokerlabdue}{Week of August 29, before the start of your lab section}

\newcommand{\keyboardlabnumber}{2}
\newcommand{\keyboardlabcollaboration}{\individualeffort}
\newcommand{\keyboardlabdue}{Week of January 31, before the start of your lab section}

\newcommand{\pointerlabnumber}{3}
\newcommand{\pointerlabcollaboration}{\individualeffort}
\newcommand{\pointerlabdue}{Week of February 7, before the start of your lab section}

\newcommand{\integerlabnumber}{4}
\newcommand{\integerlabcollaboration}{\individualeffort}
\newcommand{\integerlabdue}{Week of February 14, before the start of your lab section}

\newcommand{\floatlabnumber}{5}
\newcommand{\floatlabcollaboration}{\individualeffort}
\newcommand{\floatlabdue}{soon}

\newcommand{\addressinglabnumber}{6}
\newcommand{\addressinglabcollaboration}{\individualeffort}
\newcommand{\addressinglabdue}{Week of February 28, before the start of your lab section}

%bomblab was 7
%attacklab was 8

\newcommand{\pollinglabnumber}{9}
\newcommand{\pollinglabcollaboration}{\individualeffort}
\newcommand{\pollinglabdue}{Week of April 11, before the start of your lab section}
\newcommand{\pollinglabenvironment}{your \developmentboard-based class hardware kit}

\newcommand{\ioprelabnumber}{\pollinglabnumber-prelab}
\newcommand{\ioprelabcollaboration}{\freecollaboration}
\newcommand{\ioprelabdue}{Before the start of your lab section on April 5 or 6}

\newcommand{\interruptlabnumber}{10}
\newcommand{\interruptlabcollaboration}{\individualeffort}
\newcommand{\interruptlabdue}{Week of April 18, before the start of your lab section}
\newcommand{\interruptlabenvironment}{your \developmentboard-based class hardware kit}

\newcommand{\capstonelab}{ComboLock}    % this will come into play when we generalize capstonelab
\newcommand{\capstonelabnumber}{11}
\newcommand{\capstonelabcollaboration}{\teameffort}
\newcommand{\capstonelabdue}{Week of May 2, Before the start of your lab section\footnote{See Piazza for the due dates of teams with students from different lab sections.}}
\newcommand{\capstonelabenvironment}{your \developmentboard-based class hardware kit}

\newcommand{\memorylabnumber}{12}
\newcommand{\memorylabcollaboration}{This is an individual-effort project. You may discuss the nature of memory technologies and of memory hierarchies with classmates, but you must draw your own conclusions.}
\newcommand{\memorylabdue}{Week of May 2, at the end of your lab section}
\newcommand{\memorylabenvironment}{your \developmentboard-based class hardware kit and your account on the \textit{csce.unl.edu} Linux server}

% Labs not used this semester

\newcommand{\concurrencylabnumber}{XX}
\newcommand{\concurrencylabcollaboration}{\individualeffort}
\newcommand{\concurrencylabdue}{not this semester}

\newcommand{\ssbcwarmupnumber}{XX}
\newcommand{\ssbcwarmupcollaboration}{\freecollaboration}
\newcommand{\ssbcwarmupdue}{not this semester}

\newcommand{\ssbcpollingnumber}{XX}
\newcommand{\ssbcpollingcollaboration}{\individualeffort}
\newcommand{\ssbcpollingdue}{not this semester}

\newcommand{\ssbcinterruptnumber}{XX}
\newcommand{\ssbcinterruptcollaboration}{\individualeffort}
\newcommand{\ssbcinterruptdue}{not this semester}

\usepackage{enumitem}
\usepackage{graphicx}
\usepackage{media9}
\usepackage{addfont}
\addfont{OT1}{d7seg}{\dviiseg}
\addfont{OT1}{deseg}{\deseg}

\renewcommand{\labnumber}{\ssbcpollingnumber}
\renewcommand{\labname}{Polling to Detect Inputs on Simulated Hardware}
\renewcommand{\shortlabname}{ssbclab-polling}
\renewcommand{\collaborationrules}{\ssbcpollingcollaboration}
\renewcommand{\duedate}{\ssbcpollingdue}
\pagelayout
\begin{document}
\labidentifier

%\usepackage{fullpage}
%\usepackage{enumitem}

Due to the circumstances of this semester, we will not be using Altera
single board computers for labs. Instead, we will use a simulated single
board computer (SSBC).

In this assignment, you will use polling to implement a simple calulator on the
Simulated Single Board Computer

The instructions are written assuming you will edit and run the code on
\runtimeenvironment. If you wish, you may edit the code in a different
environment; however, you will not be able to link an executable except on
\runtimeenvironment.

\section{Getting Started}

The accompanying document, \textit{ssbc.pdf} describes how to use the SSBC
library: how to interact with the simulated SBC and how to use its inputs and
outputs in a program.

Download \textit{\shortlabname.zip} or \textit{\shortlabname.tar} from
\filesource\ and copy it to \runtimeenvironment. Once copied, unpackage the
file. You will find four files:
\begin{description}
    \item [ssbc.h] The header file for the SSBC library.
    \item [Makefile] The Makefile to build the programs for this lab. You can build the programs individually, or you can build all of them using the command \texttt{make all}.
    % \item [demo.c] A program that demonstrates using the SSBC library. Run the
    %     command \texttt{make demo} to build the program, and then run it with
    %     the command \texttt{./demo}. This file is described line-by-line in
    %     \textit{ssbc.pdf}
    % \item [warmup.c] Starter code for this lab. Run the command \texttt{make
    %     warmup} to build the program, and then run it with the command
    %     \texttt{./warmup}.
    \item [poll-calculator.c] Starter code for this lab. Run the
        command \texttt{make poll-calculator} to build the program, and then
        run it with the command \texttt{./poll-calculator}.
    % \item [interrupt-calculator.c] Starter code for part 3 of this lab. Run the command\\ \texttt{make interrupt-calculator} to build the program, and then run it with the command \texttt{./interrupt-calculator}.
\end{description}

\textbf{\textit{Note: }Do NOT use \function{printf} to print debugging
information. Use \function{ssbc_print} instead.}

\section{Use Polling to Detect Inputs}

It may not have been obvious while you were writing your warmup program, but
you cannot tell when the user has pressed the same number button twice in a
row by simply reading the BCD value from the number pad's register. This is
obvious if you run the demonstration program and pay attention to the debug
window. You will now overcome this problem by \textit{polling} the number pad
register's \textit{dirty bit}.

``Dirty bit'' is a bit used to indicate that something has changed. Here, the
dirty bit indicates that the user has pressed a number button. ``Polling'' is a
technique to periodically read a value to determine if it has changed.

\begin{figure}
    \centering
    \includegraphics[width=10cm]{AreWeThereYet}
    \caption{Polling. \tiny Image by 20th Century Fox Television}
\end{figure}

Edit \textit{poll-calculator.c} to write a program that uses the SSBC's input
and output registers to meet the following specification:

\begin{itemize}
    \item Functionality: implement a simple 2-function decimal integer
        calculator.
    \begin{itemize}
        \item Initially, $operand1$ and $operand2$ will hold the
            value $0$.
        \item When all toggle switches are in the ``off'' position, the
            seven-segment displays will be blank.
        \item When toggle switch 3 (the switch that is toggled by the `a' key)
            is moved to the ``on'' position, the program starts to build
            $operand1$. When toggle switch 3 is moved to the ``off''
            position, $operand1$ takes the value that was built.
        \item When toggle switch 2 (the switch that is toggled by the `s' key)
            is moved to the ``on'' position, the program starts to build
            $operand2$. When toggle switch 2 is moved to the ``off''
            position, $operand2$ takes the value that was built.
        \item When building an operand:
        \begin{itemize}
            \item The right-most seven-segment display (the ``ones'' place)
                will always display a digit. The remaining seven-segment
                displays will not display leading $0$s.
            \item The value being built will initially be $0$, and so {\dviiseg
                0} will be initially displayed.
            \item When the first number button is pressed after starting to
                build an operand, the ``ones'' place will display that digit.
                For example, if the user presses `5', then {\dviiseg 5} will be
                displayed.
            \item There will never be leading $0$s in the ``tens,''
                ``hundreds,'' and ``thousands'' places.
            \item When the second, third, and fourth number button is pressed
                (if a second, third, or fourth number button is pressed) then
                the existing digits will shift one decimal digit to the left,
                and the newly-entered number will go in the ``ones'' place. For
                example, if the user presses `5', `3', `0', and `9', then the
                display will show these values in sequence: \\
                \begin{tabular}{r}
                {\dviiseg 0} \\
                {\dviiseg 5} \\
                {\dviiseg 53} \\
                {\dviiseg 530} \\
                {\dviiseg 5309}
                \end{tabular}
            \item There is no way to build a negative operand.
            \item Except as noted in Section~\ref{bonus}, you may assume that
                operands are representable with four or fewer decimal digits.
                Unless you implement the bonus, we are not specifying the
                correct behavior if the user presses a fifth number button.
        \end{itemize}
        \item Number button presses will be ignored except when building
            operands.
        \item When toggle switch 1 (the switch that is toggled by the `d' key)
            is moved to the ``on'' position, the seven-segment displays will
            display the sum of $operand1 + operand2$. If the sum is $0$ then
            {\dviiseg 0} shall be displayed; otherwise, there shall be no
            leading $0$s. Except as noted in Part~\ref{bonus}, you may assume
            that the sum is representable with four or fewer decimal digits.
            Unless you implement the bonus, we are not specifying the correct
            behavior for sums greater than $9,999$.
        \item When toggle switch 0 (the switch that is toggled by the `f' key)
            is moved to the ``on'' position, the seven-segment displays will
            display the difference of $operand1 - operand2$. If the difference
            is $0$ then {\dviiseg 0} shall be displayed; otherwise, there shall
            be no leading $0$s. Except as noted in Part~\ref{bonus}, you may
            assume that the difference is non-negative; unless you implement
            the bonus, the correct behavior for negative differences is
            unspecified.
        \item You may assume that at most one toggle switch will ever be in the
            ``on'' position. We are not specifying the correct behavior if two
            toggle switches are in the ``on'' position.
    \end{itemize}
    \item Implementation
    \begin{itemize}
        \item Poll the number pad register's dirty bit to determine when a
            number button has been pressed. The SSBC library will place a $1$
            in the dirty bit whenver a number button has been pressed.
        \item After reading the number pad register's BCD value, clear the dirty
            bit by setting it to $0$.
        \item You may use any built-in C operations.
        \item \textit{Hint: } You may want to write a function that will
            convert an integer into the bit vector for the seven-segment
            displays' register. The array to generate bit vectors for a single
            digit from the warmup exercise will also be useful.
    \end{itemize}
\end{itemize}

\section*{Turn-in and Grading}

When you have completed this assignment, upload \textit{poll-calculator.c} to
\filesubmission.

This assignment is worth 35 points. \\

Rubric:
\begin{description}
\rubricitem{1}{The displays are blank when all toggle switches are in the
    ``off'' position}
\rubricitem{1}{$Operand1$ with with one digit can be built.}
\rubricitem{2}{$Operand1$ with non-repeating digits can be built,
    \textit{e.g.}, $231$.}
\rubricitem{3}{$Operand1$ with repeating digits can be built, \textit{e.g.},
    $2331$.}
\rubricitem{1}{When building $operand1$, the ``ones'' place is always displayed}
\rubricitem{2}{When building $operand1$, there are no leading $0$s in the
    ``tens,'' ``hundreds,'' or ``thousands'' places.}
\rubricitem{1}{When toggle switch 3 is moved to the ``off'' position,
    $operand1$ takes on the value built.}
\rubricitem{9}{It is possible to build $operand2$ in accordance with
    specification.}
\rubricitem{1}{When toggle switch 2 is moved to the ``off'' position,
    $operand2$ takes on the value built.}
\rubricitem{5}{When toggle switch 1 is moved to the ``on'' position, the sum of
    $operand1 + operand2$ is displayed without leading $0$s.}
\rubricitem{5}{When toggle switch 0 is moved to the ``on'' position, the
    difference of $operand1 - operand2$ is displayed without leading $0$s.}
\rubricitem{3}{Number button presses are detected by polling the dirty bit.}
\rubricitem{1}{The dirty bit is reset after obtaining the value of the number
    button pressed.}
\end{description}

\section{Extra Credit} \label{bonus}

\textbf{Up to 15 bonus points}

The calculator specification left a few edge cases unspecified. For bonus
credit, implement some or all of these specifications for the edge cases.

\textbf{Note: } You may receive bonus credit as part of this week's lab or as
part of next week's lab, but not both. \textit{If you implement the
edge cases, indicate so in the header comments of the calculator implementation
that you want evaluated for bonus credit. \textbf{We will not evaluate a
calculator implementation for bonus credit that does not state that it has the
edge cases implemented.}}

\begin{description}
\bonusitem{5}{Handle an attempt to build a too-great operand}
    \begin{itemize}
        \item When building an operand, if the user presses a fifth number
            button, the operand being built is invalid. The seven-segment
            displays shall display {\dviiseg err}.
        \item The only way to clear this error is to move the operand's toggle
            switch to the ``off'' position. When the user does so, the original
            operand's value will be preserved. For example:
        \begin{itemize}
            \item $operand1$ holds the value $53$.
            \item The user moves toggle switch 3 to the ``on'' position and
                presses `4', `5', `2', `3' which causes the display will show
                these values in sequence: \\
                \begin{tabular}{r}
                {\dviiseg 0} \\
                {\dviiseg 4} \\
                {\dviiseg 45} \\
                {\dviiseg 452} \\
                {\dviiseg 4523}
                \end{tabular}
            \item If the user were to move toggle switch 3 to the ``off''
                position, then $operand1$ would take the value $4,523$.
                Instead, the user presses `7' which causes the display to show
                {\dviiseg err}.
            \item The user now moves toggle switch 3 to the ``off'' position,
                and $operand1$ retains the value $53$.
        \end{itemize}
        \item This edge case must work for both operands to receive any credit
            for this edge case.
    \end{itemize}
\bonusitem{3}{Handle a too-great sum} \\
        If, when toggle switch 1 is moved to the ``on'' position, the sum is
        greater than $9,999$ then the seven-segment displays shall display
        {\dviiseg err}.
\bonusitem{5}{Handle negative differences} \\
        If, when toggle switch 0 is moved to the ``on'' position, the
        difference is between $-1$ and $- 999$ (inclusive) then the difference
        shall be displayed, including the negative sign. As with non-negative
        values, there shall be no leading $0$s. For example, subtracting
        $53-102$ would cause the seven-segment displays to display
        {\dviiseg -49}.
\bonusitem{2}{Handle a too-low difference} \\
        If, when toggle switch 0 is moved to the ``off'' position, the
        difference is less than $-999$ then the seven-segment displays shall
        display {\dviiseg err}.
    \begin{itemize}
        \item If you do not implement the 5-point ``Handle negative
        differences'' bonus, then you can receive the 2-point ``Handle a
        too-low difference'' by displaying {\dviiseg err} for any value less
        than $0$.
    \end{itemize}
\end{description}

\end{document}
