%%
%% InterruptLab (c) 2021-23 Christopher A. Bohn
%%
%% Licensed under the Apache License, Version 2.0 (the "License");
%% you may not use this file except in compliance with the License.
%% You may obtain a copy of the License at
%%     http://www.apache.org/licenses/LICENSE-2.0
%% Unless required by applicable law or agreed to in writing, software
%% distributed under the License is distributed on an "AS IS" BASIS,
%% WITHOUT WARRANTIES OR CONDITIONS OF ANY KIND, either express or implied.
%% See the License for the specific language governing permissions and
%% limitations under the License.
%%

%%
%% labs/common/assignment.tex
%% (c) 2021-22 Christopher A. Bohn
%%
%% Licensed under the Apache License, Version 2.0 (the "License");
%% you may not use this file except in compliance with the License.
%% You may obtain a copy of the License at
%%     http://www.apache.org/licenses/LICENSE-2.0
%% Unless required by applicable law or agreed to in writing, software
%% distributed under the License is distributed on an "AS IS" BASIS,
%% WITHOUT WARRANTIES OR CONDITIONS OF ANY KIND, either express or implied.
%% See the License for the specific language governing permissions and
%% limitations under the License.
%%

\documentclass[12pt]{article}

\usepackage{fullpage}
\usepackage{fancyhdr}
\usepackage[procnames]{listings}
\usepackage{hyperref}
\usepackage{textcomp}
\usepackage{bold-extra}
\usepackage[dvipsnames]{xcolor}
\usepackage{etoolbox}

% These are placeholder commands and will be renewed in each lab

\newcommand{\labnumber}{}
\newcommand{\labname}{Lab \labnumber\ Assignment}
\newcommand{\shortlabname}{}
\newcommand{\duedate}{}

% Individual or team effort

\newcommand{\individualeffort}{This is an individual-effort project. You may
    discuss concepts and syntax with other students, but you may discuss
    solutions only with the professor and the TAs. Sharing code with or copying
    code from another student or the internet is prohibited.}
\newcommand{\teameffort}{This is a team-effort project. You may discuss concepts
    and syntax with other students, but you may discuss solutions only with your
    assigned partner(s), the professor, and the TAs. Sharing code with or
    copying code from a student who is not on your team, or from the internet,
    is prohibited.}
\newcommand{\freecollaboration}{In addition to the professor and the TAs, you
    may freely seek help on this assignment from other students.}
\newcommand{\collaborationrules}{}

% Software engineering (if you care about that)

\providebool{allowspaghetticode}

\newcommand{\softwareengineeringfrontmatter}{
    \ifboolexpe{not bool{allowspaghetticode}}{
        \section*{No Spaghetti Code Allowed}
        In the interest of keeping your code readable, you may \textit{not} use
        any \lstinline{goto} statements, nor may you use any
        \lstinline{continue} statements, nor may you use any \lstinline{break}
        statements to exit from a loop, nor may you have any functions
        \lstinline{return} from within a loop.
    }{}
}

\newcommand{\spaghetticodepenalties}[1]{
    \ifboolexpe{not bool{allowspaghetticode}}{
        \penaltyitem{1}{for each \lstinline{goto} statement,
            \lstinline{continue} statement, \lstinline{break} statement used to
            exit from a loop, or \lstinline{return} statement that occurs within
            a loop.}
    }{}
}

% You shouldn't need to customize these,
% but you can if you like

\lstset{language=C, tabsize=4, upquote=true, basicstyle=\ttfamily}
\newcommand{\function}[1]{\textbf{\lstinline{#1}}}
\setlength{\headsep}{0.7cm}
\hypersetup{colorlinks=true}

\newcommand{\pagelayout}{
    \pagestyle{fancy}
    \fancyhf{}
    \lhead{\coursenumber}
    \chead{\ Lab \labnumber: \labname}
    \rhead{\courseterm}
    \cfoot{\shortlabname-\thepage}
}

\newcommand{\labidentifier}{
    \title{\ Lab \labnumber}
    \author{\labname}
    \date{Due: \duedate}
    \maketitle

    \textit{\collaborationrules}
}

% deprecated
\newcommand{\startdocument}{
    \pagelayout
	\begin{document}
	\labidentifier
}

\newcommand{\rubricitem}[2]{\item[\underline{\hspace{1cm}} +#1] #2}
\newcommand{\bonusitem}[2]{\item[\underline{\hspace{1cm}} Bonus +#1] #2}
\newcommand{\penaltyitem}[2]{\item[\underline{\hspace{1cm}} -#1] #2}
\newcommand{\checkoffitem}[1]{\item (\phantom{xxx}) #1}
\newcommand{\precheckoffitem}[1]{\item [] (\phantom{xxx}) #1}

%%
%% labs/common/semester.tex
%% (c) 2021-22 Christopher A. Bohn
%%
%% Licensed under the Apache License, Version 2.0 (the "License");
%% you may not use this file except in compliance with the License.
%% You may obtain a copy of the License at
%%     http://www.apache.org/licenses/LICENSE-2.0
%% Unless required by applicable law or agreed to in writing, software
%% distributed under the License is distributed on an "AS IS" BASIS,
%% WITHOUT WARRANTIES OR CONDITIONS OF ANY KIND, either express or implied.
%% See the License for the specific language governing permissions and
%% limitations under the License.
%%


% Customize the semester (or quarter) and the course number

\newcommand{\courseterm}{Spring 2022}
\newcommand{\coursenumber}{CSCE 231}

% Customize how a typical lab will be managed;
% you can always use \renewcommand for one-offs

\newcommand{\runtimeenvironment}{your account on the \textit{csce.unl.edu} Linux server}
\newcommand{\filesource}{Canvas or {\footnotesize$\sim$}cse231 on \textit{csce.unl.edu}}
\newcommand{\filesubmission}{Canvas}

% Customize for the I/O lab hardware

\newcommand{\developmentboard}{Arduino Nano}

%\newcommand{\serialprotocol}{SPI}
\newcommand{\serialprotocol}{I2C}

%\newcommand{\displaymodule}{MAX7219digits}
%\newcommand{\displaymodule}{MAX7219matrix}
\newcommand{\displaymodule}{LCD1602}

\setbool{usedisplayfont}{true}

\newcommand{\obtaininghardware}{
    The EE Shop has prepared ``class kits'' for CSCE 231; your class kit costs \$20. The EE Shop is located at 122 Scott
    Engineering Center and is open M-F 7am-4pm. You do not need an appointment. You may pay at the window with cash,
    with a personal check, or with your NCard. If you wish to pay by credit card, you must make the purchase from
    \url{https://marketplace.unl.edu/ees/engineering-class-kits/csce231-kit.html} the day before you visit the EE
    Shop.\footnote{The price listed on the website is \$18.65; after sales tax is added, your total will be \$20.}
}

% Update to reflect the CS2 course(s) at your institute

\newcommand{\cstwo}{CSCE~156, RAIK~184H, or SOFT~161}

% Do you care about software engineering?

\setbool{allowspaghetticode}{false}

% Which assignments are you using this semester, and when are they due?

\newcommand{\pokerlabnumber}{1}
\newcommand{\pokerlabcollaboration}{Except as noted in Section~\ref{StudyTheCode}, \individualeffort}
\newcommand{\pokerlabdue}{Week of January 24, before the start of your lab section}

\newcommand{\keyboardlabnumber}{2}
\newcommand{\keyboardlabcollaboration}{\individualeffort}
\newcommand{\keyboardlabdue}{Week of January 31, before the start of your lab section}

\newcommand{\pointerlabnumber}{3}
\newcommand{\pointerlabcollaboration}{\individualeffort}
\newcommand{\pointerlabdue}{Week of February 7, before the start of your lab section}

\newcommand{\integerlabnumber}{4}
\newcommand{\integerlabcollaboration}{\individualeffort}
\newcommand{\integerlabdue}{Week of February 14, before the start of your lab section}

\newcommand{\floatlabnumber}{5}
\newcommand{\floatlabcollaboration}{\individualeffort}
\newcommand{\floatlabdue}{soon}

\newcommand{\addressinglabnumber}{6}
\newcommand{\addressinglabcollaboration}{\individualeffort}
\newcommand{\addressinglabdue}{Week of February 28, before the start of your lab section}

%bomblab was 7
%attacklab was 8

\newcommand{\pollinglabnumber}{9}
\newcommand{\pollinglabcollaboration}{\individualeffort}
\newcommand{\pollinglabdue}{Week of April 11, before the start of your lab section}
\newcommand{\pollinglabenvironment}{your \developmentboard-based class hardware kit}

\newcommand{\ioprelabnumber}{\pollinglabnumber-prelab}
\newcommand{\ioprelabcollaboration}{\freecollaboration}
\newcommand{\ioprelabdue}{Before the start of your lab section on April 5 or 6}

\newcommand{\interruptlabnumber}{10}
\newcommand{\interruptlabcollaboration}{\individualeffort}
\newcommand{\interruptlabdue}{Week of April 18, before the start of your lab section}
\newcommand{\interruptlabenvironment}{your \developmentboard-based class hardware kit}

\newcommand{\capstonelab}{ComboLock}    % this will come into play when we generalize capstonelab
\newcommand{\capstonelabnumber}{11}
\newcommand{\capstonelabcollaboration}{\teameffort}
\newcommand{\capstonelabdue}{Week of May 2, Before the start of your lab section\footnote{See Piazza for the due dates of teams with students from different lab sections.}}
\newcommand{\capstonelabenvironment}{your \developmentboard-based class hardware kit}

\newcommand{\memorylabnumber}{12}
\newcommand{\memorylabcollaboration}{This is an individual-effort project. You may discuss the nature of memory technologies and of memory hierarchies with classmates, but you must draw your own conclusions.}
\newcommand{\memorylabdue}{Week of May 2, at the end of your lab section}
\newcommand{\memorylabenvironment}{your \developmentboard-based class hardware kit and your account on the \textit{csce.unl.edu} Linux server}

% Labs not used this semester

\newcommand{\concurrencylabnumber}{XX}
\newcommand{\concurrencylabcollaboration}{\individualeffort}
\newcommand{\concurrencylabdue}{not this semester}

\newcommand{\ssbcwarmupnumber}{XX}
\newcommand{\ssbcwarmupcollaboration}{\freecollaboration}
\newcommand{\ssbcwarmupdue}{not this semester}

\newcommand{\ssbcpollingnumber}{XX}
\newcommand{\ssbcpollingcollaboration}{\individualeffort}
\newcommand{\ssbcpollingdue}{not this semester}

\newcommand{\ssbcinterruptnumber}{XX}
\newcommand{\ssbcinterruptcollaboration}{\individualeffort}
\newcommand{\ssbcinterruptdue}{not this semester}

%%
%% labs/common/storylines.tex
%% (c) 2020-22 Christopher A. Bohn
%%
%% Licensed under the Apache License, Version 2.0 (the "License");
%% you may not use this file except in compliance with the License.
%% You may obtain a copy of the License at
%%     http://www.apache.org/licenses/LICENSE-2.0
%% Unless required by applicable law or agreed to in writing, software
%% distributed under the License is distributed on an "AS IS" BASIS,
%% WITHOUT WARRANTIES OR CONDITIONS OF ANY KIND, either express or implied.
%% See the License for the specific language governing permissions and
%% limitations under the License.
%%

\newcommand{\MeetArchie}{
    You're relaxing at your favorite hangout when another customer catches your attention.
    He's rather large (dare I say, \textit{mammoth}), a bit hairy, and looking frustrated in front of his laptop.
    ``I'm Archie,'' he says, ``and I'm trying to teach myself this card game called \textit{Poker}.
    I found this source code that I thought I could use to understand Poker better, but the code is incomplete, and I don't entirely understand what's there.
    Could you explain the code to me, please?'
}

\newcommand{\GetHired}{
    Archie's face lights up in a very big smile.
    ``Thanks!''
    After pausing in thought for a moment, he says, ``Say, I've got a new startup company that could really use your help.
    Are you interested?
    It'll be exciting!''
}

\usepackage{amsmath}
%\usepackage{array,color,colortbl}
\definecolor{LightGreen}{rgb}{0.88,1,0.88}
\usepackage{multicol}
\usepackage{CJKutf8}
\usepackage{gensymb}
\setlength{\columnsep}{2cm}


% \captionsetup{width=.8\linewidth}

% \lstset{language=c, numbers=left, showstringspaces=false,
%     moredelim = [s][\ttfamily]{/*}{*/} % I shouldn't need this parameter!
%     }

\renewcommand{\labnumber}{\interruptlabnumber}
\renewcommand{\labname}{Using Interrupt-Driven Input/Output}
\renewcommand{\shortlabname}{interrupt-driven i/o -- interruptlab}
\renewcommand{\collaborationrules}{\interruptlabcollaboration}
\renewcommand{\duedate}{\interruptlabdue}
\newcommand{\nano}{\developmentboard} % TODO: replace \nano with \developmentboard
\renewcommand{\runtimeenvironment}{\interruptlabenvironment}
\pagelayout
\begin{document}
    \labidentifier

    \pdfbookmark[1]{Frontmatter}{frontmatter}                   The purpose of this assignment is to give you more confidence in C programming
and to begin your exposure to the underlying bit-level representation of data.

The instructions are written assuming you will edit and run the code on
\runtimeenvironment. If you wish, you may edit and run the code
in a different environment; be sure that your compiler suppresses no warnings,
and that if you are using an IDE that it is configured for C and not C++.

\section*{Learning Objectives}

After successful completion of this assignment, students will be able to:
\begin{itemize}
    \item Use the ASCII table to determine the corresponding integer values of C
    \lstinline{char} values.
    \item Apply arithmetic operators and comparators to C \lstinline{}{char} values.
    \item Construct and use a bitmask.
    \item Use bitwise operators and bit shift operators to create and modify values.
\end{itemize}

\subsection*{Continuing Forward}

Your experience with viewing values as bit patterns will be applicable in
future labs, as will bit masks and bit operations. Some of the functions you
write in this lab will be used in the next lab.

\section*{During Lab Time}

During your lab period, the TAs will demonstrate how to read the ASCII table and
will provide a refresher on bitwise AND, bitwise OR, and left- and right-shifts.
During the remaining time, the TAs will be available to answer questions.

Before leaving lab, \textit{at a minimum} \dots


    \softwareengineeringfrontmatter

    \section*{Scenario}                                         %\DisdainfulHerb \NumberConversionTool
                                                                \LessDisdainfulHerb \RemoteControlledCar

    \section{Assignment Summary}                                Please familiarize yourself with the entire assignment before beginning.
There are three parts to this assignment.

\subsection{Why are There Letters on Telephone Keypads?}

Once upon a time, telephone exchanges were staffed by operators who would use patch cords on a switchboard to connect callers.
If you needed to make a local-area call to someone whose phone was serviced by a different exchange, then you needed to tell your operator which exchange to connect to.
Letters were assigned to digits so that easy-to-remember -- and audibly-distinctive -- mnemonics could be formed such that the first two letters of the mnemonic that correspond to the 2-digit exchange identifier.
For example, the 86 exchange would use a mnemonic that started with a `T', `U', or `V' and that has as a second letter an `M', `N', or `O' --
so 867--5309 might be ``University 7--5309''.

The presence of letters on telephone dials and (later) telephone keypads allowed for custom phone numbers that used words formed by the available letters.
For example, a bank's phone number might be 472--2265, aka 472--BANK.
Less fictionally, 1--800--FLOWERS was used by a company that partnered with florists to allow people to have bouquets delivered anywhere in the U.S\@.

When the Short Message Service protocol was introduced to allow text-based communication by taking advantage of unused bytes in the handshake between cellular phones and the cell network, naturally the letters that were already present on the keypad were used to tap out messages.

While QWERTY keyboards on smartphones have largely replaced the 10-digit keypad for text entry, the letters remain, waiting for the next clever use\dots

\subsection{Constraints} \label{subsec:constraints}

You may \textit{not} poll the matrix keypad nor the pushbuttons to determine if they have been pressed.
You must use interrupts to determine if a key or button has been pressed.
Once an interrupt has fired, you may scan the matrix keypad or read the pushbuttons to determine which key has been pressed or whether the button has been pressed or released.

You may use any features that are part of the C standard if they are supported by the compiler.
You may use the constants and functions provided in the starter code.

\subsubsection{Constraints on the Arduino core}

You may \textit{not} use any libraries, functions, macros, types, or constants from the Arduino core.

%\subsubsection{Constraints on AVR-libc}
%
%You may use any AVR-specific functions, macros, types, or constants of avr-libc.\footnote{
%    \url{https://www.nongnu.org/avr-libc/user-manual/index.html}
%}

\ifdefstring{\processor}{ATmega328P}{
    \subsubsection{Constraints on AVR-libc}

    You may not use any AVR-specific functions, macros, types, or constants of avr-libc.\footnote{\url{https://www.nongnu.org/avr-libc/user-manual/index.html}}
}{}
\ifdefstring{\processor}{RP2040}{
% TODO: parameterize this (when we eventually port to the bare-metal Arduino toolchain, and to the Pico SDK)
    \subsubsection{Constraints on MBED OS}

    You may not use any functions, macros, types or constants from MBED that are not part of the C standard.\footnote{\url{https://os.mbed.com/docs/mbed-os/v6.16/introduction/index.html}}
}{}

\subsubsection{Constraints on the CowPi library}

You may use any functions provided by the CowPi\footnote{
    \url{https://cow-pi.readthedocs.io/en/latest/library.html}
}
and the CowPi\_stdio\footnote{
    \url{https://cow-pi.readthedocs.io/en/latest/stdio.html}
} libraries,
and you may use any data structures\footnote{
    \url{https://cow-pi.readthedocs.io/en/latest/microcontroller.html}
} provided by the CowPi library.

\subsubsection{Constraints on other libraries}

You may \textit{not} use any libraries beyond those explicitly identified here.


    \section{Getting Started} \label{sec:GettingStarted}        Download the zip file or tarball from \filesource.
Once downloaded, unpackage the file and open the project in your IDE\@.

\subsection{Description of RangeFinder Files}

\subsubsection{combolock.c, interrupt\_support.h, interrupt\_support.cpp, display.h, display.cpp}

Do not edit \textit{combolock.c}, \textit{interrupt\_support.h}, \textit{interrupt\_support.cpp}, \textit{display.}, or \textit{display.cpp}.

These files contain code to simplify interrupt management, and functions to place text on the display module.

\subsubsection{rotary-encoder.h \& rotary-encoder.c}

Do not edit \textit{rotary-encoder.h}

The \textit{rotary-encoder.c} file is where you will process inputs from the rotary encoder.

\subsubsection{servomotor.h \& servomotor.c}

Do not edit \textit{servomotor.h}

The \textit{servomotor.c} file is where you will control the servomotor.

\subsection{lock-controller.h \& lock-controller.c}

Do not edit \textit{lock-controller.h}

The \textit{lock-controller.c} file is where you will implement the logic for the combination lock.

%\subsubsection{shared\_variables.h}
%
%The \textit{shared\_variables.h} header file is where you will place any types that you define and where you will externalize any global variables that need to be used by more than one \textit{.c} file.
%
%It also contains a structure that you may use to access an analog-digital converter's registers,
%and it contains meaningfully-named constants to refer to specific pins you will use in this assignment.


\subsection{Assemble the Hardware}

\textcolor{red}{\textbf{BEFORE YOU PROCEED FURTHER:}}
\begin{description}
    \checkoffitem{Add the new hardware to your Cow~Pi as described in Appendix~\ref{sec:hardwareMods-mk4b}.}
\end{description}


    \section{Cart Controller Specification} \label{sec:spec}    %! suppress = LabelConvention
\begin{enumerate}
    \item \textit{Initial Conditions:} The cart shall initially be stationary (speed 0), facing north (heading 0\textdegree), with the wheels pointing forward, with the brake off, and with the motor disengaged.
    \item \textit{No ``Reverse'' Gear:} The cart shall not be able to travel backwards: its speed must always be a non-negative value.
    \item \textit{Maximum Speed:} The cart has a maximum speed of 3,000~furlongs per fortnight.\footnote{
            A furlong is $\frac{1}{8}$ of a mile, approximately 200m. A fortnight is 14 days.
        }
        To prevent damage to the drivetrain, the system shall not allow the cart to attempt to go faster than that.
    \item \textit{Newton's First Law:} When no keys are pressed, the car shall hold its speed and shall travel in a straight line if its speed is positive.
    \item \label{spec:importantData} \textit{Important Data:} The system will keep track of the carts speed, heading, and distance travelled since the last system reset.
        \begin{enumerate}
            \item The speed will be tracked to a precision of 1~furlong per fortnight.
            \item The heading will be tracked to a precision of 0.000,001\textdegree.
                \begin{itemize}
                    \item The heading is the traditional compass heading, with right turns increasing the heading and left turns decreasing the heading, modulo~360\textdegree.
                \end{itemize}
            \item The distance will be tracked to a precision of 0.001~microfurlongs.
        \end{enumerate}
    \item \label{spec:standardDisplay} \textit{Standard Display:} Under normal operation, the display module shall always display the current speed (in furlongs per fortnight), the current heading (in degrees), and the distance travelled (in microfurlongs) since the system was last reset.
        While the system will store values more precisely, the displayed values shall be \textit{rounded down} to an integer.
        For example, a cart that has travelled 456.4 microfurlongs and is currently travelling 123.0~furlongs per fortnight at a heading of 270.8\textdegree\ would show: \\
        \display{
            \colorbox{LightGreen}{Spd\phantom{x}123fpf\phantom{fxxx}Hdg} \vspace{-1mm}\\
            \colorbox{LightGreen}{Odo\phantom{x}456\textmu fur\phantom{xx}270}
        }
        (This is an example only; no particular layout is specified.)
        \begin{itemize}
            \item The display must show all significant digits of the vehicle's speed, and the speed must be clearly indicated with a label and/or with units.
            \item The display must display at least the lower four decimal digits of the rounded distance travelled, and the distance must be clearly indicated with a label and/or with units.
                The odometer may ``roll over'' at 10,000 microfurlongs or at a greater value if space is available for more digits.
                The odometer must never show a negative distance.
            \item The display must show all significant digits of the vehicle's direction, and the direction must be clearly indicated with a label and/or with a ``degrees'' indicator (`\textdegree').
        \end{itemize}
    \item \label{spec:dPad} \textit{Controlling the Cart:} The \textbf{numeric keypad} shall serve as a \textit{directional pad}.
    \begin{enumerate}
        \item \label{spec:pressOnlyOneKey} The user will press at most one key at a time;
            the user shall never press two or more keys at the same time.
        \item When the user presses the `2' key, the cart shall accelerate if the motor is engaged.
            The increase in speed is per-press.
            Continuously holding the `2' key shall not increase the speed further;
            the only way to increase the speed further is to press the `2' key again.
            \begin{itemize}
                \item The cart shall not accelerate when the motor is disengaged.
                \item If the user attempts to accelerate to a speed faster than the maximum speed, then the speed shall be the maximum speed.
            \end{itemize}
        \item When the user presses the `8' key, the cart shall decelerate.
            The decrease in speed is per-press.
            Continuously holding the `8' key shall not decrease the speed further;
            the only way to decrease the speed further is to press the `2' key again.
            \begin{itemize}
                \item If the user attempts to decelerate to a negative speed, then the speed shall be 0.
            \end{itemize}
        \item When the user presses the `0' key, the cart shall brake.
            Braking results in the speed immediately being set to 0, and is also required to engage and disengage the motor.
        \item \label{spec:turnLeft} When the user presses the `4' key, the cart shall turn left.
            The cart shall continue to turn left while the user continues to press the `4' key.
            The turn is complete when the user releases the `4' key.
        \item \label{spec:turnRight} When the user presses the `6' key, the cart shall turn right.
            The cart shall continue to turn right while the user continues to press the `6' key.
            The turn is complete when the user releases the `6' key.
        \item Pressing any other key shall have no effect.
    \end{enumerate}
    \item \textit{Engaging/Disengaging the Motor:} The \textbf{left switch} shall be used to engage and disengage the motor.
        \begin{enumerate}
            \item The motor shall only change between being engaged and being disengaged as a result of the left switch's position when the user releases the brake (the `0' key).
            \item When the user releases the brake, if the left switch is in the \textit{left position} then the motor shall disengage.
            \item When the user releases the brake, if the left switch is in the \textit{right position} then the motor shall engage.
        \end{enumerate}
    \item \textit{Gear Selector:} The \textbf{right switch} shall be used to change between \textit{low gear} and \textit{high gear}.
        \begin{enumerate}
            \item When the right switch is in the \textit{left position}, the cart shall be in \textit{low gear}.
                Pressing the `2' button shall cause the cart to accelerate by 1~furlong per fortnight.
                Pressing the `8' button shall cause the cart to decelerate by 1~furlong per fortnight.
            \item When the right switch is in the \textit{right} position, the cart shall be in \textit{high gear}.
                Pressing the `2' button shall cause the cart to accelerate by 3~furlongs per fortnight.
                Pressing the `8' button shall cause the cart to decelerate by 3~furlongs per fortnight.
        \end{enumerate}
    \item \label{spec:TurnSignalControls} \textit{Turn Signal Controls:} The \textbf{pushbuttons} shall serve as turn signal selectors.
        \begin{itemize}
            \item The user shall indicate their preference to turn left by pressing the \textbf{left pushbutton};
                the user shall indicate their preference to turn right by pressing the \textbf{right pushbutton}.
        \end{itemize}
    \item \label{spec:TurnSignals} \textit{Turn Signals:} One function of the \textbf{LEDs} is as turn signals.
        \begin{enumerate}
            \item If the user has indicated a preference to turn left, then the \textbf{left LED} shall blink;
                if the user has indicated their preference to turn right, then the \textbf{right LED} shall blink.
            \item \label{spec:blinkRate} The blink rate shall be repeating 750ms on, 750ms off.
            \item The turn signal shall continue to blink until the user has completed the turn.
                \begin{itemize}
                    \item Per Requirements~\ref{spec:turnLeft}--\ref{spec:turnRight}, releasing the `4' or `6' key shall cause a blinking turn signal to stop blinking
                \end{itemize}
            \item An LED shall not blink when its turn signal has not been selected.
        \end{enumerate}
    \item \textit{Brake Lights:} The other function of the \textbf{LEDs} is as brake indicators.
        \begin{enumerate}
            \item If no turn signal has been selected, then both LEDs shall illuminate steadily for as long as the user is braking.
            \item If a turn signal has been selected, then the corresponding LED shall blink, and the LED for the non-selected direction shall illuminate steadily.
            \item If the user is not braking, then neither LED shall illuminate steadily.
        \end{enumerate}
    \item \textit{Turn Rate:} While the `4' or `6' key is pressed, the cart shall change its heading at the rate of $\pm$0.3\textdegree~per microfurlong.
        \begin{itemize}
            \item For example, if the cart is travelling at the rate of 500~furlongs per fortnight, then it travels 0.5~microfurlongs each nanofortnight, and so a right turn would result in a heading change of 0.15~degrees.
            \item If the cart is stationary (the speed is 0), then pressing the `4' or `6' key would not cause the cart to change its heading.
        \end{itemize}
    \item \label{spec:dataRefreshRate} \textit{Data Refresh Rate:} Every nanofortnight,\footnote{
            A nanofortnight is 1.2096ms. We will approximate this as \textbf{1.208ms}, an error under 0.15\%, resulting in a loss of fewer than 30 minutes over a fortnight.
        } the system shall update the distance that the cart has travelled and its heading.
    \item \label{spec:displayRefreshRate} \textit{Display Refresh Rate:} The system shall update the \textbf{display module} evey 256 nanofortnights.
    \item \label{spec:diagnosticDisplays} \textit{Diagnostic Displays:} The letter keys can be used to show diagnostic data.
        The display module shall be refreshed at the same rate specified in Requirement~\ref{spec:displayRefreshRate}.
        \begin{enumerate}
            \item When the user presses the `A' key, the system shall display the standard display described in Requirement~\ref{spec:standardDisplay}.
            \item When the user presses the `B' key, the system shall display the full variable used to track the cart's speed.
                Other diagnostic data may also be displayed.
            \item When the user presses the `C' key, the system shall display the full variable used to track the cart's distance.
                Other diagnostic data may also be displayed.
            \item When the user presses teh `D' key, the system shall display the full variable used to track the cart's direction.
                Other diagnostic data may also be displayed.
        \end{enumerate}
    \item \label{spec:responsive} The system shall always be responsive to user input.
        \begin{itemize}
            \item While one pushbutton is being pressed, the system does not need to respond to the other being pressed
            \item As noted in Requirement~\ref{spec:pressOnlyOneKey}, the user will never press two keys at the same time
            \item But for those exceptions, there shall be no noticeable lag when responding to an input
        \end{itemize}
\end{enumerate}

    \section{Turn Signals} \label{sec:turnSignals}              During your lab period, the TAs will guide the class through the first modifications to the starter code that you must make, described in this section.
If you do not attend your lab period, then you must complete this section on your own.
\textbf{\textit{Except during lab time, you may }not\textit{ discuss the solutions for this section with other students.}}


\subsection{Examining the Starter Code}

\subsubsection{turn\_signals.h}

A not-uncommon pattern in embedded systems is to design the system as a state machine, or as multiple state machines.
The states are represented as the value of an enumerated type.
A couple of key advantages of writing the system as a state machine or as a collection of state machines is that it's easy to change the state based on inputs, and it's easy for the parts of the system that take action to do so based on what state the state machine is in.

In \textit{turn\_signals.h} you'll see \lstinline{enum turn_directions} which enumerates the states of a small state machine.
You'll also see the variable \lstinline{indicated_turn_direction} which will store the state of that small state machine.
In the header file, the variable is modified with the \lstinline{extern} keyword, which can be thought of as causing the \lstinline{indicated_turn_direction} variable in \textit{turn\_signals.c} to be ``public'' so that the variable can be modified by code in both \textit{turn\_signals.c} and \textit{cart\_controller.c}.


\subsubsection{turn\_signals.c}

In \textit{turn\_signals.c} you'll see:
\begin{itemize}
    \item \function{initialize_turn_signals()} where you'll put code to set variables' initial values and to configure interrupts
    \item \function{handle_button_action()} where you'll put the code that should run whenever a pushbutton is pressed
    \item \function{ISR(TIMER1_COMPA_vect)} where you'll put the code that needs to run periodically
\end{itemize}


\subsection{Detecting Button Presses}

Requirement~\ref{spec:TurnSignalControls} states that the user will indicate the direction they want to turn by pressing the appropriate pushbutton to activate the turn signal.

Read Section~5 of the Cow Pi datasheet for an overview of handling interrupts for your \developmentboard.

To detect button presses (and key presses), you will take advantage of having the NAND of the pushbuttons and the NAND of the keypad columns being input to Arduino pins D2 and D3.

Add code to \function{initialize_turn_signals()} to register \function{handle_button_action()} as your handler for button presses.
Use the \function{attachInterrupt()} and \function{digitalPinToInterrupt()} functions, as described in Section~5.3 of the datasheet.
Specify \lstinline{CHANGE} mode;
we shall explain why in Section~\ref{subsubsec:debouncing}.

Your code to respond to button presses will go in the \function{handle_button_action()} function.

\begin{lstlisting}[basicstyle=\small]
void handle_button_action(void) {
  debounce_interrupt({
    static int8_t number_of_buttons_pressed = -1;
    bool left_button_is_pressed, cowpi_right_button_is_pressed;
    while ((left_button_is_pressed = cowpi_left_button_is_pressed())
             + (right_button_is_pressed = cowpi_right_button_is_pressed())
           == number_of_buttons_pressed) {}   // busy-wait through the race condition
    /* Add code to respond to button presses */

    number_of_buttons_pressed = left_button_is_pressed + right_button_is_pressed;
  });
}
\end{lstlisting}

\subsubsection{Debouncing} \label{subsubsec:debouncing}

When you place your code that responds to button presses in \function{handle_button_action()}, place it inside the braces for the \function{debounce_interrupt()} macro.
Just as the \function{debounce_byte()} function took care of debouncing when polling mechanical inputs, the \function{debounce_interrupt()} macro defined in \textit{supplement.h} takes care of (most of) the debouncing when mechanical inputs generate interrupts.
There are two catches.

The first catch is that \function{debounce_interrupt()} only works when the interrupt handler responds to both the rising and falling edges of the input.
This is why you must register \function{handle_button_action()} using the \lstinline{CHANGE} mode.
This isn't such a terrible limitation:
Determine whether the interrupt is due to a button being pressed or released.
If pressed, then comply with Requirement~\ref{spec:TurnSignalControls};
if released, then do nothing.

The second catch is that \function{debounce_interrupt()} works by ignoring interrupts that are generated due to switchbounce.
It does not -- it cannot -- ignore switchbounce that occurs while the interrupt handler is executing.
If there were only one device attached to the input pin, then switchbounce that occurs while the interrupt handler is executing would not be a problem:
if you have a \lstinline{static bool} variable to track whether the \textit{previous} interrupt was due to the button being pressed or released, then you know that the \textit{current} interrupt is due to the button moving in the other direction.

As it is, there are two pushbuttons that could cause the interrupt, and to respond correctly, you must determine \textit{which} pushbutton was pressed by examining the inputs.
There is, however, a race condition.
Consider this scenario: the user presses a button, closing the contacts, triggering an interrupt, and the interrupt handler launches.
Before reaching the code that reads the inputs to determine which button was pressed, the switch bounces, temporarily re-opening the contacts;
when the code reads the inputs, neither button appears to be pressed.
The fix is to introduce a loop that iterates until a sensible input reading is found.
This busy-wait loop is already in the starter code.

\subsubsection{Design the Interrupt Handler}

After the busy-wait loop terminates, use the \lstinline{left_button_is_pressed} and \\ \lstinline{cowpi_right_button_is_pressed} variables to assign the appropriate state to \\ \lstinline{indicated_turn_direction}.
The code in the pushbutton interrupt handler should \textit{only} make the changes that are due to changes in the pushbuttons' positions.


\subsection{Detecting Time}

Read Section~5.2 and Section~6 of the Cow Pi datasheet to familiarize yourself with timer interrupts on the \developmentboard.
Note that Sections~6.3--6.5 are generally duplicates of each other except for their tables and a warning at the start of Section~6.3;
for you now can read Sections 6.1, 6.2, 6.4, and 6.6.
Later, you can re-visit Sections~6.3 and 6.5.

You must use a timer interrupt, along with the turn signal's state, to implement Requirement~\ref{spec:TurnSignals}.
Requirement~\ref{spec:blinkRate} identifies the timer period that you need.
Use the equation in the datasheet's Section~6.2 to iterate through possible comparison values and prescalers until you arrive at a prescaler that is available for Timer1 and a comparison value that fits in the available counter bits for that timer.
When you have done so, add code to \function{initialize_turn_signals()}:
\begin{itemize}
    \item Create a \lstinline{cowpi_timer16bit_t} pointer (in the remaining steps, we will refer to this pointer as \lstinline{timer})
    \item Use the address offset listed in the datasheet's Section~6.4 to assign the appropriate address to \lstinline{timer}
    \item Use the datasheet's Tables~9--10 to select the WGM bits for ``Clear on Timer Compare'' mode and the CS bits for your prescaler;
        use these bits and Table~8 to generate a bit vector and assign that bit vector to \lstinline{timer->control}
    \item Subtract 1 from the comparison value and assign that to \lstinline{timer->compareA}
    \item Create a \lstinline{uint8_t} pointer (which we will refer to as \lstinline{timer_interrupt_masks}), and use the address offset listed in the datasheet's Section~6.6 to assign the appropriate address to \lstinline{timer_interrupt_masks}
    \item Treating the pointer as an array, index the \lstinline{timer_interrupt_masks} array with the timer number (1), and use the datasheet's Table~14 to assign the appropriate bit vector to enable the \texttt{TIMER\textit{n}\_COMPA\_vect} interrupt, where $n$ is the timer number
%    \item Create an ISR for the \lstinline{TIMER1_COMPA_vect} interrupt using the \function{ISR()} macro as described in Section~5.2 of the datasheet
\end{itemize}

You can now add code to the ISR for \lstinline{TIMER1_COMPA_vect} to implement Requirement~\ref{spec:blinkRate}.
Based on \lstinline{indicated_turn_direction}'s state, either the left LED should blink, or the right LED should blink, or neither should blink.
\ifbool{offerdecompositionhints}{
    Introducing a \lstinline{static} variable to your ISR may be helpful.
}{}


\vspace{1cm}

You are now ready to complete the remainder of this assignment on your own.
Reminders:
\begin{itemize}
    \item Declare any global variables you use as \lstinline{volatile}.
    \item \collaborationrules
\end{itemize}

    \section{Controlling the Cart} \label{sec:cartController}   \subsection{Examining the Starter Code}

\subsubsection{Starter Code Functions and Variables}

In \textit{cart\_controller.c} you'll see:
\begin{itemize}
    \item \function{initialize_cart_system()} where you'll put code to set variables' initial values and to configure interrupts
    \item \function{control_cart()} where you'll put code that should repeated execute as part of the system's main loop
    \item \function{handle_key_action()} where you'll put the code that should run whenever a key on the keypad is pressed or released
    \item \function{ISR(TIMER2_COMPA_vect)} where you'll put the code that needs to run periodically
\end{itemize}

You may, of course, add other helper functions to \textit{cart\_controller.c} as you see fit.

You will also see a global \lstinline{volatile} variable, \lstinline{display_needs_refreshed} that you can use as part of implementing Requirement~\ref{spec:displayRefreshRate}.
There are also two \lstinline{const} variables, \lstinline{MU} and \lstinline{DEGREE} that can be used if you wish to include `\textmu' or `\textdegree' as part of your display (you are  not required to use these characters, but they are available if you wish to do so).

Recall also that \textit{turn\_signals.h} has an \lstinline{extern} variable, \lstinline{indicated_turn_direction} that is shared between \textit{turn\_signals.c} and \textit{cart\_controller.c}.

You \textit{will} need to add additional global variables to pass information between \function{control_cart()}, \function{handle_key_action()}, and \function{ISR(TIMER2_COMPA_vect)}.
Be sure to declare any global variables that you introduce as \lstinline{volatile}.

\subsubsection{Determining Your Display Module's ROM Code}

You can skip this subsection if you will not use \lstinline{MU} or \lstinline{DEGREE}.

The LCD1602's datasheet\footnote{
    \url{https://www.sparkfun.com/datasheets/LCD/HD44780.pdf}
} describes the character sets for two possible ROMs, coded ``A00'' and ``A02''.\footnote{
    No, I don't know what happened to ROM A01.
}
Both character sets include the ASCII printable characters, but they have different characters for the non-ASCII values.
Initially, the \function{initialize_cart_system()} function has this code:
\begin{lstlisting}[basicstyle=\small]
void initialize_cart_system(void) {
  display_needs_refreshed = true;
  /* USE THIS CODE TO DETERMINE WHICH ROM CODE YOUR DISPLAY MODULE HAS, AND
     SET MU and DEGREE ACCORDINGLY ON LINE 55-61. THEN COMMENT-OUT THE NEXT
     EIGHT LINES AFTER THIS COMMENT */
  char label_string[] = "mu        degree";
  char candidates_string[17] = {0};
  if (display_needs_refreshed) {
    sprintf(candidates_string, "0:%c 2:%c  0:%c 2:%c", (char)0xE4, (char)0xB5, (char)0xDF, (char)0xB0);
    display_needs_refreshed = false;
    display_string(label_string, TOP_ROW);
    display_string(candidates_string, BOTTOM_ROW);
  }
  /* END OF CODE TO DETERMINE YOUR DISPLAY MODULE'S ROM CODE */

}
\end{lstlisting}
Depending on which ROM your display module has, your display module will initially show either

ROM Code 0 \\
\texttt{
    \colorbox{LightGreen}{mu\phantom{xxxxxxxxx}degree} \vspace{-1mm}\\
    \colorbox{LightGreen}{0:\textmu\phantom{x}2:\begin{CJK*}{UTF8}{goth}ォ\end{CJK*}\phantom{xx}0:\textdegree\phantom{x}2:-}
} \\
or

ROM Code 2 \\
\texttt{
    \colorbox{LightGreen}{mu\phantom{xxxxxxxxx}degree} \vspace{-1mm}\\
    \colorbox{LightGreen}{0:{\"a}\phantom{x}2:\textmu\phantom{\small{xxx}}0:{\ss}\phantom{x}2:\textdegree}
}

If you have the display corresponding to ROM Code 0, then make sure that lines~57--58 are uncommented and that lines~60--61 are commented-out:
\begin{lstlisting}[numberstyle=\color{gray}, numbers=left, firstnumber=55]
// A couple of handy non-ASCII characters if you want them for your display
// ROM CODE 0
const char MU = (char)0xE4;
const char DEGREE = (char)0xDF;
// ROM CODE 2
// const char MU = (char)0xB5;
// const char DEGREE = (char)0xB0;
\end{lstlisting}
On the other hand, if you have the display corresponding to ROM Code 2, then make sure that lines~57--58 are commented-out and that lines~60--61 are uncommented:
\begin{lstlisting}[numberstyle=\color{gray}, numbers=left, firstnumber=55]
// A couple of handy non-ASCII characters if you want them for your display
// ROM CODE 0
// const char MU = (char)0xE4;
// const char DEGREE = (char)0xDF;
// ROM CODE 2
const char MU = (char)0xB5;
const char DEGREE = (char)0xB0;
\end{lstlisting}
If you have neither, then contact the professor for the code to create custom characters for `\textmu' and `\textdegree' (if you plan to use these characters on your display).

After you have done so, comment-out the eight lines in \function{initialize_cart_system()} that generate that display.


\subsection{Detecting Key Presses and Releases}

Add code to \function{initialize_cart_system()} to register \function{handle_key_action()} as your handler for key presses and releases.
Use the \function{attachInterrupt()} and \function{digitalPinToInterrupt()} functions, as described in Section~5.3 of the datasheet.
Specify \lstinline{CHANGE} mode.

The starter code for \function{handle_key_action()} has debouncing code similar to what you saw in \function{handle_button_action()}.
After the busy-wait loop terminates, the \lstinline{key} variable will either hold a non-NUL character, which will tell you which key press triggered this interrupt,
or \lstinline{key} will hold `\\0', in which case the \lstinline{last_key} variable will tell you which key release triggered this interrupt.

\begin{lstlisting}[basicstyle=\small]
void handle_key_action(void) {
  debounce_interrupt({
    static char last_key = 0xF0;
    char key;
    while ((key = cowpi_get_keypress()) == last_key) {}   // busy-wait through the race condition
    /* Add code to respond to key presses and releases */

    last_key = key;
  });
}
\end{lstlisting}

As described in Requirements~\ref{spec:dPad} and \ref{spec:diagnosticDisplays}, your system needs to respond to key presses for some (but not all) of the keys on the numeric keypad,
and it needs to respond to key releases for only a few of the keys.
Until you design the logic for your cart controller, you might simply print a message to the console when these particular actions happen.

As a rule, you want your interrupt-handling code to run quickly.
You certainly don't want to take the time to send a message across the slow UART connection to the Arduino IDE's Serial Monitor.
When you introduce the logic for your cart controller, be sure to remove the print statements from your interrupt handler.

\subsection{Detecting Time}

You must use a timer interrupt, to implement Requirement~\ref{spec:dataRefreshRate}.
Use that same timer interrupt\ifbool{offerdecompositionhints}{, along with a \lstinline{static} variable the ISR,}{} to determine when to set the \lstinline{display_needs_refreshed} variable.

Requirement~\ref{spec:dataRefreshRate} identifies the timer period that you need.
Use the equation in the datasheet's Section~6.2 to iterate through possible comparison values and prescalers until you arrive at a prescaler that is available for Timer2 and a comparison value that fits in the available counter bits for that timer.
When you have done so, add code to \function{initialize_cart_system()}:
\begin{itemize}
    \item Create a \lstinline{cowpi_timer8bit_t} pointer (in the remaining steps, we will refer to this pointer as \lstinline{timer})
    \item Use the address offset listed in the datasheet's Section~6.5 to assign the appropriate address to \lstinline{timer}
    \item Use the datasheet's Tables~12--13 to select the WGM bits for ``Clear on Timer Compare'' mode and the CS bits for your prescaler;
    use these bits and Table~11 to generate a bit vector and assign that bit vector to \lstinline{timer->control}
    \item Subtract 1 from the comparison value and assign that to \lstinline{timer->compareA}
    \item Create a \lstinline{uint8_t} pointer (which we will refer to as \lstinline{timer_interrupt_masks}), and use the address offset listed in the datasheet's Section~6.6 to assign the appropriate address to \lstinline{timer_interrupt_masks}
    \item Treating the pointer as an array, index the \lstinline{timer_interrupt_masks} array with the timer number (2), and use the datasheet's Table~14 to assign the appropriate bit vector to enable the \texttt{TIMER\textit{n}\_COMPA\_vect} interrupt, where $n$ is the timer number
%    \item Create an ISR for the \lstinline{TIMER1_COMPA_vect} interrupt using the \function{ISR()} macro as described in Section~5.2 of the datasheet
\end{itemize}

Until you design the logic for your cart controller, you might simply leave the ISR's body empty except for the code that updates \lstinline{display_needs_refreshed}.


\subsection{Updating the Display}

As a rule, you want your interrupt-handling code to run quickly.
You certainly don't want to take the time to update the display module in your ISR\@.
Instead, include in your ISR enough code to determine when 256 nanofortnights have passed,
\ifbool{offerdecompositionhints}{
    that is, when the timer interrupt has fired 256 times,
}{}
and set the \lstinline{display_needs_refreshed} when that has happened.

Then, in the \function{control_cart()} function, find this code:
\begin{lstlisting}[basicstyle=\small]
  static char top_line[17];
  static char bottom_line[17];
  if (display_needs_refreshed) {
    display_needs_refreshed = false;
    /* Add to create the strings to be displayed */

  }
\end{lstlisting}

Place inside that \lstinline{if} statement's body any code that should run whenever the display needs to be updated.
You will probably want to use \function{sprintf()} to generate the strings to be displayed.
See Appendix~\ref{sec:formatStrings} for some discussion.

\subsection{Controlling the Cart}

Instead of hacking code together, you should first think about its design.
You might not get the design perfect initially, but you will have much less backtracking if you give thought to it first.

\subsubsection{Data Representations}

Consider, for example, Requirement~\ref{spec:importantData}.
Finding a suitable datatype for the speed should be straight-forward enough.
What about the heading and distance?
How can you maintain a precision of \textit{exactly} 0.000,001\textdegree and of \textit{exactly} 0.001~microfurlongs?

\ifbool{offerdecompositionhints}{
    An IEEE~754 binary floating point type will not allow you to maintain precisions of exactly 0.000,001\textdegree and 0.001~microfurlongs.
    An attempt to approximate those precisions will result in the display frequently showing the incorrect distance travelled -- admittedly for only 256 nanofortnights each time this happens, but you can do better.
    (Another problem is that, by default, \function{sprintf()} will not work with floating point values on microcontrollers, including the ATmega328P.
    We can change this with a compiler argument, but the Arduino IDE doesn't make it easy to do that.)
    %\paragraph{Hint} Review your notes from the part of the floating point lecture in which we discussed how decimal currency is handled. \\
    \paragraph{Hint} Review the ``Decimal Fixed Point Value Hack'' slides from the floating point lecture, and think about how to use integer types to represent 0.001~microfurlongs and 0.000,001\textdegree.
    \paragraph{Hint} There are 1,000~nanofurlongs in a microfurlong, and there are 1,000,000~microdegrees in a degree. \\
}{}

How will you keep track of whether the motor is engaged?
How will you keep track of what direction (if any) the cart is turning?
How will you keep track of which display should be displayed (see Requirement~\ref{spec:diagnosticDisplays})?

\subsubsection{Design the Displays}

Design the diagnostic displays that show the full variables you chose.
Also design the ``normal'' display that shows the speed in furlongs per fortnight, the distance in microfurlongs, and the heading in degrees.
Be sure to allow sufficient room on the ``normal'' display to show a speed up to 3,000~furlongs per fortnight, a distance up to 9,999~microfurlongs, and a heading up to 359\textdegree.

Place the code to select the display in the \function{handle_key_action()} function, and the code to generate the displays and send them to the display module in the \function{control_cart()} function.

\subsubsection{Speed}

Now think about how you will change the speed.
How will you handle the minimum and maximum speeds?
How will you handle low and high gear?
How will you handle the motor being engaged or not engaged?

What has to happen when you press the `0' key to brake?
What else has to happen when you press the brake?
What has to happen when you release the brake?
What else has to happen when you release the brake?

\ifbool{offerdecompositionhints}{
    \paragraph{Hint} The requirements state that the brake lights shall be illuminated while as the user is pressing the brake.
    You can correctly infer that once the user presses the brake, they continue to do so until they release the brake. \\
}{}

Now that you've given thought to how to control the cart's speed, write the code for accelerating, decelerating, and braking.

\subsubsection{Distance Travelled}

Now that you can change the cart's speed, you can change the distance travelled.
Add code to \function{ISR(TIMER2_COMPA_vect)} that will update the distance travelled whenever the timer interrupt fires.

\ifbool{offerdecompositionhints}{
    \paragraph{Hint} Assuming you designed the timer interrupt to fire every nanofortnight:
    \begin{align*}
        Change\_in\_distance    & = speed \times time \\
                                & = \mathtt{speed\_variable}\ \frac{furlong}{fortnight} \times 1\ nanofortnight \\
                                & = \mathtt{speed\_variable}\ nanofurlong \\
                                & = \mathtt{speed\_variable} \times 0.001\ microfurlong
    \end{align*}
}{}

\subsubsection{Turning}

While activating a turn signal indicates the intention to turn a particular direction, actually turning requires using the ``D-pad'' to indicate the turn direction and forward motion to determine how much the direction changes.
This means that the code for making turns will be spread across a couple of functions.

Don't forget that the turn signal should stop blinking when the user ends the turn.

\subsubsection{Final Review of Requirements}

Look again over the requirements in Section~\ref{sec:spec}.
Have you forgotten anything?

Check to make sure that you've removed any long-running code, such as \function{display_string()} and \function{printf()} from your interrupt handlers / ISRs.


    \section{Turn-in and Grading}                               \filesubmission

\policyforcodethatdoesnotcompile

\latepolicy

\subsection*{Rubric}

This assignment is worth 20 points.
\begin{description}
    \rubricitem{4}{\textit{problem1.c} produces the specified output.}
    \rubricitem{4}{\function{iz_digit()} in \textit{problem2.c} determines whether
    or not a character is a digit.}
    \rubricitem{4}{\function{decapitalize()} in \textit{problem2.c} converts
    uppercase letters to lowercase and leaves other characters unchanged.}
    \rubricitem{4}{\function{is_even()} in \textit{problem3.c} determines whether
    a number is even or odd.}
    \item[\hspace{1cm}]\function{produce_multiple_of_ten()} in \textit{problem3.c}
    has the following:
    \begin{description}
        \rubricitem{1}{Code to assign the value 5 to the variable \lstinline{five}}
        \rubricitem{1}{Code to divide an even number by 2}
        \rubricitem{1}{Code to subtract 1 from an odd number}
        \rubricitem{1}{Correct functionality}
    \end{description}
    \item[Penalties]
    \penaltyitem{4}{for each solution that depends on a prohibited character.}
    \penaltyitem{4}{for each solution that hard-codes a return value instead of attempting to solve the specified problem}
    \softwareengineeringpenalties
\end{description}


    \section*{Epilogue}                                         \LauraDern

    \textit{To be continued...}

%    \newpage\appendix
    \appendix

    \section{Appendix: Lab Checkoff}                            You are not required to have your assignment checked-off by a TA or the professor.
If you do not do so, then we will perform a functional check ourselves.
In the interest of making grading go faster, we are offering a small bonus % to get your assignment checked-off at the start of your scheduled lab time immediately after it is due.
% Because checking off all students during lab would take up most of the lab time, we are offering a slightly larger bonus
if you complete your assignment early and get it checked-off by a TA or the professor during office hours.

\subsection*{TODO}

%\begin{enumerate}
%    \precheckoffitem{Position your Cow Pi's storage box upright, a little more than 1~meter from the Cow Pi.}
%    \precheckoffitem{Place both switches in the left position.}
%    \precheckoffitem{Upload your code to your Cow Pi, and leave your code open in the IDE.}
%    \precheckoffitem{Confirm that the system detects the box and not something closer (such as a computer or the table surface).}
%
%    \checkoffitem{Show and explain to the TA how your code generates a tone with a frequency of 5kHz; that is, it has a period of 200\textmu s.}
%    \checkoffitem{Place the right switch in the right position, putting the system in Continuous Tone mode.
%        The system generates a continuous 5kHz tone.}
%    \item[] (TA, confirm that the tone is 5kHz by code inspection and by ear; confirm with the HuskerScope spectrum analyzer if you aren't sure.) \\
%        \textit{+1 There is code to generate an audible tone} \\
%        \textit{+1 The system continuously generates the tone when in Continuous Tone mode} \\
%        \textit{+2 The audible tone has a frequency of 5kHz}
%
%    \checkoffitem{Place the left switch in the right position, putting the system in Threshold Adjustment mode.
%        The system prompts for a new threshold range. \\
%        \textit{+1 The user is prompted to enter a new threshold range when the system is in Threshold Adjustment mode}}
%    \checkoffitem{Enter a range of 25, using the `\#' key to indidicate that you have fininished entering the value.
%        The system displays a helpful error message explaining that this is not a valid threshold range.
%        The system then prompts the user for a new threshold range. \\
%        \textit{+1 The user is given a helpful error message after entering an invalid threshold range} \\
%        \textit{+1 The user is re-prompted to enter a threshold range after entering an invalid threshold range}}
%    \checkoffitem{Enter a range of 450, using the `\#' key to indidicate that you have fininished entering the value.
%        The system displays a helpful error message explaining that this is not a valid threshold range.
%        The system then prompts the user for a new threshold range.}
%    \checkoffitem{Enter a range of 75, using the `\#' key to indidicate that you have fininished entering the value.
%        The system displays a message confirming the new threshold range. \\
%        \textit{+2 Valid threshold ranges are those between 50cm and 400cm, inclusive} \\
%        \textit{+1 The user is shown a confirmation message after entering a valid threshold range} \\
%        \textit{+2 The user can enter a new threshold range when the system is in Threshold Adjustment mode}}
%
%    \checkoffitem{Place the right switch in the left position, putting the system in Single Pulse mode.
%        The system might indicate that no object has been detected yet; however, this is not required information before initiating a ping.}
%    \checkoffitem{Show and explain to the TA how your code initiates a pulse.}
%    \checkoffitem{Show and explain to the TA how your code measures the length of a pulse.}
%    \checkoffitem{Show and explain to the TA how your code achieves the required precision (no greater than 1\textmu s) and accuracy (immediately detect pulse edges without waiting for code in the main loop to poll the pin).}
%    \checkoffitem{Press the pushbutton to initiate a pulse.
%        The right LED strobes once.
%        The piezodisc does not chirp.
%        The system displays the correct distance to the wall (or book or other object). \\
%        \textit{+2 There is code to initiate an ultrasound pulse} \\
%        \textit{+3 There is code to detect the length of the pulse} \\
%        \textit{+3 The pulse's length is measured to a precision of no greater than 1\textmu s} \\
%        \textit{+3 The pulse's length is measured as accurately as possible} \\
%        \textit{+2 The user can request a ping when the system is in Single Pulse mode} \\
%        \textit{+2 The distance to an object is correctly calculated from the pulse's length} \\
%        \textit{+2 When an object is detected, the system displays the distance to the object} \\
%        \textit{+2 When an object is detected in Single-Pulse mode, the system generates exactly one alarm} \\
%        \textit{+2 A strobe is an illumination of the right LED for 50ms} \\
%        \textit{+2 A strobe occurs for any detected object}}
%    \checkoffitem{Slightly change the distance between the Cow Pi and the target object.
%        Press the pushbutton to initiate a pulse.
%        The LED strobes once.
%        The piezodisc does not chirp.
%        The system displays the new distance to the wall (or book or other object). \\
%        \textit{+2 The user can request another ping when the system is in Single Pulse mode}}
%
%    \checkoffitem{Place the left switch in the left position, putting the system in Normal Operation mode.
%        The system displays the distance to the target object, and it displays an approach rate of 0cm/s.
%        The LED strobes once per seccond (100ms), but the piezodisc does not chirp. \\
%        \textit{+1 The switches control the mode of operation as specified} \\
%        \textit{+2 When an object is detected in Normal Operation mode, the system repeatedly generates alarms}}
%    \checkoffitem{Slowly move the Cow Pi closer to the wall, or slowly move the book (or other object) closer to the Cow Pi.
%        As you do so, vary the rate of approach slighly to demonstrate that the rate of approach updates.
%        The displayed distance changes with the decreasing distance to the target object.
%        The system displays a plausible, positive rate of approach that updates at least once every second. \\
%        \textit{+2 When an object is detected in Normal Operation mode the rate of approach is displayed} \\
%        \textit{+2 When an object is detected in Normal Operation mode, the rate of approach is updated at least once every second}}
%    \checkoffitem{As the distance between the Cow Pi and the target object decreases, note that:
%        \begin{description}
%            \item[When the distance falls below 100cm] the LED strobes more frequently, once every 750ms (\textthreequarters sec)
%            \item[When the distance falls below 75cm] the piezodisc chirps every 750ms
%            \item[When the distance falls below 50cm] the LED strobes, and the piezodisc chirps, every 500ms (\textonehalf sec)
%            \item[When the distance falls below 25cm] the LED strobes, and the piezodisc chirps, every 250ms (\textonequarter sec)
%            \item[When the distance falls below 10cm] the LED strobes, and the piezodisc chirps, every 125ms ($\frac{1}{8}$ sec)
%        \end{description}
%        \textit{+2 A chirp is an audible tone lasting 50ms} \\
%        \textit{+2 When the system repeatedly generates alarms, the time between alarms is as specified}}
%
%    \checkoffitem{Place the both switches in the right position, putting the system in Threshold Adjustment mode.
%        The system prompts for a new threshold range.}
%    \checkoffitem{Enter a range of 55.
%        The system displays a message confirming the new threshold range.}
%    \checkoffitem{Place the both switches in the left position, putting the system in Normal Operation mode.
%        The system displays the distance to the target object, and it displays an approach rate of 0cm/s.}
%    \checkoffitem{Slowly move the Cow Pi away from the wall, or slowly move the book (or other object) away from the Cow Pi.
%        The displayed distance changes with the decreasing distance to the target object.
%        The system displays a plausible, negative rate of approach.}
%    \checkoffitem{As the distance between the Cow Pi and the target object decreases, the alarms become less urgent.
%        Note that as the distance increases above 55cm, the piezodisc stops chirping but the LED continues to strobe. \\
%        \textit{+2 A chirp occurs for a detected object that is closer than the threshold range} \\
%        \textit{+2 A chirp only occurs for a detected object that is closer than the threshold range}}
%
%    \checkoffitem{Reorient the Cow Pi, or remove the book (or other object) so that there are no in-range objects to detect.
%        The system displays a message indicating that no object is detected.
%        The LED does not strobe, and the piezodisc does not chirp.\\
%        \textit{+3 The code correctly recognizes the that no object has been detected, if no object reflects the ultrasound pulse} \\
%        \textit{+2 When there is no in-range object, the system displays a message to that effect} \\
%        \textit{+2 A strobe only occurs for a detected object}}
%
%    \checkoffitem{Show the TA any code they have not yet examined. \\
%        \textit{+1 The code is clean, well-organized, has good variable and function names, and is otherwise understandable}}
%\end{enumerate}
%
%
%\begin{description}
%    \precheckoffitem{Establish that the code you are demonstrating is the code
%    you submitted to to \filesubmission.}
%    \begin{itemize}
%        \item If you are getting checked-off during lab time, show the TA that the
%        file was submitted before it was due.
%        \item Download the file into your ComboLab directory. If necessary,
%        rename it to \textit{ComboLab.ino}.
%    \end{itemize}
%    \precheckoffitem{Upload \textit{ComboLab.ino} to your \developmentboard\ and open the
%    Serial Monitor.}
%\end{description}
%
%\begin{enumerate}
%    \checkoffitem{The combination screen is displayed
%    \texttt{\phantom{88}-\phantom{88}-\phantom{88}}) with the cursor (two
%    decimal points) blinking in the left-most position. No numbers are
%    displayed in the combination.} \\
%    \textit{+2 The lock is locked when powered-up.} \\
%    \textit{+2 When locked, shows combo-entry display, initially without numbers
%        (power-up).} \\
%    \textit{+2 The cursor is represented with decimal points in the relevant
%    position.} \\
%    \textit{+2 The cursor blinks.}
%    \checkoffitem{Place both switches in the left position.}
%    \checkoffitem{Press the right button twice. The cursor moves into the middle
%    position and then the right-most position.} \\
%    \textit{+3 The user can move the cursor using the right button.}
%    \checkoffitem{Press the right button again. the cursor moves into the left-most
%    position.} \\
%    \textit{+1 The cursor wraps-around from the last number to the first
%    number.}
%    \checkoffitem{Press 1, then A. The display shows \texttt{1.A.-\phantom{88}-\phantom{88}}}. \\
%    \textit{+1 Combinations use 2-hex-digit numbers.} \\
%    \textit{+4 Numbers are entered using the keypad.}
%    \checkoffitem{Press the left button. The display shows \texttt{error} and then
%    \texttt{1.A.-\phantom{88}-\phantom{88}}.} \\
%    \textit{+1 Submit ``error'' combination with left button.} \\
%    \textit{+1 Incomplete combination produces error message.} \\
%    \textit{+1 After the error message, the combo-entry display returns.}
%    \checkoffitem{Finish entering an incorrect combination. The display shows all
%    three numbers, separated by dashes.} \\
%    \textit{+1 Combinations are three numbers separated by dashes.}
%    \checkoffitem{Press the left button. The display shows \texttt{badtry 1} and
%    then the combo-display display.} \\
%    \textit{+1 Submit ``bad try'' combination with left button.} \\
%    \textit{+1 Wrong combination produces bad-try message.} \\
%    \textit{+2 After the first bad try, the combo-entry display returns.}
%    \checkoffitem{Enter another incorrect combination and press the left button. The
%    display shows \texttt{badtry 2} and then the combo-display display.} \\
%    \textit{+1 The ``bad try'' number increments.} \\
%    \textit{+2 After the second bad try, the combo-entry display returns.}
%    \checkoffitem{Enter another incorrect combination and press the left button. The
%    display shows \texttt{badtry 3} and then \texttt{alert!}. The external
%    LED rapidly blinks.} \\
%    \textit{+1 After the third bad try, the sytem displays an alert message.} \\
%    \textit{+2 After the third bad try, the external LED rapidly blinks.}
%    \checkoffitem{Press buttons and keys a few times. Nothing happens except that
%    the alert message is still displayed and the LED still blinks.} \\
%    \textit{+1 After the third bad try, the system is non-responsive.}
%    \checkoffitem{Press the \developmentboard's RESET button (on the top of the \developmentboard).}
%    \checkoffitem{Enter the correct combination and press the left button. The
%    system displays \texttt{lab open}.} \\
%    \textit{+2 Submit correct combination with left button.} \\
%    \textit{+4 When the user locks the lock, it displays ``lab open.''}
%    \checkoffitem{Double-check that both switches are in the left position. Press
%    the left button. Nothing happens. Press the right button. Nothing
%    happens.} \\
%    \textit{+1 Single-presses of only one button have no effect when the lock is unlocked and the switches are not in the right position.}
%    \checkoffitem{Place both switches in the right position and press the left
%    button. The system displays \texttt{enter} then
%    \texttt{\phantom{88}-\phantom{88}-\phantom{88}}.} \\
%    \textit{+4 Start changing combo by pushing left button while both switches
%    are to the right.} \\
%    \textit{+2 When starting to change the combo, ``enter'' is displayed.} \\
%    \textit{+2 After displaying ``enter,'' the combo-entry display is shown.}
%    \checkoffitem{Enter a new combination.}
%    \checkoffitem{Press the left button. Nothing happens.} \\
%    \textit{+\textonehalf\ Left button has no effect unless left switch is
%    moved.}
%    \checkoffitem{Place the left switch in the left position and press the left
%    button. The system displays \texttt{re-enter} then
%    \texttt{\phantom{88}-\phantom{88}-\phantom{88}}.} \\
%    \textit{+4 Start confirming by moving the left switch and pressing the left
%    button.} \\
%    \textit{+2 When starting to change the combo, ``re-enter'' is displayed.} \\
%    \textit{+2 After displaying ``re-enter,'' the combo-entry display is shown.}
%    \checkoffitem{Enter a non-matching combination.}
%    \checkoffitem{Press the left button. Nothing happens.} \\
%    \textit{+\textonehalf\ Left button has no effect unless right switch is
%    moved.}
%    \checkoffitem{Place the right switch in the left position and press the left
%    button. The system displays \texttt{nochange}. Though unspecified, it is
%    acceptable to display \texttt{lab open} after displaying \texttt{nochange} for at least one second.} \\
%    \textit{+4 Compare combos by moving the right switch and pressing the left
%    button.} \\
%    \textit{+2 When the combos do not match, ``nochange'' is displayed.}
%    \checkoffitem{Double-check that both switches are in the left position. Either
%    double-click the right button \textit{or} press both buttons at the same
%    time. The system displays \texttt{closed} and then \texttt{\phantom{88}-\phantom{88}-\phantom{88}}.} \\
%    \textit{+4 Re-lock the lock using one of the specified techniques.} \\
%    \textit{+2 When locked, shows combo-entry display, initially without numbers
%        (re-locked).}
%    \checkoffitem{Enter the original, correct combination and press the left button.
%    The system displays \texttt{lab open}.} \\
%    \textit{+2 When the combos do not match, the previously-correct combo
%    remains the correct combo.}
%    \checkoffitem{Move both switches to the right, enter a new combination, move the
%    left switch to the left, and press the left button. The system displays
%    \texttt{re-enter} then \texttt{\phantom{88}-\phantom{88}-\phantom{88}}.}
%    \checkoffitem{Enter the matching combination, move the right switch to the left,
%        and press the right button. The system displays \texttt{nochange}. Though
%        unspecified, it is acceptable to display \texttt{lab open} after
%        displaying \texttt{changed} for at least one second.} \\
%    \textit{+2 When the combos match, ``changed'' is displayed.}
%    \checkoffitem{Double-check that both switches are in the left position. Either
%    double-click the right button \textit{or} press both buttons at the same
%    time. The system displays \texttt{closed} and then \texttt{\phantom{88}-\phantom{88}-\phantom{88}}.}
%    \checkoffitem{Enter the new combination and press the left button. The system
%    displays \texttt{lab open}.} \\
%    \textit{+2 When the combos match, the new combo is the correct combo.}
%    \checkoffitem{Press the \developmentboard's RESET button.}
%    \checkoffitem{Enter the new combination and press the left button. The system
%    displays \texttt{lab open}.} \\
%    \textit{+4 The correct combination persists while the Arduino is
%    powered-down.}
%\end{enumerate}

\end{document}





%Please familiarize yourself with the entire assignment before beginning.
%Section~\ref{sec:FunctionalSpecification} has the functional specification of
%the system system you will develop. Section~\ref{sec:Constraints} describes
%implementation constraints. Section~\ref{sec:Paradigm} relates interrupt-driven
%I/O to event-driven programming (which you should have learned about in \cstwo).
%Section~\ref{sec:GettingStarted} identifies code that you can reuse from PollingLab.
%Section~\ref{sec:TimerInterrupts} will guide you through configuring a hardware
%timer and responding to interrupts from that timer.
%Section~\ref{sec:ExternalInterrupts} will guide you through registering and writing
%interrupt service routines for external interrupts. Finally,
%Sections~\ref{sec:ConversionTimeout} and \ref{sec:BuildingAndConverting} offers
%suggestions for using your interrupt service routines to implement the system's
%specifications.












%\section*{Turn-in and Grading}\addcontentsline{toc}{section}{Rubric}
%
%When you have completed this assignment, upload \textit{InterruptLab.ino} to
%\filesubmission.
%
%This assignment is worth 40 points. \\
%
%Rubric:
%\begin{description}
%\item[Pushbutton Interrupts]
%\rubricitem{4}{Pushbutton presses and releases are detected with an external
%    interrupt.}
%\rubricitem{2}{The pushbuttons' interrupt handler determines which button was
%    released.}
%\rubricitem{2}{The pushbuttons' interrupt handler determines whether the left
%    button has been double-clicked.}
%\item[Matrix Keypad Interrupts]
%\rubricitem{4}{Matrix keypad presses are detected with an external interrupt.}
%\rubricitem{2}{The keypad's interrupt handler determines which key was pressed.}
%\item[Timer Interrupts]
%\rubricitem{2}{Timer interrupts for Timer1 or Timer2 are enabled and handled.}
%\rubricitem{4}{The timer interrupts configured such that, when combined with
%    other logic, the software is able to determine when exactly 7.5~seconds or
%    exactly 20~seconds have passed since the left button was double-clicked.}
%\item[Conversion Timeout]
%\rubricitem{2}{The system is able to determine when exactly 7.5~seconds have
%    passed since the left button was double-clicked.}
%\rubricitem{2}{The system is able to determine when exactly 20~seconds have
%    passed since the left button was double-clicked.}
%\item[Building Numbers]
%\rubricitem{3}{Builds a value consistent with
%    requirements~\ref{spec:initallyZero}--\ref{spec:noLeadingZeroes}.
%    (\textthreequarters\ point for each of: displaying correct positive decimal
%    values, displaying correct negative decimal values, displaying correct
%    positive hexadecimal values, displaying correct negative hexadecimal
%    values)}
%\rubricitem{1}{Negates value when left pushbutton is released. (\textonequarter\
%    point for each of: displaying correct positive decimal values, displaying
%    correct negative decimal values, displaying correct positive hexadecimal
%    values, displaying correct negative hexadecimal values)}
%\rubricitem{1}{Releasing the right pushbutton clears all digits on the Display
%    Module except the least-significant digit, which displays {\dviiseg 0}.}
%\rubricitem{1}{Detects and displays correct message when the number being built
%    is too big. (\textonehalf\ point for detecting too-big numbers;
%    \textonequarter\ point for no false detections; \textonequarter\ point for
%    displaying the correct message)}
%\item[Converting Numbers]
%\rubricitem{2}{Converts from decimal to hexadecimal when the left button is
%    double-clicked. (1 point for correctly-displayed positive numbers; 1 point
%    for correctly-displayed negative numbers)}
%\rubricitem{2}{Converts from hexadecimal to decimal when the left button is
%    double-clicked. (1 point for correctly-displayed positive numbers; 1 point
%    for correctly-displayed negative numbers)}
%\rubricitem{1}{Displays the number in its original number base 7.5~seconds or
%    20~seconds (as appropriate) after displaying the converted number.}
%\rubricitem{1}{Detects and displays correct message when the number being
%    converted to the other number base is too great for the new number base.}
%\rubricitem{1}{Displays the number in its original number base 7.5~seconds or
%    20~seconds (as appropriate) after displaying the error message.}
%\rubricitem{1}{The external LED illuminates while and only while the converted
%    number (or the convered error message) is displayed.}
%\item[Other Requirements]
%\rubricitem{1}{The number being built does not change when a button is pressed.}
%\rubricitem{1}{If the left button is double-clicked, then the number being built
%    does not change when the button is released.}
%\bonusitem{2}{Get assignment checked-off by TA or professor during office hours
%    before it is due. (You cannot get both bonuses.)}
%\bonusitem{1}{Get assignment checked-off by TA at \textit{start} of your
%    scheduled lab immediately after it is due. (Your code must be uploaded to
%    \filesubmission\ \textit{before} it is due. You cannot get both bonuses.)}
%\item[Penalties]
%\penaltyitem{0}{No penalty needed for ``Pushbutton Interrupts,'' ``Matrix Keypad
%    Interrupts,'' or ``Timer Interrupts'' portions of rubric since credit for
%    these parts of the rubric is not possible without complying with the
%    constraints in Section~\ref{sec:Constraints}.}
%\penaltyitem{4}{Code associated with conversion timeout (other than that
%    covered in the first penalty item) relies on code that violates the
%    constraints in Section~\ref{sec:Constraints}.}
%\penaltyitem{6}{Code associated with building numbers (other than that
%    covered in the first penalty item) relies on code that violates the
%    constraints in Section~\ref{sec:Constraints}.}
%\penaltyitem{8}{Code associated with converting numbers (other than that
%    covered in the first two penalty items) relies on code that violates the
%    constraints in Section~\ref{sec:Constraints}.}
%\penaltyitem{2}{Code associated with certain button actions having no effect
%    (other than that covered in the first penalty item) relies on code that
%    violates the constraints in Section~\ref{sec:Constraints}.}
%\spaghetticodepenalties{1}
%\end{description}
%%
%
%
%
%
%\section*{Appendix: Lab Checkoff}\addcontentsline{toc}{section}{Lab Checkoff}
%
%You are not required to have your assignment checked-off by a TA or the
%professor. If you do not do so, then we will perform a functional check
%ourselves. In the interest of making grading go faster, we are offering a small
%bonus to get your assignment checked-off at the start of your scheduled lab
%time immediately after it is due. Because checking off all students during lab
%would take up most of the lab time, we are offering a slightly larger bonus if
%you complete your assignment early and get it checked-off by a TA or the
%professor during office hours.
%
%\begin{description}
%\precheckoffitem{Establish that the code you are demonstrating is the code
%    you submitted to to \filesubmission.}
%    \begin{itemize}
%    \item If you are getting checked-off during lab time, show the TA that the
%        file was submitted before it was due.
%    \item Download the file into your PollingLab directory. If necessary,
%        rename it to \textit{InterruptLab.ino}.
%    \end{itemize}
%\precheckoffitem{Upload \textit{InterruptLab.ino} to your \nano\ and open the
%    Serial Monitor.}
%\end{description}
%
%\begin{enumerate}
%\checkoffitem{Show in your code that your \function{loop()} function has no
%    executable code, and that there is not an infinite loop in the
%    \function{setup()} fuction, nor in any functions called by
%    \function{setup()}.} \\
%    \textit{establishing this now allows us to conclude that the observed
%    functionality is in the ISRs or their helper functions}
%\checkoffitem{Show in your code that you registered interrupt handlers for
%    the pushbuttons and for the matrix keypad.} \\
%    \textit{make note of the interrupt condition}
%\checkoffitem{Show in your code your interrupt handlers for the pushbuttons and
%    for the matrix keypad.} \\
%    \textit{+4 pushbutton presses and releases are detected with an external
%    interrupt \\
%    \phantom{xxx}(if a single handler is registered for pushbutton CHANGEs, or \\
%    \phantom{xxx}if handlers are registered for both FALLING and RISING
%    pushbuttons)} \\
%    \textit{+4 matrix keypad presses are detected with an external interrupt}
%\checkoffitem{Show and explain how you determined the comparison value and
%    prescaler for your timer interrupts.}
%\checkoffitem{Show in your code that you had configured the timer to use
%    those comparison and prescaler values and that you had configured the
%    correct timer mode.} \\
%    \textit{+4 timer interrupts are configured correctly}
%\checkoffitem{Show in your code that you enabled the correct timer.}
%\checkoffitem{Show in your code that you created an ISR for the correct
%    timer.} \\
%    \textit{+2 timer interrupts are enabled and handled}
%\checkoffitem{Place the right switch in the left (decimal) position.}
%\checkoffitem{Press 1. One of two things will happen:}
%    \begin{itemize}
%    \item 1 appears on the Serial Monitor.
%    \item {\dviiseg 1} appears on the display module.
%    \end{itemize}
%    \textit{+2 keypad ISR determines which key was pressed}
%\checkoffitem{Press the left pushbutton. The Serial Monitor may indicate that
%    the left pushbutton is DOWN and the right pushbutton is UP, but there is
%    no other observable behavior.}
%\checkoffitem{Release the left pushbutton. One or both of two things will
%    happen:}
%    \begin{itemize}
%    \item The Serial Monitor indicates that both pushbuttons are UP.
%    \item {\dviiseg -1} appears on the display module.
%    \end{itemize}
%\checkoffitem{Press the right pushbutton. The Serial Monitor may indicate that
%    the right pushbutton is DOWN and the left pushbutton is UP, but there is
%    no other observable behavior.} \\
%    \textit{+1 The number being build does not change when a button is pressed.}
%\checkoffitem{Release right pushbutton. One or both of two things will
%    happen:}
%    \begin{itemize}
%    \item The Serial Monitor indicates that both pushbuttons are UP.
%    \item {\dviiseg 0} appears on the display module.
%    \end{itemize}
%    \textit{+2 pushbutton ISR determines which button was released}
%\checkoffitem{Place the right switch in the right (hexadecimal) position. Place
%    the right switch in the left (7.5~seconds) position.}
%\checkoffitem{Press A. {\dviiseg A} may appear on the display module.}
%\checkoffitem{Double-click the left pushbutton. One or more of three things will
%    happen:}
%    \begin{itemize}
%    \item The Serial Monitor indicates that a double-click occurred.
%    \item The display will update to {\dviiseg 10} or an attempt to do so, such
%        as {\dviiseg 0} or {\dviiseg 1}.
%    \item The external LED illuminates.
%    \end{itemize}
%    \textit{+2 pushbutton ISR determines whether the left button is
%    double-clicked} \\
%\checkoffitem{Wait 7.5~seconds. One or more of three things will happen:}
%    \begin{itemize}
%    \item The Serial Monitor indicates that 7.5~seconds have elapsed.
%    \item The display will revert to {\dviiseg A} or an attempt to do so.
%    \item The external LED deluminates.
%    \end{itemize}
%    \textit{+2 system determines when 7.5~seconds have passed} \\
%    \textit{+1 external LED illuminates while and only while converted number is
%    displayed} \\
%% \checkoffitem{Release both buttons. The Serial Monitor may briefly indicate that
%%     one button is DOWN and the other is UP, and it may indicate that both
%%     buttons are UP, but there is no other observable behavior.} \\
%    \textit{+1 the number being built does not change when the left button is
%    released during a double-click}
%\checkoffitem{Place the left switch in the left (20~seconds) position.}
%\checkoffitem{Double-click the left pushbutton. One or more of the
%    previously-mentioned indications of simultaneous button presses happen.}
%\checkoffitem{Wait 20~seconds. One or more of the previously-mentioned
%    indications of conversion timeout happen.} \\
%    \textit{+2 system determines when 20~seconds have passed}
%\checkoffitem{Place the left switch in the right (7.5~seconds) position.}
%\checkoffitem{Press and release the right pushbutton. The output is {\dviiseg 0}
%    on the display module.}
%\checkoffitem{Enter the value 0x123A. The output is {\dviiseg\phantom{8888}123A}
%    on the display module.} \\
%    \textit{+\textthreequarters\ displays correct positive hexadecimal values}
%\checkoffitem{Press the and release the left pushbutton. The output is
%    {\dviiseg FFFFEDC6} on the display module.} \\
%    \textit{+\textonequarter\ correctly negates positive hexadecimal values}
%\checkoffitem{Press and release the right pushbutton. The output is {\dviiseg 0}
%    on the display module.}
%    \textit{+1 releasing the right pushbutton clears all digits and then
%        displays {\dviiseg 0}}
%\checkoffitem{Enter the value 0xB654789C. The output is {\dviiseg B654789C}
%    on the display module.} \\
%    \textit{+\textthreequarters\ displays correct negative hexadecimal values}
%\checkoffitem{Press the and release the left pushbutton. The output is
%    {\dviiseg 49AB8764} on the display module.} \\
%    \textit{+\textonequarter\ correctly negates negative hexadecimal values}
%\checkoffitem{Place the right switch in the left (decimal) position.}
%\checkoffitem{Press and release the right pushbutton. The output is {\dviiseg 0}
%    on the display module.}
%\checkoffitem{Enter the value 1478. The output is {\dviiseg\phantom{8888}1478}
%    on the display module.} \\
%    \textit{+\textthreequarters\ displays correct positive decimal values}
%\checkoffitem{Press the and release the left pushbutton. The output is
%    {\dviiseg\phantom{888}-1478} on the display module.} \\
%    \textit{+\textonequarter\ correctly negates positive decimal values}
%\checkoffitem{Press and release the right pushbutton. The output is {\dviiseg 0}
%    on the display module.}
%\checkoffitem{Enter the value -5236. The output is {\dviiseg\phantom{888}-5236}
%    on the display module.} \\
%    \textit{+\textthreequarters\ displays correct negative decimal values}
%\checkoffitem{Press the and release the left pushbutton. The output is
%    {\dviiseg\phantom{8888}5236} on the display module.} \\
%    \textit{+\textonequarter\ correctly negates negative decimal values}
%\checkoffitem{Press and release the right pushbutton. The output is {\dviiseg 0}
%    on the display module.}
%\checkoffitem{Enter the value 12369874. The output is {\dviiseg 12369874} on
%    the display module.} \\
%    \textit{+\textonequarter\ no false ``too big'' detections}
%\checkoffitem{Press 0. The output is {\dviiseg\phantom{8}too big} on the display
%    module.} \\
%    \textit{+\textonehalf\ detects that a number is ``too big''} \\
%    \textit{+\textonequarter\ displays correct error message}
%\checkoffitem{Press and release the right pushbutton. The output is {\dviiseg 0}
%    on the display module.}
%\checkoffitem{Enter the value 1234. The output is {\dviiseg\phantom{8888}1234}
%    on the display module.}
%\checkoffitem{Double-click the left pushbutton. The output is
%    {\dviiseg\phantom{88888}4D2} on the display module.} \\
%    \textit{+1 converts positive decimal to hexadecimal}
%\checkoffitem{Wait 7.5~seconds for the conversion to timeout. The output is
%    {\dviiseg\phantom{8888}1234} on the display module.} \\
%    \textit{+\textonequarter\ the original positive decimal number displays after
%    timeout}
%\checkoffitem{Press and release the left pushbutton. The output is
%    {\dviiseg\phantom{888}-1234} on the display module.}
%\checkoffitem{Double-click the left pushbutton. The output is
%    {\dviiseg FFFFFB2E} on the display module.} \\
%    \textit{+1 converts negative decimal to hexadecimal}
%\checkoffitem{Wait 7.5~seconds for the conversion to timeout. The output is
%    {\dviiseg\phantom{888}-1234} on the display module.} \\
%    \textit{+\textonequarter\ the original negative decimal number displays after
%    timeout}
%\checkoffitem{Press and release the right pushbutton. The output is {\dviiseg 0}
%    on the display module.}
%\checkoffitem{Place the right switch in the right (hexadecimal) position.}
%\checkoffitem{Enter the value 0x56B7. The output is
%    {\dviiseg\phantom{8888}56B7} on the display module.}
%\checkoffitem{Double-click the left pushbutton. The output is
%    {\dviiseg\phantom{888}22199} on the display module.} \\
%    \textit{+1 converts positive hexadecimal to decimal}
%\checkoffitem{Wait 7.5~seconds for the conversion to timeout. The output is
%    {\dviiseg\phantom{8888}56B7} on the display module.} \\
%    \textit{+\textonequarter\ the original positive hexadecimal number displays
%    after timeout}
%\checkoffitem{Press and release the left pushbutton. The output is
%    {\dviiseg FFFFA949} on the display module.}
%\checkoffitem{Double-click the left pushbutton. The output is
%    {\dviiseg\phantom{88}-22199} on the display module.} \\
%    \textit{+1 converts negative hexadecimal to decimal}
%\checkoffitem{Wait 7.5~seconds for the conversion to timeout. The output is
%    {\dviiseg FFFFA949} on the display module.} \\
%    \textit{+\textonequarter\ the original negative hexadecimal number displays
%    after timeout}
%\checkoffitem{Press and release the right pushbutton. The output is {\dviiseg 0}
%    on the display module.}
%\checkoffitem{Enter the value 0x6A0,0000. The output is
%    {\dviiseg\phantom{8}6A00000} on the display module.}
%\checkoffitem{Double-click the left pushbutton. The output is
%    {\dviiseg\phantom{888}ERROR} on the display module.} \\
%    \textit{+\textonehalf\ detects that converted number is too great to
%    display} \\
%    \textit{+\textonequarter\ displays correct error message}
%\checkoffitem{Wait 7.5~seconds for the conversion to timeout. The output is
%    {\dviiseg\phantom{8}6A00000} on the display module.} \\
%    \textit{+1 the original number displays after timeout}
%\checkoffitem{Press and release the right pushbutton. The output is {\dviiseg 0}
%    on the display module.}
%\checkoffitem{Enter the value 0x5F5,E0FF. The output is
%    {\dviiseg\phantom{8}5F5E0FF} on the display module.}
%\checkoffitem{Double-click the left pushbutton. The output is
%    {\dviiseg 99999999} on the display module.} \\
%    \textit{+\textonequarter\ no false ``error'' detections}
%\end{enumerate}
%
%This concludes the demonstration of your system's functionality. The TAs will
%later examine your code for violations of the assignment's constraints. If your
%code looks like it is tailored for this checklist, the TAs may re-grade using a
%different checklist.
