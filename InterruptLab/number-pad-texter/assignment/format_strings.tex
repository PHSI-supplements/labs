I encourage you to look at \url{https://pubs.opengroup.org/onlinepubs/9699919799/functions/sprintf.html} for an authoritative source of information about conversion specifiers in the \function{printf} family.
Section~7.2 of \textit{The C Programming Language} on pages 153--155 is also a good source.

In PokerLab, you used the \lstinline{%d} conversion specifier to print an integer, and you used it again in PollingLab.
In KeyboardLab, you used the \lstinline{%c} conversion specifier to print a character, and you may have used \lstinline{%s} to place a string inside another string.
In PollingLab, you saw how to specify the field width and alignment by using positive or negative constants as part of the conversion specifier.

You should be able to generate any display strings you need for this lab using integer conversions, character conversions, field widths and alignment, and (if you want to get fancy) string conversions.
We offer some examples.

\subsection{Simple Examples}

The example display in Requirement~\ref{spec:standardDisplay} is explicitly only an example, with no particular layout actually specified.
That particular example would be difficult to produce in only one pass.
We can produce some similar displays easily enough.

One option would be to omit units altogether:
\begin{lstlisting}
    sprintf(top_line, "Spd %-8d Hdg", 123);
    sprintf(bottom_line, "Odo %-8d %3d", 456, 270);
\end{lstlisting} \phantom{x}\\
\display{
    \colorbox{LightGreen}{Spd\phantom{x}123\phantom{fpf}\phantom{fxxx}Hdg} \vspace{-1mm}\\
    \colorbox{LightGreen}{Odo\phantom{x}456\phantom{\textmu fur}\phantom{xx}270}
} \\

Another option might be to allow space between the values and the units:
\begin{lstlisting}
    sprintf(top_line, "Spd %-5dfpf Hdg", 123);
    sprintf(bottom_line, "Odo %-4d%cfur %3d", 456, MU, 270);
\end{lstlisting} \phantom{x}\\
\display{
    \colorbox{LightGreen}{Spd\phantom{x}123\phantom{fxxx}fpf Hdg} \vspace{-1mm}\\
    \colorbox{LightGreen}{Odo\phantom{x}456\phantom{xx}\textmu fur 270}
} \\

Or, perhaps right-justify the measurements:
\begin{lstlisting}
    sprintf(top_line, "Spd %5dfpf Hdg", 123);
    sprintf(bottom_line, "Odo %4d%cfur %3d", 456, MU, 270);
\end{lstlisting} \phantom{x}\\
\display{
    \colorbox{LightGreen}{Spd\phantom{x}\phantom{fxxx}123fpf Hdg} \vspace{-1mm}\\
    \colorbox{LightGreen}{Odo\phantom{x}\phantom{xx}456\textmu fur 270}
} \\

Yet another might be to flip the left and right sides:
\begin{lstlisting}
    sprintf(top_line, "Hdg--Spd %dfpf", 123);
    sprintf(bottom_line, "%3d--Odo %d%cfur", 270, 456, MU);
\end{lstlisting} \phantom{x}\\
\display{
    \colorbox{LightGreen}{Hdg--Spd\phantom{x}123fpf\phantom{fxx}} \vspace{-1mm}\\
    \colorbox{LightGreen}{270--Odo\phantom{x}456\textmu fur\phantom{x}}
} \\

Maybe eliminate the labels, relying on the units to identify which value is which:
\begin{lstlisting}
    sprintf(top_line, "%12d fpf", 123);
    sprintf(bottom_line, "%3d%c %7d%cfur", 270, DEGREE, 456, MU);
\end{lstlisting} \phantom{x}\\
\display{
    \colorbox{LightGreen}{\phantom{xxxxxxxxxx}123\phantom{x}fpf\phantom{.}} \vspace{-1mm}\\
    \colorbox{LightGreen}{270\textdegree\phantom{xxxxxx}456\textmu fur}
} \\

\subsection{If You Really Want to Duplicate the Example}

If you really want to duplicate the example in the specification, one approach would be to create a couple of intermediate strings:
\begin{lstlisting}
    char intermediate_string[9];
    sprintf(intermediate_string, "%dfpf", 123);
    sprintf(top_line, "Spd %-8s Hdg", intermediate_string);
    sprintf(intermediate_string, "%d%cfur", 456, MU);
    sprintf(bottom_line, "Odo %-8s %3d", intermediate_string, 270);
\end{lstlisting} \phantom{x}\\
\display{
    \colorbox{LightGreen}{Spd\phantom{x}123fpf\phantom{fxxx}Hdg} \vspace{-1mm}\\
    \colorbox{LightGreen}{Odo\phantom{x}456\textmu fur\phantom{xx}270}
}\\

\subsection{Have Some Fun With It}

The specification describes the \textit{minimum} information that needs to be displayed.
You can go beyond that.
Maybe briefly display ``VROOM!!'' when the motor engages and briefly display ``retro-encabulator secure'' when the motor disengages.
Or look at the character sets in the LCD1602 datasheet\footnote{
    \url{https://www.sparkfun.com/datasheets/LCD/HD44780.pdf}
} to introduce an icon indicating whether the motor is engaged or not.
