When you have completed this assignment, upload \textit{inputs.c}, \textit{character\_selector.c} and \textit{message\_editor.c} to
\filesubmission.

%\policyforcodethatdoesnotcompile
\subsection*{No Credit for Uncompilable Code}
If the TA cannot create an executable from your code, then your code will be assumed to have no functionality.\footnote{
    At the TA's discretion, if they can make your code compile with \textit{one} edit (such as introducing a missing semicolon) then they may do so and then assess a 10\% penalty on the resulting score.
    The TA is under no obligation to do so, and you should not rely on the TA's willingness to edit your code for grading.
    If there are multiple options for a single edit that would make your code compile, there is no guarantee that the TA will select the option that would maximize your score.
}
Before turning in your code, be sure to compile and test your code on \runtimeenvironment\ with the original driver code and the original header file(s).

\interruptlablatepolicy

\subsection*{Rubric}

This assignment is worth 30 points.
\begin{description}
    \item[Interrupts] % 12
    \rubricitem{1}{Matrix keypad presses and releases are detected with a pin change interrupt}
    \rubricitem{2}{The right pushbutton's presses are detected with a pin change interrupt}
    \rubricitem{2}{The left pushbutton's presses are detected with a pin change interrupt}
    \rubricitem{1}{Timer interrupts for TIMER2 are handled}
    \rubricitem{1}{TIMER2's interrupt period, when combined with logic that uses \lstinline{supplemental_counter}, causes the right LED to illuminate for exactly 256ms\dots}
    \rubricitem{1}{\dots and that logic is implemented}
    \rubricitem{2}{Timer interrupts for TIMER1 are handled}
    \rubricitem{2}{TIMER1's interrupt period is exactly 2000ms}
%    \rubricitem{1}{There is no long-running code (including, but not limited to, \function{display_string()} and \function{printf()}) in the interrupt handlers / ISRs, \textit{except} where the inclusion of print statements is the only mechanism to demonstrate that the interrupt handler determines which button was pressed}

    \item[System Implementation] % 18
        % 3 for display
    \rubricitem{1}{The cursor is always below the character being created}
    \rubricitem{1}{A message with fewer than 16 chracters is displayed in full}
    \rubricitem{1}{A message with more than 15 characters is displayed as the last 15 characters plus the character being created}
        % 2 for right bound
    \rubricitem{1}{A message with 63 characters can be created\dots}
    \rubricitem{1}{\dots but not a message with 64 characters}
        % 1 for LED
    \rubricitem{1}{Pressing a key causes the right LED to illuminate for about \textonequarter\ second}
        % 3 for character sequencing
    \rubricitem{1}{Starting with a blank character, pressing a number key initiates the character sequence shown in Figure~\ref{fig:keypad}}
    \rubricitem{2}{Repeated presses of the same number key cycles through the character sequence shown in Figure~\ref{fig:keypad}}
        % 3 for finalize
    \rubricitem{1}{Pressing a different key finalizes the character, advances the cursor, and initiates the appropriate character sequence for the next character}
    \rubricitem{1}{Pressing the right pushbutton finalizes the character and advances the cursor but leaves a blank character above the cursor}
    \rubricitem{1}{Pressing no buttons or keys for 2 seconds causes the system to finalize the character and to advance the cursor, leaveing a blank character above the cursor}
        % 3 for delete (including left bound)
    \rubricitem{1}{If a character is being created, then pressing the left pushbutton deletes the character being created and leaves the cursor under a blank space}
    \rubricitem{1}{If a character is not being created, the pressing the left pushbutton moves the character back one position and deletes the message's last character, leaving the cursor under a blank space\dots}
    \rubricitem{1}{\dots except that the cursor cannot retreat to a position earlier than start of the message buffer}
        % 3 for transmit
    \rubricitem{1}{Pressing the \texttt{D} key causes the message to be printed on the Serial Monitor\dots}
    \rubricitem{\textonehalf}{\dots and the message to be cleared from the display\dots}
    \rubricitem{\textonehalf}{\dots with the cursor in the left position\dots}
    \rubricitem{1}{\dots and another message can be created and ``transmitted''}
        % 0 for extraneous inputs
%    \rubricitem{0}{The \texttt{A}, \texttt{B}, \texttt{C}, \texttt{\#}, and \texttt{*} keys, and the switches, have no effect}

    \item[Bonuses]
    \bonusitem{2}{Get assignment checked-off by TA or professor during office hours before it is due (you cannot get both check-off bonuses)}
    \bonusitem{1}{Get assignment checked-off by TA at \textit{start} of your scheduled lab immediately after it is due (your code must be uploaded to \filesubmission\ \textit{before} it is due; you cannot get both bonuses)}

    \item[Penalties]
    \penaltyitem{4}{Number key presses are detected by means other than a pin change interrupt}
    \penaltyitem{3}{Pressing the \texttt{D} key is detected by means other than a pin change interrupt}
    \penaltyitem{1}{Pressing the right button is detected by means other than a pin change interrupt}
    \penaltyitem{3}{Pressing the left button is detected by means other than a pin change interrupt}
    \penaltyitem{1}{Blinking the LED is accomlished by means other than a timer interrupt}
    \penaltyitem{1}{Timing-out character generation is accomlished by means other than a timer interrupt}
    \spaghetticodepenalties{1}
\end{description}
