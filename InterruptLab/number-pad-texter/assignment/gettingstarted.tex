If you are using the Arduino IDE, use \textit{pollinglab-arduino.zip} or \textit{pollinglab-arduino.tar}.
If you are using the VS Code with PlatformIO, use \textit{pollinglab-platformio.zip} or \textit{pollinglab-platformio.tar}.
Download the zip file or tarball from \filesource.
Once you have downloaded the file, close any existing projects in your IDE, unpackage the file, and open the project in your IDE\@.

\subsection{Description of InterruptLab Files}

\subsubsection{InterruptLab.ino or InterruptLab.cpp}

Do not edit \textit{InterruptLab.ino} or \textit{InterruptLab.cpp}.

\textit{InterruptLab.ino} is in the package for the Arduino IDE, and \textit{InterruptLab.cpp} is in the package for PlatformIO\@.
Whichever you are using, this file contains the code to call all the configuration functions.

\subsubsection{output.h \& output.c}

Do not edit \textit{outputs.h} or \textit{outputs.c}.

The \textit{outputs.c} file contains the code to configure your Cow~Pi for this lab,
and it contains code to render the message correctly on the display module and to send the message to the Serial Monitor.

\subsubsection{inputs.h \& inputs.c}

Do not edit \textit{inputs.h}.

The \textit{inputs.c} file contains interrupt service routines that you will write and that you will register to handle interrupts.

\subsubsection{character\_selector.h \& character\_selector.c}

Do not edit \textit{character\_selector.h}.

The \textit{character\_selector.c} file contains functions that you will write to select the character to be added to the end of the message.

\subsubsection{message\_editor.h \& message\_editor.c}

Do not edit \textit{message\_editor.h}.

The \textit{message\_editor.c} file contains functions that you will write to manage the message and the cursor position.


\subsection{Separation of Concerns}

As with PollingLab, we attempt to provide a separation of concerns.
The dependencies between the modules is shown in Figure~\ref{fig:architecture}.

\begin{figure}
    \centering
    \begin{tikzpicture}[x=1mm, y=1mm]
        \draw (-5,10)  rectangle (15,20) ++(-10,-5)  node {inputs.c};
        \draw (30,0)  rectangle (70,10) ++(-20,-5)  node {character\_selector.c};
        \draw (30,20) rectangle (70,30) ++(-20,-5)  node {message\_editor.c};
        \draw (85,20) rectangle (105,30) ++(-10,-5) node {outputs.c};
        \draw[blue, thick, arrows={-Stealth}] (60,10) -- (60,20);
        \draw[blue, thick] (65,15) node {data};
        \draw[blue, thick, arrows={-Stealth}] (70,25) -- (85,25);
        \draw[blue, thick] (77.5,22) node {data};
        \draw[red, thick, arrows={<->}] (40,10) -- (40,20);
        \draw[red, thick] (30,15) node {commands};
        \draw[red, thick, arrows={->}] (10,10) -- (30,5);
        \draw[red, thick, arrows={->}] (10,20) -- (30,25);
        \draw[red, thick] (20, 25) node {\rotatebox{15}{commands}};
        \draw[red, thick] (20, 5) node {\rotatebox{-15}{commands}};
    \end{tikzpicture}
    \caption{High-Level Architecture of the Text Messager} \label{fig:architecture}
\end{figure}

Only \textit{outputs.c} (which you will not need to edit) interfaces with the display module and with the host computer's Serial Monitor.

Only \textit{inputs.c} responds to user inputs on the buttons and keypad.
The code in \textit{inputs.c} will instruct the character selector to reset or to update the character when the user makes particular inputs.
Similarly, the code in \textit{inputs.c} will instruct the message editor to delete the character at the end of the message or to send the message to the output.

The code in \textit{character\_generator.c} sends updated characters to the message editor and will also instruct the message editor to advance the cursor when a character is finalized.

The code in \textit{message\_editor.c} might need to instruct the character selector to finalize its character,
and it will also send the message to \textit{outputs.c} to be displayed on the display module or to be printed on the Serial Monitor.

By separating the concerns in this manner, only the message editor needs to know the details of the message,
and only the character selector needs to know which character sequence is being traversed.
We could, in principle, replace the keypad with a QWERTY keyboard, or replace the Serial Monitor with a packet radio, or make any number of other changes, and the other modules would not detect the difference.


\subsection{Interrupt-Driven Input/Output -- A Different but Familiar Programming Paradigm}

In PollingLab you used polling to determine when to respond to a pushbutton or a key on the keypad being pressed.
In this lab, you will rely on interrupts to let your code know when it needs to respond to a pushbutton or a key on the keypad being pressed, and you will rely on interrupts to let your code know when a certain amount of time has passed.

\vspace{.25cm}

Most of the code you wrote in the earlier lab assignments for this course used imperative programming:
you, the programmer, had full control over changes to the program state.
In PollingLab, you stretched this model to something resembling shared-memory concurrent programming:
you still had full control over changes to the program state of your program's flow of control,
but your program periodically read variables (memory-mapped input registers) that could be changed by another flow of control (the physical world).

In this lab assignment, you will write code that reacts to things happening in the physical world;
your code program state will not change except in reaction to interrupts.
Conceptually, this isn't very different from event-driven programming that you learned in \cstwo.
While configuring the hardware timer is a little more complex than configuring a GUI framework's timer, handling a timer interrupt is very much like writing an \function{onTimeout()} event handler.
Similarly, handling a change in a pushbutton's position is very much like writing an \function{onMouseClick} event handler.
Your code is not focused on \textit{when} the button is pressed or released, nor even on \textit{detecting} that a button has been pressed or released;
it is focused solely on \textit{what should happen} when the button is pressed or released.