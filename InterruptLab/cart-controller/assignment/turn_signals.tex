During your lab period, the TAs will guide the class through the first modifications to the starter code that you must make, described in this section.
If you do not attend your lab period, then you must complete this section on your own.
\textbf{\textit{Except during lab time, you may }not\textit{ discuss the solutions for this section with other students.}}


\subsection{Examining the Starter Code}

\subsubsection{turn\_signals.h}

A not-uncommon pattern in embedded systems is to design the system as a state machine, or as multiple state machines.
The states are represented as the value of an enumerated type.
A couple of key advantages of writing the system as a state machine or as a collection of state machines is that it's easy to change the state based on inputs, and it's easy for the parts of the system that take action to do so based on what state the state machine is in.

In \textit{turn\_signals.h} you'll see \lstinline{enum turn_directions} which enumerates the states of a small state machine.
You'll also see the variable \lstinline{indicated_turn_direction} which will store the state of that small state machine.
In the header file, the variable is modified with the \lstinline{extern} keyword, which can be thought of as causing the \lstinline{indicated_turn_direction} variable in \textit{turn\_signals.c} to be ``public'' so that the variable can be modified by code in both \textit{turn\_signals.c} and \textit{cart\_controller.c}.


\subsubsection{turn\_signals.c}

In \textit{turn\_signals.c} you'll see:
\begin{itemize}
    \item \function{initialize_turn_signals()} where you'll put code to set variables' initial values and to configure interrupts
    \item \function{handle_button_action()} where you'll put the code that should run whenever a pushbutton is pressed
    \item \function{ISR(TIMER1_COMPA_vect)} where you'll put the code that needs to run periodically
\end{itemize}


\subsection{Detecting Button Presses}

Requirement~\ref{spec:TurnSignalControls} states that the user will indicate the direction they want to turn by pressing the appropriate pushbutton to activate the turn signal.

Read Section~5 of the Cow Pi datasheet for an overview of handling interrupts for your \developmentboard.

To detect button presses (and key presses), you will take advantage of having the NAND of the pushbuttons and the NAND of the keypad columns being input to Arduino pins D2 and D3.

Add code to \function{initialize_turn_signals()} to register \function{handle_button_action()} as your handler for button presses.
Use the \function{attachInterrupt()} and \function{digitalPinToInterrupt()} functions, as described in Section~5.3 of the datasheet.
Specify \lstinline{CHANGE} mode;
we shall explain why in Section~\ref{subsubsec:debouncing}.

Your code to respond to button presses will go in the \function{handle_button_action()} function.

\begin{lstlisting}[basicstyle=\small]
void handle_button_action(void) {
  debounce_interrupt({
    static int8_t number_of_buttons_pressed = -1;
    bool left_button_is_pressed, cowpi_right_button_is_pressed;
    while ((left_button_is_pressed = cowpi_left_button_is_pressed())
             + (right_button_is_pressed = cowpi_right_button_is_pressed())
           == number_of_buttons_pressed) {}   // busy-wait through the race condition
    /* Add code to respond to button presses */

    number_of_buttons_pressed = left_button_is_pressed + right_button_is_pressed;
  });
}
\end{lstlisting}

\subsubsection{Debouncing} \label{subsubsec:debouncing}

When you place your code that responds to button presses in \function{handle_button_action()}, place it inside the braces for the \function{debounce_interrupt()} macro.
Just as the \function{debounce_byte()} function took care of debouncing when polling mechanical inputs, the \function{debounce_interrupt()} macro defined in \textit{supplement.h} takes care of (most of) the debouncing when mechanical inputs generate interrupts.
There are two catches.

The first catch is that \function{debounce_interrupt()} only works when the interrupt handler responds to both the rising and falling edges of the input.
This is why you must register \function{handle_button_action()} using the \lstinline{CHANGE} mode.
This isn't such a terrible limitation:
Determine whether the interrupt is due to a button being pressed or released.
If pressed, then comply with Requirement~\ref{spec:TurnSignalControls};
if released, then do nothing.

The second catch is that \function{debounce_interrupt()} works by ignoring interrupts that are generated due to switchbounce.
It does not -- it cannot -- ignore switchbounce that occurs while the interrupt handler is executing.
If there were only one device attached to the input pin, then switchbounce that occurs while the interrupt handler is executing would not be a problem:
if you have a \lstinline{static bool} variable to track whether the \textit{previous} interrupt was due to the button being pressed or released, then you know that the \textit{current} interrupt is due to the button moving in the other direction.

As it is, there are two pushbuttons that could cause the interrupt, and to respond correctly, you must determine \textit{which} pushbutton was pressed by examining the inputs.
There is, however, a race condition.
Consider this scenario: the user presses a button, closing the contacts, triggering an interrupt, and the interrupt handler launches.
Before reaching the code that reads the inputs to determine which button was pressed, the switch bounces, temporarily re-opening the contacts;
when the code reads the inputs, neither button appears to be pressed.
The fix is to introduce a loop that iterates until a sensible input reading is found.
This busy-wait loop is already in the starter code.

\subsubsection{Design the Interrupt Handler}

After the busy-wait loop terminates, use the \lstinline{left_button_is_pressed} and \\ \lstinline{cowpi_right_button_is_pressed} variables to assign the appropriate state to \\ \lstinline{indicated_turn_direction}.
The code in the pushbutton interrupt handler should \textit{only} make the changes that are due to changes in the pushbuttons' positions.


\subsection{Detecting Time}

Read Section~5.2 and Section~6 of the Cow Pi datasheet to familiarize yourself with timer interrupts on the \developmentboard.
Note that Sections~6.3--6.5 are generally duplicates of each other except for their tables and a warning at the start of Section~6.3;
for you now can read Sections 6.1, 6.2, 6.4, and 6.6.
Later, you can re-visit Sections~6.3 and 6.5.

You must use a timer interrupt, along with the turn signal's state, to implement Requirement~\ref{spec:TurnSignals}.
Requirement~\ref{spec:blinkRate} identifies the timer period that you need.
Use the equation in the datasheet's Section~6.2 to iterate through possible comparison values and prescalers until you arrive at a prescaler that is available for Timer1 and a comparison value that fits in the available counter bits for that timer.
When you have done so, add code to \function{initialize_turn_signals()}:
\begin{itemize}
    \item Create a \lstinline{cowpi_timer16bit_t} pointer (in the remaining steps, we will refer to this pointer as \lstinline{timer})
    \item Use the address offset listed in the datasheet's Section~6.4 to assign the appropriate address to \lstinline{timer}
    \item Use the datasheet's Tables~9--10 to select the WGM bits for ``Clear on Timer Compare'' mode and the CS bits for your prescaler;
        use these bits and Table~8 to generate a bit vector and assign that bit vector to \lstinline{timer->control}
    \item Subtract 1 from the comparison value and assign that to \lstinline{timer->compareA}
    \item Create a \lstinline{uint8_t} pointer (which we will refer to as \lstinline{timer_interrupt_masks}), and use the address offset listed in the datasheet's Section~6.6 to assign the appropriate address to \lstinline{timer_interrupt_masks}
    \item Treating the pointer as an array, index the \lstinline{timer_interrupt_masks} array with the timer number (1), and use the datasheet's Table~14 to assign the appropriate bit vector to enable the \texttt{TIMER\textit{n}\_COMPA\_vect} interrupt, where $n$ is the timer number
%    \item Create an ISR for the \lstinline{TIMER1_COMPA_vect} interrupt using the \function{ISR()} macro as described in Section~5.2 of the datasheet
\end{itemize}

You can now add code to the ISR for \lstinline{TIMER1_COMPA_vect} to implement Requirement~\ref{spec:blinkRate}.
Based on \lstinline{indicated_turn_direction}'s state, either the left LED should blink, or the right LED should blink, or neither should blink.
\ifbool{offerdecompositionhints}{
    Introducing a \lstinline{static} variable to your ISR may be helpful.
}{}


\vspace{1cm}

You are now ready to complete the remainder of this assignment on your own.
Reminders:
\begin{itemize}
    \item Declare any global variables you use as \lstinline{volatile}.
    \item \collaborationrules
\end{itemize}