\subsection{Examining the Starter Code}

\subsubsection{Starter Code Functions and Variables}

In \textit{cart\_controller.c} you'll see:
\begin{itemize}
    \item \function{initialize_cart_system()} where you'll put code to set variables' initial values and to configure interrupts
    \item \function{control_cart()} where you'll put code that should repeated execute as part of the system's main loop
    \item \function{handle_key_action()} where you'll put the code that should run whenever a key on the keypad is pressed or released
    \item \function{ISR(TIMER2_COMPA_vect)} where you'll put the code that needs to run periodically
\end{itemize}

You may, of course, add other helper functions to \textit{cart\_controller.c} as you see fit.

You will also see a global \lstinline{volatile} variable, \lstinline{display_needs_refreshed} that you can use as part of implementing Requirement~\ref{spec:displayRefreshRate}.
There are also two \lstinline{const} variables, \lstinline{MU} and \lstinline{DEGREE} that can be used if you wish to include `\textmu' or `\textdegree' as part of your display (you are  not required to use these characters, but they are available if you wish to do so).

Recall also that \textit{turn\_signals.h} has an \lstinline{extern} variable, \lstinline{indicated_turn_direction} that is shared between \textit{turn\_signals.c} and \textit{cart\_controller.c}.

You \textit{will} need to add additional global variables to pass information between \function{control_cart()}, \function{handle_key_action()}, and \function{ISR(TIMER2_COMPA_vect)}.
Be sure to declare any global variables that you introduce as \lstinline{volatile}.

\subsubsection{Determining Your Display Module's ROM Code}

You can skip this subsection if you will not use \lstinline{MU} or \lstinline{DEGREE}.

The LCD1602's datasheet\footnote{
    \url{https://www.sparkfun.com/datasheets/LCD/HD44780.pdf}
} describes the character sets for two possible ROMs, coded ``A00'' and ``A02''.\footnote{
    No, I don't know what happened to ROM A01.
}
Both character sets include the ASCII printable characters, but they have different characters for the non-ASCII values.
Initially, the \function{initialize_cart_system()} function has this code:
\begin{lstlisting}[basicstyle=\small]
void initialize_cart_system(void) {
  display_needs_refreshed = true;
  /* USE THIS CODE TO DETERMINE WHICH ROM CODE YOUR DISPLAY MODULE HAS, AND
     SET MU and DEGREE ACCORDINGLY ON LINE 55-61. THEN COMMENT-OUT THE NEXT
     EIGHT LINES AFTER THIS COMMENT */
  char label_string[] = "mu        degree";
  char candidates_string[17] = {0};
  if (display_needs_refreshed) {
    sprintf(candidates_string, "0:%c 2:%c  0:%c 2:%c", (char)0xE4, (char)0xB5, (char)0xDF, (char)0xB0);
    display_needs_refreshed = false;
    display_string(label_string, TOP_ROW);
    display_string(candidates_string, BOTTOM_ROW);
  }
  /* END OF CODE TO DETERMINE YOUR DISPLAY MODULE'S ROM CODE */

}
\end{lstlisting}
Depending on which ROM your display module has, your display module will initially show either

ROM Code 0 \\
\texttt{
    \colorbox{LightGreen}{mu\phantom{xxxxxxxxx}degree} \vspace{-1mm}\\
    \colorbox{LightGreen}{0:\textmu\phantom{x}2:\begin{CJK*}{UTF8}{goth}ォ\end{CJK*}\phantom{xx}0:\textdegree\phantom{x}2:-}
} \\
or

ROM Code 2 \\
\texttt{
    \colorbox{LightGreen}{mu\phantom{xxxxxxxxx}degree} \vspace{-1mm}\\
    \colorbox{LightGreen}{0:{\"a}\phantom{x}2:\textmu\phantom{\small{xxx}}0:{\ss}\phantom{x}2:\textdegree}
}

If you have the display corresponding to ROM Code 0, then make sure that lines~57--58 are uncommented and that lines~60--61 are commented-out:
\begin{lstlisting}[numberstyle=\color{gray}, numbers=left, firstnumber=55]
// A couple of handy non-ASCII characters if you want them for your display
// ROM CODE 0
const char MU = (char)0xE4;
const char DEGREE = (char)0xDF;
// ROM CODE 2
// const char MU = (char)0xB5;
// const char DEGREE = (char)0xB0;
\end{lstlisting}
On the other hand, if you have the display corresponding to ROM Code 2, then make sure that lines~57--58 are commented-out and that lines~60--61 are uncommented:
\begin{lstlisting}[numberstyle=\color{gray}, numbers=left, firstnumber=55]
// A couple of handy non-ASCII characters if you want them for your display
// ROM CODE 0
// const char MU = (char)0xE4;
// const char DEGREE = (char)0xDF;
// ROM CODE 2
const char MU = (char)0xB5;
const char DEGREE = (char)0xB0;
\end{lstlisting}
If you have neither, then contact the professor for the code to create custom characters for `\textmu' and `\textdegree' (if you plan to use these characters on your display).

After you have done so, comment-out the eight lines in \function{initialize_cart_system()} that generate that display.


\subsection{Detecting Key Presses and Releases}

Add code to \function{initialize_cart_system()} to register \function{handle_key_action()} as your handler for key presses and releases.
Use the \function{attachInterrupt()} and \function{digitalPinToInterrupt()} functions, as described in Section~5.3 of the datasheet.
Specify \lstinline{CHANGE} mode.

The starter code for \function{handle_key_action()} has debouncing code similar to what you saw in \function{handle_button_action()}.
After the busy-wait loop terminates, the \lstinline{key} variable will either hold a non-NUL character, which will tell you which key press triggered this interrupt,
or \lstinline{key} will hold `\\0', in which case the \lstinline{last_key} variable will tell you which key release triggered this interrupt.

\begin{lstlisting}[basicstyle=\small]
void handle_key_action(void) {
  debounce_interrupt({
    static char last_key = 0xF0;
    char key;
    while ((key = cowpi_get_keypress()) == last_key) {}   // busy-wait through the race condition
    /* Add code to respond to key presses and releases */

    last_key = key;
  });
}
\end{lstlisting}

As described in Requirements~\ref{spec:dPad} and \ref{spec:diagnosticDisplays}, your system needs to respond to key presses for some (but not all) of the keys on the numeric keypad,
and it needs to respond to key releases for only a few of the keys.
Until you design the logic for your cart controller, you might simply print a message to the console when these particular actions happen.

As a rule, you want your interrupt-handling code to run quickly.
You certainly don't want to take the time to send a message across the slow UART connection to the Arduino IDE's Serial Monitor.
When you introduce the logic for your cart controller, be sure to remove the print statements from your interrupt handler.

\subsection{Detecting Time}

You must use a timer interrupt, to implement Requirement~\ref{spec:dataRefreshRate}.
Use that same timer interrupt\ifbool{offerdecompositionhints}{, along with a \lstinline{static} variable the ISR,}{} to determine when to set the \lstinline{display_needs_refreshed} variable.

Requirement~\ref{spec:dataRefreshRate} identifies the timer period that you need.
Use the equation in the datasheet's Section~6.2 to iterate through possible comparison values and prescalers until you arrive at a prescaler that is available for Timer2 and a comparison value that fits in the available counter bits for that timer.
When you have done so, add code to \function{initialize_cart_system()}:
\begin{itemize}
    \item Create a \lstinline{cowpi_timer8bit_t} pointer (in the remaining steps, we will refer to this pointer as \lstinline{timer})
    \item Use the address offset listed in the datasheet's Section~6.5 to assign the appropriate address to \lstinline{timer}
    \item Use the datasheet's Tables~12--13 to select the WGM bits for ``Clear on Timer Compare'' mode and the CS bits for your prescaler;
    use these bits and Table~11 to generate a bit vector and assign that bit vector to \lstinline{timer->control}
    \item Subtract 1 from the comparison value and assign that to \lstinline{timer->compareA}
    \item Create a \lstinline{uint8_t} pointer (which we will refer to as \lstinline{timer_interrupt_masks}), and use the address offset listed in the datasheet's Section~6.6 to assign the appropriate address to \lstinline{timer_interrupt_masks}
    \item Treating the pointer as an array, index the \lstinline{timer_interrupt_masks} array with the timer number (2), and use the datasheet's Table~14 to assign the appropriate bit vector to enable the \texttt{TIMER\textit{n}\_COMPA\_vect} interrupt, where $n$ is the timer number
%    \item Create an ISR for the \lstinline{TIMER1_COMPA_vect} interrupt using the \function{ISR()} macro as described in Section~5.2 of the datasheet
\end{itemize}

Until you design the logic for your cart controller, you might simply leave the ISR's body empty except for the code that updates \lstinline{display_needs_refreshed}.


\subsection{Updating the Display}

As a rule, you want your interrupt-handling code to run quickly.
You certainly don't want to take the time to update the display module in your ISR\@.
Instead, include in your ISR enough code to determine when 256 nanofortnights have passed,
\ifbool{offerdecompositionhints}{
    that is, when the timer interrupt has fired 256 times,
}{}
and set the \lstinline{display_needs_refreshed} when that has happened.

Then, in the \function{control_cart()} function, find this code:
\begin{lstlisting}[basicstyle=\small]
  static char top_line[17];
  static char bottom_line[17];
  if (display_needs_refreshed) {
    display_needs_refreshed = false;
    /* Add to create the strings to be displayed */

  }
\end{lstlisting}

Place inside that \lstinline{if} statement's body any code that should run whenever the display needs to be updated.
You will probably want to use \function{sprintf()} to generate the strings to be displayed.
See Appendix~\ref{sec:formatStrings} for some discussion.

\subsection{Controlling the Cart}

Instead of hacking code together, you should first think about its design.
You might not get the design perfect initially, but you will have much less backtracking if you give thought to it first.

\subsubsection{Data Representations}

Consider, for example, Requirement~\ref{spec:importantData}.
Finding a suitable datatype for the speed should be straight-forward enough.
What about the heading and distance?
How can you maintain a precision of \textit{exactly} 0.000,001\textdegree and of \textit{exactly} 0.001~microfurlongs?

\ifbool{offerdecompositionhints}{
    An IEEE~754 binary floating point type will not allow you to maintain precisions of exactly 0.000,001\textdegree and 0.001~microfurlongs.
    An attempt to approximate those precisions will result in the display frequently showing the incorrect distance travelled -- admittedly for only 256 nanofortnights each time this happens, but you can do better.
    (Another problem is that, by default, \function{sprintf()} will not work with floating point values on microcontrollers, including the ATmega328P.
    We can change this with a compiler argument, but the Arduino IDE doesn't make it easy to do that.)
    %\paragraph{Hint} Review your notes from the part of the floating point lecture in which we discussed how decimal currency is handled. \\
    \paragraph{Hint} Review the ``Decimal Fixed Point Value Hack'' slides from the floating point lecture, and think about how to use integer types to represent 0.001~microfurlongs and 0.000,001\textdegree.
    \paragraph{Hint} There are 1,000~nanofurlongs in a microfurlong, and there are 1,000,000~microdegrees in a degree. \\
}{}

How will you keep track of whether the motor is engaged?
How will you keep track of what direction (if any) the cart is turning?
How will you keep track of which display should be displayed (see Requirement~\ref{spec:diagnosticDisplays})?

\subsubsection{Design the Displays}

Design the diagnostic displays that show the full variables you chose.
Also design the ``normal'' display that shows the speed in furlongs per fortnight, the distance in microfurlongs, and the heading in degrees.
Be sure to allow sufficient room on the ``normal'' display to show a speed up to 3,000~furlongs per fortnight, a distance up to 9,999~microfurlongs, and a heading up to 359\textdegree.

Place the code to select the display in the \function{handle_key_action()} function, and the code to generate the displays and send them to the display module in the \function{control_cart()} function.

\subsubsection{Speed}

Now think about how you will change the speed.
How will you handle the minimum and maximum speeds?
How will you handle low and high gear?
How will you handle the motor being engaged or not engaged?

What has to happen when you press the `0' key to brake?
What else has to happen when you press the brake?
What has to happen when you release the brake?
What else has to happen when you release the brake?

\ifbool{offerdecompositionhints}{
    \paragraph{Hint} The requirements state that the brake lights shall be illuminated while as the user is pressing the brake.
    You can correctly infer that once the user presses the brake, they continue to do so until they release the brake. \\
}{}

Now that you've given thought to how to control the cart's speed, write the code for accelerating, decelerating, and braking.

\subsubsection{Distance Travelled}

Now that you can change the cart's speed, you can change the distance travelled.
Add code to \function{ISR(TIMER2_COMPA_vect)} that will update the distance travelled whenever the timer interrupt fires.

\ifbool{offerdecompositionhints}{
    \paragraph{Hint} Assuming you designed the timer interrupt to fire every nanofortnight:
    \begin{align*}
        Change\_in\_distance    & = speed \times time \\
                                & = \mathtt{speed\_variable}\ \frac{furlong}{fortnight} \times 1\ nanofortnight \\
                                & = \mathtt{speed\_variable}\ nanofurlong \\
                                & = \mathtt{speed\_variable} \times 0.001\ microfurlong
    \end{align*}
}{}

\subsubsection{Turning}

While activating a turn signal indicates the intention to turn a particular direction, actually turning requires using the ``D-pad'' to indicate the turn direction and forward motion to determine how much the direction changes.
This means that the code for making turns will be spread across a couple of functions.

Don't forget that the turn signal should stop blinking when the user ends the turn.

\subsubsection{Final Review of Requirements}

Look again over the requirements in Section~\ref{sec:spec}.
Have you forgotten anything?

Check to make sure that you've removed any long-running code, such as \function{display_string()} and \function{printf()} from your interrupt handlers / ISRs.
