\subsection{Printing a IEEE~754-Compliant Value} \label{subsec:printing}

Examine the stub for \function{ieee754_to_string()}.
This function prints its \lstinline{ieee754_t} argument bit vector in hexadecimal, and then it is supposed to print the bit vector as a binary floating point value.
Making use of the query functions that you wrote, the stub prints the sign and handles the special cases.

Find this code:

\begin{lstlisting}
// The number is either Normal or Subnormal
unsigned int integer = 0;
uint32_t fraction = 0;
int exponent = 0;
char fraction_string[40];
/* DETERMINE THE INTEGER PORTION, THE FRACTION, AND THE EXPONENT */
\end{lstlisting}

Assign the appropriate values to \lstinline{integer}, \lstinline{fraction}, and \lstinline{exponent}.
The \lstinline{integer} is the implicit integer portion of the significand.
The \lstinline{fraction} is simply the fraction bits as they appear in the \lstinline{number}.
The \lstinline{exponent} is the two's complement representation of the exponent that you get after removing the bias from the \lstinline{number}'s \texttt{E} field.
Do not make any assignments to \lstinline{fraction_string} -- this is a buffer that will be used later.
The subsequent \function{sprintf()} call uses these variables to print the bit vector as a binary floating point value.

\paragraph*{Check Your Work}

Compile and run \texttt{\textbf{\textit{./floatlab}}}, trying a few values, starting with values whose IEEE~754 representation are easy to check.
For example:

\begin{verbatim}
        Enter a value, a two-operand arithmetic expression,
            "denormalize <value> <change exponent amount>",
            "renormalize <value> <change exponent amount>",
            or "quit": 1
        0x3f800000	+1.0000,0000,0000,0000,0000,000_{2} x 2^{0}

        Enter a value, ... or "quit": .25
        0x3e800000	+1.0000,0000,0000,0000,0000,000_{2} x 2^{-2}

        Enter a value, ... or "quit": 15.625
        0x417a0000	+1.1111,0100,0000,0000,0000,000_{2} x 2^{3}
\end{verbatim}

Try a few more on your own.
You can also try creating some bit vectors to see that the sign, significand, and exponent are all what you expect, based on your understanding of the IEEE~754 format.
For example:

\begin{verbatim}
        Enter a value, ... or "quit": 0xF22AAAAA
        0xf22aaaaa	-1.0101,0101,0101,0101,0101,010_{2} x 2^{101}
\end{verbatim}

Don't forget to check subnormal numbers, too.
For example:

\begin{verbatim}
        Enter a value, ... or "quit": 5e-40
        0x000571cc	+0.0000,1010,1110,0011,1001,100_{2} x 2^{-126}
\end{verbatim}

\subsection{\texttt{unnormal\_t} and its functions} \label{subsec:unnormal}

The \lstinline{unnormal_t} data type is a structure with fields for the sign bit, the integer portion, the fractional portion, and the exponent.
It also has flags for Not-a-Number and for Infinity.
Unlike the IEEE~754 format, an \lstinline{unnormal_t} value can have more than one bit in its integer portion.

You should \textit{not} access the \lstinline{unnormal_t} fields directly.
Instead, use the \function{unnormal()} macro to create an \lstinline{unnormal_t} value, the accessor functions that we provide to retrieve the fields, and the modifier functions that we provide to make adjustments in a value-preserving manner.

\textit{\textbf{NOTE:}} while you typically will see non-primitive variables passed by reference to C functions, the \lstinline{unnormal_t} functions use pass-by-value.
This is considerably less efficient in both time and memory (and there are other downsides beyond the scope of what you've learned so far);
however, passing the \lstinline{unnormal_t} variables by value ensures that there will be no aliased variables and also that memory management will be handled by the program stack.
This means that the variables passed as arguments to functions will be unchanged, and if an \lstinline{unnormal_t} variable is returned then it will be a modified copy of the original.
A consequence of this is that \textit{you must make sure that you always use the value returned by a function if you expect to use the effect of the function.}

You can, and should, look at the functions' signatures in \textit{unnormal.h}.
We summarize the functions here:

\begin{itemize}
    \item Creating and printing an \lstinline{unnormal_t} value (all necessary calls to these functions are in the starter code)
    \begin{description}
        \item[unnormal()] returns an \lstinline{unnormal_t} value (\textit{not} a pointer) based on the arguments provided.
            The argument list is a series of dot-prefixed named fields (such as \lstinline{.sign=0, .infinity=1}) whose meanings are the obvious ones from the class discussion about floating point numbers.
            \textit{\textbf{Note: }} the \lstinline{.exponent} argument, if included, is expected to be a two's complement value.
            \textit{\textbf{Note: }} the \lstinline{.fraction} argument, if included, is expected to be the numerator of $\frac{.fraction\ argument}{2^{64}}$ (\textit{i.e.}, the $2^{-1}$ bit is $bit_{63}$)\@.
        \item[unnormal\_to\_string] returns a string representation of the value.
            Because of the number of bits available in the \lstinline{unnormal_t} fields, the significand is represented in base-16, though the exponent is the exponent of 2.
    \end{description}
    \item Accessors
    \begin{description}
        \item[get\_sign()] returns 0 if the value is positive, 1 if the value is negative
        \item[get\_integer()] returns a \lstinline{uint64_t} storing the unsigned integer representation of the value's integer portion
        \item[get\_fraction()] returns a \lstinline{uint64_t} storing the numerator of $\frac{get\_fraction()}{2^{64}}$ (\textit{i.e.}, the $2^{-1}$ bit is $bit_{63}$)
        \item[get\_exponent()] returns a \lstinline{int16_t} storing the two's complement representation of the exponent
        \item[is\_infinite()] returns 0 if the value is finite (or NaN), 1 if the value is $\pm\infty$
        \item[is\_not\_a\_number()] returns 0 if the value is a valid number, 1 if the value is not a number
    \end{description}
    \item Bit shifts (some of these functions are equivalent to each other, to support whichever mental model works for you)
    \begin{description}
        \item[shift\_left\_once()] aka \function{decrement_exponent()}, aka \function{move_binary_point_to_the_right()} -- shifts the significand's bits to the left by one position, having the effect of moving the binary point to the right and decreasing the exponent by one
        \item[shift\_right\_once()] aka \function{increment_expoennt()}, aka \function{move_binary_point_to_the_left()} -- shifts the significand's bits to the right by one position, having the effect of moving the binary point to the left and increasing the exponent by one
        \item[shift\_left()] shifts the significand's bits to the left by a specified amount
        \item[shift\_right()] shifts the significand's bits to the right by a specified amount
    \end{description}
    \item Alignment functions (shifts bits by specifying the desired result instead of the amount)
    \begin{description}
        \item[place\_all\_bits\_in\_integer()] aka \function{prepare_for_arithmetic()} shifts the significand such that the least-significant \lstinline{1} bit is in the $2^0$ position, with a corresponding change in the exponent (the fraction, of course, will be 0)
        \item[set\_integer()] if possible, shifts the significand such that the resulting integer portion is the specified value, with a corresponding changes in the fraction and exponent
        \item[set\_exponent()] shifts the significand, with a corresponding change in the exponent, such that the resulting exponent is the specified value
    \end{description}
    \item Static Warnings (warnings based on the current bit positions)
    \begin{description}
        \item[multiplication\_is\_not\_recommended()] indicates that there are \lstinline{1} bits to the right of the binary point:
                if there are \lstinline{1} bits in the fraction, then there will be extra bookkeeping that you will be responsible for;
                we recommend that you attempt multiplication only when all \lstinline{1} bits are in the integer portion
        \item[multiplication\_is\_unreliable()] indicates that there are \lstinline{1} bits far enough to the left of the binary point that multiplication could yield a product whose integer portion exceeds the available bits
        \item[addition\_is\_unreliable()] indicates that there are \lstinline{1} bits far enough to the left of the binary point that addition could yield a sum whose integer portion exceeds the available bits
    \end{description}
    \item Dynamic Warnings (warnings that result from the last function call)
    \begin{description}
        \item[shift\_overflowed()] indicates that one or more \lstinline{1} bit shifted to the left beyond the available bits
        \item[shift\_underflowed()] indicates that one or more \lstinline{1} bit shifted to the right beyond the available bits
        \item[operation\_was\_not\_performed()] indicates that the previous function did not have the desired result, such as attempting to shift by a negative amount or attempt to set an integer value whose bits are not present in the significand
        \item[created\_number\_is\_improbable()] indicates that a call to \function{unnormal()} was made with all of the fraction's \lstinline{1} far enough from the binary point that it is unlikely to have been the intended value (because it is \textit{possible} that the requested fraction is also the intended fraction, an Unnormal value with the requested fraction was created)
    \end{description}
    \item Prediction Functions (warnings that indicate what will happen in the next operation)
    \begin{description}
        \item[fraction\_will\_carry\_into\_integer\_on\_addition()] indicates that if the next operation is addition with the specified values, then adding the fractions will carry into the integer portion, requiring you to add 1 to the sum's integer
        \item[fraction\_will\_borrow\_from\_integer\_on\_subtraction()] indicates that if the next operation is subtraction with the specified values, then subtracting the fractions will require ``borrowing'' from the integer portion, requiring you to subtract 1 to the difference's integer
        \item[left\_shift\_will\_make\_multiplication\_unreliable()] indicates that if the next function is \function{left_shift_once()} (or one of its aliases) then after that function call, \function{multiplication_is_unreliable()} will return \lstinline{true}
        \item[left\_shift\_will\_make\_addition\_unreliable()] indicates that if the next function is \function{left_shift_once()} (or one of its aliases) then after that function call, \\ \function{addition_is_unreliable()} will return \lstinline{true}
        \item[left\_shift\_will\_overflow()] indicates that if the next function is \function{left_shift_once()} (or one of its aliases) then after that function call, \function{shift_overflowed()} will return \lstinline{true}
        \item[right\_shift\_will\_underflow()] indicates that if the next function is \function{right_shift_once()} (or one of its aliases) then after that function call, \function{shift_underflowed()} will return \lstinline{true}
    \end{description}
\end{itemize}

\subsection{Denormalization}

The \function{denormalize()} function converts an IEEE~754-compliant \lstinline{ieee754_t} value into a format that facilitates arithmetic.
As described in Section~\ref{subsec:unnormal}, the \lstinline{unnormal_t} structure that is returned by \function{denormalize()} has fields for the sign bit, the integer portion, the fractional portion, and the exponent.
It also has flags for Not-a-Number and for Infinity.
Unlike the IEEE~754 format, an \lstinline{unnormal_t} value can have more than one bit in its integer portion.

In general, you can use code very similar to the code you wrote for \function{ieee754_to_string()} to extract the \lstinline{ieee754_t} value's bit fields.
The principal difference is that \function{ieee754_to_string()} expected you to use the \lstinline{fraction} exactly as you found it in the \lstinline{ieee754_t} value,
but in \function{denormalize()} you will need to left-shift the \lstinline{fraction} by several places such that the $2^{-1}$ bit is in $bit_{63}$ (the most-significant bit) of an \lstinline{uint64_t} bit vector, the $2^{-2}$ bit is in $bit_{62}$, and so on.
This is because the \function{unnormal()} function call expects the \lstinline{.fraction} named field to be the numerator of
\[\frac{.fraction\ argument}{2^{64}}\]

\subsubsection*{Examples}

Suppose that the number is $68588.0_{10} = 1.0000,1011,1110,11_{2} \times 2^{16}$.
In the IEEE~754 normal form, this is 0x4785'F600 = 0b0\underline{100'0111'1}000'0101'1111'0110'0000'0000 (we have underlined the $E$ field to help you follow the discussion).

In the \lstinline{unnormal_t} data structure, the \lstinline{.integer} is 1, the \lstinline{.exponent} is 16, and the \lstinline{.fraction} is 0x0BEC'0000'0000'0000.
Mathematically, that is the \lstinline{.fraction} field because the fraction field needs to be the numerator of
\[\frac{BEC,0000,0000,0000_{16}}{1,0000,0000,0000,0000_{16}}\]
That is why you need to left-shift the \lstinline{fraction} by several places (so that the $2^{-1}$ bit is in the most-significant bit) before passing it to the \function{denormalize()}'s function.

To see this from a practical perspective, let us consider some of the functions defined for \lstinline{unnormal_t}.
In these examples, assume that all numbers are stored in \lstinline{unnormal_t} data structures.
\begin{itemize}
    \item \textbf{\texttt{shift\_left\_once($1.0000,1011,1110,11_{2} \times 2^{16}$)}} will return a \textit{copy} of the original \lstinline{unnormal_t} data structure, except that the significand's bits have been left-shifted by one, and the fraction has been decreased by one;
        specifically, it will return \\ $10.0001,0111,1101,1_{2} \times 2^{15}$.
        (The original data structure will be unchanged.)
    \item \textbf{\texttt{shift\_right($10.0001,0111,1101,1_{2} \times 2^{15}$), 4}} will return \\
        $0.0010,0001,0111,1101,1_{2} \times 2^{19}$: the significand's bits have been right-shifted by four, and the fraction has increased by four.
    \item \textbf{\texttt{place\_all\_bits\_in\_integer($0.0010,0001,0111,1101,1_{2} \times 2^{19}$)}} will return \\
        $100,0010,1111,1011.0_{2} \times 2^{2}$.
    \item \textbf{\texttt{set\_exponent($100,0010,1111,1011.0_{2} \times 2^{2}$, 0)}} will return  \\
        $1,0000,1011,1110,1100.0_{2} \times 2^{0}$.
    \item \textbf{\texttt{set\_integer($1,0000,1011,1110,1100.0_{2} \times 2^{0}$, 1)}} will return  \\
        $1.0000,1011,1110,11_{2} \times 2^{16}$.
\end{itemize}


\paragraph*{Check Your Work}

We have provided a function that will print a \lstinline{unnormal_t} value.
When you run \texttt{\textbf{\textit{./floatlab}}}, you can specify that you want to denormalize a value, such as \texttt{\textbf{\textit{denormalize 12.375}}}.
The program will then print the value, based on the \lstinline{unnormal_t} fields.
Because there are 64 bits available for the integer portion and another 64 bits for the fractional portion, the \lstinline{unnormal_t} value will be printed in base-16:

\begin{verbatim}
        Enter a value, a two-operand arithmetic expression,
            "denormalize <value> <change exponent amount>",
            "renormalize <value> <change exponent amount>",
            or "quit": denormalize 12.375
        +0000000000000001.8c00000000000000_{16} x 2^{3}
\end{verbatim}

Because $12.375_{10} = 1.1000,11_{2} \times 2^3$, we can see that $1.8\mathrm{C}_{16} \times 2^3$ is correct.

When denormalizing, you can optionally specify an amount to change the exponent.
For example:

\begin{verbatim}
    Enter ... "denormalize <value> <change exponent amount>", ...
        or "quit": denormalize 12.375 -4
    +0000000000000018.c000000000000000_{16} x 2^{-1}

    Enter ... "denormalize <value> <change exponent amount>", ...
        or "quit": denormalize 12.375 3
    +0000000000000000.3180000000000000_{16} x 2^{6}
\end{verbatim}

If your \function{denormalize()} function works correctly without the exponent adjustment, then it will work with the exponent adjustment.
This feature is not useful to you \textit{now}, but you may find it useful if you need to debug your \function{normalize()} function.
