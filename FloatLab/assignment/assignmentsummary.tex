In this assignment, you will implement a 32-bit FPU\@.
You will implement floating point arithmetic using integer operations and bit-level operations, an encapsulated data type and a small amount of other starter code that we will provide, and other code that you will write.
Specifically, the FPU's operations are:
\begin{itemize}
    \item Some query functions to determine characteristics of a floating point value
    \item Normalization and denormalization
    \item Negation
    \item Addition and Subtraction
    \item Multiplication
    \item Division, only when the dividend's significand is a multiple of the divisor's significand
\end{itemize}
You will \textit{not} be required to implement a square root function, which is also commonly found in FPUs.

We have provided an encapsulated data type, \lstinline{unnormal_t}, to simplify some of the steps that will be necessary in your arithmetic functions.

\subsection{Constraints}

You may not use C's built-in floating point types or operations.
Specifically, you may not declare any variable to be of type \lstinline{float} or \lstinline{double}, you may not cast any value to either of these types, nor may you create floating point constants.

You may not use unions.
You may not use inline assembly code.

You may not use any libraries other than the \textit{libc} library.
This prohibition includes, but is not limited to, C's \textit{math} library.
If you find yourself \lstinline{#include}ing any header files other than \textit{fpu.h}, \textit{unnormal.h}, \textit{assert.h}, \textit{stdbool.h}, \textit{stdint.h}, and \textit{stdio.h}, you are probably violating this constraint.

The time-complexity of all operations must be linear (or constant) in the number of bits required to represent the values.
Inefficient solutions, such as (but not limited to) implementing multiplication by repeatedly adding the multiplicand to itself $multiplier$ times, or implementing division by repeatedly subtracting the divisor from the dividend, are prohibited.

You may not change the signatures of any functions that you are required to complete for this assignment.
If you write helper functions that are not required, then they may have any signature that you deem necessary
(your helper functions must also comply with the assignment's constraints).

You are permitted to use C's integer comparators, integer arithmetic, bitwise operations, bit shift operations.
You may cast between integer types.
You may use loops, conditionals, function calls, structs, and arrays.
And, of course, you may use any of the provided starter code and any code that you write by yourself for this assignment.

You can check whether you're declaring a \lstinline{float} or \lstinline{double}, and whether you're defining a \lstinline{union} by running the constraint-checking Python script: \\
\texttt{python constraint-check.py integerlab.json} \\
You can check whether you're creating a floating point constant with a simple \texttt{grep} command: \\
\verb#grep -e "\d\." -e "\.\d" fpu.c# \\
(Note that this \texttt{grep} command will detect false-positives in strings and comments, including two false-positives in the starter code.)

\ifbool{offerdecompositionhints}{
    \subsection{Problem Decomposition}

    \dots
}{}
