\subsection*{Normalize}

The \function{normalize()} function converts an \lstinline{unnormal_t} value into an IEEE~754-compliant format.
The function stub already handles zero and the flagged cases of Not-a-Number and Infinity.
The stub also handles the sign bit.

Your task is to handle:
\begin{itemize}
    \item Normal numbers, both those that already have exactly $1$ in the integer portion and those that need to be adjusted.
    \item Subnormal numbers, both those that can be directly converted and those that need to be adjusted.
    \item Cases that you will not be able to test until later:
    \begin{itemize}
        \item Numbers too great to be represented as normal numbers.
        \item Numbers too small to be represented as subnormal numbers.
        \item Rounding (when implemented, follow the IEEE~754 default of ``round-to-nearest-even.'')
    \end{itemize}
\end{itemize}

For the first two tasks, you will probably make use of \lstinline{unnormal_t}'s \function{set_integer()} and \function{set_exponent()} functions in addition to the functions that access the structure's fields.
Don't forget that the bit vector returned by \function{get_fraction()} is the numerator of $\frac{get\_fraction()}{2^{64}}$ and that \function{get_exponent()} returns the two's complement representation of the exponent.

\subsubsection*{Check Your Work}

When you run \texttt{\textbf{\textit{./floatlab}}}, you can specify that you want to renormalize a value, such as \texttt{\textbf{\textit{renormalize 12.375}}} and \texttt{\textbf{\textit{renormalize 12.375 6}}}.
The program will first denormalize the value.
It will then adjust the exponent by the specified amount (if an amount is specified).
Then it will send the result to \function{normalize()}.
Finally, it will print the original value and the \lstinline{ieee754_t} value returned by \function{normalize}.

For example:

\begin{verbatim}
Enter ... "renormalize <value> <change exponent amount>", ...
    or "quit": renormalize 12.375
expected: 12.3750000000_{10}	0x41460000	+1.1000,1100,0000,0000,0000,000_{2} x 2^{3}
actual:   12.3750000000_{10}	0x41460000	+1.1000,1100,0000,0000,0000,000_{2} x 2^{3}

Enter ... "renormalize <value> <change exponent amount>", ...
    or "quit": renormalize 12.375 6
expected: 12.3750000000_{10}	0x41460000	+1.1000,1100,0000,0000,0000,000_{2} x 2^{3}
actual:   12.3750000000_{10}	0x41460000	+1.1000,1100,0000,0000,0000,000_{2} x 2^{3}

Enter ... "renormalize <value> <change exponent amount>", ...
    or "quit": renormalize 0x00055000
expected: 0.0000000000_{10}	0x00055000	+0.0000,1010,1010,0000,0000,000_{2} x 2^{-126}
actual:   0.0000000000_{10}	0x00055000	+0.0000,1010,1010,0000,0000,000_{2} x 2^{-126}
\end{verbatim}