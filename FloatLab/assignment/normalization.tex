\subsection*{Encode}

The \function{encode()} function converts an \lstinline{unnormal_t} value into an IEEE~754-compliant format.
The function stub already handles zero and the flagged cases of Not-a-Number and Infinity.
The stub also handles the sign bit.

Your task is to handle:
\begin{itemize}
    \item Normal numbers, both those that already have exactly $1$ in the integer portion and those that need to be adjusted.
    \item Subnormal numbers, both those that can be directly converted and those that need to be adjusted.
    \item Cases that you will not be able to test until later:
    \begin{itemize}
        \item Numbers too great to be represented as normal numbers.
        \item Numbers too small to be represented as subnormal numbers.
        \item Rounding (when implemented, follow the IEEE~754 default of ``round-to-nearest-even.'')
    \end{itemize}
\end{itemize}

%For the first two tasks, you will probably make use of \lstinline{unnormal_t}'s \function{set_integer()} and \function{set_exponent()} functions in addition to the functions that access the structure's fields.
For the first task, the starter code already calls \function{set_integer(1)} for you, placing exactly \lstinline{1} to the left of the binary point.
For the second task, you will probably make use of \lstinline{unnormal_t}'s \function{set_exponent()} function.
For both of the first two tasks, you will also need to use \lstinline{unnormal_t}'s functions that access the structure's fields.
Don't forget that the bit vector returned by \function{get_fraction()} is the numerator of $\frac{get\_fraction()}{2^{64}}$ and that \function{get_exponent()} returns the two's complement representation of the exponent.

\ifbool{offerdecompositionhints}{
    \paragraph{Hint}
    In the \function{encode()} function, when your program reaches \\
    \lstinline{/* GENERATE THE APPROPRIATE BIT VECTOR AND PLACE IT IN RESULT */} \\
    then you know that \lstinline{number}'s ``true value'' is a non-zero, finite number.

    \begin{itemize}
        \item If the number is too great to be represented as a normal number, then it is indistinguishable from infinity.
            Suppose that the number \textit{can} be represented as a normal number -- what is the greatest exponent possible?
            Is \lstinline{number}'s exponent greater than that?
        \item If the number is too small to be represented as a normal number, then it might be representable as a subnormal number.
            Suppose that the number \textit{can} be represented as a normal number -- what is the least exponent possible?
            Is \lstinline{number}'s exponent less than that?
    \end{itemize}
}{}

\subsubsection*{Check Your Work}

When you run \texttt{\textbf{\textit{./floatlab}}}, you can specify that you want to recode a value, such as \texttt{\textbf{\textit{recode 12.375}}} and \texttt{\textbf{\textit{recode 12.375 6}}}.
The program will first decode the value.
It will then adjust the exponent by the specified amount (if an amount is specified).
Then it will send the result to \function{encode()}.
Finally, it will print the original value and the \lstinline{ieee754_t} value returned by \function{encode}.

For example:

\begin{verbatim}
Enter ... "recode <value>",
    or "quit": recode 12.375
expected: 12.3750000000_{10}	0x41460000	+1.1000,1100,0000,0000,0000,000_{2} x 2^{3}
actual:   12.3750000000_{10}	0x41460000	+1.1000,1100,0000,0000,0000,000_{2} x 2^{3}

Enter ... "recode <value>",
    or "quit": recode 0x00055000
expected: 0.0000000000_{10}	0x00055000	+0.0000,1010,1010,0000,0000,000_{2} x 2^{-126}
actual:   0.0000000000_{10}	0x00055000	+0.0000,1010,1010,0000,0000,000_{2} x 2^{-126}
\end{verbatim}

Try some other numbers to check that your \function{encode()} function correctly encodes the value that's in an \lstinline{unnormal_t} data structure.
