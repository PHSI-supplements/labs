\subsection{Adjust\_Exponent}

The \function{adjust_exponent()} function changes a \lstinline{unnormal_t} variable's exponent while preserving its value.
For example, these values are all equal but have different exponents:
\[
    1.1000,11_{2} \times 2^3 = 1,1000.11_{2} \times 2^{-1} = 0.0011,0001,1_{2} \times 2^6
\]

Write \function{adjust_exponent()} such that:
\begin{itemize}
    \item A positive \lstinline{amount} causes the \lstinline{number}'s exponent to increase.
        The virtual binary point moves to the left by \lstinline{amount} by taking that many least significant bit(s) from the integer portion (right-shifting the other integer bits) and making it/them the most significant bit(s) in the fractional portion (right-shifting the existing fractional bits).
    \item A negative \lstinline{amount} causes the \lstinline{number}'s exponent to decrease.
        The virtual binary point moves to the right by \lstinline{-amount} by taking that many most significant bit(s) from the fractional portion (left-shifting the other fractional bits) and making it/them the least significant bit(s) in the integer portion (left-shifting the existing fractional bits).
\end{itemize}

You can now use \function{adjust_exponent()} to cause two values to have the same exponent, or you can use the function to align the significant bits.
The \function{adjust_exponent()} function will not be graded directly.

\paragraph*{Check Your Work}

When you run \texttt{\textbf{\textit{./floatlab}}} and specify that you want to denormalize a value, you can also specify the amount by which you want to change the exponent.
(The amount must be an integer value.) For example:

\begin{verbatim}
    Enter ... "denormalize <value> <change exponent amount>", ...
        or "quit": denormalize 12.375 -4
    +0000000000000018.c000000000000000_{16} x 2^{-1}

    Enter ... "denormalize <value> <change exponent amount>", ...
        or "quit": denormalize 12.375 3
    +0000000000000000.3180000000000000_{16} x 2^{6}
\end{verbatim}

\subsection{Normalize}

The \function{normalize()} function converts an \lstinline{unnormal_t} value into an IEEE~754-compliant format.
The function stub already handles zero and the flagged cases of Not-a-Number and Infinity.
The stub also handles the sign bit.

Your task is to handle:
\begin{itemize}
    \item Normal numbers, both those that already have exactly $1$ in the integer portion and those that need to be adjusted.
    \item Subnormal numbers, both those that can be directly converted and those that need to be adjusted.
    \item Cases that you will not be able to test until later:
    \begin{itemize}
        \item Numbers too great to be represented as normal numbers.
        \item Numbers too small to be represented as subnormal numbers.
        \item Rounding (when implemented, follow the IEEE~754 default of ``round-to-nearest-even.'')
    \end{itemize}
\end{itemize}

\subsubsection*{Check Your Work}

When you run \texttt{\textbf{\textit{./floatlab}}}, you can specify that you want to renormalize a value, such as \texttt{\textbf{\textit{renormalize 12.375}}} and \texttt{\textbf{\textit{renormalize 12.375 6}}}.
The program will first denormalize the value.
It will then adjust the exponent by the specified amount (if an amount is specified).
Then it will send the result to \function{normalize()}.
Finally, it will print the original value and the \lstinline{ieee754_t} value returned by \function{normalize}.

For example:

\begin{verbatim}
Enter ... "renormalize <value> <change exponent amount>", ...
    or "quit": renormalize 12.375
expected: 12.3750000000_{10}	0x41460000	+1.1000,1100,0000,0000,0000,000_{2} x 2^{3}
actual:   12.3750000000_{10}	0x41460000	+1.1000,1100,0000,0000,0000,000_{2} x 2^{3}

Enter ... "renormalize <value> <change exponent amount>", ...
    or "quit": renormalize 12.375 6
expected: 12.3750000000_{10}	0x41460000	+1.1000,1100,0000,0000,0000,000_{2} x 2^{3}
actual:   12.3750000000_{10}	0x41460000	+1.1000,1100,0000,0000,0000,000_{2} x 2^{3}

Enter ... "renormalize <value> <change exponent amount>", ...
    or "quit": renormalize 0x00055000
expected: 0.0000000000_{10}	0x00055000	+0.0000,1010,1010,0000,0000,000_{2} x 2^{-126}
actual:   0.0000000000_{10}	0x00055000	+0.0000,1010,1010,0000,0000,000_{2} x 2^{-126}
\end{verbatim}