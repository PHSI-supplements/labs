Download \textit{\shortlabname.zip} or \textit{\shortlabname.tar} from \filesource\ and copy it to \runtimeenvironment.
Once copied, unpackage the file.
Three of the four files (\textit{fpu.h}, \textit{fpu.c}, and \textit{floatlab.c}) contain the starter code for this assignment.
The last file (\textit{Makefile}) tells the \texttt{make} utility how to compile the code.
To compile the program, type:

\texttt{make}

This will produce an executable file called \textit{floatlab}.

When you run the program with the command \texttt{\textbf{\textit{./floatlab}}}, you will be prompted:

\begin{verbatim}
    Enter a value, a two-operand arithmetic expression,
        "denormalize <value> <change exponent amount>",
        "renormalize <value> <change exponent amount>",
        or "quit":
\end{verbatim}

When you enter a value, if it is prepended with \texttt{\textbf{\textit{0x}}} then the parser will treat it as a bit vector;
otherwise, the parser will treat it as a floating point value.

For now, type \texttt{\textbf{\textit{quit}}} to exit the program.

\subsection{Description of FloatLab Files}

\subsubsection{floatlab.c}

Do not edit \textit{floatlab.c}.

This file contains the driver code for the lab, as well as a couple of helper functions.
It parses your input, calls the appropriate arithmetic function, and displays the output.

\subsubsection{fpu.h}\label{subsubsec:fpu.h}

Do not edit \textit{fpu.h}.

This header file contains two type definitions:
\begin{description}
    \item[ieee754\_t] is a 32-bit bit vector that represents a floating point number.
    The bit vector is to be interpreted as though it were a single-precision floating point value in the IEEE~754 format.
    The value may be in the normal form, in the subnormal form, or one of the special cases (NaN, infinity, or zero).
    \item[unnormal\_t] is a structure to hold the components of a floating point number that does not have to be in the IEEE~754 format.
    We expect that you will find the structure useful when performing arithmetic.
    The fields contain enough bits for you to produce exact results that can later be rounded to fit into the normal form.
    While you are not required to do so, we recommend that you place the ``ones'' bit ($2^0$) of the integer portion in the least significant bit of the \lstinline{integer_portion} field, the ``halves'' bit ($2^{-1}$) of the fractional portion in the most significant bit of the \lstinline{fractional_portion} field, and the remaining fractional bits in their appropriate locations relative to the ``halves'' bit.
    We recommend that you store absolute values of the integer and fractional portions, using the 1-bit \lstinline{sign} field to indicate whether the value is positive or negative.
    We further recommend that you use the \lstinline{exponent} field to store the exponent as a two's complement integer and not as a biased integer.
    There are \lstinline{is_infinite} and \lstinline{is_nan} 1-bit fields available to indicate special values.
\end{description}
The header file also contains several function declarations.
The requirements for these functions will be discussed later in this assignment.

\subsubsection{fpu.c}

This file contains stubs for constants and functions you need to create.
