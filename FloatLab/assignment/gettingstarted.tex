Download \textit{\shortlabname.zip} or \textit{\shortlabname.tar} from \filesource\ and copy it to \runtimeenvironment.
Once copied, unpackage the file.
Five of the six files (\textit{fpu.h}, \textit{fpu.c}, \textit{unnormal.h}, \textit{unnormal.c} and \textit{floatlab.c}) contain the starter code for this assignment.
The last file (\textit{Makefile}) tells the \texttt{make} utility how to compile the code.
To compile the program, type:

\texttt{make}

This will produce an executable file called \textit{floatlab}.

When you run the program with the command \texttt{\textbf{\textit{./floatlab}}}, you will be prompted:

\begin{verbatim}
    Enter a value, a two-operand arithmetic expression,
        "denormalize <value> <change exponent amount>",
        "renormalize <value> <change exponent amount>",
        or "quit":
\end{verbatim}

When you enter a value, if it is prepended with \texttt{\textbf{\textit{0x}}} then the parser will treat it as a bit vector;
otherwise, the parser will treat it as a floating point value.

For now, type \texttt{\textbf{\textit{quit}}} to exit the program.

\subsection{Description of FloatLab Files}

\subsubsection{floatlab.c}

Do not edit \textit{floatlab.c}.

This file contains the driver code for the lab, as well as a couple of helper functions.
It parses your input, calls the appropriate arithmetic function, and displays the output.

\subsubsection{fpu.h}\label{subsubsec:fpu.h}

Do not edit \textit{fpu.h}.

This header file contains a type definition.
\textbf{\texttt{ieee754\_t}} is a 32-bit bit vector that represents a floating point number.
The bit vector is to be interpreted as though it were a single-precision floating point value in the IEEE~754 format.
The value may be in the normal form, the subnormal form, or one of the special cases (NaN, infinity, or zero).

The header file also contains several function declarations.
The requirements for these functions will be discussed later in this assignment.

\subsubsection{unnormal.h \& unnormal.c}

Do not edit \textit{unnormal.h}.
Do not edit \textit{unnormal.c}.

These files provide an encapsulation of the \textbf{\texttt{unnormal\_t}} data type.
The data type and its functions are described in Section~\ref{subsec:unnormal}.

\subsubsection{fpu.c}

This file contains stubs for constants and functions you need to create.
\begin{itemize}
    \item Constants
    \begin{description}
        \item[SIGN\_BIT\_MASK] can be used to determine whether an \lstinline{ieee754_t} value is positive or negative
        \item[EXPONENT\_BITS\_MASK] can be used to isolate the bits used to encode the exponent in an \lstinline{ieee754_t} value
        \item[FRACTION\_BITS\_MASK] can be used to isolate the bits used for the fraction in an \lstinline{ieee754_t} value
        \item[EXPONENT\_BIAS] is the single-precision exponent bias
        \item[NUMBER\_OF\_FRACTION\_BITS] is the number of bits used for the fraction field in a single-precision floating point number
        \item[NAN] is a bit vector for an \lstinline{ieee754_t} value that is not a number
        \item[INFINITY] is the bit vector for an \lstinline{ieee754_t} value that is too great to be represented with the available bits
    \end{description}
    \item Query functions
    \begin{description}
        \item[is\_infinity()] reports whether an \lstinline{ieee754_t} value is positive or negative infinity
        \item[is\_nan()] reports whether an \lstinline{ieee754_t} value is a legal NaN bit vector
        \item[is\_zero()] reports whether an \lstinline{ieee754_t} value is positive or negative zero
        \item[is\_negative()] reports whether an \lstinline{ieee754_t} value is negative
    \end{description}
    \item Functions to examine \lstinline{ieee754_t} values
    \begin{description}
        \item[ieee754\_to\_string()] converts an \lstinline{ieee754_t} value into a meaningful string
        \item[denormalize()] converts an \lstinline{ieee754_t} value into an \lstinline{unnormal_t} value
    \end{description}
    \item Function to normalize and round \lstinline{ieee754_t} values
    \begin{description}
        \item[denormalize()] converts an \lstinline{unnormal_t} value into an \lstinline{ieee754_t} value, applying rounding as necessary
    \end{description}
    \item Arithmetic functions
    \begin{description}
        \item[add()] adds two \lstinline{ieee754_t} values
        \item[negate()] negates an \lstinline{ieee754_t} value
        \item[subtract()] uses \function{add()} and \function{negate()} to subtract an \lstinline{ieee754_t} value from another (this is a one-line function provided for you)
        \item[multiply()] multiplies two \lstinline{ieee754_t} values
        \item[divide()] divides an \lstinline{ieee754_t} value by another, under a limited set of circumstances
    \end{description}
\end{itemize}
