In this assignment, you will become more familiar with bit-level representations of floating point numbers.
You'll do this by implementing floating point arithmetic for 32-bit floating point numbers using only bitwise operators and integer arithmetic.

The instructions are written assuming you will edit and run the code on \runtimeenvironment.
If you wish, you may edit and run the code in a different environment;
be sure that your compiler suppresses no warnings, and that if you are using an IDE that it is configured for C and not C++.

\section*{Learning Objectives}

After successful completion of this assignment, students will be able to:
\begin{itemize}
    \item Identify the bit fields of an IEEE~754-compliant floating point number
    \item Obtain the value of an IEEE~754-compliant floating point number
    \item Manipulate IEEE~754-compliant floating point numbers in a meaningful way
    \item Apply IEEE~754 ``round-to-nearest-even'' rounding
\end{itemize}

\subsection*{Continuing Forward}

The familiarity you gain with the IEEE~754 format will pay off handsomely on the first exam.

\section*{During Lab Time}

During your lab period, the TAs will provide a refresher of the IEEE~754 format, with a particular emphasis on single-precision floating point numbers, and they will guide students through a discussion and discovery of useful bitmasks for this lab.
During the remaining time, the TAs will be available to answer questions.

Before leaving lab, \textit{at a minimum} complete \dots