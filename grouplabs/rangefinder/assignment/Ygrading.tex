When you have completed this assignment, upload \textit{alarm.c}, \textit{sensor.c}, \textit{user\_controls.c}, and \textit{shared\_variables.h} to \filesubmission.

%\policyforcodethatdoesnotcompile
\subsection*{No Credit for Uncompilable Code}
If the TA cannot create an executable from your code, then your code will be assumed to have no functionality.\footnote{
    At the TA's discretion, if they can make your code compile with \textit{one} edit (such as introducing a missing semicolon) then they may do so and then assess a 10\% penalty on the resulting score.
    The TA is under no obligation to do so, and you should not rely on the TA's willingness to edit your code for grading.
    If there are multiple options for a single edit that would make your code compile, there is no guarantee that the TA will select the option that would maximize your score.
}
Before turning in your code, be sure to compile and test your code on \runtimeenvironment\ with the original driver code, the original header file(s), and using the Arduino IDE\@.

\interruptlablatepolicy

\subsection*{Rubric}

This assignment is worth 60 points.

\textbf{\textit{We are still preparing the rubric.}}
\begin{description}
%    \item[Turn Signals] % 8
%    \rubricitem{1}{Pushbutton presses are detected with an external interrupt}
%    \rubricitem{1}{The pushbuttons' interrupt handler determines which button was pressed}
%    \rubricitem{1}{Timer interrupts for Timer1 are handled}
%    \rubricitem{1}{Timer1 interrupts every 750ms}
%    \rubricitem{1}{There is no long-running code (including, but not limited to, \function{display_string()} and \function{printf()}) in the interrupt handlers / ISRs, \textit{except} where the inclusion of print statements is the only mechanism to demonstrate that the interrupt handler determines which button was pressed}
%    \rubricitem{1}{Pressing the left button activates the left turn signal, and pressing the right button activates the right turn signal}
%    \rubricitem{2}{When the left (right) turn signal is active, the left (right) LED blinks by illuminating for 750ms and deluminating for 750ms, and the left (right) LED does not blink}
%
%    \item[Controlling the Cart] % 27
%    \rubricitem{2}{Matrix keypad presses and releases are detected with an external interrupt}
%    \rubricitem{1}{The keypad's interrupt handler determines which key (if any) was pressed}
%    \rubricitem{1}{The keypad's interrupt handler determines which key (if any) was released}
%    \rubricitem{2}{Timer interrupts for Timer2 are handled}
%    \rubricitem{2}{Timer2 interrupts every 1.208ms}
%    \rubricitem{1}{There is no long-running code (including, but not limited to, \function{display_string()} and \function{printf()}) in the interrupt handlers / ISRs, \textit{except} where the inclusion of print statements is the only mechanism to demonstrate that the interrupt handler determines which key was pressed or released}
%    \rubricitem{1}{Pressing the `2' key increases the speed by 1 (or 3 if in high gear) if and only if the motor is engaged}
%    \rubricitem{1}{Pressing the `8' key decreases the speed by 1 (or 3 if in high gear)}
%    \rubricitem{1}{Pressing the `0' key sets the speed to 0 and activates the brake lights}
%    \rubricitem{1}{If neither turn signal is active, then both LEDs illuminate steadily while the brake lights are active; if a turn signal is active, then the turn signal's LED will continue to blink while the opposite LED illuminates steadily}
%    \rubricitem{1}{Releasing the `0' key deactivates the brake lights}
%    \rubricitem{1}{The right switch selects low gear (left position) and high gear (right position), having an effect only when pressing the `2' or `8' key}
%    \rubricitem{1}{The left switch disengages (left position) and engages (right position) the motor, having an effect only when releasing the `0' key}
%    \rubricitem{1}{Pressing the `4' (`6') key causes the cart to turn left (right) when and only when the cart is in motion}
%    \rubricitem{1}{Releasing the `4' or `6' key causes the turn to stop and deactivates the turn signal}
%    \rubricitem{1}{Pressing the `A', `B', `C', or `D' key causes the appropriate display to appear on the display module}
%    \rubricitem{\textonehalf}{Releasing `2', `8', `A', `B', `C', or `D' has no effect}
%    \rubricitem{\textonehalf}{Pressing or releasing `1', `3', `5', `7', `9', `*', or `\#' has no effect}
%    \rubricitem{1}{The speed is tracked at $1~fpf$; that is, the `A' display and the `B' display show the same speed value}
%    \rubricitem{1}{The distance is tracked at $0.001~\mu fur$; that is, the `C' display shows 3 decimal digits to the right of the $1~\mu fur$ position}
%    \rubricitem{1}{The direction is tracked at $0.000,001\degree$; that is, the `D' display shows 6 decimal digits to the right of the $1\degree$ position}
%    \rubricitem{1}{The speed is strictly limited to $0 \leq speed \leq 3,000~fpf$, and the direction is strictly limited to $0\degree \leq speed < 360\degree$}
%    \rubricitem{1}{When turning, the direction changes by $\pm 0.3\degree$ for each microfurlong travelled while turning (the sign of the change being determined by the turn direction)}
%    \rubricitem{1}{The distance and direction are updated every 1.208ms}
%    \rubricitem{1}{The display refreshes every $256 \times 1.208ms = 309.248ms$}

    \item[Bonuses]
    \bonusitem{2}{Get assignment checked-off by TA or professor during office hours before it is due (you cannot get both check-off bonuses)}
    \bonusitem{1}{Get assignment checked-off by TA at \textit{start} of your scheduled lab immediately after it is due (your code must be uploaded to \filesubmission\ \textit{before} it is due; you cannot get both bonuses)}

    \item[Penalties]
%    \penaltyitem{1}{The turn signals are set by means other than detecting button presses in an interrupt handler}
%    \penaltyitem{2}{The blink rate is set by means other than detecting a timer interrupt}
%    \penaltyitem{9}{The directional controls and display controls are managed by means other than detecting key presses/releases in an interrupt handler}
%    \penaltyitem{3}{The time to update distance and direction, and to refresh the display, is determined by means other than detecting a timer interrupt}
    \spaghetticodepenalties{1}
\end{description}
