Download the zip file or tarball from \filesource.
Once downloaded, unpackage the file and open the project in the Arduino IDE\@.

If your display module's I$^2$C address is not 0x27, then edit \textit{i2c\_address.h}, and replace \lstinline{0x27} with the correct I$^2$C address for your display module.

\subsection{Description of RangeFinder Files}

\subsubsection{RangeFinder.ino}

Do not edit \textit{InterruptLab.ino}.

This file contains the code to configure your Cow Pi for this lab.

\subsubsection{i2c\_address.h}

This file exists solely to store the display module's I$^2$C address.
You do not need to edit it further.

\subsubsection{supplement.h}

Do not edit \textit{supplement.h}.

This file contains code that we hope will make it easier for you to focus on this assignment's learning objectives by taking care of a couple of the complexities.
These functions are not yet in the CowPi library but soon may be, depending on how well they work for you.

\subsubsection{user\_controls.h \& user\_controls.c}

Do not edit \textit{user\_controls.h}

The \textit{user\_controls.c} file is where you will get inputs from the user.

\subsubsection{sensor.h \& sensor.c}

Do not edit \textit{sensor.h}

The \textit{sensor.c} file is where you will control the ultrasonic echo sensor and compute any detected objects' distance and speed.

\subsubsection{alarm.h \& alarm.c}

Do not edit \textit{alarm.h}

The \textit{alarm.c} file is where you will control the piezoelectric disc and manage the chirping and strobing of a proximity alarm.

\subsubsection{shared\_variables.h}

The \textit{shared\_variables.h} header file is where you will place any types that you define and where you will externalize any global variables that need to be used by more than one \textit{.c} file.
