In this assignment, you will write code for \runtimeenvironment\ that will use new electronic devices to interact with the physical world.

The instructions are written assuming you will edit the code in the Arduino IDE and run it on \runtimeenvironment, constructed according to the pre-lab instructions.
If you wish, you may edit the code in a different environment; however, our ability to provide support for problems with other IDEs is limited.

\section*{Learning Objectives}

After successful completion of this assignment, students will be able to:
\begin{itemize}
    \item Work collaboratively on a hardware/software project
    \item Design and implement a simple embedded system
    \item Expand their programming knowledge by consulting documentation
\end{itemize}

\subsection*{Continuing Forward}

This penultimate lab assignment does not contribute to the final lab assignment.
By integrating elements of what you learned in this course, and by demonstrating that you can review documentation to learn on your own, to design a small embedded system, you will show how much progress you have made this semester.

\section*{During Lab Time}

During your lab period, coordinate with your group partner(s) to decide on your working arrangements.
Unless you're only going to work on the assignment when you're together, you may want to set up a private Git repository that is shared with your partner(s).
With your partner(s), modify your hardware kit as described in Section~\ref{sec:hardwareMods}.
Then, think through your system's design and begin implementing it.
The TAs will be available for questions.
