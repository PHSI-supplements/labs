%! suppress = LabelConvention
\begin{enumerate}
    \item Definitions
        \begin{description}
            \item[Threshold Range] The distance at which a detected object will cause the system to produce an audible alarm
            \item[Strobe] An illumination of the right LED for 50ms
            \item[Chirp] A 5kHz tone lasting 50ms
        \end{description}
    \item \label{spec:modes} The slide-switches shall control the mode of operation.
        \begin{enumerate}
            \item Both switches in the \textit{left position}: Normal Operation
            \item The \textbf{left switch} in the \textit{right position} and the \textbf{right switch} in the \textit{left position}: Single-Pulse Operation
            \item Both switches in the \textit{right position}: Threshold Adjustment
            \item The \textbf{left switch} in the \textit{left position} and the \textbf{right switch} in the \textit{right position}: Continuous Tone
        \end{enumerate}
    \item \label{spec:continuousTone} When the range finder and alarm system is in \textbf{Continuous Tone} mode:
        \begin{enumerate}
            \item The system shall not detect the range of nearby objects: it shall neither emit nor detect ultrasound
            \item The system shall not illuminate the right LED
            \item The system shall produce a continuous 5kHz audible tone
        \end{enumerate}
    \item \label{spec:thresholdAdjustment} When the range finder and alarm system is in \textbf{Threshold Adjustment} mode:
        \begin{enumerate}
            \item The system shall not detect the range of nearby objects: it shall neither emit nor detect ultrasound
            \item The system shall not produce an audible tone
            \item The system shall not illuminate the right LED
            \item The system shall prompt the user on the \textbf{display module} to enter the threshold range, in centimeters
                \begin{itemize}
                    \item You may assume that the user understands what the system does: the prompt must be \textit{meaningful}, but it may be \textit{succinct}
                \end{itemize}
            \item The user shall be able to enter the threshold range, in centimeters, using the numeric keypad
                \begin{enumerate}
                    \item The user will enter the range in decimal
                    \item The system shall echo the user's input, digit-by-digit, on the \textbf{display module} as they type their input
                    \item The user will indicate that they have finished entering their input by pressing the `\#' key
                \end{enumerate}
            \item After the user has completed their input, then any value less than 50cm or greater than 400cm shall be rejected as an invalid threshold;
                the system shall display a helpful error message on the \textbf{display module} and re-prompt the user to enter the threshold range
            \item After the user has completed their input, and if the input is valid, then the system shall display ``Threshold (value)cm'' on the \textbf{display module}, where ``(value)'' shall be the numeric threshold range in centimeters;
                this shall be displayed until the user takes the system out of the Threshold Adjustment mode
        \end{enumerate}
    \item \label{spec:singlePulseOperation} When the range finder and alarm system is in \textbf{Single-Pulse Operation} mode: \\
%        {\footnotesize Note: the references here to ``the pushbutton'' refer to the Cow Pi's original \textbf{left pushbutton}}
        \begin{enumerate}
            \item The system shall neither emit nor receive ultrasound, shall not produce an audible tone, and shall not illuminate an LED, \textit{until} the pushbutton has been pressed
            \item Whenever the user presses the \textbf{left pushbutton}, the system shall emit one ultrasonic pulse
            \item If the system does not receive that pulse's echo, the system shall resume waiting for the user to press the pushbutton
            \item \label{spec:singlePulseResponse} If the system does receive that pulse's echo:
                \begin{enumerate}
                    \item The system shall strobe the \textbf{right LED} once
                    \item If the object's distance is less than the threshold range then the system shall emit one chirp
                    \item The system shall then resume waiting for the user to press the pushbutton
                \end{enumerate}
        \end{enumerate}
    \item \label{spec:normalOperation} When the range finder and alarm system is in \textbf{Normal Operation} mode, the system shall repeatedly:
        \begin{enumerate}
            \item \label{spec:emitUltrasound} Emit an ultrasonic pulse and take no further action until either the system receives an echo or the system can establish that it will not receive an echo
            \item If the system can establish that it will not receive an echo:
                \begin{enumerate}
                    \item The system shall stop strobing the right LED and chirping if it previously was doing so
                    \item The system shall repeat the action in Requirement~\ref{spec:emitUltrasound}
                        {\footnotesize
                        \begin{itemize}
                             \item Repeating the action may be postponed until after necessary quiescent periods
                        \end{itemize}}
                \end{enumerate}
            \item If the system receives an echo:
                \begin{enumerate}
                    \item The system shall compute and display the object's distance, in centimeters
                    \item The system shall compute and display the object's rate of approach, in centimeters per second
                    \item The system shall strobe the \textbf{right LED} at the rate described by Table~\ref{tab:alarmPeriods}
                    \item If the object's distance is less than the threshold range, the system shall emit chirps at the rate described by Table~\ref{tab:alarmPeriods}
                    \item After the system has completed the speed and distance calculations, it shall repeat the action in Requirement~\ref{spec:emitUltrasound}
                        {\footnotesize
                        \begin{itemize}
                            \item Repeating the action may be postponed until after necessary quiescent periods
                        \end{itemize}}
                \end{enumerate}
        \end{enumerate}
    \item All mechanical inputs shall be properly debounced
    \item The system shall always be responsive to user input
        \begin{itemize}
            \item The user will never press two keys on the numeric keypad at the same time
            \item The system may ignore changes to the slide-switches while the user enters a new threshold range
            \item The system may block while displaying an error message
            \item But for those exceptions, there shall be no noticeable lag when responding to an input
        \end{itemize}
\end{enumerate}

\begin{table}
    \centering
    \begin{tabular}{||c|c|c|c||} \hline\hline
        \multirow{2}{*}{distance}   & LED on /         & LED off /         & total     \\
                                    & Piezo sounding    & Piezo silent      & period    \\ \hline\hline
        $distance \geq 250cm$          & $50ms$            & $2450ms$          & $2500ms$  \\ \hline
        $200cm \leq distance < 250cm$  & $50ms$            & $1950ms$          & $2000ms$  \\ \hline
        $150cm \leq distance < 200cm$  & $50ms$            & $1450ms$          & $1500ms$  \\ \hline
        $100cm \leq distance < 150cm$  & $50ms$            & $ 950ms$          & $1000ms$  \\ \hline
        $ 50cm \leq distance < 100cm$  & $50ms$            & $ 700ms$          & $ 750ms$  \\ \hline
        $ 25cm \leq distance <  50cm$  & $50ms$            & $ 450ms$          & $ 500ms$  \\ \hline
        $ 10cm \leq distance <  25cm$  & $50ms$            & $ 200ms$          & $ 250ms$  \\ \hline
        $distance < 10cm$           & $50ms$            & $  75ms$          & $ 125ms$  \\ \hline\hline
    \end{tabular}
    \caption{Strobe and Chirp periods for various distances to an object}\label{tab:alarmPeriods}
\end{table}