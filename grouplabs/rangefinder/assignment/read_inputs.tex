\subsection{Examining the Starter Code}

\paragraph{user\_controls.c}
The \function{initialize_controls()} function is where you'll place any alarm-related code that needs to be run once when the program starts.
The \function{manage_controls()} function is where you'll place any alarm-related code that needs to run with every iteration of the program's main loop.
You may, of course, add helper functions.

\paragraph{shared\_variables.h}
This is the only header file you will turn in, so if you need to share any \lstinline{enum}s, \lstinline{struct}s, or variables between \textit{.c} files, place them in here and not in the other header files, so that we can compile your code.
When you need to share a variable between \textit{.c} files, declare it in exactly one \textit{.c} file (preferably the one that the variable best coheres to) without the \lstinline{extern} keyword,
and then externalize it by declaring it again with the \lstinline{extern} keyword in \textit{shared\_variables.h}.
This approach will create just one global symbol for the variable in the program while making it ``visible'' to the code in the other \textit{.c} files.

\subsection{Read the Switches}

Create a way to track which mode the system is in (Requirement~\ref{spec:modes}).
Be sure to declare it not only in \textit{user\_controls.c} (without the \lstinline{extern} modifier) but also in \textit{shared\_variables.h} (with the \lstinline{extern} modifier).
\ifbool{offerdecompositionhints}{
    \textbf{Hint:} If you need to, review InterruptLab's discussion about state machines on page 9 of \textit{interruptlab.pdf}.
}{}
Add code to \function{manage_controls()} to poll the positions of the slide-switches and set the system's mode accordingly.

\subsection{Read the Pushbutton} \label{subsec:readPushbutton}

As noted in Requirement~\ref{spec:singlePulseOperation}, if the system is in Single-Pulse Operation mode and the user presses the (left) pushbutton, they want to emit exactly one ultrasound pulse to get the distance to an object.
\begin{figure}[h]
    \centering
    \includegraphics[width=10cm]{internet_images/one_ping_only}
    \caption{Verify our range to target. One ping only. \\ \tiny Image by Paramount Pictures Corporation \label{fig:onePingOnly}}
\end{figure}

Create a variable to indicate that the user has requested a ping.
Be sure to declare it not only in \textit{user\_controls.c} (without the \lstinline{extern} modifier) but also in \textit{shared\_variables.h} (with the \lstinline{extern} modifier).
Add code to \function{initialize_controls()} to set this variable initially to \lstinline{false}.
Add code to \function{manage_controls()} to set this variable to \lstinline{true} when and only when the use presses the button while in Single-Pulse Operation mode.
The variable should be set to \lstinline{true} only \textit{once} per press.
(Do \textit{not} set it to \lstinline{true} just because the user is still holding the button down.)

Do not worry about setting this variable to \lstinline{false} yet;
that will come later.

\vspace{1cm}

You can now continue to work through the remainder of the lab with your partner, perhaps pair programming,
or you can decide to have one partner work on Section~\ref{sec:distance} and the other work on Section~\ref{sec:sound}.
Regardless, you will need to work together on Section~\ref{sec:integration}, as this is when you will integrate the work from Sections~\ref{sec:distance}--\ref{sec:sound}.
