%%
%% CombinationLock GroupLab (c) 2022 Christopher A. Bohn
%%
%% Licensed under the Apache License, Version 2.0 (the "License");
%% you may not use this file except in compliance with the License.
%% You may obtain a copy of the License at
%%     http://www.apache.org/licenses/LICENSE-2.0
%% Unless required by applicable law or agreed to in writing, software
%% distributed under the License is distributed on an "AS IS" BASIS,
%% WITHOUT WARRANTIES OR CONDITIONS OF ANY KIND, either express or implied.
%% See the License for the specific language governing permissions and
%% limitations under the License.
%%

%%
%% (c) 2021 Christopher A. Bohn
%%

\documentclass[12pt]{article}

\usepackage{fullpage}
\usepackage{fancyhdr}
\usepackage[procnames]{listings}
\usepackage{hyperref}
\usepackage{textcomp}
\usepackage{bold-extra}
\usepackage[dvipsnames]{xcolor}
\usepackage{etoolbox}


% Customize the semester (or quarter) and the course number

\newcommand{\courseterm}{Spring 2022}
\newcommand{\coursenumber}{CSCE 231}

% Customize how a typical lab will be managed;
% you can always use \renewcommand for one-offs

\newcommand{\runtimeenvironment}{your account on the \textit{csce.unl.edu} Linux server}
\newcommand{\filesource}{Canvas or {\footnotesize$\sim$}cse231 on \textit{csce.unl.edu}}
\newcommand{\filesubmission}{Canvas}

% These are placeholder commands and will be renewed in each lab

\newcommand{\labnumber}{}
\newcommand{\labname}{Lab \labnumber\ Assignment}
\newcommand{\shortlabname}{}
\newcommand{\duedate}{}

% Individual or team effort

\newcommand{\individualeffort}{This is an individual-effort project. You may discuss concepts and syntax with other students, but you may discuss solutions only with the professor and the TAs. Sharing code with or copying code from another student or the internet is prohibited.}
\newcommand{\teameffort}{This is a team-effort project. You may discuss concepts and syntax with other students, but you may discuss solutions only with your assigned partner(s), the professor, and the TAs. Sharing code with or copying code from a student who is not on your team, or from the internet, is prohibited.}
\newcommand{\freecollaboration}{In addition to the professor and the TAs, you may freely seek help on this assignment from other students.}
\newcommand{\collaborationrules}{}

% Do you care about software engineering?

\providebool{allowspaghetticode}

\setbool{allowspaghetticode}{false}

\newcommand{\softwareengineeringfrontmatter}{
    \ifboolexpe{not bool{allowspaghetticode}}{
        \section*{No Spaghetti Code Allowed}
        In the interest of keeping your code readable, you may \textit{not} use
        any \lstinline{goto} statements, nor may you use any \lstinline{break}
        statements to exit from a loop, nor may you have any functions
        \lstinline{return} from within a loop.
    }{}
}

\newcommand{\spaghetticodepenalties}[1]{
    \ifboolexpe{not bool{allowspaghetticode}}{
        \penaltyitem{1}{for each \lstinline{goto} statement, \lstinline{break}
            statement used to exit from a loop, or \lstinline{return} statement
            that occurs within a loop.}
    }{}
}

% You shouldn't need to customize these,
% but you can if you like

\lstset{language=C, tabsize=4, upquote=true, basicstyle=\ttfamily}
\newcommand{\function}[1]{\textbf{\lstinline{#1}}}
\setlength{\headsep}{0.7cm}
\hypersetup{colorlinks=true}

\newcommand{\startdocument}{
    \pagestyle{fancy}
    \fancyhf{}
    \lhead{\coursenumber}
    \chead{\ Lab \labnumber: \labname}
    \rhead{\courseterm}
    \cfoot{\shortlabname-\thepage}

	\begin{document}
	\title{\ Lab \labnumber}
	\author{\labname}
	\date{Due: \duedate}
	\maketitle

    \textit{\collaborationrules}
}

\newcommand{\rubricitem}[2]{\item[\underline{\hspace{1cm}} +#1] #2}
\newcommand{\bonusitem}[2]{\item[\underline{\hspace{1cm}} Bonus +#1] #2}
\newcommand{\penaltyitem}[2]{\item[\underline{\hspace{1cm}} -#1] #2}

%%
%% labs/common/semester.tex
%% (c) 2021-22 Christopher A. Bohn
%%
%% Licensed under the Apache License, Version 2.0 (the "License");
%% you may not use this file except in compliance with the License.
%% You may obtain a copy of the License at
%%     http://www.apache.org/licenses/LICENSE-2.0
%% Unless required by applicable law or agreed to in writing, software
%% distributed under the License is distributed on an "AS IS" BASIS,
%% WITHOUT WARRANTIES OR CONDITIONS OF ANY KIND, either express or implied.
%% See the License for the specific language governing permissions and
%% limitations under the License.
%%


% Customize the semester (or quarter) and the course number

\newcommand{\courseterm}{Fall 2022}
\newcommand{\coursenumber}{CSCE 231}

% Customize how a typical lab will be managed;
% you can always use \renewcommand for one-offs

\newcommand{\runtimeenvironment}{your account on the \textit{csce.unl.edu} Linux server}
\newcommand{\filesource}{Canvas or {\footnotesize$\sim$}cse231 on \textit{csce.unl.edu}}
\newcommand{\filesubmission}{Canvas}

% Customize for the I/O lab hardware

\newcommand{\developmentboard}{Arduino Nano}
%\newcommand{\serialprotocol}{SPI}
\newcommand{\serialprotocol}{I2C}
%\newcommand{\displaymodule}{MAX7219digits}
%\newcommand{\displaymodule}{MAX7219matrix}
\newcommand{\displaymodule}{LCD1602}

\setbool{usedisplayfont}{true}

\newcommand{\obtaininghardware}{
    The EE Shop has prepared ``class kits'' for CSCE 231; your class kit costs \$30.
    The EE Shop is located at 122 Scott Engineering Center and is open M-F 7am-4pm. You do not need an appointment.
    You may pay at the window with cash, with a personal check, or with your NCard.
    The EE shop does \textit{not} accept credit cards.
}

% Update to reflect the CS2 course(s) at your institute

\newcommand{\cstwo}{CSCE~156, RAIK~184H, or SOFT~161}

% Do you care about software engineering?

\setbool{allowspaghetticode}{false}

% Which assignments are you using this semester, and when are they due?

\newcommand{\pokerlabnumber}{1}
\newcommand{\pokerlabcollaboration}{
    Sections~\ref{sec:connecting}, \ref{sec:terminology}, \ref{sec:gettingstarted}, \ref{subsec:typesofpokerhands}, and~\ref{subsec:studythecode}: \freecollaboration
    Sections~\ref{sec:completingcard} and~\ref{subsec:completepoker}: \individualeffort
}
\newcommand{\pokerlabdue}{Week of August 29, before the start of your lab section}

\newcommand{\keyboardlabnumber}{2}
\newcommand{\keyboardlabcollaboration}{\individualeffort}
\newcommand{\keyboardlabdue}{Week of January 31, before the start of your lab section}

\newcommand{\pointerlabnumber}{3}
\newcommand{\pointerlabcollaboration}{\individualeffort}
\newcommand{\pointerlabdue}{Week of February 7, before the start of your lab section}

\newcommand{\integerlabnumber}{4}
\newcommand{\integerlabcollaboration}{\individualeffort}
\newcommand{\integerlabdue}{Week of February 14, before the start of your lab section}

\newcommand{\floatlabnumber}{5}
\newcommand{\floatlabcollaboration}{\individualeffort}
\newcommand{\floatlabdue}{soon}

\newcommand{\addressinglabnumber}{6}
\newcommand{\addressinglabcollaboration}{\individualeffort}
\newcommand{\addressinglabdue}{Week of February 28, before the start of your lab section}

%bomblab was 7
%attacklab was 8

\newcommand{\pollinglabnumber}{9}
\newcommand{\pollinglabcollaboration}{\individualeffort}
\newcommand{\pollinglabdue}{Week of April 11, before the start of your lab section}
\newcommand{\pollinglabenvironment}{your \developmentboard-based class hardware kit}

\newcommand{\ioprelabnumber}{\pollinglabnumber-prelab}
\newcommand{\ioprelabcollaboration}{\freecollaboration}
\newcommand{\ioprelabdue}{Before the start of your lab section on April 5 or 6}

\newcommand{\interruptlabnumber}{10}
\newcommand{\interruptlabcollaboration}{\individualeffort}
\newcommand{\interruptlabdue}{Week of April 18, before the start of your lab section}
\newcommand{\interruptlabenvironment}{your \developmentboard-based class hardware kit}

\newcommand{\capstonelab}{ComboLock}    % this will come into play when we generalize capstonelab
\newcommand{\capstonelabnumber}{11}
\newcommand{\capstonelabcollaboration}{\teameffort}
\newcommand{\capstonelabdue}{Week of May 2, Before the start of your lab section\footnote{See Piazza for the due dates of teams with students from different lab sections.}}
\newcommand{\capstonelabenvironment}{your \developmentboard-based class hardware kit}

\newcommand{\memorylabnumber}{12}
\newcommand{\memorylabcollaboration}{This is an individual-effort project. You may discuss the nature of memory technologies and of memory hierarchies with classmates, but you must draw your own conclusions.}
\newcommand{\memorylabdue}{Week of May 2, at the end of your lab section}
\newcommand{\memorylabenvironment}{your \developmentboard-based class hardware kit and your account on the \textit{csce.unl.edu} Linux server}

% Labs not used this semester

\newcommand{\concurrencylabnumber}{XX}
\newcommand{\concurrencylabcollaboration}{\individualeffort}
\newcommand{\concurrencylabdue}{not this semester}

\newcommand{\ssbcwarmupnumber}{XX}
\newcommand{\ssbcwarmupcollaboration}{\freecollaboration}
\newcommand{\ssbcwarmupdue}{not this semester}

\newcommand{\ssbcpollingnumber}{XX}
\newcommand{\ssbcpollingcollaboration}{\individualeffort}
\newcommand{\ssbcpollingdue}{not this semester}

\newcommand{\ssbcinterruptnumber}{XX}
\newcommand{\ssbcinterruptcollaboration}{\individualeffort}
\newcommand{\ssbcinterruptdue}{not this semester}

\usepackage{enumitem}
% \usepackage{graphicx}
\usepackage{addfont}
\addfont{OT1}{d7seg}{\dviiseg}
% \addfont{OT1}{deseg}{\deseg}
% \usepackage{subfig}
% \usepackage{wrapfig}
% \usepackage{multicol}
% \usepackage{caption}
% \captionsetup{width=.8\linewidth}

% \lstset{language=c, numbers=left, showstringspaces=false,
%     moredelim = [s][\ttfamily]{/*}{*/} % I shouldn't need this parameter!
%     }



\renewcommand{\labnumber}{\capstonelabnumber}
\renewcommand{\labname}{Implementing an Electronic Combination Lock}
\renewcommand{\shortlabname}{combolock -- grouplab}
\renewcommand{\collaborationrules}{\capstonelabcollaboration}
\renewcommand{\duedate}{\capstonelabdue}
\newcommand{\nano}{\developmentboard} % TODO: replace \nano with \developmentboard
\renewcommand{\runtimeenvironment}{\capstonelabenvironment}
\pagelayout
\begin{document}
\labidentifier

In this assignment, you will collaboratively write code for \runtimeenvironment\
to implement a simple embedded system. Specifically, you will immplement an
electronic combination lock.

The instructions are written assuming you will edit the code in the Arduino IDE
and run it on \runtimeenvironment, constructed according to the pre-lab
instructions. If you wish, you may edit the code in a different environment;
however, our ability to provide support for problems with other IDEs is limited.

Please familiarize yourself with the entire assignment before beginning.
Section~\ref{sec:FunctionalSpecification} has the functional specification of
the system system you will develop. Section~\ref{sec:Constraints} describes
the few implementation constraints. Section~\ref{sec:AccessingEEProm} describes
your options for storing data on and retrieving data from the \nano's EEPROM.

\section*{Learning Objectives}

After successful completion of this assignment, students will be able to:
\begin{itemize}
\item Work collaboratively on a hardware/software project
\item Design and implement a simple embedded system
\item Expand their programming knowledge by consulting documentation
\end{itemize}

\subsection*{Continuing Forward}

This penultimate lab assignment does not contribute to the final lab assignment.
By integrating elements of what you learned in this course, and by demonstrating
that you can review documentation to learn on your own, to design a small
embedded system, you will show how much progress you have made this semester.

\section*{During Lab Time}

During your lab period, coordinate with your group partner(s) to decide on your
working arrangements. Unless you're only going to work on the assignment when
you're together, you may want to set up a private Git repository that is shared
with your partner(s). With your partner(s), think through your system's design
and begin implementing it. The TAs will be available for questions.

\softwareengineeringfrontmatter

\section{Scenario}

``I have various teams working on different projects around here to improve
security,'' Archie reminds you. He glances toward the Zoo's labs, where there's
now a guy who looks like the actor who portrayed the fictional actor who
portrayed the Norse god Odin, trying to avoid children while wistfully talking
about raising rabbits in Montana. You briefly wonder why there are children
someplace where there are also carnivorous megafauna, and then you remember that
you work at a petting zoo. ``What I need your team to do,'' Archie continues,
``is make a combination lock so that only authorized people can get into our lab
facilities.''

\section{Electronic Combination Lock Specification} \label{sec:FunctionalSpecification}

\begin{enumerate}
\item A combination shall consist of three numbers in a particular order.
    \begin{enumerate}
    \item Each of the three numbers shall consist of exactly two hexadecimal
        digits.
    \item It shall be possible for any two of the numbers to be duplicates of
        each other. It shall be possible for all three numbers to be duplicates
        of each other. For example, 98-98-98 is a valid combination.
    \item Single-digit numbers must have leading zeroes. For example,
        01-02-03 is a valid combination, but 1-2-3 is not.
    \end{enumerate}
\item When displayed on the \textbf{display module}, the combination's first
    number shall occupy the two leftmost digits; the second number shall occupy
    the middle two digits; and the third number shall occupy the two rightmost
    digits. Dashes shall be displayed between the numbers, for example:
    {\dviiseg 12-34-56}. If one or more of the positions has not had a number
    entered, then the digits for the unentered positions(s) shall be blank.
    For example, if the first number has been entered but not the second and
    third, then the display will show {\dviiseg 12-\phantom{88}-\phantom{88}}.
\item The combination's numbers shall be entered using the \textbf{matrix
    keypad}. Buttons with decimal numerals (\textit{0-9}) or alphabetic letters
    (\textit{A-D}) shall be interpreted as having the corresponding hexadecimal
    numeral; the button with the octothorp (\#) shall be interpreted as having
    the hexadecimal numeral \textit{E}; and the button with the asterisk (*)
    shall be interpreted as having the hexadecimal numeral \textit{F}.
\item When the system is powered-up, it shall be LOCKED, regardless of its state
    before losing power, and the display shall have all numbers blank:
    {\dviiseg \phantom{88}-\phantom{88}-\phantom{88}}.
\item When the system is LOCKED, the system shall display the combination-entry
    display.
\item \label{spec:enterCombination} When the system displays the
    combination-entry display, it shall display the currently-entered
    cobmination (which might not be the correct combination), and the user shall
    be able to enter and change the currently-entered combination.
    \begin{enumerate}
    \item The system shall indicate to the user the position in the combination
        that they are entering a number for, by blinking the decimal points for
        that position's two digits on-and-off every half-second. For example,
        when all numbers are unentered and the user is being prompted to enter
        the first number, the display shall show: \\
        {\dviiseg \phantom{8}.\phantom{8}.-\phantom{88}-\phantom{88}} for a
            half-second, then \\
        {\dviiseg \phantom{88}-\phantom{88}-\phantom{88}} for a half-second,
            then \\
        {\dviiseg \phantom{8}.\phantom{8}.-\phantom{88}-\phantom{88}} for a
            half-second, then \\
        {\dviiseg \phantom{88}-\phantom{88}-\phantom{88}} for a half-second, and
            so on.
    \item Hereafter, the blinking decimal points will be referred to as the
        \textit{cursor}.
    \item The cursor shall continue to blink while and after the user has
        entered numerals. For exmaple, after the user has entered ``1'' as the
        first digit of the first number, the display shall repeatedly show: \\
        {\dviiseg 1.\phantom{8}.-\phantom{88}-\phantom{88}} for a
            half-second, and \\
        {\dviiseg 1\phantom{8}-\phantom{88}-\phantom{88}} for a half-second,
            then \\
        then after the user has entered ``2'' as the second digit of the first
        number, the display shall repeatedly show: \\
        {\dviiseg 1.2.-\phantom{88}-\phantom{88}} for a
            half-second, and \\
        {\dviiseg 12-\phantom{88}-\phantom{88}} for a half-second.
    \item The user shall advance the cursor to the next number to be entered by
        pressing the \textbf{right pushbutton}.
        \begin{enumerate}
        \item The cursor shall not advance until the user presses the
            \textbf{right pushbutton}.
        \item If the cursor is positioned on the first number, then
            advancing the cursor places the cursor in the second nummber's
            position.
        \item If the cursor is in the second number's position, then
            advancing repositions the cursor in the third number's position
        \item If the cursor is in the third number's position, then
            advancing the cursor places the cursor in the first number's
            position.
        \end{enumerate}
    \item The user shall be able to change the numbers they have entered by
        advancing the cursor from position to position.
    \item The user shall assert that they have entered the combination by
        pressing the \textbf{left pushbutton}.
    \item If the user has not entered numbers for all three positions before
        pressing the \textbf{left pushbutton}, then the system shall display
        {\dviiseg error} for one second and then resume displaying
        the combination-entry display.
    \end{enumerate}
\item The \textbf{switches} shall be ignored while the system is LOCKED.
\item If the user had entered the correct combination when when they pressed
    the \textbf{left pushbutton}, then the system shall be UNLOCKED and shall
    display {\dviiseg lab open}.
\item If the user had entered an incorrect combination when they pressed the
    \textbf{left pushbutton}, then the system shall display
    {\dviiseg badtry 1} \\ {\dviiseg badtry 2} or \\ {\dviiseg badtry 3} for
    one second, depending on whether it was their first, second, or third
    attempt.
    \begin{itemize}
    \item The original specification had a space between ``bad'' and ``try.''
        You may choose instead to have the space between the words and no space
        between ``try'' and the number.
    \item You may add the middle segment to the letter `y' if you wish.
    \end{itemize}
\item After the user's first and second ``bad try'' messages, the system shall
    resume displaying the combination-entry display.
\item After the user's third ``bad try'' message, the system shall be ALARMED.
    \begin{enumerate}
    \item When the system is ALARMED, it shall display {\dviiseg alert!} and
        the \textbf{external LED} shall blink on-and-off every quarter-second.
    \item When the system is ALARMED, it shall not accept further input until
        it has been powered-down and then powered-up.
    \end{enumerate}
\item \label{spec:changeCombination} When the system is UNLOCKED, the user shall
    be able to change the combination.
    \begin{enumerate}
    \item When the user places both \textbf{switches} in the right positions and
        then pressing the \textbf{left pushbutton}, the system shall be
        CHANGING.
    \item When the system is CHANGING, it shall display {\dviiseg enter} for
        one second and then shall display the combination-entry display, and
        the user shall be able to enter the proposed new combination, with the
        following change:
        \begin{enumerate}
        \item Pressing the \textbf{left pushbutton} while the
            \textbf{left switch} is in the right position shall have no effect.
        \item If the user places the \textbf{left switch} in the left position
            and then presses the \textbf{left pushbutton}, then the system
            shall be CONFIRMING.
        \end{enumerate}
    \item When the system is CONFIRMING, it shall display {\dviiseg re-enter}
        for one second and then shall display the combination-entry display, and
        the user shall be able to enter the proposed new combination, with the
        following change:
        \begin{enumerate}
        \item Pressing the \textbf{left pushbutton} while the
            \textbf{right switch} is in the right position shall have no effect.
        \item If the user places the \textbf{right switch} in the left position
            and then presses the \textbf{left pushbutton}, then the system
            shall display {\dviiseg changed} if the confirmed
            combination matches the proposed combination, or {\dviiseg nochange}
            if the confirmed combination differs from the proposed
            combination. The system shall then be UNLOCKED.
        % \item After displaying the ``changed'' or ``no change'' message for
        %     two seconds, \dots
        \end{enumerate}
    \end{enumerate}
\item \label{spec:locking} When the system is UNLOCKED, the user shall be able
    to lock the system by one of the two methods listed below. The system shall
    be LOCKED. It shall display {\dviiseg closed} for one seceond and then shall
    display the combination-entry display.
    \begin{enumerate}
    \item One option to lock the system is placing both \textbf{switches} in the
        left positions and then pressing both \textbf{pushbuttons}
        simultaneously.
    \item The other option to lock the system is placing both \textbf{switches}
        in the left positions and then double-clicking the \textbf{right
        pushbutton}.
    \end{enumerate}
    \begin{itemize}
    \item You only need to implement \textit{one} of these locking mechanisms.
        With the Cow Pi's hardware, double-clicking is easier to detect with
        interrupts, and chording is easier to detect with polling.
    \end{itemize}
\item Except as specified in Requirements~\ref{spec:changeCombination} and
    \ref{spec:locking}, pressing the \textbf{pushbuttons} shall have no effect
    when the system is UNLOCKED.
\item \label{spec:persistentCombination} The correct combination shall be
    persistent even while the system is unpowered. For example, if the user
    changes the combination, powers-down the system, and powers-up the system,
    then the correct combination shall still be the combination that they had
    changed it to before powering-down the system.
\end{enumerate}

\section{Constraints}\label{sec:Constraints}

You may re-use code from the previous lab assignments and from \textit{cowpi.h},
or you can rewrite the necessary code using functions, macros, types, or
constants that are part of the Arduino
core;\footnote{\url{https://www.arduino.cc/reference/en/}} functions, macros,
types, or constants of
avr-libc;\footnote{\url{https://www.nongnu.org/avr-libc/user-manual/index.html}}
or one of Arduino's standard
libraries.\footnote{\url{https://www.arduino.cc/reference/en/libraries/} The
standard libraries are those under the heading, ``Official Arduino Libraries.''}
You may \textit{not} use a third-party library, not even one that the Arduino
Libraries reference page links to (if it isn't among those available in the
Arudino IDE's ``Sketch'' $\rightarrow$ ``Include Library'' menu under ``Arduino
libraries'' then you may not use it).

You can use memory-mapped I/O registers, or you can use functions provided by
the Arduino core read from and write to pins. You can use polling, or you can
use interrupts. You can have as much or as little code in the \function{loop()}
function as you deem fit.

Do \textit{not} edit \textit{cowpi.h}; all of your code must go in
\textit{ComboLock.ino}.

\section{Accessing the EEPROM} \label{sec:AccessingEEProm}

To implement Requirement~\ref{spec:persistentCombination}, you will want to save
your lock's combination in your \nano's EEPROM. The \nano\ has 1KB of EEPROM,
and you will use only a tiny fraction of this (depending on your implementation,
you will probably use either 3 or 6 bytes of the EEPROM). While the EEPROM is
not part of the ATmega328P's memory address space, you can access it, one byte
at a time, using memory-mapped I/O. If you choose to use memory-mapped I/O to access the EEPROM, please read the ATmega328P
datasheet,\footnote{\url{http://ww1.microchip.com/downloads/en/DeviceDoc/Atmel-7810-Automotive-Microcontrollers-ATmega328P_Datasheet.pdf}}
§7.4 and §7.6. Alternatively, you can use the Arduino EEPROM library. If you
choose to use the EEPROM library, please read the EEPROM library's
documentation\footnote{\url{https://docs.arduino.cc/learn/built-in-libraries/eeprom}}
and guide.\footnote{\url{https://docs.arduino.cc/learn/programming/eeprom-guide}}
We recommend that you use the \function{EEPROM.get()} and
\function{EEPROM.put()} library functions, as they work with arbitrary data
types and write to the EEPROM in the most-efficient manner possible.

\subsection*{Clearing Part of EEPROM Memory}

In the event that you forget your lock's combination (or that you need to set
the combination before your first use), we have provided
\textit{ClearEEPROMPage.ino} that will set four bytes to 0. Using the Serial
Monitor, you will be prompted to enter a number divisible by 4 (not to exceed
1023). The 4-byte EEPROM page starting at that address will then be set to 0.
If you need to clear more than four bytes, press the RESET button on top of your
\nano\ to run the program again.
\textbf{NOTE:} after you type in the number, you will send that number to the
\nano\ one of three ways:
\begin{description}
\item [Arduino 1.8] Use your mouse to click on the ``Send'' button.
\item [Arduino 2.0 on Windows or Linux] Press Control-Enter on your keyboard.
\item [Arudino 2.0 on Macintosh] Press Command-Enter on your keyboard.
\end{description}

\subsection*{Non-Performing and Toxic Team Members}

\begin{description}
\item[Non-performing team members] If a student is habitually failing to be a
contributing member of your team beyond your ability to resolve, you may
ask a TA or the professor for help resolving the problem.  In extreme
circumstances, the professor may remove that student from the team.  This
can occur only after we've discussed your team's dilemma and concurred
that no better option is available.
\item [Toxic team members] If a student's behavior is harming the other team
members' ability to learn and perform, the professor may remove that
student from the team.  This can occur only after we've discussed your
team's dilemma and concurred that no better option is available. If necessary,
the case will be referred to Student Affairs for appropriate action.
\end{description}

We will find a way for an unfired team member to demonstrate mastery of the
lab material and earn credit for the assignment without having to commplete the
full assignment on their own.

A fired team member may complete the project on his/her own, or join a group
with other fired team members (if any exist).

\section*{Turn-in and Grading}\addcontentsline{toc}{section}{Rubric}

When you have completed this assignment, upload \textit{ComboLock.ino} to
\filesubmission.

This assignment is worth 80 points. \\

Rubric:
\begin{description}
\rubricitem{2}{Combinations are displayed as three 2-hex-digit numbers separated
    by dashes.}
\rubricitem{4}{Numbers are entered using the matrix keypad.}
\rubricitem{2}{The lock is locked when powered-up.}
\rubricitem{4}{When the lock is locked, it displays the combination-entry
    display, initially showing empty numbers (only dashes).}
\rubricitem{4}{The cursor indicating which number is being entered is
    represented as blinking decimal points in the relevant number position.}
\rubricitem{4}{The user can move the cursor between number positions by pressing
    the right button.}
\rubricitem{4}{After entering a combination, the user can submit the combination
    by pressing the left button.}
\rubricitem{2}{If the user tries to submit an incomplete combination, the lock
    displays ``error'' and then returns to the combination-entry display.}
\rubricitem{4}{When the user unlocks the lock, it displays ``lab open''.}
\rubricitem{2}{When the user mis-enters the combination, it displays ``bad try''
    and the attempt number.}
\rubricitem{4}{After the first two bad tries, the user is given another
    opportunity to enter the combination.}
\rubricitem{4}{After the third bad try, the system displays ``alert!'', the
    external LED rapidly blinks, and the lock becomes unresponsive.}
\rubricitem{4}{The user can begin changing the combination by moving both
    switches to the right and pressing the left button.}
\rubricitem{4}{When the user begins changing the combination, the lock displays
    ``enter'' and then shows the combination-entry display.}
\rubricitem{4}{After entering a new combination, the user can begin confirming
    the combination by sliding the left switch to the left and pressing the left
    button.}
\rubricitem{4}{When the user begins confirming the combination, the lock
    displays ``re-enter'' and then shows the combination-entry display.}
\rubricitem{4}{After entering the confirming combination, the user can compare
    the new combination with the confirmed combination by sliding the right switch
    to the left and pressing the left button.}
\rubricitem{4}{If the user successfully confirmed the new combination, the lock
    displays ``changed'', and the new combination is the correct combination
    for future unlocking attempts.}
\rubricitem{4}{If the user failed to confirm the new combination, the lock
    displays ``nochange'', and the previously-correct combination remains the
    combination for future unlocking attempts.}
\rubricitem{4}{The user can lock the lock by moving both switches to the left
    and \textit{either} double-clicking the right button, \textit{or} pressing
    both buttons at the same time. (Only one of these methods needs to be
    implemented).}
\rubricitem{2}{The inputs that are explicitly specified as having no effect in
    certain situations are ignored in those situations.}
\rubricitem{4}{The correct combination persists while Arduino is powered-down.}
\rubricitem{2}{The source code is well-organized and is readable.}
\bonusitem{2}{Get assignment checked-off by TA or professor during office hours
    before it is due. (You cannot get both bonuses.)}
\bonusitem{1}{Get assignment checked-off by TA at \textit{start} of your
    scheduled lab immediately after it is due. (Your code must be uploaded to
    \filesubmission\ \textit{before} it is due. You cannot get both bonuses.)}
\spaghetticodepenalties{1}
\end{description}

Students' scores may be adjusted up or down as necessary if the team had an
inequitable distribution of effort.

\section*{Epilogue}

After fastening the new electronic combination lock to the lab door, Archie
smiles and tells you that this was a job well done. With all of the excitement
neatly wrapped-up and arriving at a satisfactory conclusion, you look forward to
a boring career in which there's absolutely no screaming and running for your
life.

\textit{The End...?}


% \newpage

\section*{Appendix: Lab Checkoff}\addcontentsline{toc}{section}{Lab Checkoff}

You are not required to have your assignment checked-off by a TA or the
professor. If you do not do so, then we will perform a functional check
ourselves. In the interest of making grading go faster, we are offering a small
bonus to get your assignment checked-off at the start of your scheduled lab
time immediately after it is due. Because checking off all students during lab
would take up most of the lab time, we are offering a slightly larger bonus if
you complete your assignment early and get it checked-off by a TA or the
professor during office hours.

\begin{description}
\precheckoffitem{Establish that the code you are demonstrating is the code
    you submitted to to \filesubmission.}
    \begin{itemize}
    \item If you are getting checked-off during lab time, show the TA that the
        file was submitted before it was due.
    \item Download the file into your ComboLab directory. If necessary,
        rename it to \textit{ComboLab.ino}.
    \end{itemize}
\precheckoffitem{Upload \textit{ComboLab.ino} to your \nano\ and open the
    Serial Monitor.}
\end{description}

\begin{enumerate}
\checkoffitem{The combination screen is displayed
    {\dviiseg \phantom{88}-\phantom{88}-\phantom{88}}) with the cursor (two
    decimal points) blinking in the left-most position. No numbers are
    displayed in the combination.} \\
    \textit{+2 The lock is locked when powered-up.} \\
    \textit{+2 When locked, shows combo-entry display, initially without numbers
        (power-up).} \\
    \textit{+2 The cursor is represented with decimal points in the relevant
        position.} \\
    \textit{+2 The cursor blinks.}
\checkoffitem{Place both switches in the left position.}
\checkoffitem{Press the right button twice. The cursor moves into the middle
    position and then the right-most position.} \\
    \textit{+3 The user can move the cursor using the right button.}
\checkoffitem{Press the right button again. the cursor moves into the left-most
    position.} \\
    \textit{+1 The cursor wraps-around from the last number to the first
        number.}
\checkoffitem{Press 1, then A. The display shows {\dviiseg
    1.A.-\phantom{88}-\phantom{88}}}. \\
    \textit{+1 Combinations use 2-hex-digit numbers.} \\
    \textit{+4 Numbers are entered using the keypad.}
\checkoffitem{Press the left button. The display shows {\dviiseg error} and then
    {\dviiseg 1.A.-\phantom{88}-\phantom{88}}.} \\
    \textit{+1 Submit ``error'' combination with left button.} \\
    \textit{+1 Incomplete combination produces error message.} \\
    \textit{+1 After the error message, the combo-entry display returns.}
\checkoffitem{Finish entering an incorrect combination. The display shows all
    three numbers, separated by dashes.} \\
    \textit{+1 Combinations are three numbers separated by dashes.}
\checkoffitem{Press the left button. The display shows {\dviiseg badtry 1} and
    then the combo-display display.} \\
    \textit{+1 Submit ``bad try'' combination with left button.} \\
    \textit{+1 Wrong combination produces bad-try message.} \\
    \textit{+2 After the first bad try, the combo-entry display returns.}
\checkoffitem{Enter another incorrect combination and press the left button. The
    display shows {\dviiseg badtry 2} and then the combo-display display.} \\
    \textit{+1 The ``bad try'' number increments.} \\
    \textit{+2 After the second bad try, the combo-entry display returns.}
\checkoffitem{Enter another incorrect combination and press the left button. The
    display shows {\dviiseg badtry 3} and then {\dviiseg alert!}. The external
    LED rapidly blinks.} \\
    \textit{+1 After the third bad try, the sytem displays an alert message.} \\
    \textit{+2 After the third bad try, the external LED rapidly blinks.}
\checkoffitem{Press buttons and keys a few times. Nothing happens except that
    the alert message is still displayed and the LED still blinks.} \\
    \textit{+1 After the third bad try, the system is non-responsive.}
\checkoffitem{Press the \nano's RESET button (on the top of the \nano).}
\checkoffitem{Enter the correct combination and press the left button. The
    system displays {\dviiseg lab open}.} \\
    \textit{+2 Submit correct combination with left button.} \\
    \textit{+4 When the user locks the lock, it displays ``lab open.''}
\checkoffitem{Double-check that both switches are in the left position. Press
    the left button. Nothing happens. Press the right button. Nothing
    happens.} \\
    \textit{+1 Single-presses of only one button have no effect when the lock is unlocked and the switches are not in the right position.}
\checkoffitem{Place both switches in the right position and press the left
    button. The system displays {\dviiseg enter} then
    {\dviiseg \phantom{88}-\phantom{88}-\phantom{88}}.} \\
    \textit{+4 Start changing combo by pushing left button while both switches
        are to the right.} \\
    \textit{+2 When starting to change the combo, ``enter'' is displayed.} \\
    \textit{+2 After displaying ``enter,'' the combo-entry display is shown.}
\checkoffitem{Enter a new combination.}
\checkoffitem{Press the left button. Nothing happens.} \\
    \textit{+\textonehalf\ Left button has no effect unless left switch is
    moved.}
\checkoffitem{Place the left switch in the left position and press the left
    button. The system displays {\dviiseg re-enter} then
    {\dviiseg \phantom{88}-\phantom{88}-\phantom{88}}.} \\
    \textit{+4 Start confirming by moving the left switch and pressing the left
        button.} \\
    \textit{+2 When starting to change the combo, ``re-enter'' is displayed.} \\
    \textit{+2 After displaying ``re-enter,'' the combo-entry display is shown.}
\checkoffitem{Enter a non-matching combination.}
\checkoffitem{Press the left button. Nothing happens.} \\
    \textit{+\textonehalf\ Left button has no effect unless right switch is
        moved.}
\checkoffitem{Place the right switch in the left position and press the left
    button. The system displays {\dviiseg nochange}. Though unspecified, it is
    acceptable to display {\dviiseg lab open} after displaying {\dviiseg
    nochange} for at least one second.} \\
    \textit{+4 Compare combos by moving the right switch and pressing the left
        button.} \\
    \textit{+2 When the combos do not match, ``nochange'' is displayed.}
\checkoffitem{Double-check that both switches are in the left position. Either
    double-click the right button \textit{or} press both buttons at the same
    time. The system displays {\dviiseg closed} and then {\dviiseg
    \phantom{88}-\phantom{88}-\phantom{88}}.} \\
    \textit{+4 Re-lock the lock using one of the specified techniques.} \\
    \textit{+2 When locked, shows combo-entry display, initially without numbers
        (re-locked).}
\checkoffitem{Enter the original, correct combination and press the left button.
    The system displays {\dviiseg lab open}.} \\
    \textit{+2 When the combos do not match, the previously-correct combo
        remains the correct combo.}
\checkoffitem{Move both switches to the right, enter a new combination, move the
    left switch to the left, and press the left button. The system displays
    {\dviiseg re-enter} then {\dviiseg
    \phantom{88}-\phantom{88}-\phantom{88}}.}
\checkoffitem{Enter the matching combination, move the right switch to the left,
    and press the right button. The system displays {\dviiseg nochange}. Though
    unspecified, it is acceptable to display {\dviiseg lab open} after
    displaying {\dviiseg changed} for at least one second.} \\
    \textit{+2 When the combos match, ``changed'' is displayed.}
\checkoffitem{Double-check that both switches are in the left position. Either
    double-click the right button \textit{or} press both buttons at the same
    time. The system displays {\dviiseg closed} and then {\dviiseg
    \phantom{88}-\phantom{88}-\phantom{88}}.}
\checkoffitem{Enter the new combination and press the left button. The system
    displays {\dviiseg lab open}.} \\
    \textit{+2 When the combos match, the new combo is the correct combo.}
\checkoffitem{Press the \nano's RESET button.}
\checkoffitem{Enter the new combination and press the left button. The system
    displays {\dviiseg lab open}.} \\
    \textit{+4 The correct combination persists while the Arduino is
    powered-down.}
\end{enumerate}

This concludes the demonstration of your system's functionality. The TAs will
later examine your code to assess the remaining 2 points for your source code
being well-organized and readable, to evaluate any claimed extra credit, and
for violations of the assignment's constraints. If your code looks like it is
tailored for this checklist, the TAs may re-grade using a different checklist.
\end{document}
