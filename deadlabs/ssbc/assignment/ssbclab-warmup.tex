%%
%% SSBClab (c) 2020-21 Christopher A. Bohn
%%
%% Licensed under the Apache License, Version 2.0 (the "License");
%% you may not use this file except in compliance with the License.
%% You may obtain a copy of the License at
%%     http://www.apache.org/licenses/LICENSE-2.0
%% Unless required by applicable law or agreed to in writing, software
%% distributed under the License is distributed on an "AS IS" BASIS,
%% WITHOUT WARRANTIES OR CONDITIONS OF ANY KIND, either express or implied.
%% See the License for the specific language governing permissions and
%% limitations under the License.
%%

%%
%% (c) 2021 Christopher A. Bohn
%%

\documentclass[12pt]{article}

\usepackage{fullpage}
\usepackage{fancyhdr}
\usepackage[procnames]{listings}
\usepackage{hyperref}
\usepackage{textcomp}
\usepackage{bold-extra}
\usepackage[dvipsnames]{xcolor}
\usepackage{etoolbox}


% Customize the semester (or quarter) and the course number

\newcommand{\courseterm}{Spring 2022}
\newcommand{\coursenumber}{CSCE 231}

% Customize how a typical lab will be managed;
% you can always use \renewcommand for one-offs

\newcommand{\runtimeenvironment}{your account on the \textit{csce.unl.edu} Linux server}
\newcommand{\filesource}{Canvas or {\footnotesize$\sim$}cse231 on \textit{csce.unl.edu}}
\newcommand{\filesubmission}{Canvas}

% These are placeholder commands and will be renewed in each lab

\newcommand{\labnumber}{}
\newcommand{\labname}{Lab \labnumber\ Assignment}
\newcommand{\shortlabname}{}
\newcommand{\duedate}{}

% Individual or team effort

\newcommand{\individualeffort}{This is an individual-effort project. You may discuss concepts and syntax with other students, but you may discuss solutions only with the professor and the TAs. Sharing code with or copying code from another student or the internet is prohibited.}
\newcommand{\teameffort}{This is a team-effort project. You may discuss concepts and syntax with other students, but you may discuss solutions only with your assigned partner(s), the professor, and the TAs. Sharing code with or copying code from a student who is not on your team, or from the internet, is prohibited.}
\newcommand{\freecollaboration}{In addition to the professor and the TAs, you may freely seek help on this assignment from other students.}
\newcommand{\collaborationrules}{}

% Do you care about software engineering?

\providebool{allowspaghetticode}

\setbool{allowspaghetticode}{false}

\newcommand{\softwareengineeringfrontmatter}{
    \ifboolexpe{not bool{allowspaghetticode}}{
        \section*{No Spaghetti Code Allowed}
        In the interest of keeping your code readable, you may \textit{not} use
        any \lstinline{goto} statements, nor may you use any \lstinline{break}
        statements to exit from a loop, nor may you have any functions
        \lstinline{return} from within a loop.
    }{}
}

\newcommand{\spaghetticodepenalties}[1]{
    \ifboolexpe{not bool{allowspaghetticode}}{
        \penaltyitem{1}{for each \lstinline{goto} statement, \lstinline{break}
            statement used to exit from a loop, or \lstinline{return} statement
            that occurs within a loop.}
    }{}
}

% You shouldn't need to customize these,
% but you can if you like

\lstset{language=C, tabsize=4, upquote=true, basicstyle=\ttfamily}
\newcommand{\function}[1]{\textbf{\lstinline{#1}}}
\setlength{\headsep}{0.7cm}
\hypersetup{colorlinks=true}

\newcommand{\startdocument}{
    \pagestyle{fancy}
    \fancyhf{}
    \lhead{\coursenumber}
    \chead{\ Lab \labnumber: \labname}
    \rhead{\courseterm}
    \cfoot{\shortlabname-\thepage}

	\begin{document}
	\title{\ Lab \labnumber}
	\author{\labname}
	\date{Due: \duedate}
	\maketitle

    \textit{\collaborationrules}
}

\newcommand{\rubricitem}[2]{\item[\underline{\hspace{1cm}} +#1] #2}
\newcommand{\bonusitem}[2]{\item[\underline{\hspace{1cm}} Bonus +#1] #2}
\newcommand{\penaltyitem}[2]{\item[\underline{\hspace{1cm}} -#1] #2}

%%
%% labs/common/semester.tex
%% (c) 2021-22 Christopher A. Bohn
%%
%% Licensed under the Apache License, Version 2.0 (the "License");
%% you may not use this file except in compliance with the License.
%% You may obtain a copy of the License at
%%     http://www.apache.org/licenses/LICENSE-2.0
%% Unless required by applicable law or agreed to in writing, software
%% distributed under the License is distributed on an "AS IS" BASIS,
%% WITHOUT WARRANTIES OR CONDITIONS OF ANY KIND, either express or implied.
%% See the License for the specific language governing permissions and
%% limitations under the License.
%%


% Customize the semester (or quarter) and the course number

\newcommand{\courseterm}{Fall 2022}
\newcommand{\coursenumber}{CSCE 231}

% Customize how a typical lab will be managed;
% you can always use \renewcommand for one-offs

\newcommand{\runtimeenvironment}{your account on the \textit{csce.unl.edu} Linux server}
\newcommand{\filesource}{Canvas or {\footnotesize$\sim$}cse231 on \textit{csce.unl.edu}}
\newcommand{\filesubmission}{Canvas}

% Customize for the I/O lab hardware

\newcommand{\developmentboard}{Arduino Nano}
%\newcommand{\serialprotocol}{SPI}
\newcommand{\serialprotocol}{I2C}
%\newcommand{\displaymodule}{MAX7219digits}
%\newcommand{\displaymodule}{MAX7219matrix}
\newcommand{\displaymodule}{LCD1602}

\setbool{usedisplayfont}{true}

\newcommand{\obtaininghardware}{
    The EE Shop has prepared ``class kits'' for CSCE 231; your class kit costs \$30.
    The EE Shop is located at 122 Scott Engineering Center and is open M-F 7am-4pm. You do not need an appointment.
    You may pay at the window with cash, with a personal check, or with your NCard.
    The EE shop does \textit{not} accept credit cards.
}

% Update to reflect the CS2 course(s) at your institute

\newcommand{\cstwo}{CSCE~156, RAIK~184H, or SOFT~161}

% Do you care about software engineering?

\setbool{allowspaghetticode}{false}

% Which assignments are you using this semester, and when are they due?

\newcommand{\pokerlabnumber}{1}
\newcommand{\pokerlabcollaboration}{
    Sections~\ref{sec:connecting}, \ref{sec:terminology}, \ref{sec:gettingstarted}, \ref{subsec:typesofpokerhands}, and~\ref{subsec:studythecode}: \freecollaboration
    Sections~\ref{sec:completingcard} and~\ref{subsec:completepoker}: \individualeffort
}
\newcommand{\pokerlabdue}{Week of August 29, before the start of your lab section}

\newcommand{\keyboardlabnumber}{2}
\newcommand{\keyboardlabcollaboration}{\individualeffort}
\newcommand{\keyboardlabdue}{Week of January 31, before the start of your lab section}

\newcommand{\pointerlabnumber}{3}
\newcommand{\pointerlabcollaboration}{\individualeffort}
\newcommand{\pointerlabdue}{Week of February 7, before the start of your lab section}

\newcommand{\integerlabnumber}{4}
\newcommand{\integerlabcollaboration}{\individualeffort}
\newcommand{\integerlabdue}{Week of February 14, before the start of your lab section}

\newcommand{\floatlabnumber}{5}
\newcommand{\floatlabcollaboration}{\individualeffort}
\newcommand{\floatlabdue}{soon}

\newcommand{\addressinglabnumber}{6}
\newcommand{\addressinglabcollaboration}{\individualeffort}
\newcommand{\addressinglabdue}{Week of February 28, before the start of your lab section}

%bomblab was 7
%attacklab was 8

\newcommand{\pollinglabnumber}{9}
\newcommand{\pollinglabcollaboration}{\individualeffort}
\newcommand{\pollinglabdue}{Week of April 11, before the start of your lab section}
\newcommand{\pollinglabenvironment}{your \developmentboard-based class hardware kit}

\newcommand{\ioprelabnumber}{\pollinglabnumber-prelab}
\newcommand{\ioprelabcollaboration}{\freecollaboration}
\newcommand{\ioprelabdue}{Before the start of your lab section on April 5 or 6}

\newcommand{\interruptlabnumber}{10}
\newcommand{\interruptlabcollaboration}{\individualeffort}
\newcommand{\interruptlabdue}{Week of April 18, before the start of your lab section}
\newcommand{\interruptlabenvironment}{your \developmentboard-based class hardware kit}

\newcommand{\capstonelab}{ComboLock}    % this will come into play when we generalize capstonelab
\newcommand{\capstonelabnumber}{11}
\newcommand{\capstonelabcollaboration}{\teameffort}
\newcommand{\capstonelabdue}{Week of May 2, Before the start of your lab section\footnote{See Piazza for the due dates of teams with students from different lab sections.}}
\newcommand{\capstonelabenvironment}{your \developmentboard-based class hardware kit}

\newcommand{\memorylabnumber}{12}
\newcommand{\memorylabcollaboration}{This is an individual-effort project. You may discuss the nature of memory technologies and of memory hierarchies with classmates, but you must draw your own conclusions.}
\newcommand{\memorylabdue}{Week of May 2, at the end of your lab section}
\newcommand{\memorylabenvironment}{your \developmentboard-based class hardware kit and your account on the \textit{csce.unl.edu} Linux server}

% Labs not used this semester

\newcommand{\concurrencylabnumber}{XX}
\newcommand{\concurrencylabcollaboration}{\individualeffort}
\newcommand{\concurrencylabdue}{not this semester}

\newcommand{\ssbcwarmupnumber}{XX}
\newcommand{\ssbcwarmupcollaboration}{\freecollaboration}
\newcommand{\ssbcwarmupdue}{not this semester}

\newcommand{\ssbcpollingnumber}{XX}
\newcommand{\ssbcpollingcollaboration}{\individualeffort}
\newcommand{\ssbcpollingdue}{not this semester}

\newcommand{\ssbcinterruptnumber}{XX}
\newcommand{\ssbcinterruptcollaboration}{\individualeffort}
\newcommand{\ssbcinterruptdue}{not this semester}

\usepackage{enumitem}
\usepackage{graphicx}
\usepackage{media9}
\usepackage{addfont}
\addfont{OT1}{d7seg}{\dviiseg}
\addfont{OT1}{deseg}{\deseg}

\renewcommand{\labnumber}{\ssbcwarmupnumber}
\renewcommand{\labname}{Familiarization with Simulated Hardware}
\renewcommand{\shortlabname}{ssbclab-warmup}
\renewcommand{\collaborationrules}{\ssbcwarmupcollaboration}
\renewcommand{\duedate}{\ssbcwarmupdue}
\pagelayout
\begin{document}
\labidentifier

%\usepackage{fullpage}
%\usepackage{enumitem}

Due to the circumstances of this semester, we will not be using Altera
single board computers for labs. Instead, we will use a simulated single
board computer (SSBC).

In this assignment, you will familiarize yourself with the Simulated Single Board Computer \textbf{You should be able to complete this
lab assignment during lab time.}

The instructions are written assuming you will edit and run the code on
\runtimeenvironment. If you wish, you may edit the code in a different
environment; however, you will not be able to link an executable except on
\runtimeenvironment.

\section{Getting Started}

The accompanying document, \textit{ssbc.pdf} describes how to use the SSBC
library: how to interact with the simulated SBC and how to use its inputs and
outputs in a program.

Download \textit{\shortlabname.zip} or \textit{\shortlabname.tar} from
\filesource\ and copy it to \runtimeenvironment. Once copied, unpackage the
file. You will find four files:
\begin{description}
    \item [ssbc.h] The header file for the SSBC library.
    \item [Makefile] The Makefile to build the programs for this lab. You can build the programs individually, or you can build all of them using the command \texttt{make all}.
    \item [demo.c] A program that demonstrates using the SSBC library. Run the
        command \texttt{make demo} to build the program, and then run it with
        the command \texttt{./demo}. This file is described line-by-line in
        \textit{ssbc.pdf}
    \item [warmup.c] Starter code for this lab. Run the command \texttt{make
        warmup} to build the program, and then run it with the command
        \texttt{./warmup}.
    % \item [poll-calculator.c] Starter code for part 2 of this lab. Run the
    %     command \texttt{make poll-calculator} to build the program, and then
    %     run it with the command \texttt{./poll-calculator}.
    % \item [interrupt-calculator.c] Starter code for part 3 of this lab. Run the command\\ \texttt{make interrupt-calculator} to build the program, and then run it with the command \texttt{./interrupt-calculator}.
\end{description}

\textbf{\textit{Note: }Do NOT use \function{printf} to print debugging
information. Use \function{ssbc_print} instead.}

\section{Familiarize Yourself with the Simulated Single Board Computer}

Edit \textit{warmup.c} to write a program that uses the SSBC's input and output
registers to meet the following specification:

\begin{itemize}
    \item When the left-most toggle switch is in the ``on'' position, the left
        three seven-segment displays show {\dviiseg 231}; when the
        left-most toggle switch is in the ``off'' position, the left three
        seven-segment displays are blank.
    \item When the right-most toggle switch is in the ``on'' position, the
        right-most seven-segment display displays the number corresponding to
        the last number button that was pressed. When the right-most toggle
        switch is in the ``off'' position, the right-most seven-segment display
        is blank. If a number button has not yet been pressed since launching
        the program, the right-most seven-segment display will be blank, even
        if the toggle switch is in the ``on'' position.
    \item Toggling the middle two toggle switches has no effect.
\end{itemize}

\textit{Hint:} You may want to create a ten-element array of 8-bit bit vectors
to map a decimal digit to the bit vector needed to display that digit on a
single seven-segment display.

\textit{Warning:} Blindly copying code from \textit{demo.c} into your lab
solution will be counter-productive. Use \textit{demo.c} to understand how
to use the SSBC library, and then design your lab solution.

\section*{Turn-in and Grading}

When you have completed this assignment, upload \textit{warmup.c} to
\filesubmission.

This assignment is worth 20 points. \\

Rubric:
\begin{description}
\rubricitem{4}{{\dviiseg 231} displays on the three left displays when the
    left-most toggle switch is in the ``on'' position.}
\rubricitem{2}{The left three displays are blank when the left-most toggle
    switch is in the ``off'' position.}
\rubricitem{2}{The right-most display is blank if a number button has not yet
    been pressed, regardless of the position of the right-most toggle switch.}
\rubricitem{4}{If a number button has been pressed, then the right-most display
    shows the number of the most-recent number button pressed when the right-
    most toggle switch is in the ``on'' position.}
\rubricitem{2}{The right-most display is blank when the right-most toggle
    switch is in the ``off'' position.}
\rubricitem{1}{The behaviors for the left-most toggle switch and for the right-
    most toggle switch can happen at the same time (\textit{e.g.}, {\dviiseg
    231} can be made to appear and disappear while also updating the number
    entered from the keypad).}
\rubricitem{1}{Nothing happens as a result of toggling either of the middle two
    toggle switches.}
\rubricitem{2}{The SSBC screen does not get corrupted under casual use of the
    program.}
\rubricitem{2}{The SSBC remains responsive to inputs throughout using the
    program.}
\end{description}

\end{document}
