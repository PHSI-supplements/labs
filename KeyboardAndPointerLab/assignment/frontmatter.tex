The purpose of this assignment is to give you more confidence in C programming and to begin your exposure to the underlying bit-level representation of data.
You will also gain practice with pointers and with file input/output.

The instructions are written assuming you will edit and run the code on \runtimeenvironment.
If you wish, you may edit and run the code in a different environment;
be sure that your compiler suppresses no warnings, and that if you are using an IDE that it is configured for C and not C++.

\section*{Learning Objectives}

After successful completion of this assignment, students will be able to:
\begin{itemize}
    \item Use the ASCII table to determine the corresponding integer values of C \lstinline{char} values.
    \item Apply arithmetic operators and comparators to C \lstinline{}{char} values.
    \item Construct and use a bitmask.
    \item Use bitwise operators and bit shift operators to create and modify values.
    \item Recognize the hazards of indeterminate values.
    \item Use C's string functions from \lstinline{string.h}\footnote{See \S7.8.1 and \S{}B.3 of Kernighan \& Ritchie's \textit{The C Programming Language}, 2nd ed.}.
    \item Use C's file I/O functions from \lstinline{stdio.h}\footnote{See \S7.5, \S7.7, and \S{}B1.1 \textit{The C Programming Language}, 2nd ed.}.
    \item Alias and reassign pointers.
    \item Create and traverse a linked list.
\end{itemize}

\subsection*{Continuing Forward}

Your experience with viewing values as bit patterns will be applicable in future labs, as will bit masks and bit operations.
Being able to understand the mistakes in Sections~\ref{subsec:uninitializedvariables} and~\ref{subsec:localaddresses} will help you avoid them in future labs.
Being able to work with pointers -- that is, with variables that hold addresses -- will help you specifically in future labs that use pointers but more generally in future labs that require you to think about accessing memory.

\section*{During Lab Time}

During your lab period, the TAs will demonstrate how to read the ASCII table and will provide a refresher on bitwise AND, bitwise OR, and left- and right-shifts.
This refresher will include a class discussion about why this code always outputs ``The number is 255.'' except when number is 0:

\begin{lstlisting}
#include <stdio.h>

int main() {
    int number;
    printf("Enter a number: ");
    scanf("%d", &number);
    if (number | 0xFF == number & 0xFF) {
        printf("The number is 255.\n");
    } else {
        printf("The number is not 255.\n");
    }
    return 0;
}
\end{lstlisting}

The TAs will also provide a refresher on linked lists and will describe Insertion Sort.
Finally, the TAs will also describe some string functions and some I/O functions from C's standard library.
During the remaining time, the TAs will be available to answer questions.

Before leaving lab, \textit{at a minimum} complete Sections~\ref{sec:ascii} and \ref{sec:char-num}.
If you leave lab before completing Section~\ref{sec:bit-ops}, be sure that you understand bitmasks and how to use them.

\textit{Because Labor Day creates a shortened week, you are not required to attend lab during Labor Day week.
Instead, the Tuesday and Wednesday lab sections that week will be treated as extra office hours, open to all students regardless of your lab section.}
