Download \textit{\shortlabname.zip} or \textit{\shortlabname.tar} from \filesource\ and copy it to \runtimeenvironment.
Once copied, unpackage the file.
Four of the five files (\textit{alu.h}, \textit{basetwo.c}, \textit{alu.c}, and \textit{integerlab.c}) contain the starter code for this assignment.
The last file (\textit{Makefile}) tells the \texttt{make} utility how to compile the code.
To compile the program, type:

\texttt{make}

This will produce an executable file called \textit{integerlab}.

When you run the program with the command \texttt{\textbf{\textit{./integerlab}}}, you will be prompted:

\begin{verbatim}
    Enter a one- or two-operand logical expression,
        a two-operand comparison expression, a two-operand arithmetic expression,
        "lg <value>" or "exponentiate <value>" to test your powers-of-two code,
        "is_negative <value>" to determine if 2's complement value is negative,
        "add1 <binary_value1> <binary_value2> <carry_in>" for 1-bit full adder,
        "add32 <hex_value1> <hex_value2> <carry_in>" for 32-bit ripple-carry adder,
        or "quit":
\end{verbatim}

When you enter a value, if it is prepended with \texttt{\textbf{\textit{0x}}} then the parser will parse it as a hexadecimal value;
otherwise, except as noted in Sections~\ref{subsec:one-bit-full-adder} and \ref{subsec:ripple-carry-adder}, the parser will treat it as a decimal value.

For now, type \texttt{\textbf{\textit{quit}}} to exit the program.

\subsection{Description of IntegerLab Files}

\subsubsection{integerlab.c}

Do not edit \textit{integerlab.c}.

This file contains the driver code for the lab.
It parses your input, calls the appropriate arithmetic function, and displays the output.

\subsubsection{alu.h} \label{subsubsec:alu.h}

Do not edit \textit{alu.h}.

This header file contains two type definitions:
\begin{description}
    \item[one\_bit\_adder\_t] is a structure to hold the 1-bin inputs (\lstinline{a}, \lstinline{b}, \lstinline{c_in}) and 1-bit outputs (\lstinline{sum}, \lstinline{c_out}) of a one-bit full adder.
    \item[alu\_result\_t] is a structure to hold the outputs from an arithmetic logic unit.
        Its fields are:
        \begin{itemize}
            \item \lstinline{result}, a 16-bit bit vector that is considered ``the'' result of the computation
            \item \lstinline{supplemental_result}, a 16-bit bit vector that stores additional result data from instructions that place their results in two registers
            \item \lstinline{unsigned_overflow}, a 1-bit flag to indicate whether overflow occurred when interpreting the source operands as unsigned values
            \item \lstinline{signed_overflow}, a 1-bit flag to indicate whether overflow occurred when interpreting the source operands as signed values
            \item \lstinline{divide_by_zero}, a 1-bit flag to indicate whether there was an attempt to divide by zero.
        \end{itemize}
\end{description}

The header file also contains two macros, \function{is_zero()} and \function{is_not_zero()} to bootstrap your ALU code.
These macros act like functions and return a boolean value to indicate whether an integer is 0 or not.\footnote{
    The astute student will quickly realize that \function{is_not_zero()} is not necessary and, with a little thought, will realize that they can write \function{is_zero()} as a function within the constraints of this assignment.}

The header file also contains several function declarations.
The requirements for these functions will be discussed later in this assignment.

\subsubsection{basetwo.c}

This is the first of two files that you will edit.

There are two functions in \textit{basetwo.c} that will allow you to demonstrate an understanding of powers-of-two and/or an understanding of some uses of bit shifts.
\begin{description}
    \item[lg()] returns the base-2 logarithm of its argument, assuming its argument is a positive power-of-two;
        if the argument is 0 or is not a power-of-two, then there are no guarantees about the function's return value
    \item[exponentiate()] creates a power-of-two by raising 2 to the provided exponent, assuming the exponent is a non-negative value strictly less than 32;
        if the argument is negative or is greater than 31, then there are no guarantees about the function's return value
\end{description}
These functions are inverses of each other: $x = \log_2 2^x$, and $y = 2^{\log_2 y}$.

Strictly speaking, you can write your ALU code without these functions;
however, some students in the past had difficulty finding solutions for their ALU code without obtaining a base-2 logarithm and/or calling a function to create a power-of-two.
Rather than tempt you to violate one of the assignment's constraints by calling the \textit{math} library's \function{log2()}, \function{exp2()}, and/or \function{pow()} functions, we now have you write your own code for these functions.

\subsubsection{alu.c}

This file will contain most of the code that you write, and the functions in \textit{alu.c} are in the order in which you will likely write them.
\begin{itemize}
    \item A simple check
        \begin{description}
            \item[is\_negative()] returns a boolean value to indicate whether the argument, when interpreted as a signed integer, is negative
        \end{description}
    \item Equality comparisons
        \begin{description}
            \item[equal()] returns \lstinline{true} if and only if $value1 = value2$
            \item[not\_equal()] returns \lstinline{true} if and only if $value1 \not = value2$
        \end{description}
    \item Logical operations
        \begin{description}
            \item[logical\_not()] returns the logical inverse of the argument
            \item[logical\_and()] returns the logical conjunction of the two arguments
            \item[logical\_or()] returns the logical disjunction of the two arguments
        \end{description}
    \item Addition and subtraction
        \begin{description}
            \item[one\_bit\_full\_addition()] performs addition for one bit position, determining both the sum bit and the carry-out bit
            \item[ripple\_carry\_addition()] adds two 32-bit values to each other and to a carry-in bit
            \item[add()] adds two 16-bit values to each other
            \item[subtract()] subtracts a 16-bit value from another
        \end{description}
    \item Inequality comparisons
        \begin{description}
            \item[less\_than()] returns \lstinline{true} if and only if $value1 < value2$
            \item[at\_most()] returns \lstinline{true} if and only if $value1 \leq value2$
            \item[at\_least()] returns \lstinline{true} if and only if $value1 \geq value2$
            \item[greater\_than()] returns \lstinline{true} if and only if $value1 > value2$
        \end{description}
    \item Unsigned multiplication and division
        \begin{description}
            \item[multiply\_by\_power\_of\_two()] multiplies the first argument by the second, assuming that the second argument is zero or a power of two;
                there are no guarantees if this assumption is not satisfied
            \item[unsigned\_multiply()] multiplies two 16-bit values to each other, if the arguments are interpreted as unsigned integers
            \item[unsigned\_divide()] divides a 16-bit value by another, if the arguments are interpreted as unsigned integers
        \end{description}
    \item Signed multiplication and division (bonus credit)
    \begin{description}
        \item[signed\_multiply()] multiplies two 16-bit values to each other, if the arguments are interpreted as signed integers
        \item[signed\_divide()] divides a 16-bit value by another, if the arguments are interpreted as signed integers
    \end{description}
\end{itemize}
