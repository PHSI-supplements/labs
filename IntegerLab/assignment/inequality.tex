In general, comparing two signed values to determine which (if either) is greater can be achieved with subtraction.
Consider, for example, the less-than comparison.
A simple application of algebra tells us that \[value1 < value2 \Leftrightarrow value1 - value2 < 0\]
Similarly, \[value1 = value2 \Leftrightarrow value1 - value2 = 0\]

As we noted in Section~\ref{subsec:negative}, real ALUs dedicate hardware to quickly compare values to 0.
This is an example of why that is so.
By performing this subtraction and determining the truth values of \function{is_zero()} and \function{is_negative()}, you have sufficient data to determine whether \lstinline{value1} is \function{less_than}, \function{at_most}, \function{at_least}, and/or \function{greater_than} \lstinline{value2}.
Do so and implement the four inequality functions.

Note: processor instructions using ``greater than'' and ``less than'' inequalities generally operate with signed values, as is the case in this assignment.
The corresponding unsigned inequalities tend to go by names such as ``above''/``below'' (x86) or ``higher''/``lower'' (ARM).

\subsubsection*{Check your work}

Compile and run \texttt{\textbf{\textit{./integerlab}}}, trying a few values.
For example:
\begin{verbatim}
    Enter ... a two-operand comparison expression ... or "quit": 4 < 5
    expected: (4 < 5) = 1
    actual:   (4 < 5) = 1

    Enter ... a two-operand comparison expression ... or "quit": 4 >= 5
    expected: (4 >= 5) = 0
    actual:   (4 >= 5) = 0
\end{verbatim}

Check your code with other values, comparing your actual results with the expected results.
Use positive and negative operands.
Generate both \lstinline{true} and \lstinline{false} results for each of the four signed inequality functions.
