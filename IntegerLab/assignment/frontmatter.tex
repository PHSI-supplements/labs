In this assignment, you will become more familiar with bit-level representations of integers.
You'll do this by implementing integer arithmetic for 16-bit signed and unsigned integers using only bitwise operators.

The instructions are written assuming you will edit and run the code on \runtimeenvironment.
If you wish, you may edit and run the code in a different environment;
be sure that your compiler suppresses no warnings, and that if you are using an IDE that it is configured for C and not C++.

\section*{Learning Objectives}

After successful completion of this assignment, students will be able to:
\begin{itemize}
    \item Recognize several powers of two
    \item Apply bit operations in non-trivial applications
    \item Illustrate ripple-carry binary addition
    \item Express common boolean and comparison operations in terms of other functions
    \item Recognize whether integer overflow has occurred
    \item Explain the relationship between multiplication, division, and bit shifts
    \item Express multiplication as an efficient use of other functions
\end{itemize}

\subsection*{Continuing Forward}

You used bit operations for toy applications in KeyboardLab;
in IntegerLab you will use bit operations to re-implement integer arithmetic.
This increased familiarity with bit operations will pay off handsomely in next lab and also in the I/O labs.
Your increased understanding of integer operations will improve your performance on the exam.

\section*{During Lab Time}

During your lab period, the TAs will review ripple-carry addition, overflow for unsigned and signed integers, and bitshift-based multiplication and division.
During the remaining time, the TAs will be available to answer questions.

Before leaving lab, \textit{at a minimum} complete Section~\ref{sec:utility}.
If you leave before completing Sections~\ref{subsec:one-bit-full-adder} and \ref{subsec:ripple-carry-adder}, be sure that you understand 1-bit full adders and how they can be composed to create a ripple-carry adder.
