Open \textit{basetwo.c} in your editor.
You will see the stubs of two functions there.


\subsection{exponentiate()}

This function produces a power of two.
Treating its argument as an exponent, it returns the value $2^{exponent}$ when $0 \le exponent < 32$.
If $exponent < 0$ or $exponent \ge 32$, the function must return \textit{something}, but we do not require that it return any specific value.

A characteristic of powers of two is that when represented in binary, exactly one bit is 1 and all others are 0.

You should be able to implement this function with a single line of code.


\subsection{lg()}

This function is the inverse of \function{exponentiate()}:
it produces the base-2 logarithm of its argument.
Assuming its argument is a power of two, then if $power\_of\_two = 2^{exponent}$, the function will return $exponent$.
If the argument is not a power of two, the function must return \textit{something}, but we do not require that it return any specific value.

A solution that would require only two or three lines of code is to apply bit shifts to the argument, counting the number of shifts necessary until you determine the position of the one bit that is a 1.
The problem with this simple solution is that it requires addition.
You may use this simple solution, provided that you do not use the \function{lg()} function when implementing addition (otherwise, you would create an infinite recursion).

The function stub suggests an alternative:
use a \lstinline{switch} statement, enumerating the 32 possible cases and returning the appropriate value in each case.


\subsection*{Check your work}

Compile and run \texttt{\textbf{\textit{./integerlab}}}, trying a few values.
For example:
\begin{verbatim}
    Enter a one- or two-operand logical expression,
        a two-operand comparison expression, a two-operand arithmetic expression,
        "lg <value>" or "exponentiate <value>" to test your powers-of-two code,
        "is_negative <value>" to determine if 2's complement value is negative,
        "add1 <binary_value1> <binary_value2> <carry_in>" for 1-bit full adder,
        "add32 <hex_value1> <hex_value2> <carry_in>" for 32-bit ripple-carry adder,
        or "quit": exponentiate 10
    expected: 2**10 == 0x00000400 == 1024
    actual:   2**10 == 0x00000400 == 1024

    Enter ... "lg <value>" or "exponentiate <value>" ... or "quit": lg 1024
    expected: log2 1024 == log2 0x00000400 == 10
    actual:   log2 1024 == log2 0x00000400 == 10

    Enter ... "lg <value>" or "exponentiate <value>" ... or "quit": lg 0x0400
    expected: log2 1024 == log2 0x00000400 == 10
    actual:   log2 1024 == log2 0x00000400 == 10
\end{verbatim}

The expected results come from the \textit{math} library's \function{exp2()} and \function{log2()} functions.
The actual results come, of course, from the code you wrote.

Check your code with other values, comparing your actual results with the expected results.


\vspace{0.5cm}

Open \textit{alu.c} in your editor.
You will see the stubs of several functions there.


\subsection{is\_negative()} \label{subsec:negative}

Real ALUs typically have hardware dedicated to quickly determining whether a value is 0 or not, and \textit{alu.h} includes the macros \function{is_zero()} and \function{is_not_zero)()} to serve this purpose.
Real ALUs also typically have hardware dedicated to quickly determining whether an integer, when treated as a signed value, is negative.

Implement \function{is_negative()} to determine whether its argument, when interpreted as a signed value, is negative.
The function shall return 1 when the value is negative, and 0 when it is non-negative.
You should be able to implement this function in a single line of code (but you may use more, provided you comply with the assignment's constraints).


\subsection{equal() and not\_equal()}

The general approach to comparing two values requires arithmetic, as discussed in Section~\ref{sec:inequality-comparison}.
If you do not anticipate testing the equality of two values in your arithmetic, then you can postpone implementing \function{equal()} and \function{not_equal()} until later.
On the other hand, if you think that you might need to test for equality as part of your arithmetic functions, there is a simple test for equality that does not require arithmetic.

To implement each of the \function{equal()} and \function{not_equal()} functions, you will need one 2-operand bitwise operation, either bitwise AND, bitwise OR, or bitwise XOR\@.
Recognize that so far you have only three tests you can make on the output of that bitwise operation: \function{is_zero()}, \function{is_not_zero()}, and \function{is_negative()}.
Now consider what the output of each of those three bitwise operations would be if the two operands were the same, and what the output would be if the two operands were different.
One of those six possibilities will have a predictable output that can be evaluated with one or more of the three existing tests.


\subsection*{Check your work}

    Compile and run \texttt{\textbf{\textit{./integerlab}}}, trying a few values.
    For example:
    \begin{verbatim}
    Enter ... "is_negative <value>" ... or "quit": is_negative 1
    expected: 1 (0x0001) is not negative
    actual:   1 (0x0001) is not negative

    Enter ... "is_negative <value>" ... or "quit": is_negative -1
    expected: -1 (0xFFFF) is negative
    actual:   -1 (0xFFFF) is negative

    Enter ... a two-operand comparison expression ... or "quit": 1 == 1
    expected: (1 == 1) = 1
    actual:   (1 == 1) = 1

    Enter ... a two-operand comparison expression ... or "quit": 1 != 1
    expected: (1 != 1) = 0
    actual:   (1 != 1) = 0

    Enter ... a two-operand comparison expression ... or "quit": 1 == -1
    expected: (1 == -1) = 0
    actual:   (1 == -1) = 0

    Enter ... a two-operand comparison expression ... 1 != -1
    expected: (1 != -1) = 1
    actual:   (1 != -1) = 1
\end{verbatim}

Check your code with other values, comparing your actual results with the expected results.


\vspace{0.5cm}

Implementing logical NOT, logical AND, and logical OR is not quite as simple as applying the corresponding bitwise operations, but it is very nearly so.

\subsection{logical\_not()}

When is a value considered to be \lstinline{false}?
From among the tests that you have available, one of these will return \lstinline{true} when that condition is satisfied, and \lstinline{false} when it is not.

\subsection{logical\_and() and logical\_or()}

When is a value considered to be \lstinline{true}?
From among the tests that you have available, one of these will return \lstinline{true} when that condition is satisfied, and \lstinline{false} when it is not.
Specifically, it will return a 1 or a 0, as appropriate.

You could not simply apply bitwise AND and bitwise OR to the original values because their bits might not line up -- for example, \lstinline{0x5 & 0xA == 0x0}.
Having reduced these values to a 1 or a 0, then their bits will line up, and you can now apply a bitwise operation to the results of the aforementioned test.

\textit{Note:} you are not required to preserve C's ``shortcut evaluation'' of the logical AND and logical OR operations.
Indeed, you cannot because the semantics of C's functions requires that both arguments to \function{logical_and()} and \function{logical_or()} be evaluated before your code has the opportunity to determine their truth values.


\subsection*{Check your work}

Compile and run \texttt{\textbf{\textit{./integerlab}}}, trying a few values.
For example:
\begin{verbatim}
    Enter a one- or two-operand logical expression or "quit": !0
    expected: !0 = 1
    actual:   !0 = 1

    Enter a one- or two-operand logical expression or "quit": !1
    expected: !1 = 0
    actual:   !1 = 0

    Enter a one- or two-operand logical expression or "quit": 0 && 42
    expected: 0 && 42 = 0
    actual:   0 && 42 = 0

    Enter a one- or two-operand logical expression or "quit": 0 || 73
    expected: 0 || 73 = 1
    actual:   0 || 73 = 1
\end{verbatim}

Check your code with other values, comparing your actual results with the expected results.
