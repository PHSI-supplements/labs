Before you can multiply arbitrary values, you must be able to multiply by a power of two.

\subsection{Multiplication by a Power of Two}

Strictly speaking, the \function{multiply_by_power_of_two()} function needs to do a little bit more than multiplying by a power of two.
If the \lstinline{power_of_two} argument is 0, then the function should return 0.
Otherwise, assume that it is a power of two and apply the fast multiplication technique for powers of two discussed in Chapter~3 and in lecture.

Note that the second argument is the power of two value, such as 0x0040 ($64_{10}$) or 0x2000 ($8192_{10}$) and \textit{not} the exponent of two, such as 6 or 13.
Be sure to remember that \function{multiply_by_power_of_two()} returns a 32-bit value.

\textit{Your solution for \function{multiply_by_power_of_two()} should be a constant-time solution.}
If your solution includes a loop or recursion, please review the Chapter~3 material.

\subsubsection*{Check your work}

Compile and run \texttt{\textbf{\textit{./integerlab}}}, trying a few values.
When you enter the inputs for your power-of-two multiplier, the operands will be interpreted as hexadecimal values even if you omit the leading ``0x''.
For example:
\begin{verbatim}
    Enter ... "mul2 <hex_value> <hex_power_of_two>" ...:  mul2 5 4
    expected: 0x0005 * 0x0004 = 0x00000014
    actual:   0x0005 * 0x0004 = 0x00000014

    Enter ... "mul2 <hex_value> <hex_power_of_two>" ...:  mul2 5 0
    expected: 0x0005 * 0x0000 = 0x00000000
    actual:   0x0005 * 0x0000 = 0x00000000

    Enter ... "mul2 <hex_value> <hex_power_of_two>" ...:  mul2 5 0
    expected: 0xFFFF * 0x8000 = 0x7FFF8000
    actual:   0xFFFF * 0x8000 = 0x7FFF8000
\end{verbatim}

Check your code with other values, comparing your actual results with the expected results.


\subsection{General Unsigned Multiplication}

The distributive property of multiplication tells us that if $multiplier = \sum_{i=0}^{31}multiplier_i \times 2^i$ then
\begin{align*}
    multiplicand \times multiplier  & = multiplicand \times \sum_{i=0}^{31}multiplier_i \times 2^i \\
                                    & = \sum_{i=0}^{31} multiplicand \times multiplier_i \times 2^i
\end{align*}

In the \function{unsigned_multiply()} function, use each of the \lstinline{multiplier}'s bits, in turn, as the \lstinline{power_of_two} argument to \function{multiply_by_power_of_two()} to multiply \lstinline{multiplicand}.
Add each of these intermediate products to arrive at the 32-bit product of $multiplicand \times multiplier$.

When multiplying two 16-bit operands, a real ALU will spread the 32-bit full product across two 16-bit registers.
High-level languages will only access the register containing the 16-bit product when assigning the result to the destination variable.
Assembly language, however, lets programmers access both registers.

Place the 16-bit product, the lower 16 bits of the full product, in \lstinline{product}'s \lstinline{result} field.
Place the upper 16 bits of the full product in \lstinline{product}'s \lstinline{supplemental_result} field.


\subsubsection*{Check your work}

Compile and run \texttt{\textbf{\textit{./integerlab}}}, trying a few values.

(Note that unless and until you implement signed multiplication, your ``SIGNED MULTIPLICATION'' actual results will differ from the expected results.
You are not required to implement signed multiplication.)

For example:
\begin{verbatim}
    Enter ... a two-operand arithmetic expression... or "quit": 3 * 5
    UNSIGNED MULTIPLICATION
        expected result (hexadecimal): 0x0003 * 0x0005 = 0x0000'000F
        expected result (unsigned):    3 * 5 = 15 (15)
        actual result (hexadecimal):   0x0003 * 0x0005 = 0x0000'000F
        actual result (unsigned):      3 * 5 = 15 (15)
    SIGNED MULTIPLICATION
        expected result (hexadecimal): 0x0003 * 0x0005 = 0x0000'000F
        expected result (signed):      3 * 5 = 15 (15)
        actual result (hexadecimal):   0x0003 * 0x0005 = 0x0000'0000
        actual result (signed):        3 * 5 = 0 (0)

    Enter ... a two-operand arithmetic expression... or "quit": 0x234 * 0x345
    UNSIGNED MULTIPLICATION
        expected result (hexadecimal): 0x0234 * 0x0345 = 0x0007'3404
        expected result (unsigned):    564 * 837 = 13316 (472068)
        actual result (hexadecimal):   0x0234 * 0x0345 = 0x0007'3404
        actual result (unsigned):      564 * 837 = 13316 (472068)
    SIGNED MULTIPLICATION
        expected result (hexadecimal): 0x0234 * 0x0345 = 0x0007'3404
        expected result (signed):      564 * 837 = 13316 (472068)
        actual result (hexadecimal):   0x0234 * 0x0345 = 0x0000'0000
        actual result (signed):        564 * 837 = 0 (0)
\end{verbatim}

If you are performing this lab on \runtimeenvironment, then the expected results (including the upper 16 bits) come directly from the two registers used by processor's ALU and are authoritative.

Check your code with other values, comparing your actual results with the expected results.
Generate products that fit within the lower 16 bits and products that require more.


\subsection{Unsigned Division by a Power of Two}

As discussed in Chapter~3 and in lecture, there is a fast division technique when the divisor is a power of two.
In the \function{unsigned_divide()} function, if the divisor is 0 then set the \lstinline{divide_by_zero} flag.
Otherwise, use that fast technique to implement division by a power of two.
\textit{Do not implement general division.}

When dividing integers, a real ALU will place the quotient in one register and the remainder in another.
When assigning the result to the destination variable, a high-level language will only access the register containing the quotient or the register containing the remainder, depending on whether the program called for division or the modulo operator.
Assembly language, however, lets programmers access both registers.

Place the quotient in \lstinline{quotient}'s \lstinline{result} field, and place the remainder in \lstinline{quotient}'s \lstinline{supplemental_result} field.

\textit{Your solution to determine the quotient should be a constant-time solution.}
If your solution includes a loop or recursion, please review the Chapter~3 material.


\subsubsection*{Check your work}

Compile and run \texttt{\textbf{\textit{./integerlab}}}, trying a few values.

(Note that unless and until you implement signed division, your ``SIGNED DIVISION'' actual results will differ from the expected results.
You are not required to implement signed division.)

For example:
\begin{small}\begin{verbatim}
    Enter ... a two-operand arithmetic expression... or "quit": 70 / 8
    UNSIGNED DIVISION
        expected result (hexadecimal): 0x0046 / 0x0008 = 0x0008    0x0046 % 0x0008 = 0x0006
        expected result (unsigned):    70 / 8 = 8    70 % 8 = 6
        actual result (hexadecimal):   0x0046 / 0x0008 = 0x0008    0x0046 % 0x0008 = 0x0006
        actual result (unsigned):      70 / 8 = 8    70 % 8 = 6
    SIGNED DIVISION
        expected result (hexadecimal): 0x0046 / 0x0008 = 0x0008    0x0046 % 0x0008 = 0x0006
        expected result (signed):      70 / 8 = 8    70 % 8 = 6
        actual result (hexadecimal):   0x0046 / 0x0008 = 0x0000    0x0046 % 0x0008 = 0x0000
        actual result (signed):        70 / 8 = 0    70 % 8 = 0

    Enter ... a two-operand arithmetic expression... or "quit": 0x29B / 0x40
    UNSIGNED DIVISION
        expected result (hexadecimal): 0x029B / 0x0040 = 0x000A    0x029B % 0x0040 = 0x001B
        expected result (unsigned):    667 / 64 = 10    667 % 64 = 27
        actual result (hexadecimal):   0x029B / 0x0040 = 0x000A    0x029B % 0x0040 = 0x001B
        actual result (unsigned):      667 / 64 = 10    667 % 64 = 27
    SIGNED DIVISION
        expected result (hexadecimal): 0x029B / 0x0040 = 0x000A    0x029B % 0x0040 = 0x001B
        expected result (signed):      667 / 64 = 10    667 % 64 = 27
        actual result (hexadecimal):   0x029B / 0x0040 = 0x0000    0x029B % 0x0040 = 0x0000
        actual result (signed):        667 / 64 = 0     667 % 64 = 0

    Enter ... a two-operand arithmetic expression... or "quit": 53 / 0
    UNSIGNED DIVISION
        expected result: divide-by-zero
        actual result:   divide-by-zero
    SIGNED DIVISION
        expected result: divide-by-zero
        actual result (hexadecimal):   0x0035 / 0x0000 = 0x0000    0x0035 % 0x0000 = 0x0000
        actual result (signed):        53 / 0 = 0    53 % 0 = 0
\end{verbatim}\end{small}

If you are performing this lab on \runtimeenvironment, then the expected results (including the upper 16 bits) come directly from the two registers used by processor's ALU and are authoritative.

Check your code with other values, comparing your actual results with the expected results.
Remember that the divisor must be either 0 or a power of two.
