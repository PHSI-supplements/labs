%%
%% IntegerLab (c) 2018-22 Christopher A. Bohn
%%
%% Licensed under the Apache License, Version 2.0 (the "License");
%% you may not use this file except in compliance with the License.
%% You may obtain a copy of the License at
%%     http://www.apache.org/licenses/LICENSE-2.0
%% Unless required by applicable law or agreed to in writing, software
%% distributed under the License is distributed on an "AS IS" BASIS,
%% WITHOUT WARRANTIES OR CONDITIONS OF ANY KIND, either express or implied.
%% See the License for the specific language governing permissions and
%% limitations under the License.
%%

%%
%% (c) 2021 Christopher A. Bohn
%%

\documentclass[12pt]{article}

\usepackage{fullpage}
\usepackage{fancyhdr}
\usepackage[procnames]{listings}
\usepackage{hyperref}
\usepackage{textcomp}
\usepackage{bold-extra}
\usepackage[dvipsnames]{xcolor}
\usepackage{etoolbox}


% Customize the semester (or quarter) and the course number

\newcommand{\courseterm}{Spring 2022}
\newcommand{\coursenumber}{CSCE 231}

% Customize how a typical lab will be managed;
% you can always use \renewcommand for one-offs

\newcommand{\runtimeenvironment}{your account on the \textit{csce.unl.edu} Linux server}
\newcommand{\filesource}{Canvas or {\footnotesize$\sim$}cse231 on \textit{csce.unl.edu}}
\newcommand{\filesubmission}{Canvas}

% These are placeholder commands and will be renewed in each lab

\newcommand{\labnumber}{}
\newcommand{\labname}{Lab \labnumber\ Assignment}
\newcommand{\shortlabname}{}
\newcommand{\duedate}{}

% Individual or team effort

\newcommand{\individualeffort}{This is an individual-effort project. You may discuss concepts and syntax with other students, but you may discuss solutions only with the professor and the TAs. Sharing code with or copying code from another student or the internet is prohibited.}
\newcommand{\teameffort}{This is a team-effort project. You may discuss concepts and syntax with other students, but you may discuss solutions only with your assigned partner(s), the professor, and the TAs. Sharing code with or copying code from a student who is not on your team, or from the internet, is prohibited.}
\newcommand{\freecollaboration}{In addition to the professor and the TAs, you may freely seek help on this assignment from other students.}
\newcommand{\collaborationrules}{}

% Do you care about software engineering?

\providebool{allowspaghetticode}

\setbool{allowspaghetticode}{false}

\newcommand{\softwareengineeringfrontmatter}{
    \ifboolexpe{not bool{allowspaghetticode}}{
        \section*{No Spaghetti Code Allowed}
        In the interest of keeping your code readable, you may \textit{not} use
        any \lstinline{goto} statements, nor may you use any \lstinline{break}
        statements to exit from a loop, nor may you have any functions
        \lstinline{return} from within a loop.
    }{}
}

\newcommand{\spaghetticodepenalties}[1]{
    \ifboolexpe{not bool{allowspaghetticode}}{
        \penaltyitem{1}{for each \lstinline{goto} statement, \lstinline{break}
            statement used to exit from a loop, or \lstinline{return} statement
            that occurs within a loop.}
    }{}
}

% You shouldn't need to customize these,
% but you can if you like

\lstset{language=C, tabsize=4, upquote=true, basicstyle=\ttfamily}
\newcommand{\function}[1]{\textbf{\lstinline{#1}}}
\setlength{\headsep}{0.7cm}
\hypersetup{colorlinks=true}

\newcommand{\startdocument}{
    \pagestyle{fancy}
    \fancyhf{}
    \lhead{\coursenumber}
    \chead{\ Lab \labnumber: \labname}
    \rhead{\courseterm}
    \cfoot{\shortlabname-\thepage}

	\begin{document}
	\title{\ Lab \labnumber}
	\author{\labname}
	\date{Due: \duedate}
	\maketitle

    \textit{\collaborationrules}
}

\newcommand{\rubricitem}[2]{\item[\underline{\hspace{1cm}} +#1] #2}
\newcommand{\bonusitem}[2]{\item[\underline{\hspace{1cm}} Bonus +#1] #2}
\newcommand{\penaltyitem}[2]{\item[\underline{\hspace{1cm}} -#1] #2}

%%
%% labs/common/semester.tex
%% (c) 2021-22 Christopher A. Bohn
%%
%% Licensed under the Apache License, Version 2.0 (the "License");
%% you may not use this file except in compliance with the License.
%% You may obtain a copy of the License at
%%     http://www.apache.org/licenses/LICENSE-2.0
%% Unless required by applicable law or agreed to in writing, software
%% distributed under the License is distributed on an "AS IS" BASIS,
%% WITHOUT WARRANTIES OR CONDITIONS OF ANY KIND, either express or implied.
%% See the License for the specific language governing permissions and
%% limitations under the License.
%%


% Customize the semester (or quarter) and the course number

\newcommand{\courseterm}{Fall 2022}
\newcommand{\coursenumber}{CSCE 231}

% Customize how a typical lab will be managed;
% you can always use \renewcommand for one-offs

\newcommand{\runtimeenvironment}{your account on the \textit{csce.unl.edu} Linux server}
\newcommand{\filesource}{Canvas or {\footnotesize$\sim$}cse231 on \textit{csce.unl.edu}}
\newcommand{\filesubmission}{Canvas}

% Customize for the I/O lab hardware

\newcommand{\developmentboard}{Arduino Nano}
%\newcommand{\serialprotocol}{SPI}
\newcommand{\serialprotocol}{I2C}
%\newcommand{\displaymodule}{MAX7219digits}
%\newcommand{\displaymodule}{MAX7219matrix}
\newcommand{\displaymodule}{LCD1602}

\setbool{usedisplayfont}{true}

\newcommand{\obtaininghardware}{
    The EE Shop has prepared ``class kits'' for CSCE 231; your class kit costs \$30.
    The EE Shop is located at 122 Scott Engineering Center and is open M-F 7am-4pm. You do not need an appointment.
    You may pay at the window with cash, with a personal check, or with your NCard.
    The EE shop does \textit{not} accept credit cards.
}

% Update to reflect the CS2 course(s) at your institute

\newcommand{\cstwo}{CSCE~156, RAIK~184H, or SOFT~161}

% Do you care about software engineering?

\setbool{allowspaghetticode}{false}

% Which assignments are you using this semester, and when are they due?

\newcommand{\pokerlabnumber}{1}
\newcommand{\pokerlabcollaboration}{
    Sections~\ref{sec:connecting}, \ref{sec:terminology}, \ref{sec:gettingstarted}, \ref{subsec:typesofpokerhands}, and~\ref{subsec:studythecode}: \freecollaboration
    Sections~\ref{sec:completingcard} and~\ref{subsec:completepoker}: \individualeffort
}
\newcommand{\pokerlabdue}{Week of August 29, before the start of your lab section}

\newcommand{\keyboardlabnumber}{2}
\newcommand{\keyboardlabcollaboration}{\individualeffort}
\newcommand{\keyboardlabdue}{Week of January 31, before the start of your lab section}

\newcommand{\pointerlabnumber}{3}
\newcommand{\pointerlabcollaboration}{\individualeffort}
\newcommand{\pointerlabdue}{Week of February 7, before the start of your lab section}

\newcommand{\integerlabnumber}{4}
\newcommand{\integerlabcollaboration}{\individualeffort}
\newcommand{\integerlabdue}{Week of February 14, before the start of your lab section}

\newcommand{\floatlabnumber}{5}
\newcommand{\floatlabcollaboration}{\individualeffort}
\newcommand{\floatlabdue}{soon}

\newcommand{\addressinglabnumber}{6}
\newcommand{\addressinglabcollaboration}{\individualeffort}
\newcommand{\addressinglabdue}{Week of February 28, before the start of your lab section}

%bomblab was 7
%attacklab was 8

\newcommand{\pollinglabnumber}{9}
\newcommand{\pollinglabcollaboration}{\individualeffort}
\newcommand{\pollinglabdue}{Week of April 11, before the start of your lab section}
\newcommand{\pollinglabenvironment}{your \developmentboard-based class hardware kit}

\newcommand{\ioprelabnumber}{\pollinglabnumber-prelab}
\newcommand{\ioprelabcollaboration}{\freecollaboration}
\newcommand{\ioprelabdue}{Before the start of your lab section on April 5 or 6}

\newcommand{\interruptlabnumber}{10}
\newcommand{\interruptlabcollaboration}{\individualeffort}
\newcommand{\interruptlabdue}{Week of April 18, before the start of your lab section}
\newcommand{\interruptlabenvironment}{your \developmentboard-based class hardware kit}

\newcommand{\capstonelab}{ComboLock}    % this will come into play when we generalize capstonelab
\newcommand{\capstonelabnumber}{11}
\newcommand{\capstonelabcollaboration}{\teameffort}
\newcommand{\capstonelabdue}{Week of May 2, Before the start of your lab section\footnote{See Piazza for the due dates of teams with students from different lab sections.}}
\newcommand{\capstonelabenvironment}{your \developmentboard-based class hardware kit}

\newcommand{\memorylabnumber}{12}
\newcommand{\memorylabcollaboration}{This is an individual-effort project. You may discuss the nature of memory technologies and of memory hierarchies with classmates, but you must draw your own conclusions.}
\newcommand{\memorylabdue}{Week of May 2, at the end of your lab section}
\newcommand{\memorylabenvironment}{your \developmentboard-based class hardware kit and your account on the \textit{csce.unl.edu} Linux server}

% Labs not used this semester

\newcommand{\concurrencylabnumber}{XX}
\newcommand{\concurrencylabcollaboration}{\individualeffort}
\newcommand{\concurrencylabdue}{not this semester}

\newcommand{\ssbcwarmupnumber}{XX}
\newcommand{\ssbcwarmupcollaboration}{\freecollaboration}
\newcommand{\ssbcwarmupdue}{not this semester}

\newcommand{\ssbcpollingnumber}{XX}
\newcommand{\ssbcpollingcollaboration}{\individualeffort}
\newcommand{\ssbcpollingdue}{not this semester}

\newcommand{\ssbcinterruptnumber}{XX}
\newcommand{\ssbcinterruptcollaboration}{\individualeffort}
\newcommand{\ssbcinterruptdue}{not this semester}

\usepackage{cancel}

\renewcommand{\labnumber}{\integerlabnumber}
\renewcommand{\labname}{Integer Representation and Arithmetic Lab}
\renewcommand{\shortlabname}{integerlab}
\renewcommand{\collaborationrules}{\integerlabcollaboration}
\renewcommand{\duedate}{\integerlabdue}
\pagelayout
\begin{document}
\labidentifier

%\usepackage{fullpage}
%\usepackage{enumitem}

%% TODO: remove bonus credit (or at least make it selectable on/off) and
%% introduce comparators and logical operators; update parser


In this assignment, you will become more familiar with bit-level
representations of integers.  You'll do this by implementing integer arithmetic
for 16-bit signed and unsigned integers using only bitwise operators.

The instructions are written assuming you will edit and run the code on
\runtimeenvironment. If you wish, you may edit and run the code
in a different environment; be sure that your compiler suppresses no warnings,
and that if you are using an IDE that it is configured for C and not C++.

\section*{Learning Objectives}

After successful completion of this assignment, students will be able to:
\begin{itemize}
\item Apply bit operations in non-trivial applications
\item Illustrate ripple-carry binary addition
\item Recognize whether integer overflow has occurred
\item Explain the relationship between multiplication, division, and bit shifts
\item Express multiplication as an efficient use of other functions

\end{itemize}

\subsection*{Continuing Forward}

You used bit operations for toy applications in KeyboardLab; in IntegerLab you
will use bit operations to re-implement integer arithmetic. This increased
familiarity with bit operations will pay off handsomely in next lab and
also in the I/O labs. Your increased understanding of integer operations will
improve your performance on the exam.

\section*{During Lab Time}

During your lab period, the TAs will review ripple-carry addition, overflow for
unsigned and signed integers, and bit shift-based multiplication and division.
During the remaining time, the TAs will be available to answer questions.

\softwareengineeringfrontmatter

\section{Scenario}

All work at the Pleistocene Petting Zoo has stopped while Archie tries to find
a $\cancelto{\mathrm{reasonable}}{\mathrm{gullible}}$ insurance company. Rather
than furloughing staff, he's asked everybody to help out with his other startup
companies for a week or two. He specifically asked that you help out with
Eclectic Electronics.

Herb Bee, the chief engineer, explains that Eclectic Electronics is developing
a patent-pending C-licon tool that will convert C code into an integrated
circuit that has the same functionality as the original C code. To test it out,
he tasked you with writing the code to implement an Arithmetic Logic Unit
(ALU). Your task will be to implement integer addition, subtraction,
multiplication, and division. Because bitwise operations and bit shift
operations have been implemented, you will be able to use C's bitwise and bit
shift operators, but because arithmetic operations have not yet been
implemented, you cannot use C's arithmetic operators. Because C library
functions generally make use of arithmetic operations (which have not yet been
implemented), you cannot use library functions.

\section{Getting Started}

Download \textit{\shortlabname.zip} or \textit{\shortlabname.tar} from
\filesource\ and copy it to \runtimeenvironment. Once copied, unpackage the
file. Three of the five files (\textit{alu.h}, \textit{alu.c}, and
\textit{integerlab.c}) contain the starter code for this assignment. The fourth
file (\textit{integergrader.c}) contains code to run your code through the lab's
rubric. The last file (\textit{Makefile}) tells the \texttt{make} utility how
to compile the code. To compile the program, type:

\texttt{make}

This will produce an executable file called \texttt{integerlab}.
%  If you're
% using your own computer and you don't have \texttt{make} available to you, then
% you can compile the program by typing:
%
% \texttt{gcc -std=c99 -Wall -o integerlab integerlab.c alu.c}

\section{Description of IntegerLab Files and Tasks}

\subsection{alu.h}

Do not edit \textit{alu.h}.

This header file contains the function declarations for \function{add()},
\function{subtract()}, \function{multiply()}, and \function{divide()}.  It also
declares a global variable:
\begin{description}
\item[is\_signed] This boolean is used to indicate whether the functions should
    treat the values as signed integers or as unsigned integers.
\end{description}
Finally, it contains three type defintions for arithmetic results:
\begin{description}
\item[addition\_subtraction\_result] This structure has two fields. The
    \lstinline{result} field is to store the sum or difference (as
    appropriate). The \lstinline{overflow} field should be set to be
    \lstinline{true} if the full answer does not fit in the 16-bit
    \lstinline{result} and \lstinline{false} if the full answer does fit.
\item[multiplication\_result] This structure has three fields. The
    \lstinline{product} field is to store the lowest 16 bits of the product. The
    \lstinline{full_product} field is to store the full 32-bit product. The
    \lstinline{overflow} field should be set to be \lstinline{true} if the full
    answer does not fit in the 16-bit \lstinline{product} and \lstinline{false}
    if the full answer does fit.
\item[division\_result] This structure has three fields. The
    \lstinline{quotient} field is to store the integer quotient, and the
    \lstinline{remainder} field is to store the integer remainder.
    Mathematically, $dividend \div divisor = quotient +
    \frac{remainder}{divisor}$. The \lstinline{division_by_zero} field should
    be set to \texttt{true} if the \function{divide()} function cannot compute
    the quotient because the divisor is 0 and \lstinline{false} otherwise.
\end{description}

\subsection{integerlab.c}

Do not edit \textit{integerlab.c}.

This file contains the driver code for the lab.  It parses your input, calls
the appropriate arithmetic function, and displays the output.

\subsection{integergrader.c}

Do not edit \textit{integergrader.c}.

This file contains alternate driver code for the lab.  It generates inputs for
each of the test cases, calls the appropriate arithmetic function, and displays
the result. After all test cases have been run, an initial score will be
calculated (this score is subject to change due to violating the assignment's
requirements).

\subsection{alu.c}

This file contains stubs for the four functions you need to edit.  Add your
name in comments as indicated, and write the code.  In addition to the four
functions, you may add helper functions to make your code more modular; you may
only place these helper functions in \textit{alu.c}.

When you implement these functions, you may NOT use C's arithmetic operators:
+\ \ ++\ \ +=\ \ -\ \ -{}-\ \ -=\ \ $*$\ \ /\ \ \%. You may not use
floating-point operators as a substitute for integer operators. You may use
only functions that you write yourself). If you use \lstinline{printf}
statements for debugging instead of using a debugger, remove them before
turning in your code. \textit{You will receive no credit for functions that use
a prohibited operator or function.}  You may only use bitwise and $\&$, bitwise
or $|$, bitwise exclusive-or \^{ }, bitwise complement $\sim$, and bit shifts
$<<$ $>>$.

\textbf{Hints:}
\begin{itemize}
\item The value 0x8, if right-shifted one position becomes 0x4 which is
    logically \lstinline{true}. If right-shifted by one position a second time,
    the value becomes 0x2 which is logically \lstinline{true}. If right-shifted
    by one position a third time, the value becomes 0x1 which is logically
    \lstinline{true}. If right-shifted by one position a fourth time, the value
    becomes 0x0 which is logically \lstinline{false}. If you generalize this
    idea, you may find a way to control a loop without an arithmetic operator.
\item After you have written the \function{add()} function, you may use it in
    other functions to control loops and for other other purposes.
\end{itemize}

\subsubsection*{add()} Takes two 16-bit integers and adds them.  The sum should
be stored as a 16-bit value in \lstinline{result}.  If \lstinline{is_signed}
is true, treat all values as signed integers; otherwise, treat all values as
unsigned integers.  If addition overflowed, set \lstinline{overflow} to
\lstinline{true}.
\begin{itemize}
\item Addition must work for both signed and unsigned integers.
\item You may find it beneficial for another part of the lab if you implement a
    32-bit full adder; you can have \function{add()} call the 32-bit full
    adder.
\end{itemize}

\subsubsection*{subtract()} Takes two 16-bit integers and subtracts the second
from the first.  The difference should be stored as a 16-bit value in
\lstinline{result}.  If \lstinline{is_signed} is true, treat all values as
signed integers; otherwise, treat all values as unsigned integers.  If
subtraction overflowed, set \lstinline{overflow} to \lstinline{true}.
\begin{itemize}
\item Subtraction must work for both signed and unsigned integers.
\end{itemize}

\subsubsection*{multiply()} Takes two 16-bit integers and multiplies them.  The
lowest 16 bits of the product should be stored in \lstinline{product}, and the
full product should be stored in \lstinline{full_product} as a 32-bit value.
If the full product doesn't fit in the 16-bit \lstinline{result} then set
\lstinline{overflow} to \lstinline{true}.
\begin{itemize}
\item Only implement multiplication for unsigned integers.  You do not need to
    implement multiplication for signed integers.
\item Your multiplication algorithm MUST be polynomial in the number of bits.
    \textit{You will receive no credit for multiplication if your algorithm is
    superpolynomial.} The brute-force approach of repeatedly adding
    \lstinline{multiplicand} to itself \lstinline{multiplier} times is a
    $\mathcal{O}(2^n)$ algorithm, where $n$ is the number of bits.
\item For full credit, be able to multiply any two non-negative integers that
    fit in 16 bits; for partial credit, be able to multiply by a power-of-two.
\end{itemize}

\subsubsection*{divide()}  Takes two 16-bit integers and divides the first by
the second.  The integer quotient should be stored in \lstinline{quotient}, and
the remainder should stored in \lstinline{remainder}.  If the divisor is zero,
then set \lstinline{division_by_zero} to \lstinline{true} and provide any value
as the quotient and remainder.
\begin{itemize}
\item Only implement division for unsigned integers.  You do not need to
    implement division for signed integers.
\item Your Division algorithm MUST be polynomial in the number of bits.
    \textit{You will receive no credit for division if your algorithm is
    superpolynomial.} The brute-force approach of repeatedly subtracting
    \lstinline{divisor} from \lstinline{dividend} is a $\mathcal{O}(2^n)$
    algorithm, where $n$ is the number of bits.
\item For full credit, be able to divide by a power-of-two; for bonus credit,
    be able to divide by an arbitrary non-negative integer.
\end{itemize}


\section{Running IntegerLab}

After you've compiled the program, you can run it as
\texttt{./integerlab~unsigned} to perform arithmetic on unsigned integers or as
\texttt{./integerlab~signed} to perform arithmetic on signed integers.  You
will be prompted to input a simple two-operator arithmetic expression.  After
you do so, the result of the computation will be printed and then you'll be
prompted to enter another arithmetic expression.  For example:
\begin{verbatim}
Input a simple two-operator arithmetic expression: 50+3
50 + 3 = 53
Input a simple two-operator arithmetic expression:
\end{verbatim}
This will continue until you enter a blank line, at which point the program
will terminate.

You can enter the inputs as either decimal or as hexidecimal.  If at least one
input is hexidecimal, then the output will be hexidecimal.  For example:
\begin{verbatim}
Input a simple two-operator arithmetic expression: 55 + 0x4
0x37 + 0x4 = 0x3b
\end{verbatim}

We suspect that you'll mostly use decimal inputs/outputs; however, being able to
use hexidecimal inputs/outputs may help you with debugging.

\section*{Turn-in and Grading}

When you have completed this assignment, upload \textit{alu.c} to
\filesubmission.

This assignment is worth 40 points. \\

Run \texttt{./integerlab unsigned} (or run \texttt{./integergrader})
\begin{description}
\rubricitem{2}{Satisfies additive identity; for example, 5+0 = 5}
\rubricitem{2}{Performs addition; for example, 32+10 = 42}
\rubricitem{2}{Sums between $2^{15}$ and $2^{16}-1$ do not overflow; for
    example, 30000+5000 = 35000}
\rubricitem{2}{Sums greater than $2^{16}-1$ do overflow; for example,
    60000+6000 = 464 and reports Overflow}
\rubricitem{2}{Satisfies subtractive identity; for example, 5-0 = 5}
\rubricitem{2}{Performs subtraction; for example, 40000-300 = 39700}
\rubricitem{2}{Differences of zero do not overflow; for example, 10-10 = 0}
\rubricitem{2}{Negative differences do overflow; for examlple, 2-3  = 65535 and
    reports Overflow}
\rubricitem{1}{Satisfies multiplicative identity; for example, 3*1 = 3}
\rubricitem{1}{Satisfies multiplicative zero; for example, 3*0 = 0}
\rubricitem{1}{Performs multiplication when multiplier is a power of two; for
    example, 3*4 = 12}
\rubricitem{1}{Performs multiplication when multiplier is not a power of two;
    for example, 3*5 = 15}
\rubricitem{1}{Products less than $2^{16}$ do not overflow when multiplier is a
    power of two; for example, 3000*16 = 48000}
\rubricitem{1}{Products less than $2^{16}$ do not overflow when multiplier is
    not a power of two; for example, 3000*20 = 60000}
\rubricitem{1}{Products greater than $2^{16}-1$ do overflow when multiplier is
    a power of two; for example, 3000*32 = 30464 and reports Overflow with the
    full answer 0x17700}
\rubricitem{1}{Products greater than $2^{16}-1$ do overflow when multiplier is
    not a power of two; for example, 3000*25 = 9464 and reports Overflow with
    the full answer 0x124f8}
\rubricitem{1}{Satisfies divisive identity; for example, 8/1 = 8}
\rubricitem{1}{A value divides itself once; for example, 8/8 = 1}
\rubricitem{1}{Satisfies divisive zero; for example, 0/8 = 0}
\rubricitem{1}{Reports division by zero; for example, 8/0 reports Division by
    Zero}
\rubricitem{1}{Divides a power of two by another power of two; for example,
    32/4 = 8}
\rubricitem{1}{Divides an arbitrary non-negative integer by a power of two; for
    example, 30/4 = 7 remainder 2}
\bonusitem{1}{Divides an arbitrary non-negative integer by one of its factors;
    for example, 30/5 = 6}
\bonusitem{1}{Divides an arbitrary non-negative integer by an arbitrary
    integer; for example, 32/5 = 6 remainder 2}
\end{description}


Run \texttt{./integerlab signed} (or run \texttt{./integergrader})
\begin{description}
\rubricitem{1}{Satisfies additive identity; for example, 5+0 = 5}
\rubricitem{1}{Performs addition with positive values; for example, 32+10 = 42}
\rubricitem{1}{Sums less than $2^{15}$ do not overflow; for example,
    30000+2000 = 32000}
\rubricitem{1}{Sums greater than $2^{15}-1$ do overflow; for example,
    30000+3000 = -32536 and reports Overflow}
\rubricitem{1}{Performs addition with a negative value; for example, -2+3 = 1}
\rubricitem{1}{Satisfies subtractive identity; for example, 5-0 = 5}
\rubricitem{1}{Performs subtraction; for example 200-50 = 150}
\rubricitem{1}{Can subtract a greater value from a lesser without overflowing;
    for example, 10-20 = -10}
\rubricitem{1}{Can subtract from a negative value; for example, -10-10 = -20}
\rubricitem{1}{Differences beyond $-2^{15}$ overflow; for example,
    -30000-3000 = 32536 and reports Overflow}
\end{description}

\textbf{Penalties.} Search \textit{alu.c} for arithmetic operators
(+ - * / \%), and examine algorithms.
\begin{description}
\penaltyitem{13}{The \function{add()} function (or a helper function for
    \function{add()}) uses an arithmetic operator (+ - * / \%) or a function
    you did not write}
\penaltyitem{13}{The \function{subtract()} function (or a helper function for
    \function{subtract()}) uses an arithmetic operator (+ - * / \%) or a
    function you did not write}
\penaltyitem{8}{The \function{multiply()} function (or a helper function for
    \function{multiply()}) uses an arithmetic operator (+ - * / \%) or a
    function you did not write,
    \textit{and/or} \function{multiply()} uses a superpolynomial multiplication
    algorithm, such as but not limited to brute-force repeated addition}
\penaltyitem{6}{\textbf(also no bonus) The \function{divide()} function (or a
    helper function for \function{divide()}) uses an arithmetic operator
    (+ - * / \%) or a function you did not write,
    \textit{and/or} \function{divide()} uses a superpolynomial division
    algorithm, such as but not limited to brute-force repeated subtraction}
\spaghetticodepenalties{1}
\end{description}

\section*{Epilogue}

Herb smiles as he hands you the the test results from the latest integrated
circuit fab batch. ``C-licon successfully turned your code into an ALU. Nicely
done!'' I think maybe it's time to use it to introduce some optimizations into
the Floating Point Unit (FPU) on our experimental microprocessor.

\textit{To be continued...}

\end{document}
