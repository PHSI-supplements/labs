%%
%% IntegerLab (c) 2018-23 Christopher A. Bohn
%%
%% Licensed under the Apache License, Version 2.0 (the "License");
%% you may not use this file except in compliance with the License.
%% You may obtain a copy of the License at
%%     http://www.apache.org/licenses/LICENSE-2.0
%% Unless required by applicable law or agreed to in writing, software
%% distributed under the License is distributed on an "AS IS" BASIS,
%% WITHOUT WARRANTIES OR CONDITIONS OF ANY KIND, either express or implied.
%% See the License for the specific language governing permissions and
%% limitations under the License.
%%

%%
%% labs/common/assignment.tex
%% (c) 2021-22 Christopher A. Bohn
%%
%% Licensed under the Apache License, Version 2.0 (the "License");
%% you may not use this file except in compliance with the License.
%% You may obtain a copy of the License at
%%     http://www.apache.org/licenses/LICENSE-2.0
%% Unless required by applicable law or agreed to in writing, software
%% distributed under the License is distributed on an "AS IS" BASIS,
%% WITHOUT WARRANTIES OR CONDITIONS OF ANY KIND, either express or implied.
%% See the License for the specific language governing permissions and
%% limitations under the License.
%%

\documentclass[12pt]{article}

\usepackage{fullpage}
\usepackage{fancyhdr}
\usepackage[procnames]{listings}
\usepackage{hyperref}
\usepackage{textcomp}
\usepackage{bold-extra}
\usepackage[dvipsnames]{xcolor}
\usepackage{etoolbox}

% These are placeholder commands and will be renewed in each lab

\newcommand{\labnumber}{}
\newcommand{\labname}{Lab \labnumber\ Assignment}
\newcommand{\shortlabname}{}
\newcommand{\duedate}{}

% Individual or team effort

\newcommand{\individualeffort}{This is an individual-effort project. You may
    discuss concepts and syntax with other students, but you may discuss
    solutions only with the professor and the TAs. Sharing code with or copying
    code from another student or the internet is prohibited.}
\newcommand{\teameffort}{This is a team-effort project. You may discuss concepts
    and syntax with other students, but you may discuss solutions only with your
    assigned partner(s), the professor, and the TAs. Sharing code with or
    copying code from a student who is not on your team, or from the internet,
    is prohibited.}
\newcommand{\freecollaboration}{In addition to the professor and the TAs, you
    may freely seek help on this assignment from other students.}
\newcommand{\collaborationrules}{}

% Software engineering (if you care about that)

\providebool{allowspaghetticode}

\newcommand{\softwareengineeringfrontmatter}{
    \ifboolexpe{not bool{allowspaghetticode}}{
        \section*{No Spaghetti Code Allowed}
        In the interest of keeping your code readable, you may \textit{not} use
        any \lstinline{goto} statements, nor may you use any
        \lstinline{continue} statements, nor may you use any \lstinline{break}
        statements to exit from a loop, nor may you have any functions
        \lstinline{return} from within a loop.
    }{}
}

\newcommand{\spaghetticodepenalties}[1]{
    \ifboolexpe{not bool{allowspaghetticode}}{
        \penaltyitem{1}{for each \lstinline{goto} statement,
            \lstinline{continue} statement, \lstinline{break} statement used to
            exit from a loop, or \lstinline{return} statement that occurs within
            a loop.}
    }{}
}

% You shouldn't need to customize these,
% but you can if you like

\lstset{language=C, tabsize=4, upquote=true, basicstyle=\ttfamily}
\newcommand{\function}[1]{\textbf{\lstinline{#1}}}
\setlength{\headsep}{0.7cm}
\hypersetup{colorlinks=true}

\newcommand{\pagelayout}{
    \pagestyle{fancy}
    \fancyhf{}
    \lhead{\coursenumber}
    \chead{\ Lab \labnumber: \labname}
    \rhead{\courseterm}
    \cfoot{\shortlabname-\thepage}
}

\newcommand{\labidentifier}{
    \title{\ Lab \labnumber}
    \author{\labname}
    \date{Due: \duedate}
    \maketitle

    \textit{\collaborationrules}
}

% deprecated
\newcommand{\startdocument}{
    \pagelayout
	\begin{document}
	\labidentifier
}

\newcommand{\rubricitem}[2]{\item[\underline{\hspace{1cm}} +#1] #2}
\newcommand{\bonusitem}[2]{\item[\underline{\hspace{1cm}} Bonus +#1] #2}
\newcommand{\penaltyitem}[2]{\item[\underline{\hspace{1cm}} -#1] #2}
\newcommand{\checkoffitem}[1]{\item (\phantom{xxx}) #1}
\newcommand{\precheckoffitem}[1]{\item [] (\phantom{xxx}) #1}

%%
%% labs/common/semester.tex
%% (c) 2021-22 Christopher A. Bohn
%%
%% Licensed under the Apache License, Version 2.0 (the "License");
%% you may not use this file except in compliance with the License.
%% You may obtain a copy of the License at
%%     http://www.apache.org/licenses/LICENSE-2.0
%% Unless required by applicable law or agreed to in writing, software
%% distributed under the License is distributed on an "AS IS" BASIS,
%% WITHOUT WARRANTIES OR CONDITIONS OF ANY KIND, either express or implied.
%% See the License for the specific language governing permissions and
%% limitations under the License.
%%


% Customize the semester (or quarter) and the course number

\newcommand{\courseterm}{Spring 2022}
\newcommand{\coursenumber}{CSCE 231}

% Customize how a typical lab will be managed;
% you can always use \renewcommand for one-offs

\newcommand{\runtimeenvironment}{your account on the \textit{csce.unl.edu} Linux server}
\newcommand{\filesource}{Canvas or {\footnotesize$\sim$}cse231 on \textit{csce.unl.edu}}
\newcommand{\filesubmission}{Canvas}

% Customize for the I/O lab hardware

\newcommand{\developmentboard}{Arduino Nano}

%\newcommand{\serialprotocol}{SPI}
\newcommand{\serialprotocol}{I2C}

%\newcommand{\displaymodule}{MAX7219digits}
%\newcommand{\displaymodule}{MAX7219matrix}
\newcommand{\displaymodule}{LCD1602}

\setbool{usedisplayfont}{true}

\newcommand{\obtaininghardware}{
    The EE Shop has prepared ``class kits'' for CSCE 231; your class kit costs \$20. The EE Shop is located at 122 Scott
    Engineering Center and is open M-F 7am-4pm. You do not need an appointment. You may pay at the window with cash,
    with a personal check, or with your NCard. If you wish to pay by credit card, you must make the purchase from
    \url{https://marketplace.unl.edu/ees/engineering-class-kits/csce231-kit.html} the day before you visit the EE
    Shop.\footnote{The price listed on the website is \$18.65; after sales tax is added, your total will be \$20.}
}

% Update to reflect the CS2 course(s) at your institute

\newcommand{\cstwo}{CSCE~156, RAIK~184H, or SOFT~161}

% Do you care about software engineering?

\setbool{allowspaghetticode}{false}

% Which assignments are you using this semester, and when are they due?

\newcommand{\pokerlabnumber}{1}
\newcommand{\pokerlabcollaboration}{Except as noted in Section~\ref{StudyTheCode}, \individualeffort}
\newcommand{\pokerlabdue}{Week of January 24, before the start of your lab section}

\newcommand{\keyboardlabnumber}{2}
\newcommand{\keyboardlabcollaboration}{\individualeffort}
\newcommand{\keyboardlabdue}{Week of January 31, before the start of your lab section}

\newcommand{\pointerlabnumber}{3}
\newcommand{\pointerlabcollaboration}{\individualeffort}
\newcommand{\pointerlabdue}{Week of February 7, before the start of your lab section}

\newcommand{\integerlabnumber}{4}
\newcommand{\integerlabcollaboration}{\individualeffort}
\newcommand{\integerlabdue}{Week of February 14, before the start of your lab section}

\newcommand{\floatlabnumber}{5}
\newcommand{\floatlabcollaboration}{\individualeffort}
\newcommand{\floatlabdue}{soon}

\newcommand{\addressinglabnumber}{6}
\newcommand{\addressinglabcollaboration}{\individualeffort}
\newcommand{\addressinglabdue}{Week of February 28, before the start of your lab section}

%bomblab was 7
%attacklab was 8

\newcommand{\pollinglabnumber}{9}
\newcommand{\pollinglabcollaboration}{\individualeffort}
\newcommand{\pollinglabdue}{Week of April 11, before the start of your lab section}
\newcommand{\pollinglabenvironment}{your \developmentboard-based class hardware kit}

\newcommand{\ioprelabnumber}{\pollinglabnumber-prelab}
\newcommand{\ioprelabcollaboration}{\freecollaboration}
\newcommand{\ioprelabdue}{Before the start of your lab section on April 5 or 6}

\newcommand{\interruptlabnumber}{10}
\newcommand{\interruptlabcollaboration}{\individualeffort}
\newcommand{\interruptlabdue}{Week of April 18, before the start of your lab section}
\newcommand{\interruptlabenvironment}{your \developmentboard-based class hardware kit}

\newcommand{\capstonelab}{ComboLock}    % this will come into play when we generalize capstonelab
\newcommand{\capstonelabnumber}{11}
\newcommand{\capstonelabcollaboration}{\teameffort}
\newcommand{\capstonelabdue}{Week of May 2, Before the start of your lab section\footnote{See Piazza for the due dates of teams with students from different lab sections.}}
\newcommand{\capstonelabenvironment}{your \developmentboard-based class hardware kit}

\newcommand{\memorylabnumber}{12}
\newcommand{\memorylabcollaboration}{This is an individual-effort project. You may discuss the nature of memory technologies and of memory hierarchies with classmates, but you must draw your own conclusions.}
\newcommand{\memorylabdue}{Week of May 2, at the end of your lab section}
\newcommand{\memorylabenvironment}{your \developmentboard-based class hardware kit and your account on the \textit{csce.unl.edu} Linux server}

% Labs not used this semester

\newcommand{\concurrencylabnumber}{XX}
\newcommand{\concurrencylabcollaboration}{\individualeffort}
\newcommand{\concurrencylabdue}{not this semester}

\newcommand{\ssbcwarmupnumber}{XX}
\newcommand{\ssbcwarmupcollaboration}{\freecollaboration}
\newcommand{\ssbcwarmupdue}{not this semester}

\newcommand{\ssbcpollingnumber}{XX}
\newcommand{\ssbcpollingcollaboration}{\individualeffort}
\newcommand{\ssbcpollingdue}{not this semester}

\newcommand{\ssbcinterruptnumber}{XX}
\newcommand{\ssbcinterruptcollaboration}{\individualeffort}
\newcommand{\ssbcinterruptdue}{not this semester}

%%
%% labs/common/storylines.tex
%% (c) 2020-22 Christopher A. Bohn
%%
%% Licensed under the Apache License, Version 2.0 (the "License");
%% you may not use this file except in compliance with the License.
%% You may obtain a copy of the License at
%%     http://www.apache.org/licenses/LICENSE-2.0
%% Unless required by applicable law or agreed to in writing, software
%% distributed under the License is distributed on an "AS IS" BASIS,
%% WITHOUT WARRANTIES OR CONDITIONS OF ANY KIND, either express or implied.
%% See the License for the specific language governing permissions and
%% limitations under the License.
%%

\newcommand{\MeetArchie}{
    You're relaxing at your favorite hangout when another customer catches your attention.
    He's rather large (dare I say, \textit{mammoth}), a bit hairy, and looking frustrated in front of his laptop.
    ``I'm Archie,'' he says, ``and I'm trying to teach myself this card game called \textit{Poker}.
    I found this source code that I thought I could use to understand Poker better, but the code is incomplete, and I don't entirely understand what's there.
    Could you explain the code to me, please?'
}

\newcommand{\GetHired}{
    Archie's face lights up in a very big smile.
    ``Thanks!''
    After pausing in thought for a moment, he says, ``Say, I've got a new startup company that could really use your help.
    Are you interested?
    It'll be exciting!''
}


\renewcommand{\labnumber}{\integerlabnumber}
\renewcommand{\labname}{Integer Representation and Arithmetic Lab}
\renewcommand{\shortlabname}{integerlab}
\renewcommand{\collaborationrules}{\integerlabcollaboration}
\renewcommand{\duedate}{\integerlabdue}

\pagelayout
\begin{document}
\labidentifier\

\pdfbookmark[1]{Frontmatter}{frontmatter}                                                               The purpose of this assignment is to give you more confidence in C programming
and to begin your exposure to the underlying bit-level representation of data.

The instructions are written assuming you will edit and run the code on
\runtimeenvironment. If you wish, you may edit and run the code
in a different environment; be sure that your compiler suppresses no warnings,
and that if you are using an IDE that it is configured for C and not C++.

\section*{Learning Objectives}

After successful completion of this assignment, students will be able to:
\begin{itemize}
    \item Use the ASCII table to determine the corresponding integer values of C
    \lstinline{char} values.
    \item Apply arithmetic operators and comparators to C \lstinline{}{char} values.
    \item Construct and use a bitmask.
    \item Use bitwise operators and bit shift operators to create and modify values.
\end{itemize}

\subsection*{Continuing Forward}

Your experience with viewing values as bit patterns will be applicable in
future labs, as will bit masks and bit operations. Some of the functions you
write in this lab will be used in the next lab.

\section*{During Lab Time}

During your lab period, the TAs will demonstrate how to read the ASCII table and
will provide a refresher on bitwise AND, bitwise OR, and left- and right-shifts.
During the remaining time, the TAs will be available to answer questions.

Before leaving lab, \textit{at a minimum} \dots


\softwareengineeringfrontmatter

\section*{Scenario}                                                                                     \OnLoanToEclecticElectronics

\section{Assignment Summary}                                                                            Please familiarize yourself with the entire assignment before beginning.
There are three parts to this assignment.

\subsection{Why are There Letters on Telephone Keypads?}

Once upon a time, telephone exchanges were staffed by operators who would use patch cords on a switchboard to connect callers.
If you needed to make a local-area call to someone whose phone was serviced by a different exchange, then you needed to tell your operator which exchange to connect to.
Letters were assigned to digits so that easy-to-remember -- and audibly-distinctive -- mnemonics could be formed such that the first two letters of the mnemonic that correspond to the 2-digit exchange identifier.
For example, the 86 exchange would use a mnemonic that started with a `T', `U', or `V' and that has as a second letter an `M', `N', or `O' --
so 867--5309 might be ``University 7--5309''.

The presence of letters on telephone dials and (later) telephone keypads allowed for custom phone numbers that used words formed by the available letters.
For example, a bank's phone number might be 472--2265, aka 472--BANK.
Less fictionally, 1--800--FLOWERS was used by a company that partnered with florists to allow people to have bouquets delivered anywhere in the U.S\@.

When the Short Message Service protocol was introduced to allow text-based communication by taking advantage of unused bytes in the handshake between cellular phones and the cell network, naturally the letters that were already present on the keypad were used to tap out messages.

While QWERTY keyboards on smartphones have largely replaced the 10-digit keypad for text entry, the letters remain, waiting for the next clever use\dots

\subsection{Constraints} \label{subsec:constraints}

You may \textit{not} poll the matrix keypad nor the pushbuttons to determine if they have been pressed.
You must use interrupts to determine if a key or button has been pressed.
Once an interrupt has fired, you may scan the matrix keypad or read the pushbuttons to determine which key has been pressed or whether the button has been pressed or released.

You may use any features that are part of the C standard if they are supported by the compiler.
You may use the constants and functions provided in the starter code.

\subsubsection{Constraints on the Arduino core}

You may \textit{not} use any libraries, functions, macros, types, or constants from the Arduino core.

%\subsubsection{Constraints on AVR-libc}
%
%You may use any AVR-specific functions, macros, types, or constants of avr-libc.\footnote{
%    \url{https://www.nongnu.org/avr-libc/user-manual/index.html}
%}

\ifdefstring{\processor}{ATmega328P}{
    \subsubsection{Constraints on AVR-libc}

    You may not use any AVR-specific functions, macros, types, or constants of avr-libc.\footnote{\url{https://www.nongnu.org/avr-libc/user-manual/index.html}}
}{}
\ifdefstring{\processor}{RP2040}{
% TODO: parameterize this (when we eventually port to the bare-metal Arduino toolchain, and to the Pico SDK)
    \subsubsection{Constraints on MBED OS}

    You may not use any functions, macros, types or constants from MBED that are not part of the C standard.\footnote{\url{https://os.mbed.com/docs/mbed-os/v6.16/introduction/index.html}}
}{}

\subsubsection{Constraints on the CowPi library}

You may use any functions provided by the CowPi\footnote{
    \url{https://cow-pi.readthedocs.io/en/latest/library.html}
}
and the CowPi\_stdio\footnote{
    \url{https://cow-pi.readthedocs.io/en/latest/stdio.html}
} libraries,
and you may use any data structures\footnote{
    \url{https://cow-pi.readthedocs.io/en/latest/microcontroller.html}
} provided by the CowPi library.

\subsubsection{Constraints on other libraries}

You may \textit{not} use any libraries beyond those explicitly identified here.


\section{Getting Started}                                                                               Download the zip file or tarball from \filesource.
Once downloaded, unpackage the file and open the project in your IDE\@.

\subsection{Description of RangeFinder Files}

\subsubsection{combolock.c, interrupt\_support.h, interrupt\_support.cpp, display.h, display.cpp}

Do not edit \textit{combolock.c}, \textit{interrupt\_support.h}, \textit{interrupt\_support.cpp}, \textit{display.}, or \textit{display.cpp}.

These files contain code to simplify interrupt management, and functions to place text on the display module.

\subsubsection{rotary-encoder.h \& rotary-encoder.c}

Do not edit \textit{rotary-encoder.h}

The \textit{rotary-encoder.c} file is where you will process inputs from the rotary encoder.

\subsubsection{servomotor.h \& servomotor.c}

Do not edit \textit{servomotor.h}

The \textit{servomotor.c} file is where you will control the servomotor.

\subsection{lock-controller.h \& lock-controller.c}

Do not edit \textit{lock-controller.h}

The \textit{lock-controller.c} file is where you will implement the logic for the combination lock.

%\subsubsection{shared\_variables.h}
%
%The \textit{shared\_variables.h} header file is where you will place any types that you define and where you will externalize any global variables that need to be used by more than one \textit{.c} file.
%
%It also contains a structure that you may use to access an analog-digital converter's registers,
%and it contains meaningfully-named constants to refer to specific pins you will use in this assignment.


\subsection{Assemble the Hardware}

\textcolor{red}{\textbf{BEFORE YOU PROCEED FURTHER:}}
\begin{description}
    \checkoffitem{Add the new hardware to your Cow~Pi as described in Appendix~\ref{sec:hardwareMods-mk4b}.}
\end{description}


\section{Utility Functions, Equality Comparisons, and Logical Boolean Operations}\label{sec:utility}    \begin{description}
    \checkoffitem{Open \textit{basetwo.c} in your editor.
        You will see the stubs of two functions there.}
\end{description}


\subsection{exponentiate()}

This function produces a power of two.
Treating its argument as an exponent, it returns the value $2^{exponent}$ when $0 \le exponent < 32$.
If $exponent < 0$ or $exponent \ge 32$, the function must return \textit{something}, but we do not require that it return any specific value.

A characteristic of powers of two is that when represented in binary, exactly one bit is 1 and all others are 0.

\begin{description}
    \checkoffitem{Implement the \function{exponentiate()} function.}
\end{description}
You should be able to implement this function with a single line of code,
but you may use more than one line.


\subsection{lg()}

This function is the inverse of \function{exponentiate()}:
it produces the base-2 logarithm of its argument.
Assuming its argument is a power of two, then if $power\_of\_two = 2^{exponent}$, the function will return $exponent$.
If the argument is not a power of two, the function must return \textit{something}, but we do not require that it return any specific value.

%A solution that would require only two or three lines of code is to apply bit shifts to the argument, counting the number of shifts necessary until you determine the position of the one bit that is a 1.
%The problem with this simple solution is that it requires addition.
%You may use this simple solution, provided that you do not use the \function{lg()} function when implementing addition (otherwise, you would create an infinite recursion).

There are some very short solutions that will work if you have already implemented arithmetic.
Since you haven't, the function stub suggests an alternative:
use a \lstinline{switch} statement, enumerating the 32 possible cases and returning the appropriate value in each case.

\begin{description}
    \checkoffitem{Implement the \function{lg()} function.}
\end{description}


\subsection*{Check your work}

\begin{description}
    \checkoffitem{Compile and run \texttt{\textbf{\textit{./integerlab}}}, trying a few values.}
    \begin{itemize}
        \item Note that you will receive a warning for an unused variable in \function{ripple_carry_addition()};
            this is okay for now
    \end{itemize}
\end{description}
For example:
\begin{verbatim}
    Enter a one- or two-operand logical expression,
        a two-operand comparison expression, a two-operand arithmetic expression,
        "lg <value>" or "exponentiate <value>" to test your powers-of-two code,
        "is_negative <value>" to determine if 2's complement value is negative,
        "add1 <binary_value1> <binary_value2> <carry_in>" for 1-bit full adder,
        "add32 <hex_value1> <hex_value2> <carry_in>" for 32-bit ripple-carry adder,
        or "quit": exponentiate 10
    expected: 2**10 == 0x00000400 == 1024
    actual:   2**10 == 0x00000400 == 1024

    Enter ... "lg <value>" or "exponentiate <value>" ... or "quit": lg 1024
    expected: log2 1024 == log2 0x00000400 == 10
    actual:   log2 1024 == log2 0x00000400 == 10

    Enter ... "lg <value>" or "exponentiate <value>" ... or "quit": lg 0x0400
    expected: log2 1024 == log2 0x00000400 == 10
    actual:   log2 1024 == log2 0x00000400 == 10
\end{verbatim}

The expected results come from the \textit{math} library's \function{exp2()} and \function{log2()} functions.
The actual results come, of course, from the code you wrote.

\begin{description}
    \checkoffitem{Check your code with other values, comparing your actual results with the expected results.}
    \checkoffitem{Run the constraint checker: \texttt{python constraint-check.py integerlab.json}}
\end{description}


\vspace{1cm}

\begin{description}
    \checkoffitem{Open \textit{alu.c} in your editor.
    You will see the stubs of several functions there.}
\end{description}


\subsection{is\_negative()} \label{subsec:negative}

Real ALUs typically have hardware dedicated to quickly determining whether a value is 0 or not, and \textit{alu.h} includes the macros \function{is_zero()} and \function{is_not_zero)()} to serve this purpose.
Real ALUs also typically have hardware dedicated to quickly determine whether an integer, when treated as a signed value, is negative.

\begin{description}
    \checkoffitem{Implement \function{is_negative()} to determine whether its argument, when interpreted as a signed value, is negative.}
\end{description}
The function shall return 1 when the value is negative, and 0 when it is non-negative.
You should be able to implement this function in a single line of code (but you may use more, provided you comply with the assignment's constraints).


\subsection{equal() and not\_equal()}

The general approach to comparing two values requires arithmetic, as discussed in Section~\ref{sec:inequality-comparison}.
If you do not anticipate testing the equality of two values in your arithmetic, then you can postpone implementing \function{equal()} and \function{not_equal()} until later.
On the other hand, if you think that you might need to test for equality as part of your arithmetic functions, there is a simple test for equality that does not require arithmetic.

To implement each of the \function{equal()} and \function{not_equal()} functions, you will need one 2-operand bitwise operation, either bitwise AND, bitwise OR, or bitwise XOR\@.
Recognize that so far you have only three tests you can make on the output of that bitwise operation: \function{is_zero()}, \function{is_not_zero()}, and \function{is_negative()}.
\begin{description}
    \checkoffitem{Consider what the output of each of those three bitwise operations would be if the two operands were the same, and what the output would be if the two operands were different.}
\end{description}
One of those six possibilities will have a predictable output that can be evaluated with one or more of the three existing tests.
\begin{description}
    \checkoffitem{Implement \function{equal()} to return \lstinline{true} if and only if its two arguments are the same value.}
    \checkoffitem{Implement \function{not_equal()} to return \lstinline{true} if and only its two arguments are not the same value.}
\end{description}

\subsection*{Check your work}

\begin{description}
    \checkoffitem{Compile and run \texttt{\textbf{\textit{./integerlab}}}, trying a few values.}
    \begin{itemize}
        \item Note that you will receive a warning for an unused variable in \function{ripple_carry_addition()};
            this is okay for now
    \end{itemize}
\end{description}
For example:
\begin{verbatim}
    Enter ... "is_negative <value>" ... or "quit": is_negative 1
    expected: 1 (0x0001) is not negative
    actual:   1 (0x0001) is not negative

    Enter ... "is_negative <value>" ... or "quit": is_negative -1
    expected: -1 (0xFFFF) is negative
    actual:   -1 (0xFFFF) is negative

    Enter ... a two-operand comparison expression ... or "quit": 1 == 1
    expected: (1 == 1) = 1
    actual:   (1 == 1) = 1

    Enter ... a two-operand comparison expression ... or "quit": 1 != 1
    expected: (1 != 1) = 0
    actual:   (1 != 1) = 0

    Enter ... a two-operand comparison expression ... or "quit": 1 == -1
    expected: (1 == -1) = 0
    actual:   (1 == -1) = 0

    Enter ... a two-operand comparison expression ... 1 != -1
    expected: (1 != -1) = 1
    actual:   (1 != -1) = 1
\end{verbatim}

\begin{description}
    \checkoffitem{Check your code with other values, comparing your actual results with the expected results.}
    \checkoffitem{Run the constraint checker: \texttt{python constraint-check.py integerlab.json}}
\end{description}


\vspace{1cm}

Implementing logical NOT, logical AND, and logical OR is not quite as simple as applying the corresponding bitwise operations, but it is very nearly so.

\subsection{logical\_not()}

When is a value considered to be \lstinline{false}?
From among the tests that you have available, one of these will return \lstinline{true} when that condition is satisfied, and \lstinline{false} when it is not.

\begin{description}
    \checkoffitem{Implement \function{logical_not()} to return \lstinline{true} if and only if its two arguments are considered to be \lstinline{true}.}
\end{description}

\subsection{logical\_and() and logical\_or()}

When is a value considered to be \lstinline{true}?
From among the tests that you have available, one of these will return \lstinline{true} when that condition is satisfied, and \lstinline{false} when it is not.
Specifically, it will return a 1 or a 0, as appropriate.

You cannot simply apply bitwise AND and bitwise OR to the original values because their bits might not line up -- for example, \lstinline{0x5 & 0xA == 0x0}.
After you reduce these values to a 1 or a 0, then their bits will line up, and tjen you can apply a bitwise operation to the results of the aforementioned test.

\begin{description}
    \checkoffitem{Implement \function{logical_and()} to return \lstinline{true} if and only if its two arguments are considered to be \lstinline{true}.}
    \checkoffitem{Implement \function{logical_or()} to return \lstinline{true} if and only if at least one of its two arguments is considered to be \lstinline{true}.}
\end{description}

\textit{Note:} you are not required to preserve C's ``shortcut evaluation'' of the logical AND and logical OR operations.
Indeed, you cannot because the semantics of C's functions requires that both arguments to \function{logical_and()} and \function{logical_or()} be evaluated before your code has the opportunity to determine their truth values.

\subsection*{Check your work}

\begin{description}
    \checkoffitem{Compile and run \texttt{\textbf{\textit{./integerlab}}}, trying a few values.}
    \begin{itemize}
        \item Note that you will receive a warning for an unused variable in \function{ripple_carry_addition()};
            this is okay for now
    \end{itemize}
\end{description}
For example:
\begin{verbatim}
    Enter a one- or two-operand logical expression or "quit": !0
    expected: !0 = 1
    actual:   !0 = 1

    Enter a one- or two-operand logical expression or "quit": !1
    expected: !1 = 0
    actual:   !1 = 0

    Enter a one- or two-operand logical expression or "quit": 0 && 42
    expected: 0 && 42 = 0
    actual:   0 && 42 = 0

    Enter a one- or two-operand logical expression or "quit": 0 || 73
    expected: 0 || 73 = 1
    actual:   0 || 73 = 1
\end{verbatim}

\begin{description}
    \checkoffitem{Check your code with other values, comparing your actual results with the expected results.}
    \checkoffitem{Run the constraint checker: \texttt{python constraint-check.py integerlab.json}}
\end{description}


\section{Addition and Subtraction}                                                                      Now that you've warmed up to bitwise operations and implementing operations without using C's built-in operations, let us turn your attention to arithmetic.
Before you can add two $n$-bit values, you must be able to add two 1-bit values.

\subsection{One Bit Full Adder} \label{subsec:one-bit-full-adder}

In the \function{one_bit_full_addition()} function, you will implement a 1-bit full adder;
that is, an adder that takes two operand bits and a carry-in bit, and it produces a sum bit and a carry-out bit.

The function takes one argument, a structure containing five fields.
As described in Section~\ref{subsubsec:alu.h}, these five fields are the operand bits \lstinline{a} and \lstinline{b}, the carry-in bit \lstinline{c_in}, the sum bit \lstinline{sum}, and the carry-out bit \lstinline{c_out}.
When the structure is passed in to the function, only \lstinline{a}, \lstinline{b}, and \lstinline{c_in} are populated.
Your task is to populate the \lstinline{sum} and \lstinline{c_out} fields, and return the structure.

Implement a 1-bit full adder using bitwise operations.
Because the fields are guaranteed to be strictly 1 or 0, you do not need to apply any of the test functions to reduce them to 1 or 0.

\subsubsection*{Check your work}

Compile and run \texttt{\textbf{\textit{./integerlab}}}, trying all possible values.
When you enter the inputs for your 1-bit full adder, only the least significant bit of each operand will be used.
For example:
\begin{verbatim}
    Enter ... "add1 <binary_value1> <binary_value2> <carry_in>" ...: add1 0 0 0
    expected: 0 + 0 + 0 = 0 carry 0
    actual:   0 + 0 + 0 = 0 carry 0

    Enter ... "add1 <binary_value1> <binary_value2> <carry_in>" ...: add1 0 0 1
    expected: 0 + 0 + 1 = 1 carry 0
    actual:   0 + 0 + 1 = 1 carry 0
\end{verbatim}

Check your code with all eight possible inputs, comparing your actual results with the expected results.


\subsection{Ripple-Carry Adder} \label{subsec:ripple-carry-adder}

Use your 1-bit full adder to implement a 32-bit ripple-carry adder.
As a reminder, in a ripple-carry adder, the carry-out bit from bit position $n$ becomes the carry-in bit for bit position $n+1$.

Use whatever code that you need, that does not violate any of this assignment's constraints, to populate the input fields of a \lstinline{one_bit_adder_t} variable and pass that variable to \function{onez_bit_full_addition()}.
Use the \lstinline{sum} field to contribute to the 32-bit sum and the \lstinline{c_out} bit as the \lstinline{c_in} bit of the next bit position.
Repeatedly do this until you have added all 32-bit positions, resulting in the 32-bit sum.

\subsubsection*{Check your work}

Compile and run \texttt{\textbf{\textit{./integerlab}}}, trying a few values.
When you enter the inputs for your 32-bit adder, the operands will be interpreted as hexadecimal values even if you omit the leading ``0x'', and only the least-significant bit of the carry-in will be used.
For example:
\begin{verbatim}
    Enter ... "add32 <hex_value1> <hex_value2> <carry_in>" ...: add32 0x1a 0x22 0
    expected: 0x0000001A + 0x00000022 + 0 = 0x0000003C
    actual:   0x0000001A + 0x00000022 + 0 = 0x0000003C

    Enter ... "add32 <hex_value1> <hex_value2> <carry_in>" ...: add32 1a 22 1
    expected: 0x0000001A + 0x00000022 + 1 = 0x0000003D
    actual:   0x0000001A + 0x00000022 + 1 = 0x0000003D
\end{verbatim}

Check your code with other values, comparing your actual results with the expected results.


\subsection{16-Bit Addition}

The \function{add()} function, along with the other arithmetic functions, returns an \lstinline{alu_result_t} structure.

Having implemented a 32-bit adder, you can use it for your 16-bit addition function.
The 16-bit sum will be the lower 16 bits of the 32-bit adder's sum;
place this sum in the \lstinline{alu_result_t} variable's \lstinline{result} field.

Your remaining task for addition is to examine the 32-bit sum to determine whether your \textit{16-bit} addition overflowed.
Assume that the operands are unsigned 16-bit integers and determine whether overflow occurred;
set the \lstinline{alu_result_t} variable's \lstinline{unsigned_overflow} flag accordingly.
Now assume that hte operands are signed 16-bit integers and determine whether overflow occurred;
set the \lstinline{alu_result_t} variable's \lstinline{signed_overflow} flag accordingly.


\subsubsection*{Check your work}

Compile and run \texttt{\textbf{\textit{./integerlab}}}, trying a few values.
For example:
\begin{verbatim}
    Enter ... a two-operand arithmetic expression... or "quit": 3 + 15
    UNSIGNED ADDITION
        expected result (hexadecimal): 0x0003 + 0x000F = 0x0012
        expected result (unsigned):    3 + 15 = 18	overflow: false
        actual result (hexadecimal):   0x0003 + 0x000F = 0x0012
        actual result (unsigned):      3 + 15 = 18	overflow: false
    SIGNED ADDITION
        expected result (hexadecimal): 0x0003 + 0x000F = 0x0012
        expected result (signed):      3 + 15 = 18	overflow: false
        actual result (hexadecimal):   0x0003 + 0x000F = 0x0012
        actual result (signed):        3 + 15 = 18	overflow: false

    Enter ... a two-operand arithmetic expression... or "quit": 0x6000 + 0x3000
    UNSIGNED ADDITION
        expected result (hexadecimal): 0x6000 + 0x3000 = 0x9000
        expected result (unsigned):    24576 + 12288 = 36864	overflow: false
        actual result (hexadecimal):   0x6000 + 0x3000 = 0x9000
        actual result (unsigned):      24576 + 12288 = 36864	overflow: false
    SIGNED ADDITION
        expected result (hexadecimal): 0x6000 + 0x3000 = 0x9000
        expected result (signed):      24576 + 12288 = -28672	overflow: true
        actual result (hexadecimal):   0x6000 + 0x3000 = 0x9000
        actual result (signed):        24576 + 12288 = -28672	overflow: true
\end{verbatim}

If you are performing this lab on \runtimeenvironment, then the expected overflow flags are obtained directly from flags set in the processor's ALU and are authoritative.\footnote{
    If you are not performing this lab on \runtimeenvironment\ and receive the compile-time warning ``Some of the code to determine the *expected* supplemental\_result and *expected* flags is not yet defined'' then the expected overflow flags should not be trusted.
}

Check your code with other values, comparing your actual results with the expected results.
Use positive and negative operands.
Generate sums that produce signed overflow, sums that produce unsigned overflow, and sums that produce neither.


\subsection{16-Bit Subtraction}

Having implemented a 32-bit adder, you can use it for your 16-bit subtraction function.
If you use the adder as discussed in Chapter~3 and in lecture, the 16-bit difference will be the lower 16 bits of the 32-bit adder's sum;
place this difference in the \lstinline{alu_result_t} variable's \lstinline{result} field.

Your challenges for this function will be using the ripple-carry adder to perform subtraction, and detecting signed- and unsigned overflow.
Note that C will promote 16-bit operands to 32 bits when they're passed as arguments the 32-bit adder;
this promotion includes extending the sign bit into the upper 16 bits.
If you prevent this sign extension (that is, if you ensure the upper 16 bits are all 0s) then you will be able to apply the overflow rules discussed in Chapter~3 and in lecture.


\subsubsection*{Check your work}

Compile and run \texttt{\textbf{\textit{./integerlab}}}, trying a few values.
For example:
\begin{verbatim}
    Enter ... a two-operand arithmetic expression... or "quit": 15 - 25
    UNSIGNED SUBTRACTION
        expected result (hexadecimal): 0x000F - 0x0019 = 0xFFF6
        expected result (unsigned):    15 - 25 = 65526	overflow: true
        actual result (hexadecimal):   0x000F - 0x0019 = 0xFFF6
        actual result (unsigned):      15 - 25 = 65526	overflow: true
    SIGNED SUBTRACTION
        expected result (hexadecimal): 0x000F - 0x0019 = 0xFFF6
        expected result (signed):      15 - 25 = -10	overflow: false
        actual result (hexadecimal):   0x000F - 0x0019 = 0xFFF6
        actual result (signed):        15 - 25 = -10	overflow: false

    Enter ... a two-operand arithmetic expression... or "quit": 0x100 - 0x7F
    UNSIGNED SUBTRACTION
        expected result (hexadecimal): 0x0100 - 0x007F = 0x0081
        expected result (unsigned):    256 - 127 = 129	overflow: false
        actual result (hexadecimal):   0x0100 - 0x007F = 0x0081
        actual result (unsigned):      256 - 127 = 129	overflow: false
    SIGNED SUBTRACTION
        expected result (hexadecimal): 0x0100 - 0x007F = 0x0081
        expected result (signed):      256 - 127 = 129	overflow: false
        actual result (hexadecimal):   0x0100 - 0x007F = 0x0081
        actual result (signed):        256 - 127 = 129	overflow: false
\end{verbatim}

As with addition, if you are performing this lab on \runtimeenvironment, then the expected overflow flags are obtained directly from flags set in the processor's ALU\@.

Check your code with other values, comparing your actual results with the expected results.
Use positive and negative operands.
Generate differences that produce signed overflow, sums that produce unsigned overflow, and sums that produce neither.


\section{Inequality Comparison Functions}\label{sec:inequality-comparison}                              In general, comparing two values to determine which (if either) is greater can be achieved with subtraction.
Consider, for example, the less-than comparison.
A simple application of algebra tells us that \[value1 < value2 \Leftrightarrow value1 - value2 < 0\]
Similarly, \[value1 = value2 \Leftrightarrow value1 - value2 = 0\]

As we noted in Section~\ref{subsec:negative}, real ALUs dedicate hardware to quickly compare values to 0.
This is an example of why that is so.
By performing this subtraction and determining the truth values of \function{is_zero()} and \function{is_negative()}, you have sufficient data to determine whether \lstinline{value1} is \function{less_than}, \function{at_most}, \function{at_least}, and/or \function{greater_than} \lstinline{value2}.
Do so and implement the four inequality functions.


\subsubsection*{Check your work}

Compile and run \texttt{\textbf{\textit{./integerlab}}}, trying a few values.
For example:
\begin{verbatim}
    Enter ... a two-operand comparison expression ... or "quit": 4 < 5
    expected: (4 < 5) = 1
    actual:   (4 < 5) = 1

    Enter ... a two-operand comparison expression ... or "quit": 4 >= 5
    expected: (4 >= 5) = 0
    actual:   (4 >= 5) = 0
\end{verbatim}

Check your code with other values, comparing your actual results with the expected results.
Use positive and negative operands.
Generate both \lstinline{true} and \lstinline{false} results for each of the four inequality functions.


\section{Unsigned Multiplication and Division}                                                          Before you can multiply arbitrary values, you must be able to multiply by a power of two.

\subsection{Multiplication by a Power of Two}

Strictly speaking, the \function{multiply_by_power_of_two()} function needs to do a little bit more than multiplying by a power of two.
If the \lstinline{power_of_two} argument is 0, then the function should return 0.
Otherwise, assume that it is a power of two and apply the fast multiplication technique for powers of two discussed in Chapter~3 and in lecture.
Be sure to remember that \function{multiply_by_power_of_two()} returns a 32-bit value.

\textit{Your solution for \function{multiply_by_power_of_two()} should be a constant-time solution.}
If your solution includes a loop or recursion, please review the Chapter~3 material.


\subsection{General Unsigned Multiplication}

The distributive property of multiplication tells us that if $multiplier = \sum_{i=0}^{31}multiplier_i \times 2^i$ then
\begin{align*}
    multiplicand \times multiplier  & = multiplicand \times \sum_{i=0}^{31}multiplier_i \times 2^i \\
                                    & = \sum_{i=0}^{31} multiplicand \times multiplier_i \times 2^i
\end{align*}

In the \function{unsigned_multiply()} function, use each of the \lstinline{multiplier}'s bits, in turn, as the \lstinline{power_of_two} argument to \function{multiply_by_power_of_two()} to multiply \lstinline{multiplicand}.
Add each of these intermediate products to arrive at the 32-bit product of $multiplicand \times multiplier$.

When multiplying two 16-bit operands, a real ALU will spread the 32-bit full product across two 16-bit registers.
High-level languages will only access the register containing the 16-bit product when assigning the result to the destination variable.
Assembly language, however, lets programmers access both registers.

Place the 16-bit product, the lower 16 bits of the full product, in \lstinline{product}'s \lstinline{result} field.
Place the upper 16 bits of the full product in \lstinline{product}'s \lstinline{supplemental_result} field.


\subsubsection*{Check your work}

Compile and run \texttt{\textbf{\textit{./integerlab}}}, trying a few values.

(Note that unless and until you implement signed multiplication, your ``SIGNED MULTIPLICATION'' actual results will differ from the expected results.
You are not required to implement signed multiplication.)

For example:
\begin{verbatim}
    Enter ... a two-operand arithmetic expression... or "quit": 3 * 5
    UNSIGNED MULTIPLICATION
        expected result (hexadecimal): 0x0003 * 0x0005 = 0x0000'000F
        expected result (unsigned):    3 * 5 = 15 (15)
        actual result (hexadecimal):   0x0003 * 0x0005 = 0x0000'000F
        actual result (unsigned):      3 * 5 = 15 (15)
    SIGNED MULTIPLICATION
        expected result (hexadecimal): 0x0003 * 0x0005 = 0x0000'000F
        expected result (signed):      3 * 5 = 15 (15)
        actual result (hexadecimal):   0x0003 * 0x0005 = 0x0000'0000
        actual result (signed):        3 * 5 = 0 (0)

    Enter ... a two-operand arithmetic expression... or "quit": 0x234 * 0x345
    UNSIGNED MULTIPLICATION
        expected result (hexadecimal): 0x0234 * 0x0345 = 0x0007'3404
        expected result (unsigned):    564 * 837 = 13316 (472068)
        actual result (hexadecimal):   0x0234 * 0x0345 = 0x0007'3404
        actual result (unsigned):      564 * 837 = 13316 (472068)
    SIGNED MULTIPLICATION
        expected result (hexadecimal): 0x0234 * 0x0345 = 0x0007'3404
        expected result (signed):      564 * 837 = 13316 (472068)
        actual result (hexadecimal):   0x0234 * 0x0345 = 0x0000'0000
        actual result (signed):        564 * 837 = 0 (0)
\end{verbatim}

If you are performing this lab on \runtimeenvironment, then the expected results (including the upper 16 bits) come directly from the two registers used by processor's ALU and are authoritative.

Check your code with other values, comparing your actual results with the expected results.
Generate products that fit within the lower 16 bits and products that require more.


\subsection{Unsigned Division by a Power of Two}

As discussed in Chapter~3 and in lecture, there is a fast division technique when the divisor is a power of two.
In the \function{unsigned_divide()} function, if the divisor is 0 then set the \lstinline{divide_by_zero} flag.
Otherwise, use that fast technique to implement division by a power of two.
\textit{Do not implement general division.}

When dividing integers, a real ALU will place the quotient in one register and the remainder in another.
When assigning the result to the destination variable, a high-level language will only access the register containing the quotient or the register containing the remainder, depending on whether the program called for division or the modulo operator.
Assembly language, however, lets programmers access both registers.

Place the quotient in \lstinline{quotient}'s \lstinline{result} field, and place the remainder in \lstinline{quotient}'s \lstinline{supplemental_result} field.

\textit{Your solution to determine the quotient should be a constant-time solution.}
If your solution includes a loop or recursion, please review the Chapter~3 material.


\subsubsection*{Check your work}

Compile and run \texttt{\textbf{\textit{./integerlab}}}, trying a few values.

(Note that unless and until you implement signed division, your ``SIGNED DIVISION'' actual results will differ from the expected results.
You are not required to implement signed division.)

For example:
\begin{small}\begin{verbatim}
    Enter ... a two-operand arithmetic expression... or "quit": 70 / 8
    UNSIGNED DIVISION
        expected result (hexadecimal): 0x0046 / 0x0008 = 0x0008    0x0046 % 0x0008 = 0x0006
        expected result (unsigned):    70 / 8 = 8    70 % 8 = 6
        actual result (hexadecimal):   0x0046 / 0x0008 = 0x0008    0x0046 % 0x0008 = 0x0006
        actual result (unsigned):      70 / 8 = 8    70 % 8 = 6
    SIGNED DIVISION
        expected result (hexadecimal): 0x0046 / 0x0008 = 0x0008    0x0046 % 0x0008 = 0x0006
        expected result (signed):      70 / 8 = 8    70 % 8 = 6
        actual result (hexadecimal):   0x0046 / 0x0008 = 0x0000    0x0046 % 0x0008 = 0x0000
        actual result (signed):        70 / 8 = 0    70 % 8 = 0

    Enter ... a two-operand arithmetic expression... or "quit": 0x29B / 0x40
    UNSIGNED DIVISION
        expected result (hexadecimal): 0x029B / 0x0040 = 0x000A    0x029B % 0x0040 = 0x001B
        expected result (unsigned):    667 / 64 = 10    667 % 64 = 27
        actual result (hexadecimal):   0x029B / 0x0040 = 0x000A    0x029B % 0x0040 = 0x001B
        actual result (unsigned):      667 / 64 = 10    667 % 64 = 27
    SIGNED DIVISION
        expected result (hexadecimal): 0x029B / 0x0040 = 0x000A    0x029B % 0x0040 = 0x001B
        expected result (signed):      667 / 64 = 10    667 % 64 = 27
        actual result (hexadecimal):   0x029B / 0x0040 = 0x0000    0x029B % 0x0040 = 0x0000
        actual result (signed):        667 / 64 = 0     667 % 64 = 0

    Enter ... a two-operand arithmetic expression... or "quit": 53 / 0
    UNSIGNED DIVISION
        expected result: divide-by-zero
        actual result:   divide-by-zero
    SIGNED DIVISION
        expected result: divide-by-zero
        actual result (hexadecimal):   0x0035 / 0x0000 = 0x0000    0x0035 % 0x0000 = 0x0000
        actual result (signed):        53 / 0 = 0    53 % 0 = 0
\end{verbatim}\end{small}

If you are performing this lab on \runtimeenvironment, then the expected results (including the upper 16 bits) come directly from the two registers used by processor's ALU and are authoritative.

Check your code with other values, comparing your actual results with the expected results.
Remember that the divisor must be either 0 or a power of two.


\section{Signed Multiplication and Division (Bonus Credit)}                                             You have the opportunity to earn a small amount of bonus credit.

Addition uses the same assembly code instruction for both signed and unsigned integers, as does subtraction.
Indeed, these instructions perform exact same actions regardless of whether the integers will be interpreted as signed or unsigned, which is why the overflow conditions for both are flagged.

Multiplication and division, however, have separate instructions for signed and unsigned integers.
This is because the logic for unsigned multiplication and division only produce correct results for positive numbers, and so the unsigned implementations cannot be used for negative integers.
The signed implementations cannot be used for unsigned integers because they treat half of the possible unsigned integers as though they were negative, yielding incorrect results.

A simple patch would be to keep track of which operands are negative, negate those operands so that they are positive, apply the unsigned implementation, and then negate the result as necessary.
Using that patch technique will not earn you bonus credit.
\textit{To earn bonus credit, you must address the underlying reason that the signed implementations need to be different.}

\subsection{Signed Multiplication}

If we only cared about the 16-bit product, the lower 16 bits of the full 32-bit product, then the unsigned implementation works for both signed and unsigned integers.
The upper 16 bits, however, differ when \function{is_negative()} is true.
For example:
\begin{verbatim}
    Enter ... a two-operand arithmetic expression... or "quit": -3 * 5
    UNSIGNED MULTIPLICATION
        expected result (hexadecimal): 0xFFFD * 0x0005 = 0x0004'FFF1
        expected result (unsigned):    65533 * 5 = 65521 (327665)
        ...
    SIGNED MULTIPLICATION
        expected result (hexadecimal): 0xFFFD * 0x0005 = 0xFFFF'FFF1
        expected result (signed):      -3 * 5 = -15 (-15)
        ...
\end{verbatim}

If you chose to implement signed multiplication then step through your unsigned multiplication to see if you can find where it breaks down for negative operands.
Implement \function{signed_multiply()} to correctly handle negative numbers when the arguments are interpreted as signed integers.

\textit{Reminder: you may not change the signatures of any functions declared in }alu.h\textit{; however, you may implement other helper functions if you wish.}

Check your work with several values, both great and small.

\subsection{Signed Division}

Recall that the semantics of integer division are that the fractional portion of the quotient be truncated;
that is, the quotient is rounded toward zero.
The fast division technique for powers of two, however, has the effect of rounding toward negative infinity.
This is fine for positive quotients, but it rounds negative quotients in the wrong direction.

For example, if we used unsigned fast division for signed division then we would see this:
\begin{small}\begin{verbatim}
    Enter ... a two-operand arithmetic expression... or "quit": -14 / 4
    ...
    SIGNED DIVISION
        expected result (hexadecimal): 0xFFF2 / 0x0004 = 0xFFFD    0xFFF2 % 0x0004 = 0xFFFE
        expected result (signed):      -14 / 4 = -3    -14 % 4 = -2
        actual result (hexadecimal):   0xFFF2 / 0x0004 = 0xFFFC    0xFFF2 % 0x0004 = 0x0002
        actual result (signed):        -14 / 4 = -4    -14 % 4 = 2
\end{verbatim}\end{small}

If you chose to implement signed division then in your implementation of \function{signed_divide()}, whenever the dividend is negative you need to introduce a bias toward positive infinity.
This bias needs to be sufficient so that when the fast division technique rounds non-integer quotients toward negative infinity, it ends up rounding to the correct quotient -- but do so without overcorrecting.
The other precaution you need to take is to ensure that when you apply the fast division technique, you preserve the sign bit.

Check your work with several values, both great and small.


\section{Turn-in and Grading}                                                                           \filesubmission

\policyforcodethatdoesnotcompile

\latepolicy

\subsection*{Rubric}

This assignment is worth 20 points.
\begin{description}
    \rubricitem{4}{\textit{problem1.c} produces the specified output.}
    \rubricitem{4}{\function{iz_digit()} in \textit{problem2.c} determines whether
    or not a character is a digit.}
    \rubricitem{4}{\function{decapitalize()} in \textit{problem2.c} converts
    uppercase letters to lowercase and leaves other characters unchanged.}
    \rubricitem{4}{\function{is_even()} in \textit{problem3.c} determines whether
    a number is even or odd.}
    \item[\hspace{1cm}]\function{produce_multiple_of_ten()} in \textit{problem3.c}
    has the following:
    \begin{description}
        \rubricitem{1}{Code to assign the value 5 to the variable \lstinline{five}}
        \rubricitem{1}{Code to divide an even number by 2}
        \rubricitem{1}{Code to subtract 1 from an odd number}
        \rubricitem{1}{Correct functionality}
    \end{description}
    \item[Penalties]
    \penaltyitem{4}{for each solution that depends on a prohibited character.}
    \penaltyitem{4}{for each solution that hard-codes a return value instead of attempting to solve the specified problem}
    \softwareengineeringpenalties
\end{description}


\section*{Epilogue}                                                                                     \SuccessfulALU

\textit{To be continued...}

\end{document}
