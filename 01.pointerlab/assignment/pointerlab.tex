%%
%% PointerLab (c) 2021 Christopher A. Bohn
%%

%%
%% labs/common/assignment.tex
%% (c) 2021-22 Christopher A. Bohn
%%
%% Licensed under the Apache License, Version 2.0 (the "License");
%% you may not use this file except in compliance with the License.
%% You may obtain a copy of the License at
%%     http://www.apache.org/licenses/LICENSE-2.0
%% Unless required by applicable law or agreed to in writing, software
%% distributed under the License is distributed on an "AS IS" BASIS,
%% WITHOUT WARRANTIES OR CONDITIONS OF ANY KIND, either express or implied.
%% See the License for the specific language governing permissions and
%% limitations under the License.
%%

\documentclass[12pt]{article}

\usepackage{fullpage}
\usepackage{fancyhdr}
\usepackage[procnames]{listings}
\usepackage{hyperref}
\usepackage{textcomp}
\usepackage{bold-extra}
\usepackage[dvipsnames]{xcolor}
\usepackage{etoolbox}

% These are placeholder commands and will be renewed in each lab

\newcommand{\labnumber}{}
\newcommand{\labname}{Lab \labnumber\ Assignment}
\newcommand{\shortlabname}{}
\newcommand{\duedate}{}

% Individual or team effort

\newcommand{\individualeffort}{This is an individual-effort project. You may
    discuss concepts and syntax with other students, but you may discuss
    solutions only with the professor and the TAs. Sharing code with or copying
    code from another student or the internet is prohibited.}
\newcommand{\teameffort}{This is a team-effort project. You may discuss concepts
    and syntax with other students, but you may discuss solutions only with your
    assigned partner(s), the professor, and the TAs. Sharing code with or
    copying code from a student who is not on your team, or from the internet,
    is prohibited.}
\newcommand{\freecollaboration}{In addition to the professor and the TAs, you
    may freely seek help on this assignment from other students.}
\newcommand{\collaborationrules}{}

% Software engineering (if you care about that)

\providebool{allowspaghetticode}

\newcommand{\softwareengineeringfrontmatter}{
    \ifboolexpe{not bool{allowspaghetticode}}{
        \section*{No Spaghetti Code Allowed}
        In the interest of keeping your code readable, you may \textit{not} use
        any \lstinline{goto} statements, nor may you use any
        \lstinline{continue} statements, nor may you use any \lstinline{break}
        statements to exit from a loop, nor may you have any functions
        \lstinline{return} from within a loop.
    }{}
}

\newcommand{\spaghetticodepenalties}[1]{
    \ifboolexpe{not bool{allowspaghetticode}}{
        \penaltyitem{1}{for each \lstinline{goto} statement,
            \lstinline{continue} statement, \lstinline{break} statement used to
            exit from a loop, or \lstinline{return} statement that occurs within
            a loop.}
    }{}
}

% You shouldn't need to customize these,
% but you can if you like

\lstset{language=C, tabsize=4, upquote=true, basicstyle=\ttfamily}
\newcommand{\function}[1]{\textbf{\lstinline{#1}}}
\setlength{\headsep}{0.7cm}
\hypersetup{colorlinks=true}

\newcommand{\pagelayout}{
    \pagestyle{fancy}
    \fancyhf{}
    \lhead{\coursenumber}
    \chead{\ Lab \labnumber: \labname}
    \rhead{\courseterm}
    \cfoot{\shortlabname-\thepage}
}

\newcommand{\labidentifier}{
    \title{\ Lab \labnumber}
    \author{\labname}
    \date{Due: \duedate}
    \maketitle

    \textit{\collaborationrules}
}

% deprecated
\newcommand{\startdocument}{
    \pagelayout
	\begin{document}
	\labidentifier
}

\newcommand{\rubricitem}[2]{\item[\underline{\hspace{1cm}} +#1] #2}
\newcommand{\bonusitem}[2]{\item[\underline{\hspace{1cm}} Bonus +#1] #2}
\newcommand{\penaltyitem}[2]{\item[\underline{\hspace{1cm}} -#1] #2}
\newcommand{\checkoffitem}[1]{\item (\phantom{xxx}) #1}
\newcommand{\precheckoffitem}[1]{\item [] (\phantom{xxx}) #1}

\usepackage{graphicx}
\usepackage{caption}
\usepackage{wrapfig}
\usepackage{ulem}
\usepackage{cancel}
\usepackage{multicol}
\captionsetup{width=.8\linewidth}
\lstset{language=c, alsolanguage=java, numbers=left, showstringspaces=false,
    moredelim = [s][\ttfamily]{/*}{*/} % I shouldn't need this parameter!
    }
%%
%%
% Credit where credit is due. Skipping line numbers from
% https://tex.stackexchange.com/questions/476100/lstlisting-line-number-gaps
\makeatletter
\let\orig@lstnumber=\thelstnumber

\newcommand\lstsetnumber[1]{\gdef\thelstnumber{#1}}
\newcommand\lstresetnumber{\global\let\thelstnumber=\orig@lstnumber}
\makeatother
%%
%%

\renewcommand{\labnumber}{3}
\renewcommand{\labname}{Pointer Manipulation Lab}
\renewcommand{\shortlabname}{pointerlab}
\renewcommand{\collaborationrules}{\individualeffort}
%\renewcommand{\duedate}{See \filesubmission\ for the due date}
\renewcommand{\duedate}{Week of February 7, before the start of your lab section}

% Update to reflect the CS2 course(s) at your institute
\newcommand{\cstwo}{CSCE~156, RAIK~184H, or SOFT~161}

\startdocument
% \begin{document}

The purpose of this assignment is to give you more confidence in C programming
and to begin your exposure to the underlying bit-level representation of data.
You will also gain practice with pointers and with file input/output.

The instructions are written assuming you will edit and run the code on
\runtimeenvironment. If you wish, you may edit and run the code
in a different environment; be sure that your compiler suppresses no warnings,
and that if you are using an IDE that it is configured for C and not C++.

\section*{Learning Objectives}

After successful completion of this assignment, students will be able to:
\begin{itemize}
\item Recognize the hazards of indeterminate values.
\item Use C's string functions from \lstinline{string.h}\footnote{See \S7.8.1 and \S{}B.3 of Kernighan \& Ritchie's \textit{The C Programming Language}, 2nd ed.}.
\item Use C's file I/O functions from \lstinline{stdio.h}\footnote{See \S7.5, \S7.7, and \S{}B1.1 \textit{The C Programming Language}, 2nd ed.}.
\item Alias and reassign pointers.
\item Create and traverse a linked list.
\end{itemize}

\subsection*{Continuing Forward}

Being able to understand the mistakes in
Sections~\ref{sec:uninitializedVariables} and \ref{sec:localAddresses} will
help you avoid them in future labs. Being able to work with pointers -- that
is, with variables that hold addresses -- will help you specifically in future
labs that use pointers but more generally in future labs that require you to
think about accessing memory.

\section*{During Lab Time}

During your lab period, the TAs will provide a refresher on linked lists and
will describe Insertion Sort. The TAs will also describe some string functions
and some I/O functions from C's standard library. During the remaining time,
the TAs will be available to answer questions.

\softwareengineeringfrontmatter

\section*{Scenario}

Working at the Pleistocene Petting Zoo certainly is proving to be interesting.
You're glad that you don't have to worry about the problem of the giant sloths
very slowly chasing their handlers, but now it seems that Archie has decided to
try to write a program or two. At a glance, his code is smellier than the wooly
rhinoceros' enclosure. But you take a closer look anyway to try to understand
why his code acts strangely.

\section{Stray Values in Memory}

\subsection{Pleistocene Petting Zoo Marquee}\label{sec:uninitializedVariables}

Archie shows you his first program, which he hoped would be used to greet
guests, but it doesn't always work right:

\begin{lstlisting}
/***********************************************************************
 * This program will output
 **         Welcome to the
 **    Pleistocene Petting Zoo!
 **
 ** Get ready for hands-on excitement on the count of three! 1.. 2.. 3..
 ** Have fun!
 * With brief pauses during the "Get ready" line.
 ***********************************************************************/

#include <stdio.h>
#include <unistd.h>

void splash_screen() {
    const char *first_line = "\t     Welcome to the\n";
    const char *second_line = "\tPleistocene Petting Zoo!\n";
    printf("%s%s\n", first_line, second_line);
}

void count() {
    int i;
    sleep(1);
    printf("Get ready for hands-on excitement on the count of three! ");
    while (i < 3) {
        fflush(stdout);
        sleep(1);
        i++;
        printf("%d.. ", i);
    }
    printf("\nHave fun!\n");
}

int main() {
    splash_screen();
    count();
    return 0;
}
\end{lstlisting}

Sometimes the output was what he expected:
\begin{verbatim}
         Welcome to the
    Pleistocene Petting Zoo!

 Get ready for hands-on excitement on the count of three! 1.. 2.. 3..
 Have fun!
\end{verbatim}

But sometimes the output was missing the
``\texttt{1.. 2.. 3..}'':
\begin{verbatim}
         Welcome to the
    Pleistocene Petting Zoo!

 Get ready for hands-on excitement on the count of three!
 Have fun!
\end{verbatim}

What mistake did Archie make? What change to \textit{one} line will fix
Archie's bug? Place your answers in \textit{answers.txt}.

\subsection{Math Doesn't Work Right \dots Or Does It?}\label{sec:localAddresses}

``Wow,'' Archie says. ``I can't believe that I missed that. Maybe you can help
me with some other code that I wrote for one of my other start-ups. Here it
is.'' Archie shows you the code:

\begin{lstlisting}
/***********************************************************************
 * This program will add two numbers and then it will multiply two other
 * numbers. Finally, it will subtract the second result from the first
 * result.
 ***********************************************************************/

#include <stdio.h>

int *add(int a, int b) {
    int result = a + b;
    return &result;
}

int *multiply(int p, int q) {
    int result[1];
    result[0] = p * q;
    return result;
}

int main() {
    int *sum = add(4, 5);
    printf("sum = %d\n", *sum);
    int *product = multiply(2, 3);
    printf("product = %d\n", *product);
    printf("sum - product = %d\n", *sum - *product);
    return 0;
}
\end{lstlisting}

Archie explains that when he compiles the program with the \textbf{clang}
compiler and then runs it, he gets this output:

\begin{verbatim}
sum = 9
product = 6
sum - product = 0
\end{verbatim}

And when he compiles the program with the \textbf{gcc} compiler and then runs
it, the program terminates with a segmentation fault. ``I'm pretty sure that
the segmentation fault is just gcc trying to protect me from trusting an
incorrect answer -- but why does the code produce an incorrect answer?''

What mistake did Archie make? Why does \lstinline{*sum - *product} produce the
value 0? Place your answer in \textit{answers.txt}.


\section{Challenge and Response}

You plug in your shiny, new keyboard, tune your satellite radio to the
Greatest Hits of the 1920s, and settle in to solving a more interesting problem.

To protect against corporate espionage, you are responsible for writing code
for a challenge-and-response system. Anybody can challenge anyone else in the
Pleistocene Petting Zoo's non-public areas by providing the name of a book and
a word contained within the book, and the person being challenged must respond
with another word from that book, based on certain rules:
\begin{itemize}
\item All of the book's words are sorted alphabetically without regard to
    capitalization (for example, ``hello'' occurs after ``Hear'' and before
    ``HELP'')
\item The challenge word occurs \textit{occurrences} times in the book
\item If \textit{occurrences} is an even number then the response word is the
    word \textit{occurrences} places \textbf{before} the challenge word in the
    alphabetized list; if the challenge word is less than \textit{occurrences}
    places from the start of the list then the response word is the first word
    in the list
\item If \textit{occurrences} is an odd number then the response word is the
    word \textit{occurrences} places \textbf{after} the challenge word in the
    alphabetized list; if the challenge word is less than \textit{occurrences}
    places from the end of the list then the response word is the last word
    in the list
\end{itemize}

Here is a simple example. Suppose the words in the specified book are:

\begin{center}
\begin{tabular}{cc}
\textit{word} & \textit{occurrences} \\ \hline
apple       & 7 \\
banana      & 4 \\
carrot      & 15 \\
date        & 3 \\
eggplant    & 2 \\
fig         & 6 \\
granola     & 9 \\
horseradish & 9 \\
ice         & 6 \\
jelly       & 3 \\
kale        & 1 \\
lemon       & 2 \\
mango       & 8 \\
naan        & 7 \\
orange      & 5 \\
pineapple   & 1 \\
quinoa      & 11 \\
raisin      & 4 \\
spaghetti   & 10 \\
tomato      & 12 \\
\end{tabular}
\end{center}
If the challenge word is ``horseradish'' then because horseradish occurs 9
times in the book, the response word is ``quinoa,'' which is 9 places in the
list after ``horseradish.'' If the challenge word is ``eggplant'' then the
response is ``carrot,'' which is 2 places earlier in the list than
``eggplant.'' If the challenge word is ``banana'' then the response word is
``apple,'' which is the first word in the list. If the challenge word is
``quinoa'' then the response word is ``tomato,'' the last word in the list.

You break the problem down into four sub-problems:

\begin{enumerate}
\item Designing the Data Structure and Its Algorithms
\item Alphabetizing Words
\item Inserting Words
\item Responding to a Challenge
\end{enumerate}

\subsection*{The Books}

Four small ``books'' are included with the starter code:

\begin{itemize}
\item ``Animals'' (sorted, 7 words)
\item ``Plants'' (unsorted, 7 words)
\item ``Cars'' (sorted; 74 words)
\item ``Food'' (unsorted; 125 words)
\end{itemize}

Two real books have also been reduced to one word per line:\footnote{The text
for these books was obtained from \href{http://www.gutenberg.org/}{Project
Gutenberg}. In accordance with Paragraph~1.C of the
\href{http://www.gutenberg.org/policy/license}{Project Gutenberg License}, all
references to Project Gutenberg have been removed from the text that we are
distributing. (Removing the references to Project Gutenberg was also necessary
to ensure that \textit{only} the words from the books are used for the
challenge-and-response system.)}

\begin{itemize}
\item Mary Shelly's \textit{Frankenstein; Or, The Modern Promethius} (filename
    ``Frankenstein'') \url{https://www.gutenberg.org/ebooks/84} (sorted; 74,363
    words)
\item Arthur Conan Doyle's \textit{The Lost World} (filename ``TheLostWorld'')
    \url{https://www.gutenberg.org/ebooks/139} (unsorted; 77,268 words)
\end{itemize}

The very small files of 7 words can be useful for debugging, and the moderate-
sized files of 74-125 words should give you confidence in the correctness of
your solution. The real books of more than 74,000 words will be used for
grading (as will smaller problem instances). The files marked as
\textit{sorted} have all of their words already in alphabetically sorted order,
ignoring capitalization; the files marked as \textit{unsorted} do not have
their words sorted (the words in ``Plants'' and ``Food'' are in a randomly-
selected order; the words in ``TheLostWorld'' appear in the order that they
appear in the original \textit{The Lost World}).

Each book file, ``\textit{file}'', has a corresponding
``\textit{file}-table.md'' that contains a Markdown-formatted table of the
challenge words, the number of occurrences for each challenge word, and the
corresponding response word. You may use these files to confirm the correctness
of your solution.

Throughout the assignment, we note that if building the list takes
more than a few seconds, there is a bug in your code; for context, we can build
the list for \textit{Frankenstein} in under 1.9 seconds and the list for
\textit{The Lost World} in under 2.4 seconds. We can locate a word (or
determine the absence of a word) in the \textit{Frankenstein} list in under
0.5ms and in the \textit{The Lost World} list in under 0.9ms.

\subsection{Preparation}

Copy your files \textit{problem2.c} and \textit{problem3.c} from KeyboardLab
into the directory that has \textit{pointerlab.c} and
\textit{challenge-response.c}. If you did not successfully complete
KeyboardLab, we will provide alternate implementations for you.

\subsection{Designing the Data Structure and Its Algorithms, part 1}

You decide that a linked list is the best data structure option for the
challenge-and-response system. You probably learned about linked lists in
\cstwo; however, we will provide a refresher.

\subsubsection{Singly-Linked List}\label{sec:SinglyLinkedList}

For now, you will design a linked list that will work for challenge words with
an odd numbers of occurrences; that is, challenge words whose response word is
later in the list than the challenge word.

A \textit{linked list} is linear collection of data. Like an array, each
element (or \textit{node}) has a particular position in the list, and when you
iterate over the list, you always access the elements in the same order every
time (unless you change or re-order the elements).

In an array, the elements are contiguous in memory, and you can access a
specific element by indexing the array (or, equivalently, performing pointer
arithmetic). In a linked list, however, the nodes can be in arbitrary locations
in memory, and the nodes are connected by references (in C, pointers). You can
access a specific element only by following pointers from one node to the next
until you reach the desired node.

The simplest linked list is a \textit{singly-linked list}. A node consists of a
\textit{payload} (the data that we care about) and a reference to the
\textit{next} node.

\begin{figure}[h]
    \centering
    \includegraphics[scale=0.5]{singly-linked-list}
    \caption{Nodes in a singly-linked list consist of the payload data and a reference that points to the next node.}
\end{figure}

A linked list's greatest advantage over an array is that inserting and removing
a node at an arbitrary location takes constant time, whereas inserting an
element into an array (assuming there is sufficient memory allocated for the
array ) or removing an element from an array requires moving all of the
elements that follow the element's index. Inserting a new node, $C$ between
adjacent nodes $A$ and $B$ (where $B = A.next$) requires connecting $C.next$ to
$B$ and re-assigning $A.next$ to $C$.

\begin{figure}[h]
    \centering
    \includegraphics[scale=0.5]{sll-insertion}
    \caption{Inserting a new node into a singly-linked list only requires assignments to the affected \textit{next} pointers.}
\end{figure}

As with an array, you do need to maintain a variable that points to the list.
Conventionally, this is a reference to the \textit{head} of the list. (Note
that if a new node is inserted before the current head node, then the new node
becomes the head of the list, and your \lstinline{head} variable would need to
be updated.) It is not uncommon to also maintain a reference to the
\textit{tail} of the list.

\subsubsection{Equivalent Java Code}

In Java, you probably wouldn't implement your own linked list; instead, you
would use \lstinline{java.util.LinkedList}, which has been available since
J2SE~1.2. A list of hypothetical \lstinline{Payload} objects would be created
with:
\begin{lstlisting}[numbers=none]
    List<Payload> payloads = new LinkedList<>;
\end{lstlisting}

C doesn't have a built-in linked list data type, so you will need to design
one. Let us consider what a custom linked list would look like in Java.

\begin{lstlisting}[mathescape=true]
public class Node {
    private final String word;          $\label{code:javaWord}$
    private int occurrences;            $\label{code:javaOccurrences}$
    private Node next;                  $\label{code:javaNext}$
    private Node previous;              $\label{code:javaPrevious}$

    public Node(String word) {$\lstsetnumber{\ldots}$
        ...$\lstresetnumber\setcounter{lstnumber}{11}$
    }

    public void insertAfter(Node existingNode) {$\lstsetnumber{\ldots}$
        ...$\lstresetnumber\setcounter{lstnumber}{22}$
    }

    public void insertBefore(Node existingNode) {$\lstsetnumber{\ldots}$
        ...$\lstresetnumber\setcounter{lstnumber}{33}$
    }$\lstsetnumber{\ldots}$
    ...$\lstresetnumber\setcounter{lstnumber}{98}$
}
\end{lstlisting}

Creating and inserting a new node would look something like this:

\begin{lstlisting}[firstnumber=200, mathescape=true]
    Node node = new Node("eggplant");   $\label{code:newNode}$
    Node otherNode = ... // code to determine where the new node goes
    node.insertAfter(otherNode);        $\label{code:javaInsertAfter}$
\end{lstlisting}

Recall that in Java, all variables except primitive types (such as
\lstinline{occurrences} on line~\ref{code:javaOccurrences}) are references. This
means that the \lstinline{next} field on line~\ref{code:javaNext} is a reference
to another Node, just as we described in Section~\ref{sec:SinglyLinkedList}.
The payload is the \lstinline{word} and how many \lstinline{occurrences} the
word has, exactly what we need for the challenge-and-response system.

Recall also that in java, the \lstinline{new} keyword allocates space for the
new object, and the constructor call -- \lstinline{Node("eggplant")} --
initializes the object.

\subsubsection{C Implementation}

In \textit{challenge-response.h}, you'll see a \lstinline{struct} with the same
fields as our Java example:

\lstinputlisting[linerange=19-24, firstnumber=36]{../starter-code/challenge-response.h}

In \textit{challenge-response.c}, you'll also see the \function{create_node()}
function:

\lstinputlisting[linerange=23-29, firstnumber=23]{../starter-code/challenge-response.c}

As you can see, it allocates space for a new node using \function{malloc()}.
The code that you will need to add to it will copy the \lstinline{word}
argument into the \lstinline{word} field and set an appropriate initial value
for the \lstinline{occurrences} field. Since we don't yet know where this node
will go, set the \lstinline{next} and \lstinline{previous} pointers to
\lstinline{NULL}.

The other function you need to write now is \function{insert_after()}:

\lstinputlisting[linerange=31-36, firstnumber=31]{../starter-code/challenge-response.c}

As the name and documentation indicate, you need to add code that will update
the nodes' \lstinline{next} pointers so that \lstinline{new_node} is placed in
the list immediately after \lstinline{existing_node}. For now, you can ignore
the \lstinline{previous} pointers.

After you have implemented \function{create_node()} and
\function{insert_after()}, go to the \function{main()} function in
textit{pointerlab.c} and un-comment the call to
\function{test_linked_list_functions()}.

\lstinputlisting[linerange=54-55, firstnumber=54]{../starter-code/pointerlab.c}

Build the executable with the command: \texttt{make pointerlab}. Be sure to fix
both errors and warnings. When the program compiles without generating any
warnings or errors, run it. The output should indicate a list with the nodes in
the order of ``first node,'' ``fourth node,'' ``second node,'' and ``third
node.''

\subsection{Alphabetize Words}

In KeyboardLab Problem~2, you wrote code to convert uppercase letters to
lowercase letters. Add code to \function{word_to_lowercase()} that calls that
function to convert all letters in a word to lowercase letters (do not copy the
\function{to_lowercase()} function into \textit{challenge-response.c}; we will
link to the function in \textit{problem2.c}).

\lstinputlisting[linerange=48-62, firstnumber=48]{../starter-code/challenge-response.c}

The starter code includes a function to compare two words (you do not need to
write this function) but it assumes that both words are completely lowercase.

If you did not complete KeyboardLab, then place this code in your
\textit{problem2.c} file:

\begin{lstlisting}
#include <ctype.h>

char decapitalize(char character) {
    return tolower(character);
}
\end{lstlisting}

\subsection{Inserting Words, part 1}

Comment-out (or delete) the call to \function{test_linked_list_functions()}.

For this sub-problem, the user will be prompted to enter the name of a book,
which will be the filename of a file that contains all of the book's words. All
punctuation has already been removed from the files, and each line in the file
contains exactly one word. For now, you will design the code to
work with files whose contents are already sorted.

\lstinputlisting[linerange=67-88, firstnumber=67]{../starter-code/challenge-response.c}

Add code to \function{insert_word()} and \function{build_list()} to read the
specified file one line at a time.\footnote{See \S7.5 and \S{}B1.1.1 of
Kernighan \& Ritchie's \textit{The C Programming Language}, 2nd ed. for
\function{fopen()} and \function{fclose()}, and \S7.7 and \S{}B1.1.4 for
\function{fgets()}.} For each word, convert it to lowercase, and then traverse
the list to find the appropriate place for the word. (Note that there will not
be a list to traverse when your code reads the first word!) If the word is not
in the list then create a node for that word and insert it into the list at the
correct location. If there is already a node containing that word, then
increment that node's variable that tracks the number of occurrences. Be sure
that \textit{only} the word is placed in a node; specifically, do not include a
newline character nor any other characters that are not part of the word.

Build the program and correct all warnings and errors. When the program
compiles without generating any warnings or errors, run it. If your program
requires more than a few seconds to build the list, there is a bug in your
code. If your program does not produce the expected output, the
\function{print_list()} utility function will help you see the list that your
created.

\subsection{Respond to a Challenge, part 1}

You now have implemented enough of the other sub-problems that you can write the
code to respond to a challenge if the number of occurrences is odd.

\lstinputlisting[linerange=93-111, firstnumber=93]{../starter-code/challenge-response.c}

After the word list is complete (after you have inserted all words in the
file), the user will be prompted to enter the challenge word. Add code to
\function{respond()} that traverses the word list to find that word. If the word
is not present in the list, return ``(word) is not present!'', where ``(word)''
is the challenge word.

In KeyboardLab Problem~3, you wrote code to determine whether an integer value
is even. Add code to \function{respond()} such that if the challenge word is
present in the list and the number of occurrences is even, the function returns
``(word) has (number) of occurrences'', where ``(word)'' is the challenge word
and ``(number)'' is the number of occurrences. If the number of occurrences is
odd then use the number of occurrences recorded in that word's node to find the
response word as described in the challenge-and-response rules, and return that
word. (Do not copy the \function{is_even()} function into \textit{challenge-
response.c}; we will link to the function in \textit{problem3.c}.)

If you did not complete KeyboardLab, then place this code in your
\textit{problem3.c} file:

\begin{lstlisting}
int is_even(int value) {
    return !(value % 2);
}
\end{lstlisting}

If your program does not provide the response word nearly instantaneously,
there is a bug in your code.

\subsection*{Next Steps}

You have now completed a significant portion of the assignment: you can build
the word list if the words in the file are already sorted, and you can provide
the response to a challenge word if the number of occurrences is odd. The next
steps will require modifying some of your code, so this would be a good time to
make a backup copy of \textit{challenge-response.c} or to commit it to a
private Git repository.

In Section~\ref{sec:DoublyLinkedList} you will implement a
\textit{doubly-linked list} which will allow you to write code to respond to a
challenge word if the number of occurrences is even. In
Section~\ref{sec:InsertionSort} you will implement the \textit{Insertion Sort}
algorithm which will allow you to build a list even when the words in the file
are not already sorted.

You may tackle either of the remaining tasks before tackling the other.

\subsection{Designing the Data Structure and Its Algorithms, part 2, and \\
    Respond to a Challenge, part 2: \\
    Doubly-Linked List}\label{sec:DoublyLinkedList}

A \textit{doubly-linked list} is a linked list with the property that each node
maintains a link not only to the \lstinline{next} node but also a link to the
\lstinline{previous} node. In C, these links are pointers.

\begin{figure}[h]
    \centering
    \includegraphics[scale=0.5]{doubly-linked-list}
    \caption{Nodes in a singly-linked list consist of the payload data and references that point to the previous and next nodes.}
\end{figure}

Inserting new node, $C$ between adjacent nodes $A$ and $B$ (where $B = A.next$
and $A = B.previous$) requires connecting $C.previous$ to $A$ and $C.next$ to
$B$, and re-assigning $A.next$ to $C$ and $B.prevous$ to $C$.

\begin{figure}[h]
    \centering
    \includegraphics[scale=0.5]{dll-insertion}
    \caption{Inserting a new node into a doubly-linked list requires assignments to the affected \textit{previous} and \textit{next} pointers.}
\end{figure}

Modify your \function{insert_after()} function to update not only the
\lstinline{next} pointers but also the \lstinline{previous} pointers so that
\lstinline{new_node} is placed between \lstinline{existing_node} and the node
that originally was located immediately after \lstinline{existing_node}.

After the program compiles without warnings or errors, you may want to use
\function{print_list} to confirm that the \lstinline{previous} pointers are
updated correctly.

Now implement \function{insert_before()}:

\lstinputlisting[linerange=48-53, firstnumber=48]{../starter-code/challenge-response.c}

If your program requires more than a few seconds to build the list, there is a
bug in your code.

Update your \function{respond()} function so that when the number of
occurrences is even, the returned string is not ``(word) has (number) of
occurrences'' but is instead the appropriate response word as described in the
challenge-and-response rules.

You can now provide the response to a challenge word regardless of whether the
number of occurrences is even or odd. This would be a good time to make a
backup copy of \textit{challenge-response.c} or to commit it to a private Git
repository.

\subsection{Inserting Words, part 2: Insertion Sort}\label{sec:InsertionSort}

While you probably learned about sorting in \cstwo, you may not have learned
about \textit{Insertion Sort}. If you did learn about Insertion Sort, you
probably learned that it's a $\mathcal{O}(n^2)$ algorithm that is less
efficient than $\mathcal{O}(n\mathrm{log}n)$ sorting algorithms such as
Merge Sort and Quick Sort. Insertion Sort has a particular advantage in that it
can be applied \textit{as the list is built}, making for a much simpler and
less error-prone implementation than a different sort that requires the list to
already be built.

The Insertion Sort algorithm reads an input and then traverses a sorted list to
find the proper location in the sorted list for the input. The input is then
inserted into the list at that location.

Your current implementation of \function{insert_word()} is based off of the
assumption that the word is either at the tail of the list or is not present in
the list. Update \function{insert_word()} so that it looks for the word in the
sorted list. If the word is found, then update the number of occurrences as
before. If the word is not present in its proper location in the sorted list,
then insert a new node for that word at its proper location in the sorted list.

If your program requires more than a few seconds to build the list, there is a
bug in your code.

You can now provide build the word list regardless of whether the words in the
file are already sorted. This would be a good time to make a backup copy of
\textit{challenge-response.c} or to commit it to a private Git repository.

\section*{Turn-in and Grading}

When you have completed this assignment, upload \textit{answers.txt} and
\textit{challenge-response.c} to \filesubmission.

This assignment is worth 30 points.
\begin{description}
\rubricitem{2}{Student's answers in \textit{answers.txt} demonstrate an
    understanding of the bug in Section~\ref{sec:uninitializedVariables}'s code
    and how to correct it.}
\rubricitem{2}{Student's answer in \textit{answers.txt} demonstrate an
    understanding of the bug in Section~\ref{sec:localAddresses}'s code.}
\rubricitem{1}{\function{create_node} creates and initializes a
    \lstinline{struct node} as specified.} % Depending on the insert_word
    % code, 0 or 1 is a reasonable initial value for the occurrences field
\item[\hspace{1cm}]\function{insert_after} correctly places a new node in a
    list by updating:
    \begin{description}
    \rubricitem{2}{\lstinline{next} pointers (singly- and doubly-linked lists).}
    \rubricitem{2}{\lstinline{previous} pointers (doubly-linked lists).}
    \end{description}
\rubricitem{2}{\function{insert_before} correctly places a new node in a
    doubly-linked list.}
\rubricitem{2}{\function{word_to_lowercase} returns a copy of the input string
    with uppercase letters replaced with lowercase letters}
\item[\hspace{1cm}]\function{insert_word}:
    \begin{description}
    \rubricitem{2}{creates a new node at the end of the list when the word
        properly belongs at the end of the list and is not already present in
        the list.}
    \rubricitem{2}{does not create a new node but instead updates the number of
        occurrences, when the word is present at the end of the list.}
    \rubricitem{2}{creates a new node at the appropriate location in the list,
        regardless of where its appropriate location is, when it is not already
        present in the list.}
    \rubricitem{2}{does not create a new node but instead updates the number of
        occurrences when the word is present, regardless of  its location in
        the list.}
    \end{description}
\item[\hspace{1cm}]\function{build_list} opens a file for reading, builds a
    list by reading one line at a time and the word that is read to
    \function{insert_word()}, and closes the file after the last line has been
    read, when:
    \begin{description}
    \rubricitem{2}{the words in the file are pre-sorted, and the
        \lstinline{next} pointers are updated (\textit{i.e.}, part 1 of the
        sub-problems are complete).}
    \rubricitem{1}{the words in the file are not pre-sorted, and the
        \lstinline{next} pointers are updated (\textit{i.e.},
        Section~\ref{sec:InsertionSort} is complete).}
    \rubricitem{1}{the words in the file are pre-sorted, and the
        \lstinline{previous} pointers are updated (\textit{i.e.},
        Section~\ref{sec:DoublyLinkedList} is complete).}
    \rubricitem{1}{the words in the file are not pre-sorted, and the
        \lstinline{previous} pointers are updated (\textit{i.e.}, part 2 of the
        sub-problems are complete).}
    \end{description}
\item[\hspace{1cm}]\function{respond} produces the correct response word in accordance with the specified rules when:
    \begin{description}
    \rubricitem{1}{the words in the file are pre-sorted, and the
        \lstinline{next} pointers are updated (\textit{i.e.}, part 1 of the
        sub-problems are complete).}
    \rubricitem{1}{the words in the file are not pre-sorted, and the
        \lstinline{next} pointers are updated (\textit{i.e.},
        Section~\ref{sec:InsertionSort} is complete).}
    \rubricitem{1}{the words in the file are pre-sorted, and the
        \lstinline{previous} pointers are updated (\textit{i.e.},
        Section~\ref{sec:DoublyLinkedList} is complete).}
    \rubricitem{1}{the words in the file are pre-sorted, and the
        \lstinline{previous} pointers are updated (\textit{i.e.}, part 2 of the
        sub-problems are complete).}
    \end{description}
\item[Penalties]
\penaltyitem{1}Newline characters are included in the word strings when
    building a list.
\spaghetticodepenalties{1}
\end{description}

\section*{Epilogue}

You hear somebody enter the room. ``\textit{Frankenstein}, `boat','' is the
challenge, and she answers, ``borne.'' Archie introduces you to the new
arrival, ``Lil, this is our new developer, the one who wrote the app we just
used.'' He turns to you: ``This is Lilith Redd from business operations.'' He
turns back to her and continues, ``Lil, what's the good word?''

``The word isn't good, I'm afraid. I just heard back from the insurance
company.''

\textit{To be continued...}

\end{document}
