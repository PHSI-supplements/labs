The purpose of this assignment is to give you more confidence in C programming and to begin your exposure to pointers and to file input/output.

The instructions are written assuming you will edit and run the code on \runtimeenvironment.
If you wish, you may edit and run the code in a different environment;
be sure that your compiler suppresses no warnings, and that if you are using an IDE that it is configured for C and not C++.

\section*{Learning Objectives}

After successful completion of this assignment, students will be able to:
\begin{itemize}
    \item Recognize the hazards of indeterminate values.
    \item Use C's string functions from \lstinline{string.h}\footnote{See \S7.8.1 and \S{}B.3 of Kernighan \& Ritchie's \textit{The C Programming Language}, 2nd ed.}.
    \item Use C's file I/O functions from \lstinline{stdio.h}\footnote{See \S7.5, \S7.7, and \S{}B1.1 \textit{The C Programming Language}, 2nd ed.}.
    \item Alias and reassign pointers.
    \item Create and traverse a linked list.
\end{itemize}

\subsection*{Continuing Forward}

Being able to understand the mistakes in Sections~\ref{subsec:uninitializedvariables} and~\ref{subsec:localaddresses} will help you avoid them in future labs.
Being able to work with pointers -- that is, with variables that hold addresses -- will help you specifically in future labs that use pointers but more generally in future labs that require you to think about accessing memory.

\section*{During Lab Time}

During your lab period, the TAs will provide a refresher on linked lists and will describe Insertion Sort.
The TAs will also describe some string functions and some I/O functions from C's standard library.
During the remaining time, the TAs will be available to answer questions.

Before leaving lab, \textit{at a minimum} \dots
