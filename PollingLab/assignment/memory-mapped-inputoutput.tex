In this section, you will use the data structures provided by the CowPi library to access the \developmentboard's memory-mapped I/O registers.
All functions that you edit in this part of the assignment are in \textit{io\_functions.c}
You will use the same test code from Section~\ref{sec:GettingStarted} to test your code.

\textit{Remember that to start the I/O test code, the \textbf{left switch} must be in the \underline{right} position when the \developmentboard\ restarts.}

You can complete this part of the assignment before or after implementing the system from Section~\ref{sec:SimpleSystem}.
You must, however, complete Section~\ref{sec:LabTime} first.
If you have already implemented the system from Section~\ref{sec:SimpleSystem}, you may want to make a backup copy of your program now.


\subsection{Simple Inputs and Outputs} \label{subsec:simpleIO}

Because you have already completed Section~\ref{sec:LabTime}, you can use the \lstinline{ioports} pointer as an array of \lstinline{cowpi_ioport_t} structures, which you can index using the \lstinline{D0_D7}, \lstinline{D8_D13}, and \lstinline{D14_D19} named constants.
Use the \lstinline{.input} field with bitmasks to read inputs.
Use the \lstinline{.output} field with bitmasks and the read-modify-write pattern to write outputs.
During lab, the TAs covered reading from inputs and using the read-modify-write pattern to write outputs;
however, if you missed lab or have forgotten, you can find examples in Section~4.1.1 of the Cow Pi datasheet.

\subsubsection{Simple Inputs}

Locate the \function{left_button_is_pressed()}, \function{right_button_is_pressed()}, \\ \function{left_switch_in_right_position()}, and  \function{right_switch_in_right_position()} functions.
In the starter code, these functions are implemented by calling functions in the CowPi library.

\begin{itemize}
    \item Replace the calls to CowPi library functions with code that uses the \lstinline{ioports} array to ascertain the controls' positions.
    \item Do \textit{not} remove the calls to \function{debounce_byte()}!
\end{itemize}

\paragraph{Test your code}

Place the left switch in the right position and upload the program to your \developmentboard.
The output will include the positions of the left and right buttons (``U'' for up and ``D'' for down) and of the left and right switches (``L'' for left position and ``R'' for right position).


\subsubsection{Simple Outputs}

Locate the \function{set_left_led()} and \function{set_right_led()} functions.
In the starter code, these functions are implemented by calling functions in the CowPi library.

\begin{itemize}
    \item Modify these two functions to replace the calls to CowPi library functions with code that uses the \lstinline{ioports} array to turn the LEDs on or off.
    \item Use the read-modify-write pattern to do so, so that you do not change any pins that you do not intend to change.\footnote{Even on ``input'' pins, changing the ``output'' register has an effect.}
    \item No debouncing code is necessary because these functions do not read from mechanical switches.
\end{itemize}

\paragraph{Test your code}

Place the left switch in the right position and upload the program to your \developmentboard.
If both buttons are pressed then the left LED will illuminate, and if both switches are in the right position then the right LED will illuminate.

\vspace{1cm}

This would be a good time to make a backup copy of your program.


\subsection{Scanned Input} \label{subsec:ScannedInput}

Locate the \function{get_keypress()} function.
The implementation in the starter code determines whether there has been a change on the keypad.
If so, it calls the CowPi library's \function{cowpi_get_keypress()} function.
The \function{cowpi_get_keypress()} function returns a \lstinline{char} corresponding to the character on the pressed key,
but the \function{get_keypress()} function is supposed to return a \lstinline{uint8_t} that has the hexadecimal value of the pressed key (considering \# as 0xE and * as 0xF).
Because of this, the rest of the implementation is a \lstinline{switch} statement to obtain the correct value from the \function{keys} nested array.

Read Section~1.2 of the Cow Pi datasheet.
After you have done so:

\begin{itemize}
    \item Preserve the debouncing code in \function{get_keypress()}, and
    \item Replace the starter code's call to \function{cowpi_get_keypress()} and \lstinline{switch} statement with code that scans the keyboard.
        \begin{itemize}
            \item Use the \lstinline{ioports} array to read from and write to the appropriate pins,
            \item Use \lstinline{delayMicroseconds(1)} for the delay shown on line 4 of the pseudocode in Section~1.2.2 of the datasheet, and
            \item Use the \lstinline{keys} nested array to obtain the correct integer value for the key at a given row and column.
        \end{itemize}
\end{itemize}

\ifbool{offerdecompositionhints}{
    \subsubsection*{Suggestions}
    Look at the pseudocode in the datasheet's Section~1.2.2 and relate it to the ``theory of operation'' in the datasheet's Section~1.2.1.
    Even though the pseudocode is expressed as nested loops, you do not need to implement it that way.
    We have seen implementations that have an outer loop for the rows and an inner loop for the columns,
    and we have seen implementations that have a loop for the rows but use a \lstinline{switch} statement or chained \lstinline{if} statements to enumerate the possibilities for the columns.
    With four rows and four columns, both are viable solutions.

    The key insights are:
    \begin{itemize}
        \item Each of the row vectors you output will have exactly one ``0'' bit;
            the placement of that ``0'' corresponds to exactly one of the rows in the \lstinline{keys} nested array, and
        \item Each of the column vectors you read as input, if it is not 0xF, will have exactly one ``0'' bit;
            the placement of that ``0'' corresponds to exactly one of the columns in the \lstinline{keys} nested array
    \end{itemize}

    Even though you \textit{can} set the loop conditions to terminate the loop as soon as the pressed key has been determined,\footnote{
        I am \textit{not} saying you can \lstinline{return} from inside the loop or \lstinline{break} out of the loop;
        I am saying you can write a \lstinline{while} condition that includes testing whether \lstinline{key_pressed} is still 0xFF.
    } it would be better to keep your loop conditions simple and allow the code to examine all sixteen keys.
    With only sixteen keys, there isn't much time savings to be had by ending the loop as soon as the key has been found,
    and in a ``real'' embedded system, our time budget would have to allow enough time to examine all sixteen keys and so any ``savings'' gained by ending the loop early would be lost to sitting idle later.
}{}

\subsubsection*{Test your code}

Place the left switch in the right position and upload the program to your \developmentboard.
The output will include the key that was pressed (if any), a hyphen if no key is being pressed, or a question mark if \function{get_keypress()} returns an invalid value.

\vspace{1cm}

This would be a good time to make a backup copy of your program.


\subsection{Display Module} \label{subsec:DisplayModule}

You will use a pointer to a \lstinline{cowpi_i2c_t} structure to access the memory-mapped I/O registers for the \developmentboard's I$^2$C (aka ``TWI'') hardware,
and you will use those registers to send data to the display module.

Read Section~1.3 of the Cow Pi datasheet to familiarize yourself with the physical characteristics of the display module.
Read the datasheet's Section~4.2 to familiarize yourself with the I$^2$C registers (Sections~4.2.1--4.2.2), with the I$^2$C handshake and transmission sequence (Section~4.2.3), and with the data format and the steps required by the display module itself (Section~4.2.4). After you have done so:

\begin{itemize}
    \item Uncomment the second line of the starter code's \function{initialize_io()} function, and
    \item Assign the appropriate address to the \lstinline{i2c} pointer on that line.
\end{itemize}

You can now use the \lstinline{i2c} pointer to operate the I$^2$C hardware.
Now locate the \function{send_halfbyte()} function.

\begin{itemize}
    \item Implement the pseudocode from the datasheet's Section~4.2.3, to
    \item Send the three bytes of data described in the datasheet's Section~4.2.4.
        \begin{itemize}
            % \item You may handle the error conditions show in the pseudocode in whatever manner you see fit; it would be best to do so in a manner that allows you to determine \textit{which} error is occurring.
            \item Use \lstinline{delayMicroseconds(1)} to implement the pulse described in Section~4.2.4 of the datasheet.
        \end{itemize}
\end{itemize}

After you are satisfied that you have implemented the \function{send_halfbyte()} function correctly:

\begin{itemize}
    \item Uncomment the last line of the starter code's \function{initialize_io()} function to register \function{send_halfbyte()} with the CowPi library's LCD1602 code.
\end{itemize}

All of the functions in the CowPi library that send characters or commands to the display module will now make use of your \function{send_halfbyte()} function.

\ifbool{offerdecompositionhints}{
    \subsubsection*{Suggestions}
    Altogether, you will make six transmissions in the \function{send_halfbyte()} function: the start bit, the serial adapter's address, three bytes meant for the display module itself, and the stop bit.
    The start and stop bits don't have data associated with them, but the other four transmissions do.
    After each transmission (except the stop bit), you need to wait until the transmission is complete before initiating the next transmission;
    you can achieve this with a \textit{busy wait}.
    As discussed in lecture, a busy wait is simply a do-nothing loop that blocks the program while some condition holds.

    The bit vector created in line~6 of the the pseudocode in the datasheet's Section~4.2.3 consists of bits that must be set to 1 in the control register for all of these transmissions.
    When transmitting the start and stop bits, the relevant bit is also set to 1 in the control register (pseudocode line~9 and 29).
    Otherwise, only the basic bit vector is written to the control register.

    The first datum to transmit after the start bit is the serial adapter's address (properly shifted) so that the adapter knows the remaining transmissions are meant for it.
    The serial adapter will pass the data all transmissions to the display module until it receives the stop bit.
    Even though the pseudocode characterizes those transmissions as a loop over the data maent for the display module, it will be easier to implement sequentially.
    Two of the three bytes meant for the display module are identical, and the remaining byte differs only only in one bit.
    The first ensures the halfbyte, backlight, read/write, and command bits are at their correct values before a latching pulse starts.
    The second maintains those values and creates the latching pulse; a \function{delayMicroseconds()} call holds the pulse.
    Finally, the third byte ensures those values are still present as the latching pulse ends.

    \vspace{0.5cm}

    There are two aspects to this task that some students found frustrating in the past:
    \begin{itemize}
        \item Some of the required bitshifts may seem like make-work -- for example, line~16 of the pseudocode shifts the serial adapter's address to the left by one bit, but after that the least significant bit is always 0
            \begin{itemize}
                \item These bits that are seemingly constant appear so only because this assignment uses one specific part of a very flexible protocol
                \item In other uses of the I$^2$C protocol, the bits that are constant \textit{in this assignment} serve other purposes
            \end{itemize}
        \item If some part of your implementation has a bug, you aren't provided with much data you can use to form hypotheses during debugging
            \begin{itemize}
                \item ``Not much data'' is not the same as ``no data''
                    \begin{itemize}
                        \item If a busy-wait doesn't terminate, the problem is probably with your implementation of the I$^2$C protocol -- double-check the bit vectors placed in the control register, and possibly the bit vector placed in the data register when you transmit the serial adapter's address
                        \item If the function terminates but the display module doesn't update, the problem is probably with the bit vectors placed in the data register that the serial adapter will pass along to the display module, or with the ``latch'' pulse
                    \end{itemize}
                \item You can always get help from the professor or a TA with any aspect of this assignment, and you can discuss \textit{concepts and syntax} with other students;
                    \textit{without showing other students your code}, asking other students questions like these, and answering them, is acceptable:
                    \begin{itemize}
                        \item ``The datasheet says we want a `1' in bit 7, a `1' in bit 2, and `0's in the other bits; that looks like this, right?'' and then write a binary or hexadecimal bit vector on paper or a whiteboard
                        \item ``The datasheet says to shift a `1' seven places to the left, and then shift a `1' two places to the left, and then take those two results and bitwise-OR them together; would you help me remember how to do a left-shift?''
                    \end{itemize}
            \end{itemize}
    \end{itemize}
}{}

\subsubsection*{Test your code}

Place the left switch in the right position and upload the program to your \developmentboard.
If the test code displays its outputs on the display module, then you implemented \function{send_halfbyte()} correctly.
If the test code displays its outputs on the Serial Monitor but not the display module, or if the test code does not run at all, then there is an error in your code.

\vspace{1cm}

If you decide to work on other parts of the assignment before finishing \function{send_halfbyte()} then comment-out the call to \function{cowpi_lcd1602_set_send_halfbyte_function()} in \function{initialize_io()}.
