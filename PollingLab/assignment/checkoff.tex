You are not required to have your assignment checked-off by a TA or the professor.
If you do not do so, then we will perform a functional check ourselves.
In the interest of making grading go faster, we are offering a small bonus to get your assignment checked-off at the start of your scheduled lab time immediately after it is due.
Because checking off all students during lab would take up most of the lab time, we are offering a slightly larger bonus if you complete your assignment early and get it checked-off by a TA or the professor during office hours.

\begin{enumerate}
%\precheckoffitem{Establish that the code you are demonstrating is the code
%    you submitted to to \filesubmission.}
%    \begin{itemize}
%    \item If you are getting checked-off during lab time, show the TA that the file was submitted before it was due.
%    \item Download the file into your number\_builder directory.
%        If necessary, rename it to \textit{number\_builder.ino}.
%    \end{itemize}


    \checkoffitem{Show the TA your \lstinline{keys} nested array, your \function{initialize_io()} function, and your \function{key_movement_detected()} function. \\
        \textit{+1 The \lstinline{key} nested array is correctly populated} \\
        \textit{+\textonehalf\ The correct address is assigned to \lstinline{ioports}} \\
        \textit{+\textonehalf\ The correct address is assigned to \lstinline{i2c}} \\
        \textit{+2 The \function{key_movement_detected} function is correctly implemented}}
    \item [] (TA make note of whether the assignment to \lstinline{cowpi_hd44780_send_halfbyte} at the end of \function{initialize_io()} is commented-out or not.)
    \checkoffitem{Show the TA your implementations of the Input/Output functions.}
    \precheckoffitem{TA make note of which functions are implemented using memory-mapped I/O and which are not.
        Do not award credit for functions that do not use memory-mapped I/O.}
    \precheckoffitem{If the \function{send_halfbyte()} function was attempted but the assignment to \lstinline{cowpi_hd44780_send_halfbyte} is commented-out then
        make note to consider \function{send_halfbyte()} for partial credit later.}

    \item [] \textbf{Memory-Mapped Input/Output}
    \precheckoffitem{Place the left switch in the right position and upload your code to your \developmentboard.}
    \checkoffitem{Show that the display module is displaying the output. \\
        \textit{+6 The \function{send_halfbyte()} function is correctly implemented}}
    \precheckoffitem{If the display module does not display the output, then comment-out the assignment to \lstinline{cowpi_hd44780_send_halfbyte} and re-upload your code to your \developmentboard.
        TA make a note to consider \function{send_halfbyte()} for partial credit later.}
    \checkoffitem{Demonstrate that both pushbuttons' positions are correctly detected. \\
        \textit{+1 The \function{xx_button_is_pressed()} functions are correctly implemented}}
    \checkoffitem{Demonstrate that both switches' positions are correctly detected. \\
        \textit{+1 The \function{xx_switch_is_in_right_position()} functions are correctly implemented}}
    \checkoffitem{Demonstrate that when and only when both pushbuttons are pressed, the left LED (the built-in LED) illuminates.
        Demonstrate that when and only when both switches are in the right position, the right LED (the external LED) illuminates. \\
        \textit{+2 The \function{set_left_led()}  function is correctly implemented} \\
        \textit{+1 The \function{set_right_led()} function is correctly implemented}}
    \checkoffitem{Demonstrate that each of the keys on the keypad is correctly detected and that the absence of a keypress is also reported correctly. \\
        \textit{+8 The \function{get_keypress()} function is correctly implemented}}

    \item [] \textbf{Number Builder}
    \precheckoffitem{If necessary, replace any other non-functioning memory-mapped I/O code with a call to the corresponding function from the CowPi library (see the starter code) and re-upload your code to your \developmentboard.}
    \precheckoffitem{Place the left switch in the left position and press the RESET button on your \developmentboard.}
    \checkoffitem{Place both switches in the left position (left justified mode, decimal number base).}
    \checkoffitem{Press 2, then 3. The right LED illuminates for \textonehalf\ second with each keypress. \\
        Left-justified on the display module is shown: \\
        \display{
            \colorbox{LightGreen}{2\phantom{xxxxxxxxxxxxxxx}} \vspace{-1mm}\\
            \colorbox{LightGreen}{0x2\phantom{xxxxxxxxxxxxx}}
        } \\
    and then: \\
    \display{
        \colorbox{LightGreen}{23\phantom{xxxxxxxxxxxxxx}} \vspace{-1mm}\\
        \colorbox{LightGreen}{0x17\phantom{xxxxxxxxxxxx}}
    } \\
        \textit{+1 Right LED illumination occurs as specified} \\
        \textit{+1 Left justified mode is implemented correctly} \\
        \textit{+1 The first digit is displayed in the correct position in this mode, making a blank display no longer blank.} \\
        \textit{+1 Subsequent digits update the number correctly in this mode}}
    \checkoffitem{Press B. The display is unchanged. \\
        \textit{+\textonehalf\ Decimal number base (positive values) is implemented correctly}}
    \checkoffitem{Press the right pushbutton. The display clears, and the left LED lluminates for \textonehalf\ second: \\
        \display{
            \colorbox{LightGreen}{\phantom{xxxxxxxxxxxxxxxx}} \vspace{-1mm}\\
            \colorbox{LightGreen}{\phantom{xxxxxxxxxxxxxxxx}}
        } \\
        \textit{+1 Left LED illumination occurs as specified} \\
        \textit{+1 The right pushbutton clears the display}}
    \checkoffitem{Press 0.The right LED illuminates for \textonehalf\ second with each keypress. \\ Left-justified on the display module is shown: \\
        \display{
            \colorbox{LightGreen}{0\phantom{xxxxxxxxxxxxxxx}} \vspace{-1mm}\\
            \colorbox{LightGreen}{0\phantom{xxxxxxxxxxxxxxx}}
        } \\
        \textit{+1 The right pushbutton sets the value to 0}}
    \checkoffitem{Press the right pushbutton. The display clears, and the left LED lluminates for \textonehalf\ second.}
    \checkoffitem{Place both switches in the right position (right justified mode, hexadecimal number base).}
    \checkoffitem{Press 2, then 3. The right LED illuminates for \textonehalf\ second with each keypress. \\ Right-justified on the display module is shown: \\
        \display{
            \colorbox{LightGreen}{\phantom{xxxxxxxxxxxxxxx}2} \vspace{-1mm}\\
            \colorbox{LightGreen}{\phantom{xxxxxxxxxxxxx}0x2}
        } \\
        and then: \\
        \display{
            \colorbox{LightGreen}{\phantom{xxxxxxxxxxxxxx}35} \vspace{-1mm}\\
            \colorbox{LightGreen}{\phantom{xxxxxxxxxxxx}0x23}
        } \\
        \textit{+1 Right justified mode is implemented correctly.} \\
        \textit{+1 The first digit is displayed in the correct position in this mode, making a blank display no longer blank.} \\
        \textit{+1 Subsequent digits update the number correctly in this mode.}}
    \checkoffitem{Press B. \\
        \display{
            \colorbox{LightGreen}{\phantom{xxxxxxxxxxxxx}571} \vspace{-1mm}\\
            \colorbox{LightGreen}{\phantom{xxxxxxxxxxx}0x23B}
        } \\
        \textit{+\textonehalf\ Hexadecimal number base (positive values) is implemented correctly.}}
    \checkoffitem{Press the left pushbutton. The left LED illuminates for \textonehalf\ second. The display shows: \\
        \display{
            \colorbox{LightGreen}{\phantom{xxxxxxxxxxxx}-571} \vspace{-1mm}\\
            \colorbox{LightGreen}{\phantom{xxxxxx}0xFFFFFDC5}
        } \\
        \textit{+\textonehalf\ The left pushbutton negates positive values in hexadecimal mode.}}
    \precheckoffitem{If hexadecimal negation does not work, then clear the number with the right pushbutton and then attempt to directly enter the negative value 0xFFFFFDC5.
        While being able to directly enter a negative hex value is a sensible thing to be able to do, it requires handling the end case of transitioning from a 7-hex-digit positive value (0xFFFFFDC) to an 8-hex-digit negative value (0xFFFFFDC5).
        This edge case is not specifically identified in the specification, and so we normally will not test for it;
        however, if a negative hex value cannot be produced through negation then direct entry of a negative hex value is the only other way to test how the number builder handles negative hex values.}
    \checkoffitem{Press A. \\
        \display{
            \colorbox{LightGreen}{\phantom{xxxxxxxxxxx}-9126} \vspace{-1mm}\\
            \colorbox{LightGreen}{\phantom{xxxxx}0xFFFFFDC5A}
        } \\
        \textit{+\textonehalf\ Hexadecimal number base (negative values) is implemented correctly.}}
    \checkoffitem{Press the left pushbutton. \label{step:negateHex} \\
        \display{
            \colorbox{LightGreen}{\phantom{xxxxxxxxxxxx}9126} \vspace{-1mm}\\
            \colorbox{LightGreen}{\phantom{xxxxxxxxxx}0x23A6}
        } \\
        \textit{+\textonehalf\ The left pushbutton negates negative values in hexadecimal mode.}}
    \precheckoffitem{If hexadecimal negation does not work, then clear the number with the right pushbutton and then enter the negative value 0x23A6.}
    \checkoffitem{Press 7, 8, 9, C. \\
        \display{
            \colorbox{LightGreen}{\phantom{xxxxxxx}598112412} \vspace{-1mm}\\
            \colorbox{LightGreen}{\phantom{xxxxxx}0x23A6789C}
        }}
    \checkoffitem{Press D. The system displays: \\
        \display{
            \colorbox{LightGreen}{\phantom{xxxxxx}TOO\phantom{xxxxxx}} \vspace{-1mm}\\
            \colorbox{LightGreen}{\phantom{xxxxxx}BIG!\phantom{xxxxxx}}
        } \\
        \textit{+\textonequarter\ Detects too-big positive hexadecimal numbers.} \\
        \textit{+\textonehalf\ Displays the correct ``too big'' error message.}}
    \checkoffitem{Press the right pushbutton to clear the display and reset the value to 0.}
    \checkoffitem{Generate the value 0x98765432 by one of two means:}
        \begin{multicols}{2}
            Indirectly:
            First enter the value 0x6789ABCE \\
                \display{
                    \colorbox{LightGreen}{\phantom{xxxxxx}1737075662} \vspace{-1mm}\\
                    \colorbox{LightGreen}{\phantom{xxxxxx}0x6789ABCE}
                } \\
            Then negate it with the left pushbutton \\
                \display{
                    \colorbox{LightGreen}{\phantom{xxxxx}-1737075662} \vspace{-1mm}\\
                    \colorbox{LightGreen}{\phantom{xxxxxx}0x98765432}
                }

        \columnbreak

            Directly: enter the value 0x98765432 \\
            \display{
                \colorbox{LightGreen}{\phantom{xxxxx}-1737075662} \vspace{-1mm}\\
                \colorbox{LightGreen}{\phantom{xxxxxx}0x98765432}
            }
        \end{multicols}
    \checkoffitem{Press 1. \\
        \display{
            \colorbox{LightGreen}{\phantom{xxxxxx}TOO\phantom{xxxxxx}} \vspace{-1mm}\\
            \colorbox{LightGreen}{\phantom{xxxxxx}BIG!\phantom{xxxxxx}}
        } \\
        \textit{+\textonequarter\ Detects too-big negative hexadecimal numbers.} \\
        \textit{+\textonequarter\ No false ``too big'' detection of hexadecimal numbers.}}
    \checkoffitem{Press the right pushbutton to clear the display and reset the value to 0.}
    \checkoffitem{Place the right swtich in the left position (decimal number base).}
    \checkoffitem{Enter 1, 2, 3, 4, 5. \\
        \display{
            \colorbox{LightGreen}{\phantom{xxxxxxxxxxx}12345} \vspace{-1mm}\\
            \colorbox{LightGreen}{\phantom{xxxxxxxxxx}0x3039}
        }}
    \checkoffitem{Press the left pushbutton. \\
        \display{
            \colorbox{LightGreen}{\phantom{xxxxxxxxxx}-12345} \vspace{-1mm}\\
            \colorbox{LightGreen}{\phantom{xxxxxx}0xFFFFCFC7}
        } \\
        \textit{+\textonehalf\ The left pushbutton negates positive values in decimal mode.}}
    \checkoffitem{Enter 6, 7, 8, 9, 0. \\
        \display{
            \colorbox{LightGreen}{\phantom{xxxxx}-1234567890} \vspace{-1mm}\\
            \colorbox{LightGreen}{\phantom{xxxxxx}0xB669FD2E}
        } \\
        \textit{+\textonehalf\ Decimal number base (negative values) is implemented correctly.}}
    \checkoffitem{Press the left pushbutton. \\
        \display{
            \colorbox{LightGreen}{\phantom{xxxxxx}1234567890} \vspace{-1mm}\\
            \colorbox{LightGreen}{\phantom{xxxxxx}0x499602D2}
        } \\
        \textit{+\textonehalf\ The left pushbutton negates negative values in decimal mode.}}
    \checkoffitem{Press Press 1. \\
        \display{
            \colorbox{LightGreen}{\phantom{xxxxxx}TOO\phantom{xxxxxx}} \vspace{-1mm}\\
            \colorbox{LightGreen}{\phantom{xxxxxx}BIG!\phantom{xxxxxx}}
        } \\
        \textit{+\textonequarter\ Detects too-big positive decimal numbers.}}
    \checkoffitem{Clear the number with the right pushbutten and enter 1, 2, 3, 4, 5, 6, 7, 8, 9, 0. \\
        \display{
            \colorbox{LightGreen}{\phantom{xxxxxx}1234567890} \vspace{-1mm}\\
            \colorbox{LightGreen}{\phantom{xxxxxx}0x499602D2}
        }}
    \checkoffitem{Press the left pushbutton. \\
        \display{
            \colorbox{LightGreen}{\phantom{xxxxx}-1234567890} \vspace{-1mm}\\
            \colorbox{LightGreen}{\phantom{xxxxxx}0xB669FD2E}
        }}
    \checkoffitem{Press 0. \\
        \display{
            \colorbox{LightGreen}{\phantom{xxxxxx}TOO\phantom{xxxxxx}} \vspace{-1mm}\\
            \colorbox{LightGreen}{\phantom{xxxxxx}BIG!\phantom{xxxxxx}}
        } \\
        \textit{+\textonequarter\ Detects too-big negative decimal numbers.} \\
        \textit{+\textonequarter\ No false ``too big'' detection of decimal numbers.}}
    \checkoffitem{Clear the number with the right pushbutton, and \textbf{rapidly} (strictly less than \textonehalf~second between presses) enter 1, 2, 3, left pushbutton, left pushbutton, left pushbutton, 4, 5, 6. \\
        The display updates as fast as you can press the keys and buttons.
        The right LED stays illuminated between keypresses until \textonehalf~second after the 3 is pressed, and again \textonehalf~second after the 6 is pressed.
        The left LED stays illuminated between button presses until \textonehalf~second after the third button press.
        At the end, the display shows: \\
        \display{
            \colorbox{LightGreen}{\phantom{xxxxxxxxx}-123456} \vspace{-1mm}\\
            \colorbox{LightGreen}{\phantom{xxxxxx}0xFFFE1DC0}
        } \\
        \textit{+1 The system is responsive and does not block after an input.}}
    \item [] \textbf{Overall} \\
    \textit{+\textonehalf\ A single button press is treated as a single press.} \\
    \textit{+\textonehalf\ A single key press is treated as a single press.}
\end{enumerate}

This concludes the demonstration of your system's functionality.
The TAs will later examine your code for violations of the assignment's constraints.
If your code looks like it is tailored for this checklist, the TAs may re-grade using a different checklist.
