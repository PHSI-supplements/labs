You are not required to have your assignment checked-off by a TA or the professor.
If you do not do so, then we will perform a functional check ourselves.
In the interest of making grading go faster, we are offering a small bonus to get your assignment checked-off at the start of your scheduled lab time immediately after it is due.
Because checking off all students during lab would take up most of the lab time, we are offering a slightly larger bonus if you complete your assignment early and get it checked-off by a TA or the professor during office hours.

\begin{enumerate}
%\precheckoffitem{Establish that the code you are demonstrating is the code
%    you submitted to to \filesubmission.}
%    \begin{itemize}
%    \item If you are getting checked-off during lab time, show the TA that the file was submitted before it was due.
%    \item Download the file into your number\_builder directory.
%        If necessary, rename it to \textit{number\_builder.ino}.
%    \end{itemize}


    \checkoffitem{Show the TA your \lstinline{keys} nested array, your \function{initialize_io()} function, and your \function{key_movement_detected()} function. \\
        \textit{+1 The \lstinline{key} nested array is correctly populated} \\
        \textit{+1 The correct address is assigned to \lstinline{ioports}} \\
        \textit{+1 The correct address is assigned to \lstinline{i2c}} \\
        \textit{+2 The \function{key_movement_detected} function is correctly implemented}}
    \item [] (TA make note of whether the call to \function{cowpi_lcd1602_set_send_halfbyte_function()} at the end of \function{initialize_io()} is commented-out or not.)
    \checkoffitem{Show the TA your implementations of the Input/Output functions.}
    \item [] (TA make note of which functions are implemented using memory-mapped I/O and which are not.
        Do not award credit for functions that do not use memory-mapped I/O.
        If the call to \function{cowpi_lcd1602_set_send_halfbyte_function()} then do not award credit for the \function{send_halfbyte()} function now;
        instead, make note of whether the student attempted to implement \function{send_halfbyte()} for later consideration of partial credit.)

    \item [] \textbf{Memory-Mapped Input/Output}
    \precheckoffitem{Place the left switch in the right position and upload your code to your \developmentboard.}
    \checkoffitem{Show that the display module is displaying the output. \\
        \textit{+6 The \function{send_halfbyte()} function is correctly implemented}}
    \item [] (If the display module does not display the output, then comment-out the call to \function{cowpi_lcd1602_set_send_halfbyte_function()} and re-upload your code to your \developmentboard.
        TA make a note to consider \function{send_halfbyte()} for partial credit later.)
    \checkoffitem{Demonstrate that both pushbuttons' positions are correctly detected. \\
        \textit{+1 The \function{xx_button_is_pressed()} functions are correctly implemented}}
    \checkoffitem{Demonstrate that both switches' positions are correctly detected. \\
        \textit{+1 The \function{xx_switch_is_in_right_position()} functions are correctly implemented}}
    \checkoffitem{Demonstrate that when and only when both pushbuttons are pressed, the left LED (the built-in LED) illuminates.
        Demonstrate that when and only when both switches are in the right position, the right LED (the external LED) illuminates. \\
        \textit{+2 The \function{set_xx_led()} functions are correctly implemented}}
    \checkoffitem{Demonstrate that each of the keys on the keypad is correctly detected and that the absence of a keypress is also reported correctly. \\
        \textit{+8 The \function{get_keypress()} function is correctly implemented}}

    \item [] \textbf{Number Builder}
    \precheckoffitem{If necessary, replace any other non-functioning memory-mapped I/O code with a call to the corresponding function from the CowPi library (see the starter code) and re-upload your code to your \developmentboard.}
    \precheckoffitem{Place the left switch in the left position and press the RESET button on your \developmentboard.}
    \checkoffitem{Place both switches in the left position (left justified mode, decimal number base).}
    \checkoffitem{Press 2, then 3. The right LED illuminates for \textonehalf\ second with each keypress. \\
        Left-justified in the upper-left corner of the display module is shown: \\
        \display{
            \colorbox{LightGreen}{2\phantom{xxxxxxxxxxxxxxx}} \vspace{-1mm}\\
            \colorbox{LightGreen}{\phantom{xxxxxxxxxxxxxxxx}}
        } \\
    and then: \\
    \display{
        \colorbox{LightGreen}{23\phantom{xxxxxxxxxxxxxx}} \vspace{-1mm}\\
        \colorbox{LightGreen}{\phantom{xxxxxxxxxxxxxxxx}}
    } \\
        \textit{+1 Right LED illumination occurs as specified} \\
        \textit{+1 Left justified mode is implemented correctly} \\
        \textit{+1 The first digit is displayed in the correct position in this mode, making a blank display no longer blank.} \\
        \textit{+1 Subsequent digits update the number correctly in this mode}}
    \checkoffitem{Press B. The display is unchanged. \\
        \textit{+\textonehalf\ Decimal number base (positive values) is implemented correctly}}
    \checkoffitem{Press the right pushbutton. The left LED lluminates for \textonehalf\ second. \\
        Left-justified in the lower-left corner of the display module is shown: \\
        \display{
            \colorbox{LightGreen}{\phantom{xxxxxxxxxxxxxxxx}} \vspace{-1mm}\\
            \colorbox{LightGreen}{23\phantom{xxxxxxxxxxxxxx}}
        } \\
        One second later, the display clears. \\
        \textit{+1 Left LED illumination occurs as specified} \\
        \textit{+1 The right pushbutton causes a copy of the number to be displayed on the bottom row} \\
        \textit{+1 The right pushbutton causes the display to clear after one second}}
    \checkoffitem{Place both switches in the right position (right justified mode, hexadecimal number base).}
    \checkoffitem{Press 2, then 3. The right LED illuminates for \textonehalf\ second with each keypress. \\ Right-justified in the upper-right corner of the display module is shown: \\
        \display{
            \colorbox{LightGreen}{\phantom{xxxxxxxxxxxxx}0x2} \vspace{-1mm}\\
            \colorbox{LightGreen}{\phantom{xxxxxxxxxxxxxxxx}}
        } \\
        and then: \\
        \display{
            \colorbox{LightGreen}{\phantom{xxxxxxxxxxxx}0x23} \vspace{-1mm}\\
            \colorbox{LightGreen}{\phantom{xxxxxxxxxxxxxxxx}}
        } \\
        \textit{+1 Right justified mode is implemented correctly.} \\
        \textit{+1 The first digit is displayed in the correct position in this mode, making a blank display no longer blank.} \\
        \textit{+1 Subsequent digits update the number correctly in this mode.}}
    \checkoffitem{Press B. \\
        \display{
            \colorbox{LightGreen}{\phantom{xxxxxxxxxxx}0x23B} \vspace{-1mm}\\
            \colorbox{LightGreen}{\phantom{xxxxxxxxxxxxxxxx}}
        } \\
        \textit{+\textonehalf\ Hexadecimal number base (positive values) is implemented correctly.}}
    \checkoffitem{Press the left pushbutton. The left LED illuminates for \textonehalf\ second. The display shows, right-justified in the upper-right corner: \\
        \display{
            \colorbox{LightGreen}{\phantom{xxxxxx}0xFFFFFDC5} \vspace{-1mm}\\
            \colorbox{LightGreen}{\phantom{xxxxxxxxxxxxxxxx}}
        } \\
        \textit{+\textonehalf\ The left pushbutton negates positive hexadecimal values.}}
    \precheckoffitem{If hexadecimal negation does not work, then clear the number with the right pushbutton and then attempt to directly enter the negative value 0xFFFFFDC5.
        While being able to directly enter a negative hex value is a sensible thing to be able to do, it requires handling the end case of transitioning from a 7-hex-digit positive value (0xFFFFFDC) to an 8-hex-digit negative value (0xFFFFFDC5).
        This edge case is not specifically identified in the specification, and so we normally will not test for it;
        however, if a negative hex value cannot be produced through negation then direct entry of a negative hex value is the only other way to test how the number builder handles negative hex values.}
    \checkoffitem{Press A. \\
        \display{
            \colorbox{LightGreen}{\phantom{xxxxx}0xFFFFFDC5A} \vspace{-1mm}\\
            \colorbox{LightGreen}{\phantom{xxxxxxxxxxxxxxxx}}
        } \\
        \textit{+\textonehalf\ Hexadecimal number base (negative values) is implemented correctly.}}
    \checkoffitem{Press the left pushbutton. \label{step:negagteHex} \\
        \display{
            \colorbox{LightGreen}{\phantom{xxxxxxxxxxxx}23A6} \vspace{-1mm}\\
            \colorbox{LightGreen}{\phantom{xxxxxxxxxxxxxxxx}}
        } \\
        \textit{+\textonehalf\ The left pushbutton negates negative hexadecimal values.}}
    \precheckoffitem{If hexadecimal negation does not work, then clear the number with the right pushbutton and then enter the negative value 0x23A6.}
    \checkoffitem{Press 7, 8, 9, C. \\
        \display{
            \colorbox{LightGreen}{\phantom{xxxxxx}0x23A6789C} \vspace{-1mm}\\
            \colorbox{LightGreen}{\phantom{xxxxxxxxxxxxxxxx}}
        }}
    \checkoffitem{Press D. The system displays: \\
        \display{
            \colorbox{LightGreen}{\phantom{xxxxx}too big\phantom{xxxxx}} \vspace{-1mm}\\
            \colorbox{LightGreen}{\phantom{xxxxxxxxxxxxxxxx}}
        } \\
        (the specification does not specify alignment) \\
        \textit{+\textonequarter\ Detects too-big positive hexadecimal numbers.} \\
        \textit{+\textonehalf\ Displays the correct ``too big'' error message.}}
    \checkoffitem{Press the right pushbutton and wait for the display to clear.}
    \checkoffitem{Generate the value 0x98765432 by one of two means:}
        \begin{multicols}{2}
            Indirectly:
            First enter the value 0x6789ABCE \\
                \display{
                    \colorbox{LightGreen}{\phantom{xxxxxx}0x6789ABCE} \vspace{-1mm}\\
                    \colorbox{LightGreen}{\phantom{xxxxxxxxxxxxxxxx}}
                } \\
            Then negate it with the left pushbutton \\
                \display{
                    \colorbox{LightGreen}{\phantom{xxxxxx}0x98765432} \vspace{-1mm}\\
                    \colorbox{LightGreen}{\phantom{xxxxxxxxxxxxxxxx}}
                }

        \columnbreak

            Directly: enter the value 0x98765432 \\
            \display{
                \colorbox{LightGreen}{\phantom{xxxxxx}0x98765432} \vspace{-1mm}\\
                \colorbox{LightGreen}{\phantom{xxxxxxxxxxxxxxxx}}
            }
        \end{multicols}
    \checkoffitem{Press 1. \\
    \display{
        \colorbox{LightGreen}{\phantom{xxxxx}too big\phantom{xxxxx}} \vspace{-1mm}\\
        \colorbox{LightGreen}{\phantom{xxxxxxxxxxxxxxxx}}
    } \\
        \textit{+\textonequarter\ Detects too-big negative hexadecimal numbers.} \\
        \textit{+\textonequarter\ No false ``too big'' detection of hexadecimal numbers.}}
    \checkoffitem{Press the right pushbutton and wait for the display to clear.}
    \checkoffitem{Place the right swtich in the left position (decimal number base).}
    \checkoffitem{Enter 1, 2, 3, 4, 5. \\
        \display{
            \colorbox{LightGreen}{\phantom{xxxxxxxxxxx}12345} \vspace{-1mm}\\
            \colorbox{LightGreen}{\phantom{xxxxxxxxxxxxxxxx}}
        }}
    \checkoffitem{Press the left pushbutton. \\
        \display{
            \colorbox{LightGreen}{\phantom{xxxxxxxxxx}-12345} \vspace{-1mm}\\
            \colorbox{LightGreen}{\phantom{xxxxxxxxxxxxxxxx}}
        } \\
        \textit{+\textonehalf\ The left pushbutton negates positive decimal values.}}
    \checkoffitem{Enter 6, 7, 8, 9, 0. \\
        \display{
            \colorbox{LightGreen}{\phantom{xxxxx}-1234567890} \vspace{-1mm}\\
            \colorbox{LightGreen}{\phantom{xxxxxxxxxxxxxxxx}}
        } \\
        \textit{+\textonehalf\ Decimal number base (negative values) is implemented correctly.}}
    \checkoffitem{Press the left pushbutton. \\
        \display{
            \colorbox{LightGreen}{\phantom{xxxxxx}1234567890} \vspace{-1mm}\\
            \colorbox{LightGreen}{\phantom{xxxxxxxxxxxxxxxx}}
        } \\
        \textit{+\textonehalf\ The left pushbutton negates negative decimal values.}}
    \checkoffitem{Press Press 1. \\
        \display{
            \colorbox{LightGreen}{\phantom{xxxxx}too big\phantom{xxxxx}} \vspace{-1mm}\\
            \colorbox{LightGreen}{\phantom{xxxxxxxxxxxxxxxx}}
        } \\
        \textit{+\textonequarter\ Detects too-big positive decimal numbers.}}
    \checkoffitem{Clear the number with the right pushbutten and enter 1, 2, 3, 4, 5, 6, 7, 8, 9, 0. \\
        \display{
            \colorbox{LightGreen}{\phantom{xxxxxx}1234567890} \vspace{-1mm}\\
            \colorbox{LightGreen}{\phantom{xxxxxxxxxxxxxxxx}}
        }}
    \checkoffitem{Press the left pushbutton. \\
        \display{
            \colorbox{LightGreen}{\phantom{xxxxx}-1234567890} \vspace{-1mm}\\
            \colorbox{LightGreen}{\phantom{xxxxxxxxxxxxxxxx}}
        }}
    \checkoffitem{Press 0. \\
        \display{
            \colorbox{LightGreen}{\phantom{xxxxx}too big\phantom{xxxxx}} \vspace{-1mm}\\
            \colorbox{LightGreen}{\phantom{xxxxxxxxxxxxxxxx}}
        } \\
        \textit{+\textonequarter\ Detects too-big negative decimal numbers.} \\
        \textit{+\textonequarter\ No false ``too big'' detection of decimal numbers.}}
\item [] \textbf{Overall} \\
    \textit{+\textonehalf\ A single button press is treated as a single press.} \\
    \textit{+\textonehalf\ A single key press is treated as a single press.}
\end{enumerate}

This concludes the demonstration of your system's functionality.
The TAs will later examine your code for violations of the assignment's constraints.
If your code looks like it is tailored for this checklist, the TAs may re-grade using a different checklist.

The TAs also will later check your code for the bonus points available for not using \function{sprintf()} or other functions in the \function{printf()} family.
If \texttt{grep} indicates that such calls only occur in \function{test_io()} then you will receive that bonus: \\
\begin{verbatim}
    % grep printf PollingLab/*c
    PollingLab/io_functions.c:    // We can use sprintf to create the string
    PollingLab/io_functions.c:    sprintf(test_io_data, " %c    %c %c   %c %c ",
    PollingLab/io_functions.c:    printf("%s\n", test_io_header);
    PollingLab/io_functions.c:    printf("%s\n\n", test_io_data);
    %
\end{verbatim}