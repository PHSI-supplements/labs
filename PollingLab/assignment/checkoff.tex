You are not required to have your assignment checked-off by a TA or the
professor. If you do not do so, then we will perform a functional check
ourselves. In the interest of making grading go faster, we are offering a small
bonus to get your assignment checked-off at the start of your scheduled lab
time immediately after it is due. Because checking off all students during lab
would take up most of the lab time, we are offering a slightly larger bonus if
you complete your assignment early and get it checked-off by a TA or the
professor during office hours.

\begin{enumerate}
\precheckoffitem{Establish that the code you are demonstrating is the code
    you submitted to to \filesubmission.}
    \begin{itemize}
    \item If you are getting checked-off during lab time, show the TA that the
        file was submitted before it was due.
    \item Download the file into your number\_builder directory. If necessary,
        rename it to \textit{number\_builder.ino}.
    \end{itemize}
\checkoffitem{Show the TA your implementations of the Input/Output functions.}
\item [] (TA make note of which functions are implemented using memory-mapped I/O and which are not. Do not award credit for functions that do not use memory-mapped I/O. Pay attention to the debouncing code, making sure that it is not implemented with calls to \function{delay()}.)
\checkoffitem{Show that debouncing is correctly implemented. \\
    \textit{+1 The inputs are debounced in software without using blocking calls.}}
\item [] \textbf{Memory-Mapped Input/Output}
\checkoffitem{Ensure that the \function{loop()} function calls \function{test_io()} and not \function{build_number()}. Upload \textit{number\_builder.ino} to your \nano.}
\checkoffitem{Show that the display module is displaying the output. \\
    \textit{+8 The \function{send_halfbyte()} function is correctly implemented.}}
\item [] (If the display module does not display the output, then the remainder of the memory-mapped I/O items can be checked on the serial monitor.)
\checkoffitem{Demonstrate that both pushbuttons' positions are correctly detected. \\
    \textit{+2 The \function{xx_button_is_pressed()} functions are correctly implemented.}}
\checkoffitem{Demonstrate that both switches' positions are correctly detected. \\
    \textit{+2 The \function{xx_switch_in_xx_position()} functions are correctly implemented.}}
\checkoffitem{Demonstrate that when and only when both pushbuttons are pressed, the left LED (the built-in LED) illuminates. Demonstrate that when and only when both switches are in the right position, the right LED (the external LED) illuminates. \\
    \textit{+2 The \function{set_xx_led()} functions are correctly implemented.}}
\checkoffitem{Demonstrate that each of the keys on the keypad is correctly detected and that the absence of a keypress is also reported correctly. \\
    \textit{+8 The \function{get_keypress()} function is correctly implemented.}}
\item [] \textbf{Number Builder}
\precheckoffitem{If the display module does not display the output, then comment-out the line in \function{initalize_io()} that registers \function{send_halfbyte()} with the CowPi library.}
\precheckoffitem{Replace any other non-functioning I/O code with a call to the corresponding function from the CowPi library (see the starter code). \textit{Do NOT remove the debouncing code.}}
\checkoffitem{Ensure that the \function{loop()} function calls \function{build_number()} and not \function{test_io()}. Upload \textit{number\_builder.ino} to your \nano.}
\checkoffitem{Place both switches in the left position (left justified mode, decimal number base).}
\checkoffitem{Press 2, then 3. The right LED illuminates for \textonehalf\ second with each keypress. \\ Left-justified in the upper-left corner of the display module is shown: \\ \display{2} \\ and then: \\ \display{23} \\
    \textit{+1 Right LED illumination occurs as specified.} \\
    \textit{+1 Left justified mode is implemented correctly.} \\
    \textit{+1 The first digit is displayed in the correct position in this mode, making a blank display no longer blank.} \\
    \textit{+1 Subsequent digits update the number correctly in this mode.}}
\checkoffitem{Press B. The display is unchanged. \\
    \textit{+\textonehalf\ Decimal number base (positive values) is implemented correctly.}}
\checkoffitem{Press the right pushbutton. The left LED lluminates for \textonehalf\ second. \\ Left-justified in the lower-left corner of the display module is shown: \\ \display{23} \\ One second later, the display clears. \\
    \textit{+1 Left LED illumination occurs as specified.} \\
    \textit{+1 The right pushbutton causes a copy of the number to be displayed on the bottom row.} \\
    \textit{+1 The right pushbutton causes the display to clear after one second.}}
\checkoffitem{Place both switches in the right position (right justified mode, hexadecimal number base).}
\checkoffitem{Press 2, then 3. The right LED illuminates for \textonehalf\ second with each keypress. \\ Right-justified in the upper-right corner of the display module is shown: \\ \parbox{3cm}{\raggedleft\display{0x2} \\ and then: \\ \display{0x23}} \\
    \textit{+1 Right justified mode is implemented correctly.} \\
    \textit{+1 The first digit is displayed in the correct position in this mode, making a blank display no longer blank.} \\
    \textit{+1 Subsequent digits update the number correctly in this mode.}}
\checkoffitem{Press B. \\ \parbox{3cm}{\raggedleft\display{0x23B}} \\
    \textit{+\textonehalf\ Hexadecimal number base (positive values) is implemented correctly.}}
\checkoffitem{Press the left pushbutton. The left LED illuminates for  \textonehalf\ second. The display show, right-justified in the upper-right corner: \\
    \parbox{3cm}{\raggedleft\display{0xFFFFFDC5}} \\
    \textit{+\textonehalf\ The left pushbutton negates positive hexadecimal values.}}
\checkoffitem{Press A. \\ \parbox{3cm}{\raggedleft\display{0xFFFFDC5A}} \\
    \textit{+\textonehalf\ Hexadecimal number base (negative values) is implemented correctly.}}
\checkoffitem{Press the left pushbutton. \\
    \parbox{3cm}{\raggedleft\display{0x23A6}} \\
    \textit{+\textonehalf\ The left pushbutton negates negative hexadecimal values.}}
\checkoffitem{Press 7, 8, 9, C. \\
    \parbox{3cm}{\raggedleft\display{0x23A67}} \\
    \parbox{3cm}{\raggedleft\display{0x23A678}} \\
    \parbox{3cm}{\raggedleft\display{0x23A6789}} \\
    \parbox{3cm}{\raggedleft\display{0x23A6789C}}}
\checkoffitem{Press D. The system displays: \\ \display{too big} \\
    \textit{+\textonequarter\ Detects too-big positive hexadecimal numbers.} \\
    \textit{+\textonehalf\ Displays the correct ``too big'' error message.}}
\checkoffitem{Press the right pushbutton and wait for the display to clear.}
\checkoffitem{Enter 9, 8, 7, 6, 5, 4, 3, 2. \\
    \parbox{3cm}{\raggedleft\display{0x9}} \\
    \parbox{3cm}{\raggedleft\display{0x98}} \\
    \parbox{3cm}{\raggedleft\display{0x987}} \\
    \parbox{3cm}{\raggedleft\display{0x9876}} \\
    \parbox{3cm}{\raggedleft\display{0x9876}} \\
    \parbox{3cm}{\raggedleft\display{0x98765}} \\
    \parbox{3cm}{\raggedleft\display{0x987654}} \\
    \parbox{3cm}{\raggedleft\display{0x9876543}} \\
    \parbox{3cm}{\raggedleft\display{0x98765432}}}
\checkoffitem{Press 1. \\ \display{too big} \\
    \textit{+\textonequarter\ Detects too-big negative hexadecimal numbers.} \\
    \textit{+\textonequarter\ No false ``too big'' detection of hexadecimal numbers.}}
\checkoffitem{Press the right pushbutton and wait for the display to clear.}
\checkoffitem{Place the right swtich in the left position (decimal number base).}
\checkoffitem{Enter 1, 2, 3, 4, 5. \\
    \parbox{3cm}{\raggedleft\display{1}} \\
    \parbox{3cm}{\raggedleft\display{12}} \\
    \parbox{3cm}{\raggedleft\display{123}} \\
    \parbox{3cm}{\raggedleft\display{1234}} \\
    \parbox{3cm}{\raggedleft\display{12345}}}
\checkoffitem{Press the left pushbutton. \\
    \parbox{3cm}{\raggedleft\display{-12345}} \\
    \textit{+\textonehalf\ The left pushbutton negates positive decimal values.}}
\checkoffitem{Enter 6, 7, 8, 9, 0. \\
    \parbox{3cm}{\raggedleft\display{-123456}} \\
    \parbox{3cm}{\raggedleft\display{-1234567}} \\
    \parbox{3cm}{\raggedleft\display{-12345678}} \\
    \parbox{3cm}{\raggedleft\display{-123456789}} \\
    \parbox{3cm}{\raggedleft\display{-1234567890}} \\
    \textit{+\textonehalf\ Decimal number base (negative values) is implemented correctly.}}
\checkoffitem{Press the left pushbutton. \\
    \parbox{3cm}{\raggedleft\display{1234567890}} \\
    \textit{+\textonehalf\ The left pushbutton negates negative decimal values.}}
\checkoffitem{Press 1. \\ \display{too big} \\
    \textit{+\textonequarter\ Detects too-big positive decimal numbers.}}
\checkoffitem{Enter 1, 2, 3, 4, 5, 6, 7, 8, 9, 0. \\
    \parbox{3cm}{\raggedleft\display{1}} \\
    \parbox{3cm}{\raggedleft\display{12}} \\
    \parbox{3cm}{\raggedleft\display{123}} \\
    \parbox{3cm}{\raggedleft\display{1234}} \\
    \parbox{3cm}{\raggedleft\display{12345}} \\
    \parbox{3cm}{\raggedleft\display{123456}} \\
    \parbox{3cm}{\raggedleft\display{1234567}} \\
    \parbox{3cm}{\raggedleft\display{12345678}} \\
    \parbox{3cm}{\raggedleft\display{123456789}} \\
    \parbox{3cm}{\raggedleft\display{1234567890}}}
\checkoffitem{Press the left pushbutton. \\
    \parbox{3cm}{\raggedleft\display{-1234567890}}}
\checkoffitem{Press 0. \\ \display{too big} \\
    \textit{+\textonequarter\ Detects too-big negative decimal numbers.} \\
    \textit{+\textonequarter\ No false ``too big'' detection of decimal numbers.}}
\item [] \textbf{Overall} \\
    \textit{+\textonehalf\ Button presses lasting fewer than 500ms are treated as a single press.} \\
    \textit{+\textonehalf\ Key presses lasting fewer than 500ms are treated as a single press.}
\end{enumerate}

This concludes the demonstration of your system's functionality. The TAs will
later examine your code for violations of the assignment's constraints. If your
code looks like it is tailored for this checklist, the TAs may re-grade using a
different checklist\texttt{.}