\section{Lab Overview}

Please familiarize yourself with the entire assignment before beginning.
Your first action must be to populate the keypad's lookup table (Section~\ref{sec:populateKeypad}).
Next you must introduce debouncing code to the I/O functions in the starter code (Section~\ref{sec:debouncing}).
After that, there are two major parts to this assignment, and they can be completed in either order.

\subsection{Memory-Mapped Input/Output}

The starter code contains functions that provide access to the buttons, switches, keypad, LEDs, and display module.
Initially, these functions make use of functions available in the CowPi library.
In Section~\ref{sec:MemMapIO}, you will re-implement these functions using memory-mapped I/O.

\subsection{Implementing a Simple System using Polling}

In Section~\ref{sec:SimpleSystem}, you will implement a simple system that makes use of your hardware kit's buttons, switches, keypad, LEDs, and display module.
You will implement this system by polling inputs.

% Please familiarize yourself with the entire assignment before beginning.
% Section~\ref{sec:FunctionalSpecification} has the functional specification of
% the system you will develop. Section~\ref{sec:Constraints} describes
% implementation constraints. Section~\ref{sec:DemonstrationMode} guides you in
% implementing the first portion of the system, and
% Section~\ref{sec:BuildingMode} offers suggestions in implementing the second
% portion of the system.

\subsection{Constraints} \label{sec:Constraints}

You may use any features that are part of the C standard if they are supported by the compiler. You may use the constants and functions provided in the starter code (to receive credit for the memory-mapped I/O portion of this lab, you will need to re-implement the I/O functions).

You must detect inputs using polling; you may not use interrupts.

All of your code must go in \textit{number\_builder.ino}.

\subsubsection{Constraints on the Arduino core}

You may use the \function{millis()} function to implement debouncing.
You may use the \function{delayMicroseconds()} function to introduce a 1$\mu$s delay in \function{get_keypress()} and to introduce a 1$\mu$s delay in \function{send_halfbyte()}.
You may (but are not required to) use \function{Serial.print()} and \function{Serial.println()} instead of \function{printf()} if you wish.

You may not use any other libraries, functions, macros, types, or constants from the Arduino core.\footnote{\url{https://www.arduino.cc/reference/en/}}

\subsubsection{Constraints on AVR-libc}

You may not use any AVR-specific functions, macros, types, or constants of avr-libc.\footnote{\url{https://www.nongnu.org/avr-libc/user-manual/index.html}}

\subsubsection{Constraints on the CowPi library}

To receive credit for the memory-mapped I/O portion of this lab, all input and
output must be accomplished using the memory-mapped I/O registers.
You may not use the functions described in Sections~2.2--2.4 of the Cow Pi datasheet.

You may use any functions that that send characters or commands to the display module that are described in Section~2.5 of the Cow Pi datasheet.
Your \function{send_halfbyte()} function, however, must be implemented using memory-mapped I/O registers.

\subsubsection{Constraints on other libraries}

You may not use any libraries beyond those explicitly identified here.