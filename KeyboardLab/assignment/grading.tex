When you have completed this assignment, upload \textit{problem1.c},
\textit{problem2.c}, and \textit{problem3.c} to \filesubmission.

This assignment is worth 20 points.
\begin{description}
    \rubricitem{4}{\textit{problem1.c} produces the specified output.}
    \rubricitem{4}{\function{iz_digit()} in \textit{problem2.c} determines whether
    or not a character is a digit.}
    \rubricitem{4}{\function{decapitalize()} in \textit{problem2.c} converts
    uppercase letters to lowercase and leaves other characters unchanged.}
    \rubricitem{4}{\function{is_even()} in \textit{problem3.c} determines whether
    a number is even or odd.}
    \item[\hspace{1cm}]\function{produce_multiple_of_ten()} in \textit{problem3.c}
    has the following:
    \begin{description}
        \rubricitem{1}{Code to assign the value 5 to the variable \lstinline{five}}
        \rubricitem{1}{Code to divide an even number by 2}
        \rubricitem{1}{Code to subtract 1 from an odd number}
        \rubricitem{1}{Correct functionality}
    \end{description}
    \item[Penalties]
    \penaltyitem{4}{The solution to \textit{problem1.c} uses \texttt{w},
        \texttt{W}, \texttt{\textbackslash{}n}, or \texttt{\textbackslash{}t}.}
    \penaltyitem{4}{\function{iz_digit())} uses a digit other than 0 and 9, uses a
    \lstinline{switch} statement, uses more than one \lstinline{if} statement, or
    uses code from an \lstinline{#include}d header.}
    \penaltyitem{4}{\function{decapitalize()} uses a \lstinline{switch} statement,
        uses more than one \lstinline{if} statement, or uses code from an
        \lstinline{#include}d header.}
    \penaltyitem{4}{\function{is_even()} uses arithmetic.}
    \penaltyitem{4}{\function{produce_multiple_of_ten()} uses addition,
        subtraction, division, or modulo; or \function{produce_multiple_of_ten()}
        uses the literal value 5, 0x5, 05, or 0b101.}
    \spaghetticodepenalties{1}
\end{description}
