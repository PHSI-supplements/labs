The purpose of this assignment is to give you more confidence in C programming and to begin your exposure to the underlying bit-level representation of data.

The instructions are written assuming you will edit and run the code on \runtimeenvironment.
If you wish, you may edit and run the code in a different environment;
be sure that your compiler suppresses no warnings, and that if you are using an IDE that it is configured for C and not C++.

\section*{Learning Objectives}

After successful completion of this assignment, students will be able to:
\begin{itemize}
    \item Use the ASCII table to determine the corresponding integer values of C \lstinline{char} values.
    \item Apply arithmetic operators and comparators to C \lstinline{}{char} values.
    \item Construct and use a bitmask.
    \item Use bitwise operators and bit shift operators to create and modify values.
\end{itemize}

\subsection*{Continuing Forward}

Your experience with viewing values as bit patterns will be applicable in future labs, as will bit masks and bit operations.
Some of the functions you write in this lab will be used in the next lab.

\section*{During Lab Time}

During your lab period, the TAs will demonstrate how to read the ASCII table and will provide a refresher on bitwise AND, bitwise OR, and left- and right-shifts.
During the remaining time, the TAs will be available to answer questions.

Before leaving lab, \textit{at a minimum} complete Section~\ref{sec:asciiTable} and the first function in Section~\ref{sec:charactersAsNumbers}.
Also make sure that you understand how to use bitwise operations to examine specific bits and to change specific bits.
