You soon discover your first $\cancelto{\mathrm{opportunity}}{\mathrm{challenge}}$.
Archie purchased your workstation from government surplus.
Your keyboard is left over from early 2001 and is missing the letter \textit{\texttt{W}}!\footnote{In January 2001, when President Bill Clinton's staff left the White House so that President George~W. Bush's staff could move in, they removed the \textit{\texttt{W}} key from several keyboards as a prank.}
You decide to write an email requesting a new keyboard: \\
\\
\texttt{TO\tab Archie\nl} \\
\texttt{RE\tab I Need a Working Keyboard\nl} \\
\nl \\
\texttt{Please order a new keyboard for me. This one is broken.\nl } \\ \\
(Note: here, the \tab\ symbol represents the \texttt{TAB} character which is needed by the email program, and the \nl\ symbol represents a \texttt{NEWLINE} character.)

You quickly realize that you can't type this directly into your mail program because of the missing \textit{\texttt{W}} key.
So you decide to write a short program that will output the text that you want to send.
The code you would like to write is:

\begin{lstlisting}
#include <stdio.h>
#include <string.h>

char *generate_email(char *destination, size_t destination_size) {
    char *lines[3] = {
            "TO\tArchie\n",
            "RE\tI Need a Working Keyboard\n\n",
            "Please order a new keyboard for me. This one is broken.\n"
    };
    sprintf(destination, "%s", lines[0]);
    size_t remaining_space = destination_size - strlen(lines[0]);
    strncat(destination, lines[1], remaining_space);
    remaining_space = remaining_space - strlen(lines[1]);
    strncat(destination, lines[2], remaining_space);
    return destination;
}
\end{lstlisting}

Of course, the \texttt{W} and the \texttt{w} are still a problem, but you realize you can insert those characters by using their ASCII values.\footnote{Use the ASCII table in the textbook or type \texttt{man ascii} in a terminal window.}
For example,

\lstinline{printf("Hello World!\n");} \\
can be replaced with

\lstinline{printf("%s%c%s\n", "Hello ", ..., "orld!");} \\
replacing \dots\ with the ASCII value for \texttt{W}.
Recall that the first argument for \function{printf()} is a \textit{format string}: \texttt{\%s} specifies that a string should be placed at that position in the output, and \texttt{\%c} specifies that a character should be placed at that position in the output.

As you open your editor, the \textit{\texttt{\textbackslash}} key falls off the keyboard, preventing you from typing \texttt{\textbackslash t} and \texttt{\textbackslash n}.

\textit{Note: } the \function{sprintf()} function\footnote{\url{https://www.gnu.org/software/libc/manual/html_node/Formatted-Output-Functions.html\#index-sprintf}} is very much like the \function{printf()} function, except that it ``prints'' into a string.
The \function{strncat()} function\footnote{\url{https://www.gnu.org/software/libc/manual/html_node/Truncating-Strings.html\#index-strncat}} takes two string arguments and a \lstinline{size} argument.
It will place a copy of the second string at the end of the first, or it will place a copy of the first \lstinline{size} characters of the second string at the end of the first if the second string is longer than \lstinline{size}.

Edit \texttt{problem1.c} so that it produces the specified output without using the W key or the backslash key.
Build the executable with the command: \texttt{make keyboardlab1} -- be sure to fix both errors and warnings.

Run the executable with the command: \texttt{./keyboardlab1}.
This will print your email message.
If it matches the intended email message, then you will be informed tht it does.
If it does not match the intended email message, then the program will also print the intended email message so that you can compare the two to look for the differences.
If the two messages appear identical but are reported to be different lengths, then the difference might be trailing whitespace, spaces instead of tabs, or unprintable characters.

You can double-check that you aren't using the W key or the backslash key by running the constraint-checking shell script: \\
\texttt{./constraint-check.sh} \\
(If you get a ``\texttt{Permission denied}'' error message, then run the command \\
\texttt{chmod +x constraint-check.sh} \\
and then run the shell script.)

%You can check that your program has the correct output with this command: \\
%\texttt{./keyboardlab1 | diff - problem1oracle}
